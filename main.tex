%-------------------------------------------------------------------------------------
%Set up document class and basic formatting
%-------------------------------------------------------------------------------------

\documentclass[12pt]{report}
\usepackage[utf8]{inputenc}
\usepackage[a4paper, 
            vmargin = 18mm, 
            hmargin = 15mm, 
            bindingoffset = 25mm, 
            includehead,
            headheight = 10mm
            ]{geometry}
\usepackage{caption,subcaption}
\usepackage{setspace}
\usepackage{graphicx}
\graphicspath{{figure/}} % change path to make importing figures less annoying
\usepackage{fancyhdr}
\usepackage{placeins}
\usepackage{subfiles}
\usepackage{appendix}
\AtBeginEnvironment{appendices}{\crefalias{section}{appendix}}
%-------------------------------------------------------------------------------------
% Import packages to make physics, units, and fractions look pretty
%-------------------------------------------------------------------------------------

\usepackage{amsmath,amssymb,nicefrac}
\usepackage{siunitx}
\AtBeginDocument{\RenewCommandCopy\qty\SI}

\usepackage{physics}

\DeclareSIUnit{\pixel}{\text{pixel}}
\DeclareSIUnit{\fps}{\text{frames-per-second}}
\DeclareMathOperator{\Erfc}{Erfc}
\DeclareMathOperator{\Erf}{Erf}
\usepackage{centernot,cancel}


\usepackage{layouts}
%-------------------------------------------------------------------------------------
% Make the outline chapter have flexible numbers of subheading
%-------------------------------------------------------------------------------------
\usepackage{enumitem}

\setlistdepth{9}

\setlist[itemize,1]{label=\ding{108}}
\setlist[itemize,2]{label=\ding{110}}
\setlist[itemize,3]{label=\ding{117}}
\setlist[itemize,4]{label=\ding{107}}
\setlist[itemize,5]{label=$-$}
\setlist[itemize,6]{label=$*$}
\setlist[itemize,7]{label=$\ding{\bullet}$}
\setlist[itemize,8]{label=$\bullet$}
\setlist[itemize,9]{label=$\bullet$}

\renewlist{itemize}{itemize}{9}
\usepackage{pifont}
%-------------------------------------------------------------------------------------
%Set up bibliography, glossary etc.
%-------------------------------------------------------------------------------------
\usepackage{IEEEtrantools}
\usepackage[numbers]{natbib}
\bibliographystyle{IEEEtranN}

\usepackage[acronym,toc]{glossaries}
\usepackage{bibunits}
\usepackage{acronym}
\usepackage[capitalise]{cleveref}
\creflabelformat{equation}{#2\textup{#1}#3}
\usepackage{notoccite}
%-------------------------------------------------------------------------------------
% Wee function to draw a box around a region in an image and label it
%-------------------------------------------------------------------------------------

\usepackage{tikz}
\usetikzlibrary{matrix,calc}
%\annotatedFigureBoxCustom{bottom-left}{top-right}{label}{label-position}{box-color}{label-color}{border-color}{text-color}
\newcommand*\annotatedFigureBoxCustom[8]{\draw[#5,thick] (#1) rectangle (#2);\node at (#4) [fill=#6,thick,shape=circle,draw=#7,inner sep=1.5pt,font = \tiny\sffamily,text=#8] {\textbf{#3}};} % fiddle with these the shape of box, corner style, line thickness etc
%\annotatedFigureBox{bottom-left}{top-right}{label}{label-position}
\newcommand*\annotatedFigureBox[4]{\annotatedFigureBoxCustom{#1}{#2}{#3}{#4}{red}{white}{black}{black}}

\newcommand*\annotatedFigureText[4]{\node[draw=none, anchor=south west,text=#2, inner sep=0, text width=#3\linewidth,font=\sffamily] at (#1){#4};}

\newenvironment{annotatedFigure}[1]{
    \centering
    \begin{tikzpicture}
        \node[anchor=south west,inner sep=0] (image) at (0,0) {#1};
        \begin{scope}[x={(image.south east)},y={(image.north west)}]}
    {   \end{scope}
    \end{tikzpicture}}
\newcommand{\gridon}{\draw[help lines,xstep=.1,ystep=.1] (0,0) grid (1,1);
                    \foreach \x in {0,1,...,9} { \node [anchor=north] at (\x/10,0) {0.\x}; }
                    \foreach \y in {0,1,...,9} { \node [anchor=east] at (0,\y/10) {0.\y}; }
}

%-------------------------------------------------------------------------------------
% Wee function to have an abstract at the start of each chapter
%-------------------------------------------------------------------------------------
\usepackage{quoting}
% \newcommand{\chapabstract}[1]{
%     \begin{quote}
%         \singlespacing
        
%         \rule{\linewidth}{1pt}
%         #1
%         \vskip-4mm
%         \rule{\linewidth}{1pt}
% \end{quote}}

\newenvironment{chapabstract}
    {
        \quoting[leftmargin = 10mm,rightmargin = 15mm]
        \noindent\textbf{Summary.~}\itshape\ignorespaces
}
{\endquoting}

%-------------------------------------------------------------------------------------
%Put other packages here
%-------------------------------------------------------------------------------------


\usepackage{lipsum}
\newcommand{\addcite}{\textcolor{blue}{[citation needed]}}
\usepackage{algorithm}
\usepackage{algpseudocode}
\usepackage{tensor}
\usepackage[version=4]{mhchem}
\usepackage{multirow}
\usepackage{bm}

\setlength{\headheight}{42pt}
%-------------------------------------------------------------------------------------
%End preamble
%-------------------------------------------------------------------------------------
\begin{document}
\bstctlcite{IEEEexample:BSTcontrol}
%\bstctlcite{bibliography:BSTcontrol}
\pagestyle{fancy}
\fancyhead{}
\fancyhead[R]{\thepage}
\fancyhead[L]{\leftmark}

\newcommand{\thesistitle}{Spectral Fluorescence Lifetime Imaging \\ for the enhanced detection of retinal health}
\newcommand{\authorname}{Daniel Martin Geddes}
\newcommand{\quals}{MPhys}
\newcommand{\institution}{University of Glasgow}
\newcommand{\college}{College of Science and Engineering}
\newcommand{\school}{School of Physics and Astronomy}
\newcommand{\subdate}{October 2023}
\newcommand{\qualification}{Submitted in fulfilment of the requirements for the Degree of Doctor of Philosophy}

\begin{titlepage}
    \begin{center}
        \vspace*{1cm}
        \LARGE
        \textbf{\thesistitle}

        \vspace{2cm}
        \large
       
        \authorname\\
        \quals\\

        \vspace{1cm}
        \normalsize
        \qualification\\

        
        \vspace{1cm}
        \includegraphics[width = 0.25\textwidth]{figures/titlepage/UofG_Coat_of_Arms.png}\\
        \vspace{2cm}
        \institution\\
        \college\\
        \school\\
        \vspace{1.5cm}
        \subdate\\
        
    

    \end{center}
\end{titlepage}

\onehalfspacing
\chapter*{Abstract}
Early detection of retinal disease improves patient outcomes. Current imaging techniques rely on detecting structural damage in the retina after permanent damage to vision has occurred. Measurements of spectrally and temporally resolved fluorescence in the retina could enable the detection of retinal disease at the initial point of biochemical dysfunction - before patients report changes in their vision. Current publications focus on measuring qualitative changes in fluorescence lifetimes as retinal disease progress thus leaving a need for quantitative techniques for measuring metabolic biomarkers in the retina.\\
In this thesis, a new-generation SPAD-TCSPC array is utilised to construct the SFLIO device: a widefield imaging system capable of recording fluorescence lifetimes in the retina over 4 spectral bands optimised for photon-efficient unmixing of retinal biomarkers. Natural motion of the retina is compensated for and $2\times$ pixel super-sampling is demonstrated in an example of a moving \textit{Convallaria Majalis} sample mounted in an artificial eye model.
Due to high spectral and lifetime overlap of the retinal biomarkers FAD, AGE, and A2E, a new method is motivated and developed for recovering the relative concentration of retinal biomarkers from SFLIM measurements. In simulations this new technique outperforms existing lifetime unmixing techniques by multiple orders of magnitude and marginally outperforms existing spectral imaging techniques. This invites further study on the most efficient usage of available photons for enhancing chemical discrimination abilities.\\
From \textit{in-vivo} SFLIM measurements recorded of multiple anaesthetised rats it was found with high confidence $\big(p<10^{-6}\big)$ that a change in what is believed to be FAD is detected as a response to acute hypoxia. \textbf{While changes in FAD could not be mapped across the entire retina - demonstration of global changes of retinal FAD as a response to hypoxia shows promise for the SLIO device and SFLIM unmixing technique for quantifying metabolic health in the retina.}\\
To summarise, the SFLIO device demonstrated in this thesis shows promise a route towards quantitative measurements of retinal health and enabling deeper study in the the pathophysiology of retinal disease.

\chapter*{Acknowledgements}
Firstly, I'd like to thank my supervisor, Prof. Andrew R. Harvey for his continued support, and patience, throughout this research and the completion of this thesis. Beyond giving me the opportunity to carry out this research, he entertained my tendencies to turn everything into an overly convoluted mathematics problem, and mentored me to become the researcher I am today. \\
I feel incredibly fortunate to say that there are too many beloved family members and friends to thank here - their support helped me persevere through long evenings in the lab and laboured writing sessions. I'd like to thank my Mum, Carol, for continuously feeding my curiosities while I was growing up, and for her continued love and support throughout my undergraduate and postgraduate studies. 
To the friends that helped proof read this thesis and endured my unrelenting moaning and complaining about my research during days out rock climbing or trips to the pub - thank you for sticking by me. I am looking forward to having more time to spend with everyone of you.\\
I am also grateful to the funders of this research - OPTOS Plc, and CENSIS.



\tableofcontents
\listoffigures
\listoftables

\chapter*{Publications}

\begin{bibunit}
    \renewcommand{\bibsection}{}
    \makeatletter
    \renewcommand\@biblabel[1]{\ding{108}}
    \makeatother
    %\def\bibfont{\small} %might remove this
    \nocite{*}
    \putbib[pubs.bib]
    
\end{bibunit}

\chapter*{Declaration}
\subfile{chapters/declaration.tex}

% \chapter*{Outline}
% \subfile{chapters/outline.tex}

\chapter*{Glossary}\label{chap:glossary}
\subfile{chapters/acronyms.tex}


\chapter{Introduction - (*) Still Writing}\label{chap:intro}
\subfile{chapters/introduction.tex}


\chapter{Construction and Characterisation of an \textit{En-Face} Spectral Fluorescence Lifetime Imaging Ophthalmoscope (+) Done To Be Checked}\label{chap:fliodevice}
\subfile{chapters/sfliodevice.tex}

\chapter{Reconstructing Fluorescence Lifetime Images from a Moving Scene (+) Done To Be Checked}\label{chap:motionreg}
\subfile{chapters/motionreg.tex}
% \chapter{FLIM Analysis Methods}\label{chap:flimanalysis}
%\subfile{chapters/flimanalysis.tex}
\chapter{Appraisal of Techniques for Measuring Retinal FAD Concentrations (+ Done to Be Checked)}\label{chap:appFAD}
\subfile{chapters/appraisalFAD.tex}

\chapter{Development of a Tensor-based SFLIM unmixing technique for measuring retinal FAD concentrations (+) Done to be Checked}\label{chap:tensSFLIM}
\subfile{chapters/tensorSFLIM.tex}


% \chapter{Spectrally Resolved Fluorescence Lifetime Imaging (*) Need to Fix English}\label{chap:sflim}
% \subfile{chapters/sflim.tex}

\chapter{Detection of FAD in \textit{in-vivo} rat retinas (-) Under Construction}\label{chap:ratfad}
\subfile{chapters/ratFAD.tex}

\chapter{Novel Applications of TCSPC / Material Discrimination - (*) Need to add results and conclusions}\label{chap:materialdisc}
\subfile{chapters/tcspc.tex}

\chapter{Summary and Conclusions}\label{chap:conclusion}
\subfile{chapters/conclusion.tex}


\begin{appendices}
\appendix



\chapter{Safety Assessment for SFLIO instrument}\label{app:sfliosafety}
\subfile{chapters/appendices/Asflioassessment.tex}



% \chapter{Tensor Algebra for SFLIM unmixing}\label{app:tensoralgebra}
% \subfile{chapters/appendices/Btensoralgebra.tex}
\end{appendices}


\chapter*{Bibliography}
\nocite{*}
\renewcommand{\bibname}{References}
\bibliography{bibliography.bib}

\end{document}