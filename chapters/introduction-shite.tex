\section{Motivation}
The human eye is a complex organ capable of forming images over large and short distances as well as being sensitive enough to detect single photons. The eye also is susceptible to many diseases that when left undiagnosed and untreated cause irrecoverable loss of vision. In the case of Age Related Macular Degeneration (AMD) - a retinal disease that affects \SI{20}{\percent} of people - over a \num{12} month period the disease can progress from a patient first presenting with symptoms to permanent loss of some central vision~\cite{lim2012age}. Routine and regular retinal imaging is the most important tool in the early detection of retinal disease and for understanding the pathophysiology of the disease to inform more effective treatments. Retinal imaging modalities typically look at imaging structural abnormalities in the retinal surface with a wide Field-Of-View (FOV) and high spatial resolution. Current state-of-the-art devices are capable of recording images of the retina that cover \SI{82}{\percent} (\SI{200}{\degree} FOV) of the retinal surface (OPTOS Monaco, Silverstone) and using Optical Coherence Tomography (OCT) are able to resolve axial line scans of the retina layers with a \SI{7}{\micro\metre} resolution. These structural techniques currently aid opthalmologists in the diagnosis, and tracking the progression, of many retinal disease such as AMD, Stargardts disease, Macular Telegenticasia, and Diabetic Retinopathy, and retinal detachment. 
\\
Therein lies an issue with structural imaging - that when abnormalities are detected in the retina any present retinal disease has often sufficiently progressed to the point of limiting treatment options and, in some cases, at stage of some permanent vision loss. If instead the progression of disease can be tracked at the level of local biochemical imbalance, before any physical damage to the retina has occurred, then diseases can be more effectively diagnosed their progression can be better understood. 
Imaging fluroescent biomarkers gives a more direct look at the metabolic function of the retina. For example the endogenous biomarker FAD is linked to the local metabolic health in the retina, and Advanced Glycation End-Products (AGEs) are linked with the progression of diabetic retinopathy. This technique of fluorescence imaging - referred to as fundus autofluorescence imaging in literature - is widely used, qualitatively, in clinical ophthalmology for resolving reticular drusen, tears in the Retinal Pigment Epithelium (RPE), and retinitis pigmentosa~\cite{yung2016clinical}. Fluorescence imaging of the retina is hindered by several fundamental limitations. Namely, the low intensity threshold for safe exposure coupled with the feint signal produced from excited fluorophores of interest being polluted by fluorescent clutter results in high noise images with poor contrast and resolution (when compared to wide-field reflectance images). This dominant fluorescent clutter also degrades the ability to quantify the concentration of individual fluorophores and map their distribution across the retina even with advanced spectral imaging techniques.
\\
Fluorescence Lifetime Imaging (FLI) is a relatively new field of microscopy where the picosecond - nanosecond fluorescence lifetime - the average time a fluorophore remains excited - is used as a contrast medium to form images similar to how the number of detected photons is used in conventional fluorescence intensity imaging. FLI brings unique advantages in that the fluorescence lifetime is intrinsic to each fluorophore but is also invariant to the local fluorophore concentration and quantum efficiency of the fluorophore but can also reveal information about the local pH, temperature, and the local embedding matrix in cellular organisms. This in principle means that weakly fluorescent fluorophores of interest can be isolated from complex mixtures but the noise induced by low photon counts inherent to autofluorescence imaging affect the accuracy and precision of this unmixing process.
To overcome the low signal produced by endogenous fluorophore, in practice, FLI is commonly performed on samples labelled with dyes with the fluorescence lifetime, excitation and emission spectra, and the binding site specifically engineered for each application. This enables more complex dynamics and processes to be investigated such as staining specific cellular structures~\cite{maibohm2019syncrgb}, measuring viscosity and tension in tissues~\cite{ringer2017multiplexing, kashirina2020monitoring}, probing intramolecular distances and interactions using F\"orster Resonance Energy Transfer~\cite{haenni2013intramolecular}, and volumetric super-resolution imaging~\cite{shtengel2009interferometric}.
\\
In retinal imaging the emergent field of Fluorescence Lifetime Imaging Ophthalmoscopy has begun investigating the potential applications that fluorescence lifetime imaging could have in the diagnosing of retinal disease. In the literature a modified SLO records arrival times of photons from the excited fluorescence over a period of \SI{90}{s} are imaged onto two single pixel Single Photon Avalanche Diode's (SPAD), to cover two separated spectral bands, using a technique called Time Correlated Single Photon Counting~\cite{dysli2017fluorescence}. The fluorescence lifetimes are then extracted from histograms of photon arrival times to construct a map of fluorescence lifetimes across the retina. With this device the connection between measured fluorescence lifetimes and retinal disease progression was explored by qualitatively comparing lifetime maps of age and gender matched healthy patients with patients exhibiting a retinal disease such as AMD, Diabetic Retinopathy, and Stargardts disease~\cite{zinkernagel2019fluorescence}. These \textit{in-vivo} studies focus on qualitative comparisons in part due to the relative youth of the field as well as the poor understanding of the specific biochemical imbalances that occur in the retina, and the body, at the onset of retinal disease.
\\
A key deficiency in current retinal imaging modalities is the ability to quantitatively assess retinal health. Previously, measurements of vascular oxygen concentrations in retinal vasculature have been reported~\cite{stefansson2019retinal,mordant2011spectral} - using the spectral properties of oxygenated and un-oxygenated haemoglobin - however the uncertainties in the measurements of $O_{2}$ concentrations are larger than the variability that would typically be experienced throughout the progression of a retinal disease, or chronic disease such as Multiple Sclerosis. There is potential for the added dimension of the fluorescence lifetime to facilitate a similar quantitative approach to monitoring retinal metabolic health but the efficient use of the available signal will be paramount for ensuring this technology can be feasibly implemented into current diagnostic practises.

In this thesis new methods of quantitatively assessing retinal and metabolic health will be explored using temporally, and/or spectrally resolved fluorescence images of the retina recorded with a modified high-street fundus camera retrofitted with new generation SPAD arrays to pave a route to lower cost regular monitoring retinal metabolic health. In addition, other novel applications of TCSPC are explored. Specifically, the material dependent temporal broadening experienced by a reflected laser pulse is explored for the purpose of discriminating materials in a scene.


Fluorescence is radiative process where photons are emitted from a substance due to the excitation and subsequent relaxation of electronic singlet states in a substance. This process is commonly visualised using a Jablonski diagram (Fig.~\ref{figintro:jablonski}):

\begin{figure}[htbp]
    \centering
    \includegraphics[width = 0.8\textwidth]{figures/introduction/JablonskiDiagram.pdf}
    \caption{Jablonski diagram of the mechanics behind fluorescence. $S_{0}, S_{1}, S_{2}$, denoted by bold lines, represent the ground and the first two excited singlet states. Non-radiative processes such as internal conversion between vibrational states (grey dashed) and inter-system crossing (purple) allow for the emission of a fluorescence photon ($S_{1}\rightarrow S_{0}$, shown in green, or a phosphorescence photon ($ S_{1}\rightarrow T_{1}\rightarrow S_{0} $), shown in red}
    \label{figintro:jablonski}
\end{figure}
More specifically, absorption of impinged photons promotes a electron from the ground electronic singlet state, $S_{0}$, to a vibrational state within a higher electronic state, $S_{1},S_{2}$ etc. Non-radiative internal conversions relax the fluorophore back into the $S_{1}$ state. The fluorophore can then decay back into the $S_{0}$ state emitting a fluorescence photon or undergo the process of inter-system crossing to the first triplet state, $T_{1}$ and then subsequently decay back to the ground state producing a phosphorescence photon.
These radiative and non-radiative processes occur over different time scales: absorption occurs over \SI{e-15}{s}; internal conversion and intersystem crossing occur over \SI{e-12}{\second}; phosphorescence occurs over times scales from \qtyrange{e-7}{e-3}{\second}; and of key interest in this thesis, fluorescence occurs over a time range of a few nanoseconds~\cite{lakowicz2013fluorescencespectroscopybook}. 
Since fluorescence is random process - the excited state isn't depopulated at the exact same time - it is more useful to consider the average time that
the fluorophore remains in the excited state or the fluorescence lifetime. This fluorescence lifetime is intrinsic to the fluorophore and is described in terms of the concentration of fluorophores in the excited state, $S_{1}$:
\begin{equation}
    \frac{d}{dt}[S_{1}] = -\Gamma [S_{1}]
    \label{eqintro:conc}
\end{equation}
Where $\Gamma$ is the rate at which fluorescence photons are produced and is the inverse of the fluorescence lifetime ($\Gamma = \nicefrac{1}{\tau}$). By then associating the measured time evolved intensity of with the concentration $[S_{1}]$ the fluorescence decay can be modelled:
\begin{align}
    \frac{d}{dt}I(t) &= -\Gamma I(t)\\
    \implies I(t) &= I_{0}\exp(-\nicefrac{t}{\tau})
    \label{eqintro:lifetime}
\end{align}
For mixtures of fluorophores, and fluorophores with specific molecular properties, multi-exponential decays with mutliple fluorescence lifetimes are exhibited:
\begin{equation}
    I(t) = I_{0}\sum_{n>0}^{N}\alpha_{n}\exp(-\nicefrac{t}{\tau_{n}})
    \label{eqintro:multiexp}
\end{equation}
