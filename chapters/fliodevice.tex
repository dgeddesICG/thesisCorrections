%\setstretch{2}
\section{Chapter Summary}
\section{Construction of \textit{En-face} S-FLIO system}
\FloatBarrier
\subsubsection{S-FLIO Imaging System}
As was outlined in \cref{sec:objresearch} the purpose of this project is to use the principles of fluorescence lifetime imaging and spectral fluorescence imaging to further develop capabilities of chemical sensing in the retina. This will be achieved using an off-the-shelf fundus camera modified to facilitate spectrally-resolved fluorescence lifetime measurements of the retina within the constraints that the device can image an FOV of at least \SI{20}{\degree}. While this field-of-view is generally considered to be narrow in ophthalmic imaging\footnotemark this device is to serve as an early prototype to establish the efficacy this new imaging technique however if a larger field-of-view is required multiple exposures across the retina can be recorded and stitched together. The total image acquisition time is also a key constraint in designing a retinal imaging device for routine assessments of retinal health and is fundamentally limited by the safe exposure levels of the illumination source and thus creating a photon starved regime.
In the proposed SFLIO device the spectrally-resolved fluorescence lifetime measurements will be optimised using a wavelength tuneable excitation source to efficiently target key retinal fluorophores such as FAD, AGE, and A2E as well as new generation SPAD arrays which record these S-FLIM measurements in a series of snapshots. 
\footnotetext{The OPTOS California OPTOS 200tx retinal imaging systems can record images with a \SI{200}{\degree} FOV equivalent to \SI{82}{\percent} coverage of the retinal surface}

\FloatBarrier
\subsection{Conventional Fundus Camera}
The S-FLIO device developed over the course of this project is based upon an off-the-shelf fundus camera (Topcon TRC 50-DX) used routinely in ophthalmic examinations reduces the complexity of the design process in that the project was now focused on fluorescence lifetime imaging and chemical quantification of the retina rather than arduous process and optimising and constructing as retinal imaging system~\cite{dehoog2008optimal}. Further, by utilising an existing imaging ophthalmoscope the SFLIO system developed can be more readily adapted to a commercial product in line with OPTOS's existing product line where it could be sold as a standalone imaging system or as a modification or add-on to existing fundus cameras. 
The fundus camera is ordinarily capable of recording images over three different field of views - \qtylist{20;35;50}{\degree} and the ability to compensate for various for different severity of short and long-sightedness through a X to Y diopter correction. The device uses a Xenon flash lamp for white light with the necessary filters to record images in 3 different modalities: brightfield or reflectance imaging; autofluorescence imaging; and fluorescence imaging using an exogenous dye such as fluorescein or idocyanine green in retinal angiography. Alignment of the fundus camera to the patients eye is achieved using a lower intensity Tungsten inspection lamp to illuminate the retina and two small green LED's that project a pair of dots onto the patients cornea which come into focus when the fundus camera is positioned correctly in the axial direction. For both illumination sources - the Xenon flash lamp and Tungsten inspection lamp - the retina is illuminated using the concept of annular illumination, shown in \cref{fig:annularillumination}, where the retina is illuminated using only the periphery of the pupil. Using this annular ring almost all of the pupil is usable for imaging through and this forms the basis of the Gullstrand Principle whereby minimising the overlap between imaging and illumination volumes reduces the influence of back-reflected illumination light - increasing image quality~\cite{gullstrand1910neue}.
\begin{figure}
    \centering
    \includegraphics[width = 0.7\textwidth]{figures/sflio-device/AnnulusfigureTEMP.jpeg}
    \caption{The annular illumination system of the fundus camera evenly illuminates the retina through the periphery of the pupil allowing the overlap of imaging and illumination volumes to be minimised.}
    \label{fig:annularillumination}
\end{figure}
The modifications made to the existing optics to facilitate excitation of retinal fluorescence and the spectral fluorescence lifetime measurements are described in the proceeding two sections in terms of the changes made to the illumination arm and the imaging arm of the fundus camera.
\FloatBarrier
\subsection{Modification - Illumination Arm}
In the S-FLIO device, retinal fluorescence is excited by a pulsed-supercontinnum white-light source (NKT SuperK Ex-12W and Fianium) was paired with a Acousto-Optically Tuneable Filter (AOTF) to be filter the broadband emission into up to 8 separate channels each with a FWHM of $\approx\SI{1}{\nm}$ which enhances abilities to target specific retinal fluorophores using their unique excitation spectra. The fibre output of this source is coupled into the existing illumination path of the fundus camera by replacing a filter wheel - ordinarily used for spectral and fluorescence imaging - with a long-pass dichroic beam-splitter, positioned a \SI{45}{\degree}, to reflect excitation light with a wavelength shorter than \SI{505}{\nm} into the illumination path while also retaining the ability to use the inspection lamp for alignment of targets albeit now filtered to remove wavelengths shorter than \SI{505}{\nm}. To accommodate the differing Numerical Aperture (NA) of the fibre output of the laser a 1'' diameter plano-convex lens with focal length of X to first collimate the beam, and a second lens with focal length Y is then used to refocus the beam to a point at the perimeter of the clear portion first annular stop, A1 of \cref{fig:sfliooptics}, which is conjugate to the annulus projected onto the cornea and culminates in the retina being illuminated with a $\SI{20}{\degree}$. While this design approach reduces the illuminated field-of-view on the retina minimises geometric losses arising from the multiple annular stops blocking the central lobe of the laser's Gaussian beam. While simply using a laser with a higher output power would mitigate these geometric loss it would necessitate more rigorous fail-safe mechanisms in the event of any damage to the annular stops causing unsafe exposure levels.
\begin{equation}\label{eq:numericalaperture}
    NA = n\sin{\Bigg(\arctan{\bigg(\frac{D}{2f}\bigg)}\Bigg)}
\end{equation}
\begin{figure}
    \centering
    \includegraphics[width = 0.5\textwidth]{figures/sflio-device/FundusCameraOpticDiagram.pdf}
    \caption{Optical diagram of the SFLIO device where the principle rays of the illumination and imaging paths are rendered as the green and cyan rays, respectively. The modifications made to the existing optics of the fundus camera to facilitate spectral fluorescence lifetime imaging is shown in blue highlighted region and the new illumination optics are shown in the red region. Reproduced from \citeauthor{shibatafundus2003}~\cite{shibatafundus2003} where the additional optics . \textcolor{blue}{Annotate annular stops, add rays for new illumination optics, and label lens ive added with $f_{x}, f_{y} etc$}}
    \label{fig:sfliooptics}
\end{figure}
Previous design iterations, shown in \cref{fig:IlluminationV1},used a similar design methodology but instead the Xenon arc-flash lamp was replaced with a $\nicefrac{1}{2}''$-diameter dichroic beam-splitter with a cut-on wavelength of \SI{505}{\nm} to couple the laser source into the illumination path. The collector lenses, positioned in front of the Xenon lamp, used in the illumination path exhibit a high NA of \num{0.885} - having a diameter of \SI{29.5}{\mm} and focal length of \SI{7.75}{\mm} - and to couple the laser light into this illumination path a pair of $\nicefrac{1}{2}''$-diameter plano-convex lenses with focal length \SI{12.7}{\mm}, after being collimated by a \SI{50.8}{\mm} focal length lens, were selected to match this numerical aperture. This design was rejected due to the severe geometric losses due to the aforementioned annular stops removing the majority of the energy in the Gaussian Beam. Further, this implementation resulted in irregular illumination of retina (\cref{fig:IlluminationV1}b) and degraded the image quality achieved using both the super-continuum source and the inspection lamp arising from the convergence angle of the beam focused being larger than \SI{45}{\degree} and thus being cropped by the dichroic mirror.


\begin{figure}
    \centering
    \begin{annotatedFigure}{\includegraphics[width = 0.85\textwidth]{figures/sflio-device/IlluminationV1fFigure.pdf}}
    \annotatedFigureText{0.00,0.87}{black}{0.3}{a)}  
    \annotatedFigureText{0.40,0.87}{white}{0.3}{b)}
    \annotatedFigureText{0.72,0.87}{white}{0.3}{c)}
    \annotatedFigureText{0.40,0.36}{white}{0.3}{d)}
    \annotatedFigureText{0.76,0.38}{white}{0.3}{e)}
    \end{annotatedFigure}
    \caption{Initial attempt of coupling the supercontinuum excitation source into the existing illumination path of the fundus camera. The pulsed supercontinuum source is coupled into the existing optics using a $f_{1}$ to collimate the fibre output and a pair of plano-convex $f_{2,3}$ to focus this beam onto a dichroic mirror with the principle rays shown in a). The high NA of the fundus camera optics resulted in an irregular illumination pattern on an mechanical eye model, c) when compared to an image of the same mechanical eye illuminated by the existing Xenon flash lamp. In d) and e) the optics are shown attached to the 3D printed mount and integrated into the fundus camera. }
    \label{fig:IlluminationV1}
\end{figure}

% \begin{itemize}

%     \item When considering designs of the illumination arm the annular illumination system was identified as a key feature to retain for recording high quality images and maximising the efficiency of injecting the excitation source into the existing imaging path
%     \item Diagram of rough illumination arm starting at the inspection lamp to the annular mirror and the objective lens.
%     \item In integrating additional optics into the existing illumination path there was significant difficulty because of the multiple aperture stops positioned conjugate to the surface of the cornea which reject light reflected from the cornea being imaged onto the detectors. The mechanism is that by not having light focused onto the cornea, it cannot be re-imaged.
%     \item Two possible locations and design methodologies were explored to mount the necessary optics to inject the excitation light
%     \begin{itemize}
%         \item A pair of plano-convex lenses to collimate the diverging light emitted from the lasers fibre output and then focus light onto a half inch dichroic beam splitter positioned 45 degrees to the direction of the beam path. By matching the NA of the existing Xe flash lamp the original field of view can be retained with minimal disruption to surrounding optics.
%         \item It was found that due to the high NA (>1.2) of the arc lamp and the multiple annular stops this resulted in significant vignetting of central and peripheral portion of the lasers Gaussian beam profile - inducing severe geometric losses, significantly reduced field of view using both illumination sources and significantly occluding the inspection lamp. Geometric losses meant that even forming a fluorescence intensity image of an artificial eye was not achievable without long acquisition times > 2s.
%         \item The illumination concept was then redesigned such that now a larger dichroic mirror could be placed at the location of the existing filter wheel where the beam profile of the existing illumination was slowly converging - allowing greater flexibility with the positioning of the optics. Further, the FOV of the illuminated area by the laser was reduced with the view of maximising the available laser power. A consequence of this design meant that a pair of aperture stops were to be removed which although would harm image quality in the bright-field however for fluorescence this is mitigated by the emission filter attenuating any corneal back reflections at the excitation wavelength leaving an insignificant amount of fluorescence originating of the cornea.
%         \item optical diagram of the setup.
%     \end{itemize}
% \end{itemize}
\FloatBarrier
\subsection{Modification - Imaging Arm}
The imaging arm of the SFLIO uses a 
\begin{itemize}
    \item for the imaging arm a 90:10 beamsplitter is used in conjunction with 3 commercial SLR lenses to relay the image produced by the fundus camera to a Thorlabs Kiralux 8.9MP sensor which will be used to record images in either brightfield or in fluorescence for co-registration purposes and the Horiba FLIMera to record FLIM images. A linear cassette style filter wheel is fitted either in front of the FLIMera for brightfield imaging with the sCMOS detector or 
    \item table with specs of both detectors
    \item SLR lenses were chosen to maximise image fidelity - having a flatter spatial frequency response - simplifying the optical design process and allowing for easy fine tuning of the focus, and aperture.
    \item magnification and focal lengths were based on calculations in second year progression report. 
    \item The image is relay with a Vivitar f = 80mm lens, the SPAD is focused using a Nikon f = 50mm lens, the sCMOS is focused using a Cannon 135mm. 
    \item the Nikon lens is attached to the SPAD using Cmount threads tapped into a 3d printed mount that attaches to existing m3 threaded mounting holes on the FLIMera 
    \item Diagram showing the relative FOV on an image of the eye.
    \item The Horbia FLIMera SPAD array
    \begin{itemize}
        \item Horiba FLIMera is an array of 192 X 126 single pixel SPADs with on-pixel TCPSC module. Each pixel detects the photon arrival times with a 47ps resolution and correlates it to a single laser pulse over a window of 12.5ns for a 80MHz rep rate. The architecture of the SPAD is based on the Henderson SPAD cmos process where each pixel is separated into the two sectors - the photo activate area; and the counting electronics. This pixel design offers scaleability at the cost of fill factor. The FLIMera used in this SPAD has a fill factor of 13 percent but this can be increased to XX percent using a microlens array designed to focus all light that hits the pixel onto the photo active area.
        \item FLIMera acquisition modes
        \begin{itemize}
            \item Standard h5 - mode produces a file which contains an XYT data cube which span's the entire acquisition period, the integrated intensity, the FWHM of each exponential decay, and lifetime image obtained using a simple single exponential fit.
            \item RAW h5 mode - each individual photon event is recorded into a single stream and saved as a 4 byte word containing either the photon arrival coordinates - X,Y, and microtime - or a frame marker which enables full XYT data cubes to be constructed at 24khz. This mode also results in larger file sizes which grow larger for brighter samples. ex a 2 min exposure required 20GB of storage space while also requiring more intensive post processing as is described in further sections.
        \end{itemize}
        \item Because the FLIMera is still pre-production there are malfunctioning pixels which either present as dead pixels - give not histogram - and screamers which appear to have an gain applied to the histogram but is discounted from analysis due to changes in the photon statistics and also pixels which produce a signal that resembles a step function. These malfunctioning pixels are masked out using a FLIMage where there is no light source - a dark image - and a FLIMage of a scene with a flat background - orange fluorescent microscope slide. Dead pixels are detected by masking the pixels more than X standard deviations lower than the mean photons. Screamers are masked in a similar way from the dark image by masking the pixel more than Y standard deviations above the mean. Produces a mask which captures the majority of the malfunctioning pixels - the rest can be removed through thresholding any other image.
        \item Figure containing integrated intensity images of the dark frame, fluorescent slide, and the resulting mask.
        \item Additionally the pre-production nature of the FLIMera also results in each pixel not being aligned in time i.e the peak of the exponential decay is not in the same time bin across the entire array.   
    \end{itemize}
\end{itemize}
\FloatBarrier
\subsection{Safety Assessment}
\begin{itemize}
    \item The photon flux recorded in FLIMages is ultimately limited by not only the quantum efficiency of the retina in terms of producing fluorescence but also the limit of safe exposure in the retina. 
    \item There are two predominant damage mechanisms in the eye - the phototoxic effects, and photochemical effects.
    \item Phototoxic damage is caused by over exposure of short wavelength light which generates free radicals in the retina. The immune system then either sees these as foreign and removes the corrupted cells or the in extreme cases the photorecepots are over stimulated / absorbed too much energy and die directly.
    \item Photothermal damage occurs with long wavelength light whereby localised heating of the humours located at the retina surfaces cause acute cavitation destroying the retinal surface. 
    \item While human imaging was not achieved in the course of this project a full safety assessment was carried out to determine the safe exposure limits for the SFLIO device and inform the potential image integration times - which will only improve with future generations of SPAD arrays.
    \item To obtain 
\end{itemize}
\FloatBarrier
\section{Evaluation of Performance}
\subsection{Spectral Calibration}

% \begin{itemize}
%     \item We need good spatial resolution, and signal to noise ratio in the high resolution camera to account for movement of the eye
%     \item SPAD has much fewer pixels and much larger spatial resolution (influenced by the irregular sparse array architecture) and there is a magnification difference so as long as spatial res of the registration camera is 5-10X the FLIMera and we can register with a 5 pixels then our registration will be limited by he quantisation of the digital imaging system
%     \item The purpose of this device is to proof of concept the measurement of FAD in the retina using lifetimes - not construct a diffraction limited system.
%     \item to quantify the spatial resolution of the system we can use the knife edge test consisting of a fluorescent slide (gives a uniform scene) that has a slanted hard edge places in front of it all housed in an artificial eye
%     \item calibration is achieved by replacing the edge with an Allen key with known diameter. The number of pixels between the parallel edges yield the pixel to um conversion factor i.e the width and height that a single pixel subtends on the retina.
%     \item this test evaluate the spatial resolution by averaging an aliased edge (in the case of the FLIMera - pixels are small enough on the FI camera that optics is the limit not quantisation) to yield the Edge Spread Function. Differential of the ESF is the LSF and the spatial resolution is defined as the FWHM of this LSF after a Gaussian has been fitted.
%     \item This was repeated through multiple foci to ascertain where the best focus is and what is its spatial resolution
%     \item on the FI camera the spatial resolution is 40um and for the FLIMera it is 400um.
%     \item This resolution is larger than what was previously reported in the Harvey, McNaught artificial eye paper (spatial res was $\approx 15$um) which uses the same artificial eye and an optically similar imaging system (fundus camera with SLR lenses in a imaging relay configuration to focus the image onto a scientific camera.
%     \item The system is optically limited but the source is likely to originate in misalignment of the SLR lenses, other optical elements in the beam path etc. 
%     \item To attempt to ascertain the source of these issues the fundus camera was set up to image the room with some targets attached to the wall at infinity and optical elements (emission filter, beam splitter) and the spatial resolution was estimated using the rise-time method
%     \begin{itemize}
%         \item spatial res $\approx$ width in pixels between 90 and 10 percent of the fast edge
%     \end{itemize}
%     \item this was then compared to two much simpler imaging systems - and iPhone 13, and the thorcam with an SLR lens attached set to the same scene.
%     \item CONCLUSION: spatial resolution is X and this is sufficient for co-registering images with the FLIMera. There are multiple sources of aberration in the system that can be harming our spatial res that could be eliminated given the a significant time investment but since a good signal to noise ratio is favoured over diffraction limited (or near it) this is sufficient.
% \end{itemize}
\subsection{Spatial Resolution of S-FLIO device}
While the purpose of this thesis is not to construct a diffraction limited retinal imaging system but to record fluorescence lifetime images and quantitatively measure the distribution of fluorophores across the retina - a spatial resolution, with respect to images recorded on the retina with the FLIMera, sufficient to resolve features, and markers, common with retinal disease is key. For example, as has been discussed in \cref{intro:AMD}, the average diameter of drusen, a common marker of disease progression in AMD, is on the order of YY - to resolve drusen with the FLIMera the spatial resolution of the S-FLIO system must be lower than this value. Additionally, the spatial resolution of both detectors will influence the accuracy required when reconstructing images from fluorescence histograms of a retina undergoing motion. For a proportional difference in the spatial resolution $\nicefrac{\Delta r_{FLIMera}}{\Delta r_{sCMOS}} \leq M$ will require registration of images within a sub-pixel accuracy (with respect to the sCMOS detector) whereas for $\nicefrac{\Delta r_{FLIMera}}{\Delta r_{sCMOS}} > M$ the registration accuracy requirement is greater than a single pixel.
\subsubsection{Spatial Resolution vs. Pixel Size}
In a sampled imaging system the spatial resolution is not simply dictated by the size of a single pixel on the detector but instead how accurately the imaging system and detector can describe a scene e.g the retina. Due the optical characteristics of the eye and the aberrations inherent with the fundus camera and the additional optics required for fluorescence lifetime imaging as well as the sampled nature of the system this will be imperfect. Mathematically the response of the imaging system is denoted as Point-Spread Function (PSF) which describes how a point source in a scene is blurred in a recorded image. 
\begin{equation}\label{eq:psf}
    I_{IMAGE}(x,y) = PSF(x,y)\circledast I_{SCENE}(x',y')
\end{equation}
Additionally in the Fourier domain the Optical Transfer Function (OTF) or the approximately equivalent Modulation Transfer Function (MTF) can be used to described how well an imaging system resolves increasing spatial frequencies~\cite{vollmerhausen2000analysis}. Here the OTF and MTF are connected to the PSF by a Fourier transform.
\begin{align}
    J_{IMAGE}(\nu,\mu) &\approxeq MTF(\nu,\mu)\circledast K_{SCENE}(\nu',\mu')\label{eq:mtf}\\
    MTF(\nu,\mu) &= \mathcal{F}\{PSF(x,y)\}(\nu,\mu) \label{eq:PSFtoMTF}\\
    MTF(\nu) &= \mathcal{F}\{LSF(x)\}(\nu)\label{eq:LSFtoMTF}
\end{align}
Conventionally, the spatial resolution of an imaging system is defined by the full-width at half-maximum of the PSF or the spatial frequency where the MTF is half its maximum value. An example of a image blurred by a PSF is shown in \cref{fig:PSFsimulation} alongside its respective MTF.
\begin{figure}
    \centering
    \begin{annotatedFigure}{\includegraphics[width = 0.95\textwidth]{figures/sflio-device/PSFMTFexample.pdf}}
    \annotatedFigureText{-0.02,0.94}{black}{0.3}{a)}
    \annotatedFigureText{0.235,0.94}{black}{0.3}{b)}   
    \annotatedFigureText{0.495,0.94}{black}{0.3}{c)}   
    \annotatedFigureText{0.735,0.94}{black}{0.3}{d)}   
    \end{annotatedFigure}
    \caption{Here a \qtyproduct{512 x 512}{\pixel} sampled image, a) has been convolved with a Guassian PSF with a FWHM of 16 pixels c) to produced the degraded image in b) where the fine details in the background of the image can no longer be clearly resolved. The resulting MT, determined using \cref{eq:PSFtoMTF}, in d) shows the significantly reduced contrast at higher spatial frequency's due to the PSF.}
    \label{fig:PSFsimulation}
\end{figure}
\FloatBarrier
\subsubsection{Knife Edge Test}
Accurately measuring the MTF of an imaging system is often a onerous and time consuming task where standardised best practises are continually refined~\cite{ISO12233}. For the purpose of this project only estimates of the MTF are required and so the knife-edge test - often referred to as the slanted-edge test in literature - was chosen for its simplistic experimental procedure~\cite{burns2000slanted, ISO12233}. Briefly, this method allows the measuring of spatial resolution of a digital imager using an image recorded of a hard-edge positioned in the image plane and rotated a few degrees ($<\SI{10}{\degree}$) either off-vertical or off-horizontal where in detector limited imaging systems this edge appears aliased. Line profiles through the aliased edge are then aligned to produce an Edge-Spread Function (ESF) and the spatial resolution of the system can then be estimated from the LSF (\cref{eq:ESF}) or from the computed MTF. The rotation of the edge in the image plane enables the ESF to be sampled with sub-pixel precision due each line profile - through the edge - being displaced by less than a single pixel.
\begin{equation}\label{eq:ESF}
    ESF(x) = \dv{}{x}LSF(x)
\end{equation}
In the S-FLIO system this Knife edge test is performed simultaneously on both detectors using an eye phantom where the edge is comprised of a scalpel blade and fluorescent microscope slide (Thorlabs - FSK4) positioned at the retinal plane. The emission spectra of the fluorescent slide allows exciting fluorescence at \qtyrange{460}{462}{\nm} and imaging above \SI{500}{\nm} with sufficient photon flux to record high quality images in both the FLIMera and the sCMOS camera with integration times of \SI{250}{\ms} and \SI{5}{\second}, respectively. 
This process is shown in~\cref{fig:FIMTFresult,fig:FLMTFResult}, where in the analysis of the knife edge images, the location of the edge within the image is first estimated by finding the peak of the absolute value of the derivative of lines-profiles taken through the edge. Using a linear fit through these edge locations the line-profiles are aligned and binned into \SI{0.25}{\pixel} intervals to yield the ESF. As above, the LSF and MTF can now be computed from the ESF. The LSF and MTF for both detectors are then calibrated using a image recorded of an  0.05'' (\SI{1.27}{\mm}) Allen key positioned in the phantom eye where it was found one pixel on the FLIMera and sCMOS detector equated to \SI{66}{\um} and \SI{6.3}{\um} respectively. To determine the spatial resolution of the system a Gaussian is fitted through each LSF in order to reduce the influence of noise and effects of quantisation and the FWHM is calculated. For the FLIMera the spatial resolution was found to be \SI{214}{\um} and for the sCMOS detector this found to be \SI{50}{\um}. 

\begin{figure}
    \centering
    \begin{annotatedFigure}{\includegraphics[width =\textwidth]{figures/sflio-device/FIMTFFigure.pdf}}
    \annotatedFigureText{0.025, 0.965}{black}{0.3}{a)}
    \annotatedFigureText{0.66,0.965}{black}{0.3}{b)}
    \annotatedFigureText{0.66,0.67}{black}{0.3}{c)}
    \annotatedFigureText{0.025,0.39}{black}{0.3}{d)}
    \annotatedFigureText{0.35,0.39}{black}{0.3}{e)}
    \annotatedFigureText{0.66,0.39}{black}{0.3}{f)}
    \end{annotatedFigure}
    \caption{From the recorded knife edge image using the sCMOS detector (a) the edge is detected and fitted (b) using the gradient of the edge profiles, shown in (c). The edge profiles are aggregated to form the a super sampled ESF (d). The LSF is then determined from the ESF using Eq.~\ref{eq:ESF} and the spatial resolution of \SI{214}{\um} is determined from the FWHM of a fitted Gaussian (black line). The MTF, shown in (f), is then calculated from this LSF using Eq.~\ref{eq:LSFtoMTF} and is plotted with the diffraction limit of the eye.}
    \label{fig:FIMTFresult}
\end{figure}

\begin{figure}
    \centering
    \begin{annotatedFigure}{\includegraphics[width = 1\textwidth]{figures/sflio-device/FLKnifeEdgeFigure.pdf}}
    \annotatedFigureText{0.03,0.94}{black}{0.3}{a)}
    \annotatedFigureText{0.33,0.94}{black}{0.3}{b)}
    \annotatedFigureText{0.7,0.94}{black}{0.3}{c)}
    \annotatedFigureText{0.03,0.48}{black}{0.3}{d)}
    \annotatedFigureText{0.33,0.48}{black}{0.3}{e)}
    \annotatedFigureText{0.7,0.48}{black}{0.3}{e)}
    \end{annotatedFigure}
    \caption{From the recorded knife edge image using the FLIMera (a) the edge is detected and fitted (b) using the gradient of the edge profiles, shown in (c). The edge profiles are aggregated to form the a super sampled ESF (d). The LSF is then determined from the ESF using Eq.~\ref{eq:ESF} and the spatial resolution of \SI{50}{\um} is determined from the FWHM of a fitted Gaussian (black line). The MTF, shown in (f), is then calculated from this LSF using \cref{eq:LSFtoMTF} and is plotted with the diffraction limit of the eye.}
    \label{fig:FLMTFResult}
\end{figure}

To then assure that the spatial resolution was assessed at the optimal focus setting on the fundus camera the knife-edge test was repeated a what was initially deemed the best focus and then 3 additional points either side of this focus spaced in $\approx\SI{10}{degrees}$ intervals and the spatial resolution was calculated for each of the 7 points for both detectors. The results shown in \cref{fig:mtffoci} that in fact the optimal focus was actually located $\approx\SI{10}{degree}$ from what was visually determined to be the best focus. Giving the previously mentioned spatial resolutions of \SI{214}{\um} for the FLIMera and \SI{50}{\um} for the sCMOS detector at this new optimal focus and at both extremum's the the spatial resolutions are \SI{296}{\um} and \SI{328}{\um} for the FLIMera, and for sCMOS they are \SI{129}{\um} and \SI{152}{\um}. In the experiments described in later sections of this chapter the focus set to this new optimal focus with only minor adjustments being made to account for variations in the position of the eye phantom with respect to objective lens of the fundus camera.

\begin{figure}
    \centering
    \includegraphics[width = 0.8\textwidth]{figures/sflio-device/MTFFociFigure.pdf}
    \caption{The knife-edge test was repeated over a total of 7 different focii's centred on what was originally deemed to be the best focus and spaced $\approx\SI{10}{\degree}$ apart. The spatial resolution for each focii and detector was calculated and a degree 2 polynomial was fitted.}
    \label{fig:mtffoci}
\end{figure}
\paragraph*{\textbf{Appraisal of Spatial Resolution}\\}
While the spatial resolution of both detectors is sufficient to co-register retinal images, the measured value of the sCMOS detector significantly deviates from that of a similar system where an off-the-shelf fundus camera (Canon CF-60Z) was used to image the same eye phantom housing capillary tubes filled with blood for the purposes of measuring blood oxygenation in the retina~\cite{mordant2011validation}. In~\citeauthor{mordant2011validation} the spatial resolution of the system was estimated to be $\approx\SI{30}{\um}$ or $\approx\SI{3}{\pixel}$ by examining line profiles through an image of a \SI{150}{\um} outer-diameter capillary tube and computing the rise time - the number of pixels required to transition from \SI{10}{\percent} to \SI{90}{\percent} of the peak grey value within the capillary.

\begin{figure}
    \centering
    \begin{annotatedFigure}{\includegraphics[width = 0.9\textwidth]{figures/sflio-device/MordauntResolutionFigure.pdf}}
    \annotatedFigureText{0.035,0.90}{white}{0.3}{a)}
    \annotatedFigureText{0.52,0.90}{black}{0.3}{b)}
    \end{annotatedFigure}
    \caption{The line profile through a \SI{150}{\um} capillary tube filled with arterial blood and mounted in an eye phantom is shown in a) and b) record using a similar imaging system to the SFLIO system where an off-the-shelf fundus camera (Canon CF-60Z) is modified for hyper-spectral imaging for the purposes of measuring oxygen saturation of blood in the retina. In b), the number of pixels to transition from black-to-white or the rise-time, $\tau_{rise}$, is demarcated by the solid black lines and was computed as $\tau_{rise} \cong \SI{30}{\um}$. a) Reproduced from Fig. 3A of~\citeauthor{mordant2011validation}\cite{mordant2011validation}.}
    \label{fig:MordauntResolitionFigure}
\end{figure}

The likely sources of the discrepancy in spatial resolution were hypothesised to originate from: the eye-phantom; the 90:10 beam-splitter in the imaging arm; and / or the \SI{500}{\nm} long pass emission filter. A checkerboard target, with a pitch of \SI{10}{\mm}, positioned approximately \SI{3}{\metre} from the objective lens of the fundus camera - which much like imaging the eye can be considered to be imaging at infinity - was then imaged with each of the aforementioned elements removed from the imaging path. The rise-time or the black-to-white transition time was then be computed for each scenario, as shown \cref{fig:sCMOStroubleshooting}, in terms of pixels and yielded a mean rise-time of \SI{4.5}{\pixel}. Since this difference rise-time ($\SI{3}{\pixel}:\SI{4.5}{\pixel}$) is proportionally similar to the discrepancy in spatial resolution ($\SI{30}{\um}:\SI{50}{\um}$) it can be concluded that the optics of the fundus camera or the SLR lenses used in the imaging arm are the limiting factor and not the beam-splitter or emission filter. While efforts could have been made to improve the spatial resolution of the SFLIO system to match this previously reported value the goal of this project was to construct an system for quantitative imaging of the retina and not diffraction limited retinal imaging system. Furthermore, while the assessment of spatial resolution of the system was treated with a reasonable degree of rigour, the knife-edge test was implemented for its relatively high accuracy when compared to the time and complexity cost. More sensitive characterisations using Siemens Star and US Air force test targets embedded in the eye phantom could have been employed for more robust determinations of the MTF - the results of the knife-edge test were considered to be sufficient for this assessment. Additionally, the irregular pixel aspect ratio and layout of the FLIMera was not taken in to account in this analysis however there exists methods using fringe patterns, produced by a Twyman-Green interferometer, projected onto the detector to accurately sample its MTF~\cite{greivenkamp1994modulation}. In practice the SFLIO system is optically limited and not detector limited so while these methods would yield more precise and accurate measurements of the MTF and spatial resolution it would not fundamentally change the performance of the systems biochemical resolution nor the requirement for adequate image registration.

\begin{figure}
    \centering
    \begin{annotatedFigure}{\includegraphics[width = 0.8\textwidth]{figures/sflio-device/SpatialResTroubleShootFigure.pdf}}
    \annotatedFigureText{-0.02,0.99}{black}{0.3}{a)}
    \annotatedFigureText{-0.02,0.46}{black}{0.3}{c)}
    \annotatedFigureText{0.475,0.99}{black}{0.3}{b)}
    \annotatedFigureText{0.475,0.46}{black}{0.3}{d)}
    \end{annotatedFigure}
    \caption{Images recorded of a checker board patter illuminated with a white light Xenon source using the S-FLIO system to ascertain the source of degradation in the spatial resolution of the fluorescence intensity imaging arm. Line profiles were taken across the checker board pattern and the rise time of the contrast was computed. In a) A \SI{500}{\nm} long pass filter is in place for fluorescence intensity imaging with sCMOS detector. The rise time was computed to be \SI{5}{\pixel}, In b) the 90:10 beam-splitter used for simultaneous imaging with the FLIMera was removed which resulted in a lower rise time of \qtyrange{3}{4}{\pixel}. In c) the \SI{500}{\nm} long pass filter was removed, emulating brightfield imaging, and the rise time was computed as \SI{5}{\pixel}. In d) both the  \SI{500}{\nm} long pass filter and the 90:10 beam-splitter was removed resulting in a rise-time of \SI{4}{\pixel}}
    \label{fig:sCMOStroubleshooting}
\end{figure}

\FloatBarrier
\subsection{Motion Registration}
The long acquisition times of over 1 minute required to record fluorescence decays, using the FLIMera, with a SNR high enough to allow discrimination of fluorophores introduces the challenge of compensating for the natural movements of the eye that occur when fixating for long periods of time. These eye movements would result in a fluorescence lifetime image with severely reduced contrast where features such as blood vessels will not be discernible. Due to low number of photons recorded these movements cannot be corrected for using only integrated intensity images generated by the FLIMera at video rate and registered using conventional image registration algorithms to construct a single unblurred histogram.
\\
These eye movements are categorised into three different types of movement: tremors which are high frequency movements (\SI{90}{\hertz}) with amplitude on order $<1$~arc-minute; drift which is a continuous random movement that occurs between tremors and saccades but with lower frequency ($<\SI{1}{\hertz}$) but longer duration ($>\SI{0.5}{\second}$) and larger in amplitude ($\approx \SI{0.5}{degree}$); and saccades which appear as fast jerks lasting \SI{25}{\milli\second} and occurring every \SI{2}{\second} of fixation time with varying amplitude~\cite{martinez2004role}.
In the S-FLIO imaging system the affects of tremors are be ignored since their amplitude is much smaller than the angular resolution ($\approx 10$~arc-minutes) of our system tremors exhibit small enough amplitudes that they would be much smaller than the spatial resolution of the system. The resultant saccades and drift movements can be compensated by determining any translation between successive brightfield images recorded using the sCMOS camera at a high framerate (\SI{10}{\fps}) and short exposure time (\SI{30}{\milli\second}) and then applying the opposite translation to  histograms from the FLIMera that can be read out at up to \SI{12}{\kilo\hertz}.
Mathematically this is modelled using the a Affine Transform which can be used to map the co-ordinates, in image space, for each pixel on the CMOS camera and the FLIMera.
\begin{align}\label{eq:mapping}
    \begin{bmatrix}
        x\\
        y\\
        1\\
    \end{bmatrix}
     & = \mathbf{A}
     \begin{bmatrix}
         x'\\
         y'\\
         1
     \end{bmatrix}\\
    \implies I_{FLIMera}(x,y) &= \mathbf{A}I_{sCMOS}(x',y')
\end{align}
Where the affine transform is composed of four separable transforms : Translation $\mathbf{T}(\Delta x, \Delta y)$, Magnification/Scaling, $\mathbf{M}(m_{x},m_{y})$; Rotation, $\mathbf{R}(\theta)$;and Shear, $\mathbf{S}(h_{x}, h_{y})$. In the images recorded using the S-FLIO system Shear was not significantly present of either the sCMOS camera or the FLIMera and is not considered in the registration process.
\begin{equation}\label{eq:affine}
    \mathbf{A} = \mathbf{T}\mathbf{R}\mathbf{M}
\end{equation}
\begin{align}\label{eq:TransMatrices}
    T &= \begin{bmatrix}
        1 & 0 & \Delta x\\
        0 & 1 & \Delta y\\
        0 & 0 & 1
    \end{bmatrix}
    &
    R &= \begin{bmatrix}
        \cos\theta & -\sin\theta & 0\\
        \sin\theta & \sin\theta & 0\\
        0 & 0 & 1\\
    \end{bmatrix}
    &
    M &= \begin{bmatrix}
        m_{x} & 0 & 0 \\
        0 & m_{y} & 0 \\
        0 & 0 & 1\\
    \end{bmatrix}
    &
    S &= \begin{bmatrix}
        0 & h_{x} & 0\\
        h_{y} & 0 & 0\\
        0 & 0 & 1\\
    \end{bmatrix}   
\end{align}
Using this affine transform the registration process can be further into separated into two stages: The affine transform, $\hat{\mathbf{A}}$ which defines the mapping between the two detectors for a static scene ; and a time dependent affine transform relating to translational motion in the retina, $\mathbf{A}(\Delta x(t), \Delta y(t))$.

\subsection{Static Image Registration}
\subsubsection{Fourier-Merlin Method}
While feature based registration algorithms are currently employed for the purpose of registering retinal images these techniques are predominately used on collections of images with similar resolution, field of view, and magnification~\cite{chanwimaluang2006hybrid,faisan2011scanning}.  
Since the FLIMera images exhibit high noise, as well as dead and hot pixels, and are of low resolution compared to the sCMOS brightfield images the Fourier Merlin algorithm has been utilised \cite{padfield2011masked} to register these image. The Fourier-Merlin uses the phase cross-correlation method of image registration in two stages where first the magnification and scaling between the images is recovered and corrected for and then finally any translational shifts is corrected.
Briefly, the phase cross-correlation registration method identifies the translation between the two images, $k(x,y)$, and $l(x','y) = k(x+\Delta x, y + \Delta y)$ by finding the location of the peak of phase correlation, $\Omega$ between the two images. This is typically implemented using the FFT:
\begin{align}
    K(\mu, \nu) &= \mathcal{F}\{k(x,y)\} & L(\mu',\nu') &= \mathcal{F}\{l(x',y')\}\\
    \Omega(\mu,\nu) &= \frac{K L^{*}}{\lvert K L^{*} \rvert} & (\Delta x, \Delta y) &= argmax\big\{{\mathcal{F}^{-1}}\{\Omega(\mu,\nu)\}\big\}
\end{align}
To recover the magnification and scaling parameters the Fourier transform of both images is computed and then the log-polar transformation is applied (\cref{eq:logpolar}). Here, any magnification and rotation $(\Delta m ,\Delta\theta)$ between the images manifests as a translation in the log-polar transformed images along the $\rho$ and $\xi)$ axes.
\begin{gather}
    I(x,y) \rightarrow I(\rho,\xi)\\
    \begin{aligned}\label{eq:logpolar}
    \rho &= \arctan{\bigg(\frac{y}{x}\bigg)} &\xi &= \ln{(\sqrt{x^{2} + y^{2}})}
    \end{aligned}  
\end{gather}
The parameters $\Delta m$ and $\Delta\theta$ can then be recovered using \cref{eq:magrot} where $U$ and $V$ refer to the number of samples in each axis of the Fourier transformed image, and $U / k$ is the radius over which the log-polar transform is applied. 
\begin{align}\label{eq:magrot}
    \Delta\theta &= \frac{2\pi\Delta\rho}{U} & \Delta m &= \frac{1}{\exp\Delta\xi / \Gamma} & \Gamma = \frac{V}{\ln\lfloor U / k\rfloor}
\end{align}
\subsubsection{S-FLIO Device}
The static calibration between the two detectors was then found using the above process and fluorescence images recorded of a United States Air Force (USAF) test target using the FLIMera and sCMOS camera in tandem. Images were recorded using the standard excitation wavelengths (\qtyrange{460}{467}{\nm}) and a \SI{500}{\nm} long pass to filter the excitation light with an integration time of \SI{10}{\second} to ensure a high SNR. For the images recorded with the FLIMera, the measured histograms were integrated over time to yield a intensity image and the  pixel aspect ratio is corrected for by twinning adjacent columns of pixels resulting in a image with resolution \qtyproduct[product-units=single]{192x252}{\pixel}.
Before the Fourier-Merlin algorithm was applied, the images were first filtered using a difference-of-Gaussian bandpass filter and a Hanning window to increase the contrast of edge features which gives a more prominent peak in the cross-correlation process. The scaling and rotation between the images was found using the FFT and log-polar transform process (\cref{fig:USAFlp}) described above and returned values of $\Delta M = \num[round-mode = figures, round-precision = 3]{4.202255063883409}$ and $\Delta\theta = \SI[round-mode = figures, round-precision = 3]{-0.098775}{\degree}$. 

\begin{figure}
    \centering
    \begin{annotatedFigure}
    {\includegraphics[width = 0.95\textwidth]{figures/sflio-device/LogPolarUSAF.pdf}}
    \annotatedFigureText{0.16,0.86}{white}{0.3}{a)}
    \annotatedFigureText{0.39,0.86}{white}{0.3}{b)}
    \annotatedFigureText{0.63,0.86}{white}{0.3}{c)}
    \annotatedFigureText{0.16,0.38}{white}{0.3}{d)}
    \annotatedFigureText{0.39,0.38}{white}{0.3}{e)}
    \annotatedFigureText{0.63,0.38}{white}{0.3}{f)}
    \end{annotatedFigure}
    \caption{Visualisation of how the scaling and rotation between the FLIMera and sCMOS can be recovered using the Fourier-Merlin: a) and d) The input images are filtered using a difference-of-Gaussians bandpass filter and a Hanning window; b) and e) The images are then Fourier transformed; and finally the Log-Polar transform (\cref{eq:logpolar}) is applied to project the relative magnification and rotation into translations in the horizontal and vertical axes of c) and f) which are recovered using standard phase cross-correlation techniques.}
    \label{fig:USAFlp}
\end{figure}


A intermediate FLIMera image is then constructed from these $\Delta m$, and $\Delta\theta$ values as well as a translation matrix to re-centre the image using an affine warp and nearest neighbour interpolation. Until now, the image origin (0,0) is taken as the upper-left corner which results in undesired shifting of the scaled image upon the application of a scaling and rotation. To account for this a further translation matrix is used to shift the image by an amount equal to \SI{-6558}{\pixel} and \SI{-3458}{\pixel} in the horizontal and vertical axes, respectively. The translations between this intermediate FLIMera image and the sCMOS image are found using standard phase cross-correlation to be $\Delta x = \SI[round-mode = figures, round-precision = 3]{-205.9709}{\pixel}$ and $\Delta y = \SI[round-mode = figures, round-precision = 3]{-298.35}{\pixel}$.
Finally, the magnification, rotation, and translations are used to construct a single Affine transform (see \cref{eq:SFLIOaff}) and then the registered FLIMera image is then produced. The accuracy of this registration is assessed using a normalised line profile taken through an identical region in both images, as shown in \cref{fig:USAFregistered}, and calculating the cross-correlation of both line profiles. For a perfectly registered image the location of the peak would occur at \num{0}, however, for the SFLIO system this location is \num{1} indicating a registration accuracy of \num{1} or fewer pixels on sCMOS detector. This greatly exceeds the minimum accuracy needed for sub-pixel registration on the FLIMera where a registration accuracy of less than $1/\Delta m$ is required.


\begin{figure}
    \centering
    \begin{annotatedFigure}
        {\includegraphics[width = 0.85\textwidth]{figures/sflio-device/AirForceTarget-RegisteredImage.pdf}}
        \annotatedFigureText{0.04,0.94}{black}{0.4}{a)}
        \annotatedFigureText{0.04,0.50}{black}{0.4}{c)}
        \annotatedFigureText{0.52,0.94}{black}{0.4}{b)}
    \end{annotatedFigure}
    \caption{Results of static image registration the sCMOS and FLIMera on the S-FLIO device using a USAF test target and the Fourier-Merlin algorithm for recovering scale, rotation, and translation transforms. a) Shows the original, high resolution, image recorded with the sCMOS camera; b) Shows the FLIMera image after being scaled, rotated, and translated using a affine transform and nearest neighbour interpolation. The accuracy of the registration process is assessed using the cross-correlation peak of two line profiles (cyan and magenta) through identical regions of a) and b) plotted in c) and yields an accuracy of 1 pixel}
    \label{fig:USAFregistered}
\end{figure}
\begin{equation}\label{eq:SFLIOaff}
\hat{\mathbf{A}} = \begin{bmatrix}
    \num[round-mode = figures, round-precision = 3]{4.202248819333833} & \num[round-mode = figures, round-precision = 3]{-0.007244469694988512}& \num[round-mode = figures, round-precision = 3]{-6856.35}\\
    \num[round-mode = figures, round-precision = 3]{0.007244469694988512} & \num[round-mode = figures, round-precision = 3]{4.202248819333833}& \num[round-mode = figures, round-precision = 3]{-3663.9709}\\
    \num[round-mode = figures, round-precision = 3]{0} & \num[round-mode = figures, round-precision = 3]{0}& \num[round-mode = figures, round-precision = 3]{1}
\end{bmatrix}    
\end{equation}

\subsection{Movement Registration}
Correcting for eye movement in the fluorescence decays recorded with the FLIMera is achieved by detecting translational shifts in successive brightfield or fluorescence intensity images, recorded at \SI{10}{\hertz}, with phase cross-correlation and then using the affine transform, $\hat{\mathbf{A}}$, to obtain equivalent shifts on the FLIMera. These shifts are then applied to successive histograms generated by the FLIMera operating in RAW mode at the same frame rate of as the sCMOS detector and combined to form a single histogram spanning the entire acquisition period. From this histogram a high quality fluorescence lifetime image can be constructed that is now free of motion artefacts.

\subsubsection{sCMOS shifts to FLIMera shifts}
The shifts in the sCMOS detector are converted to shifts in the FLIMera detector using the attribute of the affine transform (Eq.~\ref{eq:affreg}) and considering the transform between pairs of images:
\begin{align}\label{eq:affreg}
I_{FLIMera}^{N+\epsilon}(x',y') &= I_{FLIMera}^{N}(x' + \Delta x'_{\epsilon} ,y'+ \Delta y'_{\epsilon}) = \mathbf{A}(\Delta x'_{\epsilon} ,\Delta y'_{\epsilon})I_{FLIMera}^{N}(x',y')\\
I_{sCMOS}^{N+\epsilon}(x,y) &= I_{sCMOS}^{N}(x + \Delta x_{\epsilon} ,y+ \Delta y_{\epsilon}) = \mathbf{A}(\Delta x_{\epsilon} ,\Delta y_{\epsilon})I_{sCMOS}^{N}(x,y)
\end{align}
Where $N$ represents a reference frame and $\epsilon$ is a frame recorder later in the acquisition period that has underwent a translational shift. The $N+\epsilon$ frames in each detector can be described using the existing relationship $I_{FLIMera}^{N+\epsilon}(x',y') = \hat{\mathbf{A}}^{-1}I_{sCMOS}^{N+\epsilon}(x,y)$ and the translational term, $\mathbf{A}(\Delta x_{\epsilon} ,\Delta y_{\epsilon})$, found using image registration is converted to shifts in the FLIMera using \cref{eq:affreg}:

\begin{align}
    I_{FLIMera}^{N+\epsilon} &= \hat{\mathbf{A}}^{-1}I_{sCMOS}^{N+\epsilon}(x,y)= \hat{\mathbf{A}}^{-1}\mathbf{A}(\Delta x_{\epsilon} ,\Delta y_{\epsilon})I_{sCMOS}^{N}(x,y) \nonumber\\
    I_{FLIMera}^{N+\epsilon} &= \hat{\mathbf{A}}^{-1}\mathbf{A}(\Delta x_{\epsilon} ,\Delta y_{\epsilon})\hat{\mathbf{A}}I_{FLIMera}^{N}(x',y')\nonumber\\
    \therefore \mathbf{A}(\Delta x'_{\epsilon} ,\Delta y'_{\epsilon}) &= \hat{\mathbf{A}}^{-1}\mathbf{A}(\Delta x_{\epsilon} ,\Delta y_{\epsilon})\hat{\mathbf{A}}\label{eq:FLIMerashifts}
\end{align}  
This approach can be further generalised for rotation, translation, or shear such that any affine transform, $\mathbf{A}(\Delta x, \Delta y, \theta, \Delta m, \Delta h)$, between two images recorded in the sCMOS imaging arm of the S-FLIO device can be converted into an equivalent transform of a FLIMera image:
\begin{equation}
    \mathbf{A}(\Delta x', \Delta y', \theta', \Delta m', \Delta h') = \hat{\mathbf{A}}^{-1}\mathbf{A}((\Delta x, \Delta y, \theta, \Delta m, \Delta h))\hat{\mathbf{A}}
\end{equation}
\subsubsection{\textit{Ex-Vitro} Demonstration using \textit{Convallaria majalis}}
The motion registration algorithm was demonstrated on a sample of \textit{Convallaria majalis} - referred to simply as Convallaria in the rest of the text - mounted in an eye phantom which could be moved within the X-Y plane \cref{fig:ConvallariaEyePhantom} during the acquisition period to mimic the image degradation seen in images of the human retina. Over a \SI{2}{\minute} integration period the sample, excited between \qtyrange{460}{467}{\nm}, is held still, translated in X, Y, and then X and Y for periods of \SI{30}{\second} each. During this period the emitted fluorescence, filtered using a \SI{500}{\nm} long-pass filter, fluorescence intensity images are continuously recorded - with integration time of \SI{30}{\ms} and frame rate of \SI{10}{\hertz} - in tandem with FLIM data using the FLIMera's RAW mode to readout histograms at a rate of \SI{24}{\kilo\hertz}. 
\begin{figure}
    \centering
    \includegraphics[width = 0.5\linewidth, trim = {2.25cm, 1cm, 1.5cm, 0.5cm}, clip]{figures/sflio-device/PhantomEyeFigConvallaria.pdf}
    \caption{The natural movements of a fixated human eye was simulated using an eye phantom with a sample of Convallaria positioned at the retinal plane and mounted to a X-Y translation stage. Saccades and drift motions are then mimic by moving the Convallaria sample with an amplitude and velocity commensurate with published values~\cite{martinez2004role}. The phantom eye serves as an analogue to the human eye in that it uses a $f=\SI{24}{\mm}, d = \SI{8}{\mm}$ lens positioned \SI{24}{\mm} from its retinal plane. The cavity in the phantom eye is filled with water to emulate the refractive properties of the vitreous humour.}
    \label{fig:ConvallariaEyePhantom}
\end{figure}
By initially having the laser shutter closed - sample is not illuminated - at the start of the acquisition period before illuminating the sample the FLIMera and sCMOS frames can then be synchronised by detecting the Heaviside like change in the photon flux, shown in \cref{fig:flimerascmossync}, by finding the peak of the gradient of this photon flux. For the FLIMera, the low detected photon flux and high frame rate resulted in a noise dominated signal (\cref{fig:flimerascmossync}a) where the frame at which the laser shutter was opened could not be readily discerned. A Butterworth filter is then used to remove the high frequency components of the signal revealing the step-like change from which the gradient could be calculated. When applying the Butterworth filter there is ordinarily some linear phase introduced into the filtered signal which would give a false location in the synchronisation process however, by using a forward and backward configuration, implemented using \textit{scipy}'s \texttt{filtfilt} function, the resulting filter does not introduce any unwanted phase to the filtered signal.

\begin{figure}
    \centering
    \begin{annotatedFigure}{\includegraphics[width =\textwidth]{figures/sflio-device/flimerasync.pdf}}
    \annotatedFigureText{0.015,0.93}{black}{0.4}{a)}
    \annotatedFigureText{0.015,0.46}{black}{0.4}{b)}
    \end{annotatedFigure}
    \caption{The step-like change in the photon flux per frame, $\Phi(n)$, for the the FLIMera (a) and sCMOS detector (b), shown as the black line, is detected by finding the peak of the gradient, $d\Phi(n)/dn$, shown as the red line. For the FLIMera a Butterworth filter was applied to the raw signal, shown as the grey line in (a), to enable the frame when the laser shutter was opened to be resolved.}
    \label{fig:flimerascmossync}
\end{figure}

Once this synchronisation process is complete the bright field or fluorescence intensity images recorded using the sCMOS detector are registered, using the above phase cross-correlation method, to the first filly illuminated image in the sequence. This is used to construct a single higher quality image, composed of corrected frames summed together spanning the acquisition period, \cref{fig:sCMOSreg}d from \cref{fig:sCMOSreg}b. It should be noted that although the image in \cref{fig:sCMOSreg}d is now free of motion artefacts it does appear to have poorer spatial resolution when compared to \cref{fig:sCMOSreg}a, an image composed of a single exposure recorded over a shorter \SI{2}{\second} exposure time. Each frame in the image sequence exhibits higher photon noise as well as higher relative contribution of read noise due to the higher frame rate - when compared to an equal but longer exposure - manifesting as small registration errors ($< \SI{2}{\pixel}$) and resulting in the motion corrected image appearing softer and losing some of the finer details

\begin{figure}
    \centering
    \begin{annotatedFigure}{\includegraphics[width = \linewidth, trim = {0.4cm, 1.1cm, 0.1cm, 0.3cm}, clip]{figures/sflio-device/sCMOSRegistration.pdf}}
    \annotatedFigureText{0.01,0.76}{white}{0.4}{a)}
    \annotatedFigureText{0.26,0.76}{white}{0.4}{b)}
    \annotatedFigureText{0.51,0.76}{white}{0.4}{c)}
    \annotatedFigureText{0.76,0.76}{white}{0.4}{d)}
    \end{annotatedFigure}
    \caption{Fluorescence intensity images recorded of \textit{Convallaria} mounted in a eye phantom (a-d). The first image, a), represents a single \SI{2}{\second} exposure. For b-d), the images are comprised of frames recorded at \SI{10}{\hertz} with \SI{30}{\ms} integration time with c) and d) being the result of summing multiple frames. The sample is initially held still for \SI{30}{\second} before being translated in the X and Y axes throughout the remainder of a total \SI{2}{\minute} acquisition period to mimic the typical movements the human eye undergoes while fixating on a target. b) shows a typical frame used in the registration process with grey values between 0 and 5. In c) the image is degraded by the movement over the entire acquisition period and in d) this motion is corrected for using phase-cross correlation to detect movement and an affine transform to correct the translational shifts.}
    \label{fig:sCMOSreg}
\end{figure}

The translational shifts in the fluorescence intensity images are then converted to equivalent shifts on the FLIMera using \cref{eq:FLIMerashifts} and applied to FLIMera histograms that cover the same time period as the sCMOS acquisition, $T = 1/\SI{10}{\hertz} = \SI{100}{\ms}$. When applying the shifts to each histogram a pixel mask is applied to eliminate the effect of "hot" pixels and faulty pixels on the fluorescence decay. This yields a single $XYT$ - data cube spanning the entire \SI{2}{\minute} acquisition period that as shown in integrated intensity images in \cref{fig:FLIMerareg}d is now free of motion artefacts and the vascular bundles can be clearly resolved albeit with a lower resolution when compared to the fluorescence intensity images shown in \cref{fig:sCMOSreg}. Furthermore, in these intensity images, motion of the sample causes features that would ordinarily be imaged by a masked faulty pixel (See \cref{fig:FLIMerareg}b) in a still sample now illuminates a different. Typically this would result in an uneven image since every pixel in the final image would be not be masked an equal number of times. This is accounted for by multiplying each pixel grey value by the factor, $\Gamma$, which weights the total number of frames, $N$, in a sequence with the number of times that pixel is masked, $M$ to yield the high quality image shown in \cref{fig:FLIMerareg}d.

\begin{equation}
    \Gamma(x,y) = \frac{N}{N-M(x,y)}
\end{equation}


\begin{figure}
    \centering
    \begin{annotatedFigure}{\includegraphics[width = \linewidth, trim = {0.5cm, 1cm, 0.2cm, 0},clip]{figures/sflio-device/FLIMeraRegistration.pdf}}
    \annotatedFigureText{0.015,0.77}{white}{0.4}{a)}
    \annotatedFigureText{0.345,0.77}{white}{0.4}{b)}
    \annotatedFigureText{0.675,0.77}{white}{0.4}{c)}
    \end{annotatedFigure}
    \caption{Demonstration of motion registration of time resolved fluorescence decays recorded with the FLIMera of a \textit{Convallaria} slide mounted in an eye phantom under going motion throughout a \SI{2}{\minute} acquisition period. Images are formed by integrating the histogram over time to mimic a fluorescence intensity image where: a) represents the entire \SI{2}{minute} acquisition period; b) is the initial \SI{30}{\second} where the sample is held still; and c) is the same \SI{2}{minute} period but with the motion corrected for using the movement detected in the fluorescence intensity images (see \cref{fig:sCMOSreg}).}
    \label{fig:FLIMerareg}
\end{figure}