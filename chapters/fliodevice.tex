\setstretch{2}

\section{Construction of \textit{En-face} S-FLIO system}
\subsection{Device Methodology}
The S-FLIO device is based upon an off-the-shelf fundus camera which is then modified to allow for exciting fluorescence with a pulsed laser source and  
\subsection{Illumination Arm}
\subsection{Imaging Arm}
\subsubsection{FLIMera}




\section{Evaluation of Performance}
\begin{itemize}
    \item We need good spatial resolution, and signal to noise ratio in the high resolution camera to account for movement of the eye
    \item SPAD has much fewer pixels and much larger spatial resolution (influenced by the irregular sparse array architecture) and there is a magnification difference so as long as spatial res of the registration camera is 5-10X the FLIMera and we can register with a 5 pixels then our registration will be limited by he quantisation of the digital imaging system
    \item The purpose of this device is to proof of concept the measurement of FAD in the retina using lifetimes - not construct a diffraction limited system.
    \item to quantify the spatial resolution of the system we can use the knife edge test consisting of a fluorescent slide (gives a uniform scene) that has a slanted hard edge places in front of it all housed in an artificial eye
    \item calibration is achieved by replacing the edge with an Allen key with known diameter. The number of pixels between the parallel edges yield the pixel to um conversion factor i.e the width and height that a single pixel subtends on the retina.
    \item this test evaluate the spatial resolution by averaging an aliased edge (in the case of the FLIMera - pixels are small enough on the FI camera that optics is the limit not quantisation) to yield the Edge Spread Function. Differential of the ESF is the LSF and the spatial resolution is defined as the FWHM of this LSF after a Gaussian has been fitted.
    \item This was repeated through multiple foci to ascertain where the best focus is and what is its spatial resolution
    \item on the FI camera the spatial resolution is 40um and for the FLIMera it is 400um.
    \item This resolution is larger than what was previously reported in the Harvey, McNaught artificial eye paper (spatial res was $\approx 15$um) which uses the same artificial eye and an optically similar imaging system (fundus camera with SLR lenses in a imaging relay configuration to focus the image onto a scientific camera.
    \item The system is optically limited but the source is likely to originate in misalignment of the SLR lenses, other optical elements in the beam path etc. 
    \item To attempt to ascertain the source of these issues the fundus camera was set up to image the room with some targets attached to the wall at infinity and optical elements (emission filter, beam splitter) and the spatial resolution was estimated using the rise-time method
    \begin{itemize}
        \item spatial res $\approx$ width in pixels between 90 and 10 percent of the fast edge
    \end{itemize}
    \item this was then compared to two much simpler imaging systems - and iPhone 13, and the thorcam with an SLR lens attached set to the same scene.
    \item CONCLUSION: spatial resolution is X and this is sufficient for co-registering images with the FLIMera. There are multiple sources of aberration in the system that can be harming our spatial res that could be eliminated given the a significant time investment but since a good signal to noise ratio is favoured over diffraction limited (or near it) this is sufficient.
\end{itemize}


\subsection{Motion Registration}
The long acquisition times of over 1 minute required to record fluorescence decays, using the FLIMera, with a SNR high enough to allow discrimination of fluorophores introduces the challenge of compensating for the natural movements of the eye that occur when fixating for long periods of time. These eye movements would result in a fluorescence lifetime image with severely reduced contrast where features such as blood vessels will not be discernible. Due to low number of photons recorded these movements cannot be corrected for using only integrated intensity images generated by the FLIMera at video rate and registered using conventional image registration algorithms to construct a single unblurred histogram.
\\
These eye movements are categorised into three different types of movement: tremors which are high frequency movements (\SI{90}{\hertz}) with amplitude on order $<1$~arc-minute; drift which is a continuous random movement that occurs between tremors and saccades but with lower frequency ($<\SI{1}{\hertz}$) but longer duration ($>\SI{0.5}{\second}$) and larger in amplitude ($\approx 0.5$~degrees); and saccades which appear as fast jerks lasting \SI{25}{\milli\second} and occurring every \SI{2}{\second} of fixation time with varying amplitude~\cite{martinez2004role}.
In the S-FLIO imaging system the affects of tremors are be ignored since their amplitude is much smaller than the angular resolution ($\approx 10$~arc-minutes) of our system tremors exhibit small enough amplitudes that they would be much smaller than the spatial resolution of the system. The resultant saccades and drift movements can be compensated by determining any translation between successive brightfield images recorded using the sCMOS camera at a high framerate (\SI{10}{\fps}) and short exposure time (\SI{30}{\milli\second}) and then applying the opposite translation to  histograms from the FLIMera that can be read out at up to \SI{12}{\kilo\hertz}.
Mathematically this is modelled using the a Affine Transform which can be used to map the co-ordinates, in image space, for each pixel on the CMOS camera and the FLIMera.
\begin{align}\label{eq:mapping}
    \begin{bmatrix}
        x\\
        y\\
        1\\
    \end{bmatrix}
     & = \mathbf{A}
     \begin{bmatrix}
         x'\\
         y'\\
         1
     \end{bmatrix}\\
    \implies I_{FLIMera}(x,y) &= \mathbf{A}I_{sCMOS}(x',y')
\end{align}
Where the affine transform is composed of four separable transforms : Translation $\mathbf{T}(\Delta x, \Delta y)$, Magnification/Scaling, $\mathbf{M}(m_{x},m_{y})$; Rotation, $\mathbf{R}(\theta)$;and Shear, $\mathbf{S}(h_{x}, h_{y})$. In the images recorded using the S-FLIO system Shear was not significantly present of either the sCMOS camera or the FLIMera and is not considered in the registration process.
\begin{equation}\label{eq:affine}
    \mathbf{A} = \mathbf{T}\mathbf{R}\mathbf{M}
\end{equation}
\begin{align}\label{eq:TransMatrices}
    T &= \begin{bmatrix}
        1 & 0 & \Delta x\\
        0 & 1 & \Delta y\\
        0 & 0 & 1
    \end{bmatrix}
    &
    R &= \begin{bmatrix}
        \cos\theta & -\sin\theta & 0\\
        \sin\theta & \sin\theta & 0\\
        0 & 0 & 1\\
    \end{bmatrix}
    &
    M &= \begin{bmatrix}
        m_{x} & 0 & 0 \\
        0 & m_{y} & 0 \\
        0 & 0 & 1\\
    \end{bmatrix}
    &
    S &= \begin{bmatrix}
        0 & h_{x} & 0\\
        h_{y} & 0 & 0\\
        0 & 0 & 1\\
    \end{bmatrix}   
\end{align}
Using this affine transform the registration process can be further into separated into two stages: The affine transform, $\hat{\mathbf{A}}$ which defines the mapping between the two detectors for a static scene ; and a time dependent affine transform relating to translational motion in the retina, $\mathbf{A}(\Delta x(t), \Delta y(t))$.

\subsection{Static Image Registration}
\subsubsection{Fourier-Merlin Method}
While feature based registration algorithms are currently employed for the purpose of registering retinal images these techniques are predominately used on collections of images with similar resolution, field of view, and magnification~\cite{chanwimaluang2006hybrid,faisan2011scanning}.  
Since the FLIMera images exhibit high noise, as well as dead and hot pixels, and are of low resolution compared to the sCMOS brightfield images the Fourier Merlin algorithm has been utilised \cite{padfield2011masked} to register these image. The Fourier-Merlin uses the phase cross-correlation method of image registration in two stages where first the magnification and scaling between the images is recovered and corrected for and then finally any translational shifts is corrected.
Briefly, the phase cross-correlation registration method identifies the translation between the two images, $k(x,y)$, and $l(x','y) = k(x+\Delta x, y + \Delta y)$ by finding the location of the peak of phase correlation, $\Omega$ between the two images. This is typically implemented using the FFT:
\begin{align}
    K(\mu, \nu) &= \mathcal{F}\{k(x,y)\} & L(\mu',\nu') &= \mathcal{F}\{l(x',y')\}\\
    \Omega(\mu,\nu) &= \frac{K L^{*}}{\lvert K L^{*} \rvert} & (\Delta x, \Delta y) &= argmax\big\{{\mathcal{F}^{-1}}\{\Omega(\mu,\nu)\}\big\}
\end{align}
To recover the magnification and scaling parameters the Fourier transform of both images is computed and then the log-polar transformation is applied (\cref{eq:logpolar}). Here, any magnification and rotation $(\Delta m ,\Delta\theta)$ between the images manifests as a translation in the log-polar transformed images along the $\rho$ and $\xi)$ axes.
\begin{gather}
    I(x,y) \rightarrow I(\rho,\xi)\\
    \begin{aligned}\label{eq:logpolar}
    \rho &= \arctan{\bigg(\frac{y}{x}\bigg)} &\xi &= \ln{(\sqrt{x^{2} + y^{2}})}
    \end{aligned}  
\end{gather}
The parameters $\Delta m$ and $\Delta\theta$ can then be recovered using Equation \ref{eq:magrot} where $U$ and $V$ refer to the number of samples in each axis of the Fourier transformed image, and $U / k$ is the radius over which the log-polar transform is applied. 
\begin{align}\label{eq:magrot}
    \Delta\theta &= \frac{2\pi\Delta\rho}{U} & \Delta m &= \frac{1}{\exp\Delta\xi / \Gamma} & \Gamma = \frac{V}{\ln\lfloor U / k\rfloor}
\end{align}
\subsubsection{S-FLIO Device}
The static calibration between the two detectors was then found using the above process and fluorescence images recorded of a United States Air Force (USAF) test target using the FLIMera and sCMOS camera in tandem. Images were recorded using the standard excitation wavelengths (\qtyrange{460}{467}{\nm}) and a \SI{500}{\nm} long pass to filter the excitation light with an integration time of \SI{10}{\second} to ensure a high SNR. For the images recorded with the FLIMera, the measured histograms were integrated over time to yield a intensity image and the  pixel aspect ratio is corrected for by twinning adjacent columns of pixels resulting in a image with resolution \qtyproduct[product-units=single]{192x252}{\pixel}.
Before the Fourier-Merlin algorithm was applied, the images were first filtered using a difference-of-Gaussian bandpass filter and a Hanning window to increase the contrast of edge features which gives a more prominent peak in the cross-correlation process. The scaling and rotation between the images was found using the FFT and log-polar transform process (Fig. \ref{fig:USAFlp}) described above and returned values of $\Delta M = \num[round-mode = figures, round-precision = 3]{4.202255063883409}$ and $\Delta\theta = \SI[round-mode = figures, round-precision = 3]{-0.098775}{\degree}$. 

\begin{figure}
    \centering
    \begin{annotatedFigure}
    {\includegraphics[width = 0.95\textwidth]{figures/sflio-device/LogPolarUSAF.pdf}}
    \annotatedFigureText{0.16,0.86}{white}{0.3}{a)}
    \annotatedFigureText{0.39,0.86}{white}{0.3}{b)}
    \annotatedFigureText{0.63,0.86}{white}{0.3}{c)}
    \annotatedFigureText{0.16,0.38}{white}{0.3}{d)}
    \annotatedFigureText{0.39,0.38}{white}{0.3}{e)}
    \annotatedFigureText{0.63,0.38}{white}{0.3}{f)}
    \end{annotatedFigure}
    \caption{Visualisation of how the scaling and rotation between the FLIMera and sCMOS can be recovered using the Fourier-Merlin: a) and d) The input images are filtered using a difference-of-Gaussians bandpass filter and a Hanning window; b) and e) The images are then Fourier transformed; and finally the Log-Polar transform (Eq.\ref{eq:logpolar}) is applied to project the relative magnification and rotation into translations in the horizontal and vertical axes of c) and f) which are recovered using standard phase cross-correlation techniques.}
    \label{fig:USAFlp}
\end{figure}


A intermediate FLIMera image is then constructed from these $\Delta m$, and $\Delta\theta$ values as well as a translation matrix to re-centre the image using an affine warp and nearest neighbour interpolation. Until now, the image origin (0,0) is taken as the upper-left corner which results in undesired shifting of the scaled image upon the application of a scaling and rotation. To account for this a further translation matrix is used to shift the image by an amount equal to \SI{-6558}{\pixel} and \SI{-3458}{\pixel} in the horizontal and vertical axes, respectively. The translations between this intermediate FLIMera image and the sCMOS image are found using standard phase cross-correlation to be $\Delta x = \SI[round-mode = figures, round-precision = 3]{-205.9709}{\pixel}$ and $\Delta y = \SI[round-mode = figures, round-precision = 3]{-298.35}{\pixel}$.
Finally, the magnification, rotation, and translations are used to construct a single Affine transform (see Eq.~\ref{eq:SFLIOaff}) and then the registered FLIMera image is then produced. The accuracy of this registration is assessed using a normalised line profile taken through an identical region in both images, as shown in Fig.~\ref{fig:USAFregistered}, and calculating the cross-correlation of both line profiles. For a perfectly registered image the location of the peak would occur at \num{0}, however, for the SFLIO system this location is \num{1} indicating a registration accuracy of \num{1} or fewer pixels on sCMOS detector. This greatly exceeds the minimum accuracy needed for sub-pixel registration on the FLIMera where a registration accuracy of less than $1/\Delta m$ is required.


\begin{figure}
    \centering
    \begin{annotatedFigure}
        {\includegraphics[width = 0.85\textwidth]{figures/sflio-device/AirForceTarget-RegisteredImage.pdf}}
        \annotatedFigureText{0.04,0.94}{black}{0.4}{a)}
        \annotatedFigureText{0.04,0.50}{black}{0.4}{c)}
        \annotatedFigureText{0.52,0.94}{black}{0.4}{b)}
    \end{annotatedFigure}
    \caption{Results of static image registration the sCMOS and FLIMera on the S-FLIO device using a USAF test target and the Fourier-Merlin algorithm for recovering scale, rotation, and translation transforms. a) Shows the original, high resolution, image recorded with the sCMOS camera; b) Shows the FLIMera image after being scaled, rotated, and translated using a affine transform and nearest neighbour interpolation. The accuracy of the registration process is assessed using the cross-correlation peak of two line profiles (cyan and magenta) through identical regions of a) and b) plotted in c) and yields an accuracy of 1 pixel}
    \label{fig:USAFregistered}
\end{figure}
\begin{equation}\label{eq:SFLIOaff}
\hat{\mathbf{A}} = \begin{bmatrix}
    \num[round-mode = figures, round-precision = 3]{4.202248819333833} & \num[round-mode = figures, round-precision = 3]{-0.007244469694988512}& \num[round-mode = figures, round-precision = 3]{-6856.35}\\
    \num[round-mode = figures, round-precision = 3]{0.007244469694988512} & \num[round-mode = figures, round-precision = 3]{4.202248819333833}& \num[round-mode = figures, round-precision = 3]{-3663.9709}\\
    \num[round-mode = figures, round-precision = 3]{0} & \num[round-mode = figures, round-precision = 3]{0}& \num[round-mode = figures, round-precision = 3]{1}
\end{bmatrix}    
\end{equation}

\subsection{Movement Registration}
For a moving sample the shift detected between an arbitrary frame and reference frame recorded with the sCMOS detector can be converted into an equivalent shift in the FLIMera using the previously calculated affine transform, $\hat{\mathbf{A}}$, and the be applied to that frames corresponding fluorescence decay histogram to finally a construct a high quality, movement free, fluorescence lifetime image. 
This conversion is constructed by defining the relationship between a reference frame, $N$, and a later frame $N+\epsilon$ which has shifted by some $\Delta x_{\epsilon},\Delta y_{\epsilon}$:
\begin{align}
I_{FLIMera}^{N+\epsilon}(x',y') &= I_{FLIMera}^{N}(x' + \Delta x'_{\epsilon} ,y'+ \Delta y'_{\epsilon}) = \mathbf{A}(\Delta x'_{\epsilon} ,\Delta y'_{\epsilon})I_{FLIMera}^{N}(x',y')\\
I_{sCMOS}^{N+\epsilon}(x,y) &= I_{sCMOS}^{N}(x + \Delta x_{\epsilon} ,y+ \Delta y_{\epsilon}) = \mathbf{A}(\Delta x_{\epsilon} ,\Delta y_{\epsilon})I_{sCMOS}^{N}(x,y)
\end{align}


%%% To Do %%%
% Make maths below look thesisy
For a frame $N$ we have:
\begin{align}
I_{sCMOS}^{N}(x,y) = \hat{\mathbf{A}} I_{FLIM}^{N}(x',y')
\end{align}
And for some later frame, $N + \epsilon$ we have a similar relationship:
\begin{align}
I_{sCMOS}^{N+\epsilon}(x,y) = \hat{\mathbf{A}} I_{FLIM}^{N+\epsilon}(x',y')    
\end{align}
Now we do some magic to pull out the affine transform in terms of the FLIMera:
\begin{align}
 I_{sCMOS}^{N + \epsilon}(x,y) &= I_{sCMOS}^{N}(x+\Delta x_{\epsilon}, y + \Delta y_{\epsilon}) = \mathbf{A}(\Delta x_{\epsilon}, \Delta y_{\epsilon})I_{sCMOS}^{N}(x,y) \\
 &= \mathbf{A}(\Delta x_{\epsilon}, \Delta y_{\epsilon})\hat{\mathbf{A}}I_{FLIM}^{N}(x',y')\\
I_{FLIM}^{N + \epsilon}(x',y') &= I_{FLIM}^{N}(x'+\Delta x'_{\epsilon}, y' + \Delta y'_{\epsilon}) = \mathbf{A}(\Delta x'_{\epsilon}, \Delta y'_{\epsilon})I_{FLIM}^{N}(x',y')
\end{align}
Using this system of equations we can formulate $\mathbf{A}(\Delta x'_{\epsilon}, \Delta y'_{\epsilon})$:
\begin{align}
I_{sCMOS}^{N + \epsilon}(x,y) &=\hat{\mathbf{A}} I_{FLIM}^{N + \epsilon}(x',y') \implies I_{FLIM}^{N + \epsilon}(x',y') = \hat{\mathbf{A}}^{-1}I_{sCMOS}^{N + \epsilon}(x,y)\\
I^{N + \epsilon}_{FLIM}(x',y') &= \hat{\mathbf{A}}^{-1}\mathbf{A}(\Delta x_{\epsilon}, \Delta y_{\epsilon})\hat{\mathbf{A}}I^{N}_{FLIM}(x',y') 
\end{align}

Or in simpler terms the movement in the FLIMera can be corrected using the following Affine Transform:
\begin{align}
\mathbf{A}(\Delta x'_{\epsilon}, \Delta y'_{\epsilon}) &= \hat{\mathbf{A}}^{-1}\mathbf{A}(\Delta x_{\epsilon}, \Delta y_{\epsilon})\hat{\mathbf{A}}
\end{align}

\subsubsection{Demonstration}





\begin{itemize}
    \item When imaging animals \textit{in-vivo} there will always some movement of the subjects eye in the form of quick, jerk-like, movements - saccade which see the eye move across a range of 25degrees in a period on the order of 200ms; and slow drifts from the ideal fixation point.
    \item Over the minute long acquisition period required for \textit{in-vivo} imaging this results in a blurred image devoid of any discernible features.
    \item To compensate for this the sCMOS camera is configured to record sequences of images over the FLIMera's acquisition period. The translation and rotation between consecutive frames can be extracted and can be applied to the FLIMera to result in a single high quality fluorescence lifetime image.
    \item To begin, first the mapping between the two camera must be found. 
    \begin{align*}
        I_{FLIM}(x,y) &= \mathbf{A}I_{scmos}(x',y')
    \end{align*}
    where the mapping $\mathbf{A}$ can be considered to be a simple affine transform.
    \item A generalised affine transform is composed for 4 other principle transformations in the form of Translation, Magnification/Scaling, Rotation, and Shear. In the FLIO system shear can be excluded since it is not significantly present in our images and any small amounts of shear can be embodied into rotation, scale, and translation. 
    \item Phase cross-correlation is used to perform this registration
    \item Registration was performed using images of convalaria mounted in an eye-phantom as well as an USAF target imaged at infinity that are recorded using the sCMOS camera and integrated intensity images from the FLIMera
    \begin{itemize}
        \item First, the scale and rotation transformations between the two images is calculated.
        \item Both images have a Butterworth bandpass filter and a Hanning window applied to amplify the contrast in the edges of the convalaria slide and USAF.
        \item A 2D FFT and then a log-polar transform are then applied to both images and the translation between these resulting power spectra are found
        \item A translation in the horizontal direction represents the scaling factor and a translation in the vertical direction represents a rotation between the images
        \item An interpolation based up scaling factor is applied to these images in order to increase the accuracy of the recovery of the scaling and rotation
        \item The scaling and rotation is then applied to the to FLIMera image using an affine warp
        \item the translation between images is then found using phase - cross - correlation on the now rotation and scale corrected integrated FLimage being careful to preserve the image origin as the top right corner of the image.
        \item The final Affine transform can then constructed from the recovered scale, rotation, and translation matrices which is then applied to the original, uncorrected, integrated FLIMimage. 
        \item To crudely assess the accuracy of the registration the sCMOS image and the integrated image can be mapped to the Red and Green channels of a RGB image (after converting to an 8bit image and normalising to produce similarly bright intense images)
        \item Using the USAF target, this process is repeated and using a line profile through identical regions the registration can be quantitatively compared using a cross-correlation of he resulting line profiles.
        \item Insert results
        \item static registration can be performed with single pixel precision.
    \end{itemize}
\item For a moving eye there is only translation and minute amounts of rotation between frames. 
\item A convallaria sample was positioned in the the eye phantom once again and was now slowly moved over a 1 minute acquisition time where during the first 30 seconds it was left still to allow adequate signal for static registration.
\item the translation between successive frames was calculated, was combined with the affine map previously found, and applied to the raw photon streams.
\end{itemize}
