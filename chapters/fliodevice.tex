\setstretch{2}
\FloatBarrier
\section{Construction of \textit{En-face} S-FLIO system}
\FloatBarrier
\subsection{S-FLIO System Design Methodology}
As was outlined in \cref{sec:objresearch} the purpose of this project is to use the principles of fluorescence lifetime imaging and spectral fluorescence imaging to further develop capabilities of chemical sensing in the retina. This will be achieved using an off-the-shelf fundus camera modified to facilitate spectrally-resolved fluorescence lifetime measurements of the retina within the constraints that the device can image the fovea - an FOV of at least \SI{20}{\degree}. While this field-of-view is generally considered to be narrow in ophthalmic imaging - where existing point scanning systems can image \SI{82}{\percent} of the retinal surface or a (\SI{200}{\degree} FOV~\cite{kato2019quantitative}) - this device is to serve as an early prototype to establish the efficacy this new imaging technique. To image larger field-of-views then multiple exposures across the retina can be recorded and stitched together. 
The total-image acquisition time is also a key constraint in designing a retinal imaging device for routine clinical assessments. 
In fluorescence lifetime imaging of the retina the safe exposure of the excitation source and the poor quantum efficiency of the retina brings about a high-noise and photon starve regime.
For the proposed S-FLIO device the excitation and imaging bands will be optimised to enhance the detection, and discrimination, of key fluorescent retinal biomarkers - FAD, AGE, A2E. New generation SPAD arrays will allow SFLIM measurements to be recorded in a series of snapshots which in principle can reduce image acquisition times when compared to raster scanned systems.



\FloatBarrier
\subsection{Conventional Fundus Camera}
The S-FLIO device developed in this project uses an off-the-shelf fundus camera to simplify the design and construction process allowing more time to be spent developing methods of chemical quantification in the retina. This apporach also allows for the SFLIO system to be more readily integrated into OPTOS's existing product lines where it could be sold as a standalone system or a modification for existing fundus camera.
% The S-FLIO device developed over the course of this project is based upon an off-the-shelf fundus camera (Topcon TRC 50-DX) used routinely in ophthalmic examinations reduces the complexity of the design process in that the project was now focused on fluorescence lifetime imaging and chemical quantification of the retina rather than arduous process and optimising and constructing as retinal imaging system~\cite{dehoog2008optimal}. Further, by utilising an existing imaging ophthalmoscope the SFLIO system developed can be more readily adapted to a commercial product in line with OPTOS's existing product line where it could be sold as a standalone imaging system or as a modification or add-on to existing fundus cameras. 
The fundus camera can record images over three field-of-views - \qtylist{20;35;50}{\degree} and uses a filtered white-light Xenon flash lamp to record images in either: brightfield, autofluorescence, or fluorescence imaging using fluorescein or idocyanine green dyes for retinal angiography.
% Got to here in redrafting - 01/07/24

The fundus camera is ordinarily capable of recording images over three different field of views - \qtylist{20;35;50}{\degree} and the ability to compensate for various for different severity of short and farsightedness through a \num{-22} to \num{+23} diopter correction. The device uses a Xenon flash lamp for white light with the necessary filters to record images in 3 different modalities: brightfield or reflectance imaging; autofluorescence imaging; and fluorescence imaging using an exogenous dye such as fluorescein or idocyanine green in retinal angiography. Alignment of the fundus camera to the patients eye is achieved using a lower intensity Tungsten inspection lamp to illuminate the retina and two small green LED's that project a pair of dots onto the patients cornea which come into focus when the fundus camera is positioned correctly in the axial direction. For both illumination sources - the Xenon flash lamp and Tungsten inspection lamp - the retina is illuminated using the concept of annular illumination, shown in \cref{fig:annularillumination}, where the retina is illuminated using only the periphery of the pupil. Using this annular ring almost all of the pupil is usable for imaging through and this forms the basis of the Gullstrand Principle whereby minimising the overlap between imaging and illumination volumes reduces the influence of back-reflected illumination light - increasing image quality~\cite{gullstrand1910neue}.
\begin{figure}
    \centering
    \includegraphics[width = 0.7\textwidth]{figures/sflio-device/AnnularIlluminationDiagram.pdf}
    \caption{The retina is evenly illuminated through an annulus (green) leaving the centre of the pupil unobstructed to image through (red).Adapted from~\citeauthor{terry2022trans}\cite{terry2022trans}}
    \label{fig:annularillumination}
\end{figure}
The modifications made (see \cref{fig:sflioopticsfull}) to the existing optics to facilitate excitation of retinal fluorescence and the spectral fluorescence lifetime measurements are described in the proceeding two sections in terms of the changes made to the illumination arm and the imaging arm of the fundus camera.

\begin{figure}
    \centering
    \includegraphics[width = 0.5\textwidth]{figures/sflio-device/FundusCameraOpticDiagram.pdf}
    \caption{Optical diagram of the SFLIO device where the principle rays of the illumination and imaging paths are rendered as the green and cyan rays, respectively. The modifications made to the existing optics of the fundus camera to facilitate spectral fluorescence lifetime imaging is shown in blue highlighted region and the new illumination optics are shown in the red region. Reproduced from \citeauthor{shibatafundus2003}~\cite{shibatafundus2003} where the additional illumination optics are shown in the red highlighted region and the imaging optics are shown in the blue highlighted region.}
    \label{fig:sflioopticsfull}
\end{figure}

\FloatBarrier
\subsection{Modification - Illumination Arm}
In the S-FLIO device, retinal fluorescence is excited by a pulsed-supercontinnum white-light source (NKT SuperK Ex-12W and Fianium) was paired with a Acousto-Optically Tuneable Filter (AOTF) to be filter the broadband emission into up to 8 separate channels each with a FWHM of $\approx\SI{1}{\nm}$ which enhances abilities to target specific retinal fluorophores using their unique excitation spectra. The fibre output of this source is coupled into the existing illumination path of the fundus camera by replacing a filter wheel - ordinarily used for spectral and fluorescence imaging - with a long-pass dichroic beam-splitter, positioned a \SI{45}{\degree}, to reflect excitation light with a wavelength shorter than \SI{505}{\nm} into the illumination path while also retaining the ability to use the inspection lamp for alignment of targets albeit now filtered to remove wavelengths shorter than \SI{505}{\nm}. To accommodate the differing Numerical Aperture (NA) of the fibre output of the laser a 1'' diameter plano-convex lens with focal length of \SI{100}{\mm} to first collimate the beam, and a second lens with focal length \SI{150}{\mm} is then used to refocus the beam to a point at the perimeter of the clear portion first annular stop, $A_{1}$ of \cref{fig:sfliooptics}, which is conjugate to the annulus projected onto the cornea and culminates in the retina being illuminated with a $\SI{20}{\degree}$(see \cref{fig:illuminationarea}). 

\begin{figure}
    \centering
    \begin{annotatedFigure}{\includegraphics[width = 0.8\textwidth]{figures/sflio-device/IlluminationOpticsExample.pdf}}
    \annotatedFigureText{0.02,0.84}{white}{0.3}{a)}
    \annotatedFigureText{0.53,0.84}{white}{0.3}{b)}
    \end{annotatedFigure}
    \caption{Modified illumination arm shows a reduced illuminated FOV on the retina when compared to the existing Tungsten inspection lamp. In a) an a brightfield image was recorded of a \textit{ex-vivo} enucleated rabbit eye using the Tungsten illumination source which fills the entire \SI{35}{\degree} FOV b) A fluorescence image was recorded using the modified illumination optics where fluorescence is excited between \qtyrange{460}{467}{\nm} and imaged using a \SI{500}{\nm} long-pass filter.}
    \label{fig:illuminationarea}
\end{figure}

\begin{figure}
    \centering
    \includegraphics{figures/sflio-device/IlluminationOpticsRayDiagram.pdf}
    \caption{Optical layout of the illumination arm of the SFLIO system required for illuminating the retina through the existing annular illumination system. The lens $f_{2}$ is adjusted until the a focussed point of light is produced on the patients cornea. Reproduced from \citeauthor{shibatafundus2003}~\cite{shibatafundus2003}}
    \label{fig:sfliooptics}
\end{figure}

While this design approach reduces the illuminated field-of-view on the retina minimises geometric losses arising from the multiple annular stops blocking the central lobe of the laser's Gaussian beam. Simply using a laser with a higher output power would mitigate these geometric losses however this would necessitate more rigorous fail-safe mechanisms in the event of any damage to the annular stops causing unsafe exposure levels.

\begin{equation}\label{eq:numericalaperture}
    NA = n\sin{\Bigg(\arctan{\bigg(\frac{D}{2f}\bigg)}\Bigg)}
\end{equation}

% Previous design iterations, shown in \cref{fig:IlluminationV1},used a similar design methodology but instead the Xenon arc-flash lamp was replaced with a $\nicefrac{1}{2}''$-diameter dichroic beam-splitter with a cut-on wavelength of \SI{505}{\nm} to couple the laser source into the illumination path. The collector lenses, positioned in front of the Xenon lamp, used in the illumination path exhibit a high NA of \num{0.885} - having a diameter of \SI{29.5}{\mm} and focal length of \SI{7.75}{\mm} - and to couple the laser light into this illumination path a pair of $\nicefrac{1}{2}''$-diameter plano-convex lenses with focal length \SI{12.7}{\mm}, after being collimated by a \SI{50.8}{\mm} focal length lens, were selected to match this numerical aperture. This design was rejected due to the severe geometric losses due to the aforementioned annular stops removing the majority of the energy in the Gaussian Beam. Further, this implementation resulted in irregular illumination of retina (\cref{fig:IlluminationV1}b) and degraded the image quality achieved using both the super-continuum source and the inspection lamp arising from the convergence angle of the beam focused being larger than \SI{45}{\degree} and thus being cropped by the dichroic mirror.


% \begin{figure}
%     \centering
%     \begin{annotatedFigure}{\includegraphics[width = 0.85\textwidth]{figures/sflio-device/IlluminationV1fFigure.pdf}}
%     \annotatedFigureText{0.00,0.87}{black}{0.3}{a)}  
%     \annotatedFigureText{0.40,0.87}{white}{0.3}{b)}
%     \annotatedFigureText{0.72,0.87}{white}{0.3}{c)}
%     \annotatedFigureText{0.40,0.36}{white}{0.3}{d)}
%     \annotatedFigureText{0.76,0.38}{white}{0.3}{e)}
%     \end{annotatedFigure}
%     \caption{Initial attempt of coupling the supercontinuum excitation source into the existing illumination path of the fundus camera. The pulsed supercontinuum source is coupled into the existing optics using a $f_{1}$ to collimate the fibre output and a pair of plano-convex $f_{2,3}$ to focus this beam onto a dichroic mirror with the principle rays shown in a). The high NA of the fundus camera optics resulted in an irregular illumination pattern on an mechanical eye model, c) when compared to an image of the same mechanical eye illuminated by the existing Xenon flash lamp. In d) and e) the optics are shown attached to the 3D printed mount and integrated into the fundus camera. }
%     \label{fig:IlluminationV1}
%end{figure}
\FloatBarrier
\subsection{Modification - Imaging Arm}
The imaging arm of the SFLIO device was designed to record fluorescence lifetime data using a the Horiba FLIMera SPAD array - referred to as the FLIMera herein - over a $\cong\SI{12}{\degree}$ field-of-view of the retina in parallel with a standard sCMOS detector (Thorlabs Kiralux) recording fluorescence intensity or brightfield images over a \SI{35}{\degree} field-of-view for the purpose of correcting for motion of the retina throughout the long image acquisition period. This configuration was constructed using three standard SLR lenses - chosen for their flat spatial frequency response, and ease of focus and aperture adjustment -  set up in a relay configuration, shown in \cref{fig:imagingoptics}, using lenses with focal lengths: $f=\SI{100}{\mm}$ to collimate the retinal image; $f = \SI{85}{\mm}$ to form the image on the sCMOS detector; and a $f = \SI{50}{\mm}$ lens to form the image on the FLIMera. A 2'' diameter plate beam-splitter is used to transmit \SI{10}{\percent} of the signal to the sCMOS camera and \SI{90}{\percent} to the FLIMera - this maximises the efficiency of the SFLIM measurements without degrading the quality of the registration images recorded with the sCMOS detector. Due to the difference in pixel-counts between the two detectors - $\qtyproduct{192 x 126}{\pixel}$ for the FLIMera and \qtyproduct{4096 x 2160}{\pixel} the field-of-view covered by the FLIMera was reduced by a linear factor of $\approx 3$ to improve the spatial resolution of the acquired images with the intention that if SFLIM images of larger areas of the retina the patient can fixate on their eyes away from centre.

\begin{figure}
    \centering
    \includegraphics[width = 0.9\textwidth]{figures/sflio-device/FundusCameraOpticDiagram-ImagingOptics.pdf}
    \caption{Optical layout of the imaging arm of the SFLIO system. The retinal image at the image plane, $p_1$, is relayed using SLR lenses where: $f_1$ collimates the image; $f_{2}$ forms the reflected image from the 90:10 beam-splitter (BS) onto the FLIMera; and $f_{3}$ forms the transmitted image onto the sCMOS camera for co-registration of the moving retinal images. Registration images can be recorded using either in brightfield or fluorescence using a \SI{500}{\nm} long-pass filter at $p_{1}$ }
    \label{fig:imagingoptics}
\end{figure}
For recording spectrally resolved fluorescence lifetime images 5 different emission bandpass filters - forming 2 sets of 4 filters - are fitted into the filter allowing for fluorescence images to be recorded with the FLIMera while retaining the ability to record brightfield or fluorescence intensity images (using an additional \SI{500}{\nm} long-pass filter) with the sCMOS camera for increased photon throughput. These spectral filters were chosen to maximise light throughput while minimising error in discriminating retinal fluorophores (see \cref{sec:tensband}) and the central wavelengths and bandwidths are shown in \cref{tab:sflio-filts} where filters 4a and 4b refer to two filters identified in the optimisation process with near equal unmixing performance. For \textit{in-vivo} rat imaging either 4a or 4b were used in the measurement process resulting in 4 total spectral bands.

\begin{table}
    \centering
    \begin{tabular}{|c|c|c|}
        \hline
        Filter No.  & CWL (\si{\nm}) & FWHM (\si{\nm})\\
        \hline
        1 & 500 & 24\\
        2 & 549 & 15\\
        3 & 561 & 14\\
        4a & 640 & 75\\
        4b & 675 & 50\\
        \hline
    \end{tabular}
    \caption{Spectral filters used in the SFLIO system for enhanced discrimination of retinal fluorophores from spectrally-resolved fluorescence lifetime measurements. In spectral-FLIM measurements only 4 filters are used where either filter 4a or 4b is used}
    \label{tab:sflio-filts}
\end{table}
\FloatBarrier
\subsection{Horiba FLIMera}
\subsubsection{FLIMera Architecture}
The FLIMera is used to record the widefield time-resolved fluorescence decays and comprised of an array of \qtyproduct{192 x 126}{\pixel} Single Photon Avalanche-Diodes with on pixel photon counting modules to enable TCSPC imaging across the full sensor with a temporal resolution of \SI{47}{\ps}. Based upon a similar architecture to SPAD array of \citeauthor{henderson2019192} the FLIMera has a pixel pitch of \qtyproduct{18.4 x 9.2}{\um} where half of each pixel is the photosensitive area - the SPAD  - and the other half is dedicated to the photon counting electronics an image sensor with an effective 2:1 pixel aspect ratio (see \cref{fig:SPADarch}) and fill factor of \SI{13}{\percent}~\cite{henderson2019192}. 
\begin{figure}
    \centering
    \includegraphics[width= 0.5\textwidth]{figures/sflio-device/SPADarchitecture.png}
    \caption{Schematic layout of the architecture used to produce the FLIMera sensor. The FLIMera features \qtyproduct{192 x 126}{\pixel} individual SPADs each with its own Time-to-Digital-Converter enabling TCSPC measurements with a pixel pitch of \qtyproduct{18.4 x 9.2}{\um} or equivalently a pixel aspect ratio of 2:1. Reproduced from \citeauthor{henderson2019192}\cite{henderson2019192}}
    \label{fig:SPADarch}
\end{figure}
While this fill factor is low when compared to traditional - non-TCSPC - imagers it does represent an improvement in fill-factor and scalability of pixel counts over existing TCSPC imaging solutions \cite{cusini2022historical,gersbach2010high,ulku2018512}. Further this fill factor can be improved using a combination of micro-lens arrays to focus light incident within a pixel onto the photosensitive area of the pixel (see \cref{fig:SPADlayout}b) - increasing fill-factor \SI{42}{\percent} in the case of the FLIMera \cite{henderson2019192} - as well as a multilayered, back-illuminated, pixel architecture (see \cref{fig:SPADlayout}c) where the photon counting electronics are manufactured on the layer beneath the photosensitive area of the SPAD also enabling an increased.
\begin{figure}
    \centering
    \begin{annotatedFigure}{\includegraphics[width = \textwidth]{figures/sflio-device/SPADlayoutFigure.pdf}}
    \annotatedFigureText{0.01,0.80}{black}{0.3}{a)}
    \annotatedFigureText{0.32,0.8}{black}{0.3}{b)}
    \annotatedFigureText{0.67,0.8}{black}{0.3}{c)}
    \end{annotatedFigure}
    \caption{The FLIMera features a low fill-factor of \SI{13}{\percent} due to the TDCs being located on the same plane as the SPADs (a). To improve this up to \SI{42}{\percent} a micro-lens can be placed over each pair of pixels to focus light that would ordinarily impinge a TDC - and not be detected - onto the photosensitive area of the pixel \SI{42}{\percent}. Additionally, a multilayered fabrication approach, shown in (c), where the SPADs are layered on top of the TDCs and connected using vias (red circles) - interconnects between layers - can be used to improve not only the fill-factor but potentially allow for higher pixel densities.}
    \label{fig:SPADlayout}
\end{figure}
\FloatBarrier
\subsubsection{TCSPC Acquisition Modes}
The FLIMera offers two methods of acquisition: a default mode where a single XYT data cube is constructed spanning a predetermined acquisition window; and the ``raw'' mode where instead of outputting a single XYT data cube each photon detection event is streamed from the FLIMera at a rate of \SI{24}{\kilo\hertz} to form ``frames''. These photon records take the form of a \SI{4}{\byte} word which from the least to most significant bit contains the X (\SI{1}{\byte}), Y (\SI{1}{\byte}), and time bins (\SI{1.5}{\byte}) coordinate of the photon detection with the final \SI{4}{\bit} are used as a frame marker (1111) to denote the next frame in the acquisition or a photon record. This higher frame-rate mode comes at the penalty of larger file  where a \SI{2}{\minute} integration time in ``raw'' results in a \SI{25}{\giga\byte} compared to the $\approx\SI{25}{\mega\byte}$. Primarily, this mode enables time-evolving scenes to imaged and in the case of this project the movement of the retina throughout the acquisition period can be accommodated for (see \cref{sec:motionreg}). 
\FloatBarrier
\subsubsection{FLIMera Calibration}
\FloatBarrier
\paragraph*{Dead and Hot Pixel Masking\\}\label{sec:pixelmask}
The FLIMera used in this project is still considered a pre-production model and as such the fabrication process of the sensor yields malfunctioning pixels. These malfunctioning pixels present as either ``dead'' pixels and do not produce a signal in response to impinged photons or ``screamers'' which produce a histogram resembling a DC signal. These errant pixels were removed from resulting FLIMera images using a binary pixel mask constructed by thresholding FLIMera datasets comprising of a scene with uniform fluorescence - a fluorescence microscope slide (Thorlabs - FSK4) - allowing most dead pixels and screamers to be identified. Any remaining further aberrant pixels are distinguished with a dark frame where only thermal noise is present.


\begin{figure}
    \centering
    \begin{annotatedFigure}{\includegraphics[width = 0.8\textwidth]{figures/sflio-device/FLIMERA-MASK-FIGURE.pdf}}
    \annotatedFigureText{0.015,0.94}{black}{0.3}{a)}
    \annotatedFigureText{0.5,0.94}{black}{0.3}{b)}
    \annotatedFigureText{0.015, 0.4}{black}{0.3}{c)}
    \annotatedFigureText{0.5, 0.4}{black}{0.3}{d)}
    \end{annotatedFigure}
    \caption{To correct for malfunction pixels in the FLIMEra (dead and screamers) a uniformly fluorescent scene is imaged and (a) and a binary pixel mask is pproduced by thresholding the image (b). Dead pixel (blue line) produce no output while the signal of a screamer resembles a DC signal with Heaviside activation (purple) an results in an increased number of detected photons when compared to a typical functioning pixel (green line). The thresholding process to produce the pixel mask is performed using a histogram of the integrated photon counts where dead and screamers appear outside of the normally distributed intensities in the scene.}
    \label{fig:pixel}
\end{figure}
\FloatBarrier
\paragraph*{Temporal Misalignment\\}\label{sec:taling}
Another artefact of the pre-production nature of the FLIMera is each SPAD within the array not being aligned in the time axis - the peak of the fluorescence decays do not all occur at the same time bin - and results in improper fitting of fluorescence lifetimes or in the recovery of retinal fluorophore concentrations. Correcting this misalignment can be achieved using a pixel offset mask (see \cref{fig:shiftmap}b) which which describes how many time bins each decay profile should be shifted by. This mask is produced by finding the time-location of the maxima in each pixel and subtracting it from a basis value which ensures the peak of histogram is shifted to that basis value.

\begin{figure}
    \centering
    \begin{annotatedFigure}{\includegraphics[width = \textwidth]{figures/sflio-device/ShiftMap-Figure.pdf}}
    \annotatedFigureText{0.03,0.96}{black}{0.3}{a)}
    \annotatedFigureText{0.53,0.96}{black}{0.3}{b)}
    \end{annotatedFigure}
    \caption{In (a) example histograms show the temporal misalignment within the FLIMera. These misalignment's are corrected for by using a shift map (b) which shifts contains the required shift along the time axis to align every pixel on the FLIMera}
    \label{fig:shiftmap}
\end{figure}

\FloatBarrier
\subsection{Safety Assessment}
A critical design requirement for a retinal imaging device is that can should be recorded while posing minimal risk and discomfort to the patient. In the SFLIO device two similar pulsed-supercontinuum laser sources were used throughout this project which both have the capability of producing broadband light at an intensity capable of causing irreparable damage many times over. As such, a full safety assessment was carried out, in collaboration with OPTOS, to determine the save exposure levels and the required attenuation of the laser source such that even when operating at the sources maximum intensity the exposure at the retina is at a safe level. Although \textit{in-vivo} human imaging was not achieved in the duration of this project this assessment is still useful for determining the efficacy of our device in measuring retinal FAD when compared with the laser intensity required for detecting FAD in rats \textit{in-vivo} (\cref{chap:ratfad}).
\subsubsection{Phototoxic and photothermal effects}
When the eye is exposed to light there is two distinct, wavelength dependent, mechanisms from which damage can occur to retina or the other constituent parts of the eye - phototoxic and photothermal damage. Phototoxic or photochemical damage predominantly occurs at wavelengths shorter than \SI{500}{\nm} where absorbed photons cause oxidative reactions within the cells of the RPE, choroid, and the retina causing cellular death and with sufficient exposure results in permanent loss of vision~\cite{glickman2002phototoxicity}. Further, in the extreme UV range of the visible spectrum ($<\SI{200}{\nm}$) these absorbed photons can directly cause the breaking of covalent bonds and with pulse lengths shorter than \SI{1}{\micro\second}, tissue can be ablated - forming the basis of LASIK surgery whereby the cornea is reshaped to correct the refractive power of the eyes lens in cases of myopia, hyperopia, and astigmatism~\cite{reinstein2012history}. In photothermal damage longer wavelength photons cause heating of the tissue of varying severity where damage occurs from necrosis of the tissue or in the case of severe and rapid heating the vitreous humour cavitates the the retina surface causing lesions~\cite{glickman2002phototoxicity,roberts2001ocular}.
The supercontinuum laser source used in the SFLIO device does not produce light in the extreme UV so photochemical ablation is not a concern however the risks due to photochemical oxidation or photothermal damage mechanisms are embodied within the safe exposure thresholds defined by the two different safety standard documents used to evaluate the device.

\FloatBarrier
\subsubsection{Safe Exposure Thresholds}
To evaluate the safety of the SFLIO device two independent standards were utilised: the \citeauthor{ISO15004} standard describes the evaluation of safety for ophthalmic devices demarcated as either Group 1 devices where there is no risk of ocular damage and Group 2 where some risk does exist but is still safe to use under strict guidelines; and the \citeauthor{IEC60825} standard which is used to define the safety of laser systems designated from Class 1, which are considered eye-safe, up to Class 4B where even stray reflections from diffuse surfaces can pose a risk to eyesight\cite{ISO15004, IEC60825}. In the ISO standard the instrument was evaluated against the criteria for a Group 1 continuous wave device with the IEC document being used to evaluate the instrument as a Class 1 pulsed laser source. The IEC document contains stricter criteria for a pulsed sources and, unlike the ISO standard, considers the additional damage mechanisms arising from the use of a short pulse-length and high repetition rate laser source such as micro-cavitation, and self-focusing of the laser beam~\cite{rockwell2010ultrashort}.
A detailed description of the procedure used determine the multiple safe exposure limits are listed in App.~\ref{app:sfliosafety} however for both standards the exposure limits, the measured exposure, and the resulting safety factor are given and defined here.
\paragraph*{Continuous Wave - \citeauthor{ISO15004}\\}
In the ISO standard the SFLIO instrument is evaluated as a Group 1 device where the exposure limits for this classification are defined in terms of a wavelength dependent irradiance, $E(\lambda)$, pertaining to a particular band of the visible spec each section of the eye - retina, cornea and lens, and the anterior segment. For each of these sections, where relevant, there are also weighting functions applied to account for the photothermal and photochemical damage mechanisms shown in \cref{fig:safetyspectra}c. In the safety assessment the spectral irradiance is defined using the power spectrum, $P(\lambda)$, and the area of the illuminated. 
\begin{equation}\label{eq:irrad}
    E(\lambda) = \frac{P(\lambda)}{Area} \cong \frac{P_{M}S(\lambda)}{\int_{0}^{\infty}S(\lambda)d\lambda}
\end{equation}
The power spectrum is measured by composing a measurement the spectra of the light emitted from the fundus camera with a measurement of the emitted power using a optical power meter. The optical power meter applies a wavelength-dependent bias function to over come the non-uniform response across the visible spectrum but since the bandwidth of the excitation is sufficiently narrow (FWHM $<\SI{11.5}{\nm}$ - see \ref{fig:safetyspectra}b) a single power measurement was recorded with the bias function set to be centred on \SI{468}{\nm} - the peak of the excitation. 
\begin{figure}
    \centering
    \begin{annotatedFigure}{\includegraphics[width = 0.8\textwidth]{figures/sflio-device/SafetyAssessmentSpectras.pdf}}
    \annotatedFigureText{0,0.95}{black}{0.3}{a)} 
    \annotatedFigureText{0.34,0.95}{black}{0.3}{b)} 
    \annotatedFigureText{0.68,0.95}{black}{0.3}{c)} 
    \end{annotatedFigure}
    \caption{The spectral output of the broadband output of the supercontinuum source was resampled from the calibration report supplied with the source in a) and in b) the output is filtered between \qtyrange{460}{467}{\nm} using an AOTF and the emission incident on the retina is measured using a spectrometer. The hazard weighting functions using in the assessment of the SFLIO device as a Group 1 continuous wave ophthalmic device is shown.}
    \label{fig:safetyspectra}
\end{figure}
The safety thresholds shown in \cref{tab:conthresh} were calculated with the optical power measured when the laser source was set to its maximum power output - $P_{M} = \SI{1.1}{\milli\watt}$ which exceeds the safe exposure limits for a Group 1 device by a factor \num{20}. A neutral density filter with $\text{OD} = \num{1.3}$, calculated using \cref{eq:OD}  was fitted into the illumination arm to attenuate the optical power incident on the retina to eye-safe levels of $P_{M}=\SI{46}{\micro\watt}$ and the various safe exposure thresholds were recalculated with a safety factor defined as $\eta = \nicefrac{\text{Threshold}}{\text{Value}}$.
\begin{equation}\label{eq:OD}
    P = P_{0}~10^{-\text{OD}}
\end{equation}

\begin{table}[ht]
    \centering
    \resizebox{0.98\textwidth}{!}{
    \begin{tabular}{|p{3cm}|c|c|c|c|c|c|c|}
        \hline
        \textbf{Description} & \textbf{Symbol} & \textbf{Equation} & \textbf{Threshold} & \textbf{Measured} & \textbf{Unit} & \textbf{Safety Factor} & \textbf{Ref}\\
        \hline
        Weighted retinal irradiance & $E_{A-R}$ & $\sum_{350}^{700}E(\lambda)A(\lambda)\Delta\lambda$ & \num{220} & \num {178} & $\si{\micro\watt\per\centi\metre^{2}}$ & \num{1.24} & $5.3.1.3a$ \\
        \hline
        Unweighted corneal and lenticular infrared irradiance & $E_{IR-CL}$ & $\sum_{770}^{2500}E(\lambda)\Delta\lambda$ & \num{20} & \num{0.061} & $\si{\milli\watt\per\centi\metre^{2}}$ & \num{326} & $5.3.1.4$ \\
        \hline
        Unweighted anterior segment visible and infrared radiation irradiance & $E_{VIR-AS}$ & $\sum_{380}^{1200}E(\lambda)\Delta\lambda$ & \num{4} & \num{0.0026} & $\si{\watt\per\centi\metre^{2}}$ & \num{1565} & $5.4.1.5$\\
        \hline
        Weighted retinal visible and infrared thermal irradiance & $E_{VIR-R}$ & $\sum_{380}^{1400}E(\lambda)R(\lambda)\Delta\lambda$ & \num{0.7} & \num{0.00028} &$\si{\watt\per\centi\metre^{2}}$ & \num{2458} & $5.3.1.3a$ \\
        \hline
    \end{tabular}
    }
    \caption{Safe exposure limits for a Group 1 device under the \citeauthor{ISO15004} standard and the resulting exposure measured from the SFLIO device with the laser source operating at maximum power - filtered between \qtyrange{460}{467}{\nano\metre} - and a $\text{OD} = 1.3$ neutral density filter is fitted. The safety factor is calculated as $\nicefrac{\text{Threshold}}{\text{Value}}$. In the final column the reference to the equation to calculate the safe exposure limits is given.}
    \label{tab:conthresh}
\end{table}
\paragraph*{Pulsed Source - \citeauthor{IEC60825}\\}
In the IEC standard the safe exposure limits, termed as Accessible Emission Limits (AEL), are determined for a pulsed Class 1 laser source which is deemed safe for continuous exposure for up to 8 hours\cite{IEC60825}. Here the AELs are calculated in terms of the energy deposited by: a single pulse, a single pulse within a train of pulses - weighted by a safety factor; and the emission from \SI{100}{\second} worth averaged into a single pulse. The most pessimistic AEL is chosen for evaluation, $\text{AEL}_{s.p.T}$, and is compared against the power incident of $P=\SI{1.3}{\micro\watt}$ and converted to energy deposited per pulse - $E_{pulse} = P/f$ where $f$ is the repetition rate of the source.

\begin{table}[ht]
    \centering
    \resizebox{0.98\textwidth}{!}{
    \begin{tabular}{|p{6cm}|c|c|c|c|}
    \hline
    Description & Symbol & Threshold (\si{\joule}) & Measured (\si{\joule}) & Safety Factor \\
    \hline
    Accessible Emission Limit for a single pulse &  $AEL_{single}$ & \num{7.7e-8} & \num{5.75e-13} & \num{1.3e5}\\
    \hline
    Accessible Emission Limit for a single pulse averaged over a period of 100s & $AEL_{s.p.T}$ &\num{1.1e-12} & \num{5.75e-13} & \num{1.91}\\
    \hline
    Accessible Emission Limit for a single pulse in a train of pulses & $AEL_{s.p.train}$ &\num{3.08e-8} & \num{5.75e-13} & \num{5e4}\\
    \hline
    \end{tabular}
    }
    \caption{Comparison of calculated Accessible Emission Limits (AEL's) for a Class 1 laser system, with the measured energy per pulse of the SFLIO system after attenuating emitted power onto the retina with a $\text{OD} = 1.3$ neutral density filter. The safety factor is calculated as $\nicefrac{\text{Threshold}}{\text{Value}}$}
    \label{tab:pulsedvalues}
\end{table}
\subsubsection{Summary of Safety Assessment}
In this safety assessment the SFLIO device was evaluated using two sets of safety standards where the excitation source is treated as both a continuous wave source, and a pulsed source. To ensure to safe operation when carrying out \textit{in-vivo} imaging of human eyes the $\text{OD}=1.3$ neutral density filter should be fitted to ensure that when the laser source is set to its maximum optical power output it the exposure is within the limits set out in both the \citeauthor{ISO15004} and \citeauthor{IEC60825}. Further, this magnitude of attenuate also results in a the emission produed by the SFLIO device to be lower than the limits of safe exposure by factors of 1.91 when treating the device as a Class 1 pulsed laser source and 1.24 the instrument is evaluated as a Group 1 continuous wave sourced ophthalmic device under the \citeauthor{ISO15004} and \citeauthor{IEC60825}, respectively. Nonetheless for imaging human retinas the optical power emitted from the SFLIO device should be regularly measured to ensure that these safety factors are not solely relied upon.
\FloatBarrier

\section{Spectral Calibration}
To ensure accurate and robust recovery of fluorophore concentrations from spectrally resolved fluorescence lifetime measurements the spectral response of both the imaging system - the optics of the fundus camera - and the non-rectangular response of the spectral filters must be calibrated. By not performing this calibration step the recorded spectra will deviate from the spectral output of the retinal autofluorescence signal yields false concentrations of retinal fluorophores. 
In the SFLIO device the effective response for each spectral channel was measured using a spectrometer - where the detector head was placed at the image plane of the FLIMera - and a Xenon broadband white-light source illuminating through the objective lens of the SFLIO device. To account for the response of the light source, $S(\lambda)$ and the spectrometer itself,$D(\lambda)$- which is present in all measurements of the spectra filters - a direct measurement of the light source was recorded and is used in \cref{eq:filtcal,eq:filttrans} to recover the filter transmission functions through the imaging system, $F_{n}(\lambda)$, for each of the $n$ spectral bands. 
\begin{equation}\label{eq:filttrans}
    F_{n}(\lambda) = \frac{I_{n}(\lambda)}{\alpha(\lambda)}
\end{equation}
\begin{equation}\label{eq:filtcal}
    S(\lambda)D(\lambda) = \alpha(\lambda)
\end{equation}
\begin{equation}
    I_{n}(\lambda) = S(\lambda)D(\lambda)(\lambda)F_{n}(\lambda)
\end{equation}
In \cref{fig:filtfuncs} the calibrated filter functions are shown after using \cref{eq:filtcal,eq:filttrans} and subtracting dark spectra to remove the influence of scattered background light. The non-uniformity in the spectral response of the fundus camera optics can be attributed to cumulative absorption from the anti-reflective coatings on each lens within the fundus camera itself as well as the SLR lenses and beam-splitter used to relay the retinal image to the FLIMera as well as the beam-splitter.
\begin{figure}
    \centering
    \includegraphics[width = 0.8\textwidth]{figures/sflio-device/FilterTransmissionFunctions.pdf}
    \caption{The calibrated filter functions for each of the spectral filters are plotted (coloured lines) are used to account for the non-uniform response of the refractive optics within the fundus camera (dashed line) as well as the non-rectangular response of the spectral filters.}
    \label{fig:filtfuncs}
\end{figure}
\FloatBarrier
\section{Spatial Resolution of S-FLIO device}
While the purpose of this project was not to construct and demonstrate a diffraction-limited retinal imaging system but to record fluorescence lifetime images and quantitatively measure the distribution of fluorophores across the retina - a spatial resolution, with respect to images recorded on the retina with the FLIMera, sufficient to resolve features, and markers, common with retinal disease is key. For example, as has been discussed in \cref{intro:AMD}, the average diameter of drusen, a common marker of disease progression in AMD, is on the order of YY - to resolve drusen with the FLIMera the spatial resolution of the S-FLIO system must be lower than this value. Additionally, the spatial resolution of both detectors will influence the accuracy required when reconstructing images from fluorescence histograms of a retina undergoing motion. For a proportional difference in the spatial resolution $\nicefrac{\Delta r_{FLIMera}}{\Delta r_{sCMOS}} \leq M$ will require registration of images within a sub-pixel accuracy (with respect to the sCMOS detector) whereas for $\nicefrac{\Delta r_{FLIMera}}{\Delta r_{sCMOS}} > M$ the registration accuracy requirement is greater than a single pixel.
\subsection{Spatial Resolution vs. Pixel Size}
In a sampled imaging system the spatial resolution is not simply dictated by the size of a single pixel on the detector but instead how accurately the imaging system and detector can describe a scene e.g the retina. Due the optical characteristics of the eye and the aberrations inherent with the fundus camera and the additional optics required for fluorescence lifetime imaging as well as the sampled nature of the system this will be imperfect. Mathematically the response of the imaging system is denoted as Point-Spread Function (PSF) which describes how a point source in a scene is blurred in a recorded image. 
\begin{equation}\label{eq:psf}
    I_{IMAGE}(x,y) = PSF(x,y)\circledast I_{SCENE}(x',y')
\end{equation}
Additionally in the Fourier domain the Optical Transfer Function (OTF) or the approximately equivalent Modulation Transfer Function (MTF) can be used to described how well an imaging system resolves increasing spatial frequencies~\cite{vollmerhausen2000analysis}. Here the OTF and MTF are connected to the PSF by a Fourier transform.
\begin{align}
    J_{IMAGE}(\nu,\mu) &\approxeq MTF(\nu,\mu)\circledast K_{SCENE}(\nu',\mu')\label{eq:mtf}\\
    MTF(\nu,\mu) &= \mathcal{F}\{PSF(x,y)\}(\nu,\mu) \label{eq:PSFtoMTF}\\
    MTF(\nu) &= \mathcal{F}\{LSF(x)\}(\nu)\label{eq:LSFtoMTF}
\end{align}
Conventionally, the spatial resolution of an imaging system is defined by the full-width at half-maximum of the PSF or the spatial frequency where the MTF is half its maximum value. An example of a image blurred by a PSF is shown in \cref{fig:PSFsimulation} alongside its respective MTF.
\begin{figure}
    \centering
    \begin{annotatedFigure}{\includegraphics[width = 0.95\textwidth]{figures/sflio-device/PSFMTFexample.pdf}}
    \annotatedFigureText{-0.02,0.94}{black}{0.3}{a)}
    \annotatedFigureText{0.235,0.94}{black}{0.3}{b)}   
    \annotatedFigureText{0.495,0.94}{black}{0.3}{c)}   
    \annotatedFigureText{0.735,0.94}{black}{0.3}{d)}   
    \end{annotatedFigure}
    \caption{Here a \qtyproduct{512 x 512}{\pixel} sampled image, a) has been convolved with a Guassian PSF with a FWHM of 16 pixels c) to produced the degraded image in b) where the fine details in the background of the image can no longer be clearly resolved. The resulting MT, determined using \cref{eq:PSFtoMTF}, in d) shows the significantly reduced contrast at higher spatial frequency's due to the PSF.}
    \label{fig:PSFsimulation}
\end{figure}
\FloatBarrier
\subsection{Knife Edge Test}
Accurately measuring the MTF of an imaging system is often a onerous and time consuming task where standardised best practises are continually refined~\cite{ISO12233}. For the purpose of this project only estimates of the MTF are required and so the knife-edge test - often referred to as the slanted-edge test in literature - was chosen for its simplistic experimental procedure~\cite{burns2000slanted, ISO12233}. Briefly, this method allows the measuring of spatial resolution of a digital imager using an image recorded of a hard-edge positioned in the image plane and rotated a few degrees ($<\SI{10}{\degree}$) either off-vertical or off-horizontal where in detector limited imaging systems this edge appears aliased. Line profiles through the aliased edge are then aligned to produce an Edge-Spread Function (ESF) and the spatial resolution of the system can then be estimated from the LSF (\cref{eq:ESF}) or from the computed MTF. The rotation of the edge in the image plane enables the ESF to be sampled with sub-pixel precision due each line profile - through the edge - being displaced by less than a single pixel.
\begin{equation}\label{eq:ESF}
    ESF(x) = \dv{}{x}LSF(x)
\end{equation}
In the S-FLIO system this Knife edge test is performed simultaneously on both detectors using an eye phantom where the edge is comprised of a scalpel blade and fluorescent microscope slide (Thorlabs - FSK4) positioned at the retinal plane. The emission spectra of the fluorescent slide allows exciting fluorescence at \qtyrange{460}{462}{\nm} and imaging above \SI{500}{\nm} with sufficient photon flux to record high quality images in both the FLIMera and the sCMOS camera with integration times of \SI{250}{\ms} and \SI{5}{\second}, respectively. 
This process is shown in~\cref{fig:FIMTFresult,fig:FLMTFResult}, where in the analysis of the knife edge images, the location of the edge within the image is first estimated by finding the peak of the absolute value of the derivative of lines-profiles taken through the edge. Using a linear fit through these edge locations the line-profiles are aligned and binned into \SI{0.25}{\pixel} intervals to yield the ESF. As above, the LSF and MTF can now be computed from the ESF. The LSF and MTF for both detectors are then calibrated using a image recorded of an  0.05'' (\SI{1.27}{\mm}) Allen key positioned in the phantom eye where it was found one pixel on the FLIMera and sCMOS detector equated to \SI{66}{\um} and \SI{6.3}{\um} respectively. To determine the spatial resolution of the system a Gaussian is fitted through each LSF in order to reduce the influence of noise and effects of quantisation and the FWHM is calculated. For the FLIMera the spatial resolution was found to be \SI{214}{\um} and for the sCMOS detector this found to be \SI{50}{\um}. 

\begin{figure}
    \centering
    \begin{annotatedFigure}{\includegraphics[width =\textwidth]{figures/sflio-device/FIMTFFigure.pdf}}
    \annotatedFigureText{0.025, 0.965}{black}{0.3}{a)}
    \annotatedFigureText{0.66,0.965}{black}{0.3}{b)}
    \annotatedFigureText{0.66,0.67}{black}{0.3}{c)}
    \annotatedFigureText{0.025,0.39}{black}{0.3}{d)}
    \annotatedFigureText{0.35,0.39}{black}{0.3}{e)}
    \annotatedFigureText{0.66,0.39}{black}{0.3}{f)}
    \end{annotatedFigure}
    \caption{From the recorded knife edge image using the sCMOS detector (a) the edge is detected and fitted (b) using the gradient of the edge profiles, shown in (c). The edge profiles are aggregated to form the a super sampled ESF (d). The LSF is then determined from the ESF using Eq.~\ref{eq:ESF} and the spatial resolution of \SI{214}{\um} is determined from the FWHM of a fitted Gaussian (black line). The MTF, shown in (f), is then calculated from this LSF using Eq.~\ref{eq:LSFtoMTF} and is plotted with the diffraction limit of the eye.}
    \label{fig:FIMTFresult}
\end{figure}

\begin{figure}
    \centering
    \begin{annotatedFigure}{\includegraphics[width = 1\textwidth]{figures/sflio-device/FLKnifeEdgeFigure.pdf}}
    \annotatedFigureText{0.03,0.94}{black}{0.3}{a)}
    \annotatedFigureText{0.33,0.94}{black}{0.3}{b)}
    \annotatedFigureText{0.7,0.94}{black}{0.3}{c)}
    \annotatedFigureText{0.03,0.48}{black}{0.3}{d)}
    \annotatedFigureText{0.33,0.48}{black}{0.3}{e)}
    \annotatedFigureText{0.7,0.48}{black}{0.3}{e)}
    \end{annotatedFigure}
    \caption{From the recorded knife edge image using the FLIMera (a) the edge is detected and fitted (b) using the gradient of the edge profiles, shown in (c). The edge profiles are aggregated to form the a super sampled ESF (d). The LSF is then determined from the ESF using Eq.~\ref{eq:ESF} and the spatial resolution of \SI{50}{\um} is determined from the FWHM of a fitted Gaussian (black line). The MTF, shown in (f), is then calculated from this LSF using \cref{eq:LSFtoMTF} and is plotted with the diffraction limit of the eye.}
    \label{fig:FLMTFResult}
\end{figure}

To then assure that the spatial resolution was assessed at the optimal focus setting on the fundus camera the knife-edge test was repeated a what was initially deemed the best focus and then 3 additional points either side of this focus spaced in $\approx\SI{10}{degrees}$ intervals and the spatial resolution was calculated for each of the 7 points for both detectors. The results shown in \cref{fig:mtffoci} that in fact the optimal focus was actually located $\approx\SI{10}{degree}$ from what was visually determined to be the best focus. Giving the previously mentioned spatial resolutions of \SI{214}{\um} for the FLIMera and \SI{50}{\um} for the sCMOS detector at this new optimal focus and at both extremum's the the spatial resolutions are \SI{296}{\um} and \SI{328}{\um} for the FLIMera, and for sCMOS they are \SI{129}{\um} and \SI{152}{\um}. In the experiments described in later sections of this chapter the focus set to this new optimal focus with only minor adjustments being made to account for variations in the position of the eye phantom with respect to objective lens of the fundus camera.

\begin{figure}
    \centering
    \includegraphics[width = 0.8\textwidth]{figures/sflio-device/MTFFociFigure.pdf}
    \caption{The knife-edge test was repeated over a total of 7 different focii's centred on what was originally deemed to be the best focus and spaced $\approx\SI{10}{\degree}$ apart. The spatial resolution for each focii and detector was calculated and a degree 2 polynomial was fitted.}
    \label{fig:mtffoci}
\end{figure}
\subsubsection{Appraisal of Spatial Resolution}
While the spatial resolution of both detectors is sufficient to co-register retinal images, the measured value of the sCMOS detector significantly deviates from that of a similar system where an off-the-shelf fundus camera (Canon CF-60Z) was used to image the same eye phantom housing capillary tubes filled with blood for the purposes of measuring blood oxygenation in the retina~\cite{mordant2011validation}. In~\citeauthor{mordant2011validation} the spatial resolution of the system was estimated to be $\approx\SI{30}{\um}$ or $\approx\SI{3}{\pixel}$ by examining line profiles through an image of a \SI{150}{\um} outer-diameter capillary tube and computing the rise time - the number of pixels required to transition from \SI{10}{\percent} to \SI{90}{\percent} of the peak grey value within the capillary.

\begin{figure}
    \centering
    \begin{annotatedFigure}{\includegraphics[width = 0.9\textwidth]{figures/sflio-device/MordauntResolutionFigure.pdf}}
    \annotatedFigureText{0.035,0.90}{white}{0.3}{a)}
    \annotatedFigureText{0.52,0.90}{black}{0.3}{b)}
    \end{annotatedFigure}
    \caption{The line profile through a \SI{150}{\um} capillary tube filled with arterial blood and mounted in an eye phantom is shown in a) and b) record using a similar imaging system to the SFLIO system where an off-the-shelf fundus camera (Canon CF-60Z) is modified for hyper-spectral imaging for the purposes of measuring oxygen saturation of blood in the retina. In b), the number of pixels to transition from black-to-white or the rise-time, $\tau_{rise}$, is demarcated by the solid black lines and was computed as $\tau_{rise} \cong \SI{30}{\um}$. a) Reproduced from Fig. 3A of~\citeauthor{mordant2011validation}\cite{mordant2011validation}.}
    \label{fig:MordauntResolitionFigure}
\end{figure}

The likely sources of the discrepancy in spatial resolution were hypothesised to originate from: the eye-phantom; the 90:10 beam-splitter in the imaging arm; and / or the \SI{500}{\nm} long pass emission filter. A checkerboard target, with a pitch of \SI{10}{\mm}, positioned approximately \SI{3}{\metre} from the objective lens of the fundus camera - which much like imaging the eye can be considered to be imaging at infinity - was then imaged with each of the aforementioned elements removed from the imaging path. The rise-time or the black-to-white transition time was then be computed for each scenario, as shown \cref{fig:sCMOStroubleshooting}, in terms of pixels and yielded a mean rise-time of \SI{4.5}{\pixel}. Since this difference rise-time ($\SI{3}{\pixel}:\SI{4.5}{\pixel}$) is proportionally similar to the discrepancy in spatial resolution ($\SI{30}{\um}:\SI{50}{\um}$) it can be concluded that the optics of the fundus camera or the SLR lenses used in the imaging arm are the limiting factor and not the beam-splitter or emission filter. While efforts could have been made to improve the spatial resolution of the SFLIO system to match this previously reported value the goal of this project was to construct an system for quantitative imaging of the retina and not diffraction limited retinal imaging system. Furthermore, while the assessment of spatial resolution of the system was treated with a reasonable degree of rigour, the knife-edge test was implemented for its relatively high accuracy when compared to the time and complexity cost. More sensitive characterisations using Siemens Star and US Air force test targets embedded in the eye phantom could have been employed for more robust determinations of the MTF - the results of the knife-edge test were considered to be sufficient for this assessment. Additionally, the irregular pixel aspect ratio and layout of the FLIMera was not taken in to account in this analysis however there exists methods using fringe patterns, produced by a Twyman-Green interferometer, projected onto the detector to accurately sample its MTF~\cite{greivenkamp1994modulation}. In practice the SFLIO system is optically limited and not detector limited so while these methods would yield more precise and accurate measurements of the MTF and spatial resolution it would not fundamentally change the performance of the systems biochemical resolution nor the requirement for adequate image registration.

\begin{figure}
    \centering
    \begin{annotatedFigure}{\includegraphics[width = 0.8\textwidth]{figures/sflio-device/SpatialResTroubleShootFigure.pdf}}
    \annotatedFigureText{-0.02,0.99}{black}{0.3}{a)}
    \annotatedFigureText{-0.02,0.46}{black}{0.3}{c)}
    \annotatedFigureText{0.475,0.99}{black}{0.3}{b)}
    \annotatedFigureText{0.475,0.46}{black}{0.3}{d)}
    \end{annotatedFigure}
    \caption{Images recorded of a checker board patter illuminated with a white light Xenon source using the S-FLIO system to ascertain the source of degradation in the spatial resolution of the fluorescence intensity imaging arm. Line profiles were taken across the checker board pattern and the rise time of the contrast was computed. In a) A \SI{500}{\nm} long pass filter is in place for fluorescence intensity imaging with sCMOS detector. The rise time was computed to be \SI{5}{\pixel}, In b) the 90:10 beam-splitter used for simultaneous imaging with the FLIMera was removed which resulted in a lower rise time of \qtyrange{3}{4}{\pixel}. In c) the \SI{500}{\nm} long pass filter was removed, emulating brightfield imaging, and the rise time was computed as \SI{5}{\pixel}. In d) both the  \SI{500}{\nm} long pass filter and the 90:10 beam-splitter was removed resulting in a rise-time of \SI{4}{\pixel}}
    \label{fig:sCMOStroubleshooting}
\end{figure}

\FloatBarrier
\section{Motion Co-Registration}\label{sec:motionreg}
The long acquisition times of over 1 minute required to record fluorescence decays, using the FLIMera, with a SNR high enough to allow discrimination of fluorophores introduces the challenge of compensating for the natural movements of the eye that occur when fixating for long periods of time. These eye movements would result in a fluorescence lifetime image with severely reduced contrast where features such as blood vessels will not be discernible. Due to low number of photons recorded these movements cannot be corrected for using only integrated intensity images generated by the FLIMera at video rate and registered using conventional image registration algorithms to construct a single unblurred histogram.
\\
These eye movements are categorised into three different types of movement: tremors which are high frequency movements (\SI{90}{\hertz}) with amplitude on order $<1$~arc-minute; drift which is a continuous random movement that occurs between tremors and saccades but with lower frequency ($<\SI{1}{\hertz}$) but longer duration ($>\SI{0.5}{\second}$) and larger in amplitude ($\approx \SI{0.5}{degree}$); and saccades which appear as fast jerks lasting \SI{25}{\milli\second} and occurring every \SI{2}{\second} of fixation time with varying amplitude~\cite{martinez2004role}.
In the S-FLIO imaging system the affects of tremors are be ignored since their amplitude is much smaller than the angular resolution ($\approx 10$~arc-minutes) of our system tremors exhibit small enough amplitudes that they would be much smaller than the spatial resolution of the system. The resultant saccades and drift movements can be compensated by determining any translation between successive brightfield images recorded using the sCMOS camera at a high framerate (\SI{10}{\fps}) and short exposure time (\SI{30}{\milli\second}) and then applying the opposite translation to  histograms from the FLIMera that can be read out at up to \SI{12}{\kilo\hertz}.
Mathematically this is modelled using the a Affine Transform which can be used to map the co-ordinates, in image space, for each pixel on the CMOS camera and the FLIMera.
\begin{align}\label{eq:mapping}
    \begin{bmatrix}
        x\\
        y\\
        1\\
    \end{bmatrix}
     & = \mathbf{A}
     \begin{bmatrix}
         x'\\
         y'\\
         1
     \end{bmatrix}\\
    \implies I_{FLIMera}(x,y) &= \mathbf{A}I_{sCMOS}(x',y')
\end{align}
Where the affine transform is composed of four separable transforms : Translation $\mathbf{T}(\Delta x, \Delta y)$; Magnification/Scaling, $\mathbf{M}(m_{x},m_{y})$; Rotation, $\mathbf{R}(\theta)$;and Shear, $\mathbf{S}(h_{x}, h_{y})$. In the images recorded using the S-FLIO system Shear was not significantly present of either the sCMOS camera or the FLIMera and is not considered in the registration process.
\begin{equation}\label{eq:affine}
    \mathbf{A} = \mathbf{T}\mathbf{R}\mathbf{M}
\end{equation}
\begin{align}\label{eq:TransMatrices}
    T &= \begin{bmatrix}
        1 & 0 & \Delta x\\
        0 & 1 & \Delta y\\
        0 & 0 & 1
    \end{bmatrix}
    &
    R &= \begin{bmatrix}
        \cos\theta & -\sin\theta & 0\\
        \sin\theta & \sin\theta & 0\\
        0 & 0 & 1\\
    \end{bmatrix}
    &
    M &= \begin{bmatrix}
        m_{x} & 0 & 0 \\
        0 & m_{y} & 0 \\
        0 & 0 & 1\\
    \end{bmatrix}
    &
    S &= \begin{bmatrix}
        0 & h_{x} & 0\\
        h_{y} & 0 & 0\\
        0 & 0 & 1\\
    \end{bmatrix}   
\end{align}
Using this affine transform the registration process can be further into separated into two stages: The affine transform, $\hat{\mathbf{A}}$ which defines the mapping between the two detectors for a static scene ; and a time dependent affine transform relating to translational motion in the retina, $\mathbf{A}(\Delta x(t), \Delta y(t))$.

\subsection{Static Image Registration}
\subsubsection{Fourier-Merlin Method}
While feature based registration algorithms are currently employed for the purpose of registering retinal images these techniques are predominately used on collections of images with similar resolution, field of view, and magnification~\cite{chanwimaluang2006hybrid,faisan2011scanning}.  
Since the FLIMera images exhibit high noise, as well as dead and hot pixels, and are of low resolution compared to the sCMOS brightfield images the Fourier Merlin algorithm has been utilised \cite{padfield2011masked} to register these image. The Fourier-Merlin uses the phase cross-correlation method of image registration in two stages where first the magnification and scaling between the images is recovered and corrected for and then finally any translational shifts is corrected.
Briefly, the phase cross-correlation registration method identifies the translation between the two images, $k(x,y)$, and $l(x','y) = k(x+\Delta x, y + \Delta y)$ by finding the location of the peak of phase correlation, $\Omega$ between the two images. This is typically implemented using the FFT:
\begin{align}
    K(\mu, \nu) &= \mathcal{F}\{k(x,y)\} & L(\mu',\nu') &= \mathcal{F}\{l(x',y')\}\\
    \Omega(\mu,\nu) &= \frac{K L^{*}}{\lvert K L^{*} \rvert} & (\Delta x, \Delta y) &= argmax\big\{{\mathcal{F}^{-1}}\{\Omega(\mu,\nu)\}\big\}
\end{align}
To recover the magnification and scaling parameters the Fourier transform of both images is computed and then the log-polar transformation is applied (\cref{eq:logpolar}). Here, any magnification and rotation $(\Delta m ,\Delta\theta)$ between the images manifests as a translation in the log-polar transformed images along the $\rho$ and $\xi)$ axes.
\begin{gather}
    I(x,y) \rightarrow I(\rho,\xi)\\
    \begin{aligned}\label{eq:logpolar}
    \rho &= \arctan{\bigg(\frac{y}{x}\bigg)} &\xi &= \ln{(\sqrt{x^{2} + y^{2}})}
    \end{aligned}  
\end{gather}
The parameters $\Delta m$ and $\Delta\theta$ can then be recovered using \cref{eq:magrot} where $U$ and $V$ refer to the number of samples in each axis of the Fourier transformed image, and $U / k$ is the radius over which the log-polar transform is applied. 
\begin{align}\label{eq:magrot}
    \Delta\theta &= \frac{2\pi\Delta\rho}{U} & \Delta m &= \frac{1}{\exp\Delta\xi / \Gamma} & \Gamma = \frac{V}{\ln\lfloor U / k\rfloor}
\end{align}
\subsubsection{USAD Test Target Registration}
The static calibration between the two detectors was then found using the above process and fluorescence images recorded of a United States Air Force (USAF) test target using the FLIMera and sCMOS camera in tandem. Images were recorded using the standard excitation wavelengths (\qtyrange{460}{467}{\nm}) and a \SI{500}{\nm} long pass to filter the excitation light with an integration time of \SI{10}{\second} to ensure a high SNR. For the images recorded with the FLIMera, the measured histograms were integrated over time to yield a intensity image and the  pixel aspect ratio is corrected for by twinning adjacent columns of pixels resulting in a image with resolution \qtyproduct[product-units=single]{192x252}{\pixel}.
Before the Fourier-Merlin algorithm was applied, the images were first filtered using a difference-of-Gaussian bandpass filter and a Hanning window to increase the contrast of edge features which gives a more prominent peak in the cross-correlation process. The scaling and rotation between the images was found using the FFT and log-polar transform process (\cref{fig:USAFlp}) described above and returned values of $\Delta M = \num[round-mode = figures, round-precision = 3]{4.202255063883409}$ and $\Delta\theta = \SI[round-mode = figures, round-precision = 3]{-0.098775}{\degree}$. 

\begin{figure}
    \centering
    \begin{annotatedFigure}
    {\includegraphics[width = 0.95\textwidth]{figures/sflio-device/LogPolarUSAF.pdf}}
    \annotatedFigureText{0.16,0.86}{white}{0.3}{a)}
    \annotatedFigureText{0.39,0.86}{white}{0.3}{b)}
    \annotatedFigureText{0.63,0.86}{white}{0.3}{c)}
    \annotatedFigureText{0.16,0.38}{white}{0.3}{d)}
    \annotatedFigureText{0.39,0.38}{white}{0.3}{e)}
    \annotatedFigureText{0.63,0.38}{white}{0.3}{f)}
    \end{annotatedFigure}
    \caption{Visualisation of how the scaling and rotation between the FLIMera and sCMOS can be recovered using the Fourier-Merlin: a) and d) The input images are filtered using a difference-of-Gaussians bandpass filter and a Hanning window; b) and e) The images are then Fourier transformed; and finally the Log-Polar transform (\cref{eq:logpolar}) is applied to project the relative magnification and rotation into translations in the horizontal and vertical axes of c) and f) which are recovered using standard phase cross-correlation techniques.}
    \label{fig:USAFlp}
\end{figure}


A intermediate FLIMera image is then constructed from these $\Delta m$, and $\Delta\theta$ values as well as a translation matrix to re-centre the image using an affine warp and nearest neighbour interpolation. Until now, the image origin (0,0) is taken as the upper-left corner which results in undesired shifting of the scaled image upon the application of a scaling and rotation. To account for this a further translation matrix is used to shift the image by an amount equal to \SI{-6558}{\pixel} and \SI{-3458}{\pixel} in the horizontal and vertical axes, respectively. The translations between this intermediate FLIMera image and the sCMOS image are found using standard phase cross-correlation to be $\Delta x = \SI[round-mode = figures, round-precision = 3]{-205.9709}{\pixel}$ and $\Delta y = \SI[round-mode = figures, round-precision = 3]{-298.35}{\pixel}$.
Finally, the magnification, rotation, and translations are used to construct a single Affine transform (see \cref{eq:SFLIOaff}) and then the registered FLIMera image is then produced. The accuracy of this registration is assessed using a normalised line profile taken through an identical region in both images, as shown in \cref{fig:USAFregistered}, and calculating the cross-correlation of both line profiles. For a perfectly registered image the location of the peak would occur at \num{0}, however, for the SFLIO system this location is \num{1} indicating a registration accuracy of \num{1} or fewer pixels on sCMOS detector. This greatly exceeds the minimum accuracy needed for sub-pixel registration on the FLIMera where a registration accuracy of less than $1/\Delta m$ is required.


\begin{figure}
    \centering
    \begin{annotatedFigure}
        {\includegraphics[width = 0.85\textwidth]{figures/sflio-device/AirForceTarget-RegisteredImage.pdf}}
        \annotatedFigureText{0.04,0.94}{black}{0.4}{a)}
        \annotatedFigureText{0.04,0.50}{black}{0.4}{c)}
        \annotatedFigureText{0.52,0.94}{black}{0.4}{b)}
    \end{annotatedFigure}
    \caption{Results of static image registration the sCMOS and FLIMera on the S-FLIO device using a USAF test target and the Fourier-Merlin algorithm for recovering scale, rotation, and translation transforms. a) Shows the original, high resolution, image recorded with the sCMOS camera; b) Shows the FLIMera image after being scaled, rotated, and translated using a affine transform and nearest neighbour interpolation. The accuracy of the registration process is assessed using the cross-correlation peak of two line profiles (cyan and magenta) through identical regions of a) and b) plotted in c) and yields an accuracy of 1 pixel}
    \label{fig:USAFregistered}
\end{figure}
\begin{equation}\label{eq:SFLIOaff}
\hat{\mathbf{A}} = \begin{bmatrix}
    \num[round-mode = figures, round-precision = 3]{4.202248819333833} & \num[round-mode = figures, round-precision = 3]{-0.007244469694988512}& \num[round-mode = figures, round-precision = 3]{-6856.35}\\
    \num[round-mode = figures, round-precision = 3]{0.007244469694988512} & \num[round-mode = figures, round-precision = 3]{4.202248819333833}& \num[round-mode = figures, round-precision = 3]{-3663.9709}\\
    \num[round-mode = figures, round-precision = 3]{0} & \num[round-mode = figures, round-precision = 3]{0}& \num[round-mode = figures, round-precision = 3]{1}
\end{bmatrix}    
\end{equation}

\subsection{Movement Registration}
Correcting for eye movement in the fluorescence decays recorded with the FLIMera is achieved by detecting translational shifts in successive brightfield or fluorescence intensity images, recorded at \SI{10}{\hertz}, with phase cross-correlation and then using the affine transform, $\hat{\mathbf{A}}$, to obtain equivalent shifts on the FLIMera. These shifts are then applied to successive histograms generated by the FLIMera operating in RAW mode at the same frame rate of as the sCMOS detector and combined to form a single histogram spanning the entire acquisition period. From this histogram a high quality fluorescence lifetime image can be constructed that is now free of motion artefacts.

\subsubsection{sCMOS shifts to FLIMera shifts}
The shifts in the sCMOS detector are converted to shifts in the FLIMera detector using the attribute of the affine transform (Eq.~\ref{eq:affreg}) and considering the transform between pairs of images:
\begin{align}\label{eq:affreg}
I_{FLIMera}^{N+\epsilon}(x',y') &= I_{FLIMera}^{N}(x' + \Delta x'_{\epsilon} ,y'+ \Delta y'_{\epsilon}) = \mathbf{A}(\Delta x'_{\epsilon} ,\Delta y'_{\epsilon})I_{FLIMera}^{N}(x',y')\\
I_{sCMOS}^{N+\epsilon}(x,y) &= I_{sCMOS}^{N}(x + \Delta x_{\epsilon} ,y+ \Delta y_{\epsilon}) = \mathbf{A}(\Delta x_{\epsilon} ,\Delta y_{\epsilon})I_{sCMOS}^{N}(x,y)
\end{align}
Where $N$ represents a reference frame and $\epsilon$ is a frame recorder later in the acquisition period that has underwent a translational shift. The $N+\epsilon$ frames in each detector can be described using the existing relationship $I_{FLIMera}^{N+\epsilon}(x',y') = \hat{\mathbf{A}}^{-1}I_{sCMOS}^{N+\epsilon}(x,y)$ and the translational term, $\mathbf{A}(\Delta x_{\epsilon} ,\Delta y_{\epsilon})$, found using image registration is converted to shifts in the FLIMera using \cref{eq:affreg}:

\begin{align}
    I_{FLIMera}^{N+\epsilon} &= \hat{\mathbf{A}}^{-1}I_{sCMOS}^{N+\epsilon}(x,y)= \hat{\mathbf{A}}^{-1}\mathbf{A}(\Delta x_{\epsilon} ,\Delta y_{\epsilon})I_{sCMOS}^{N}(x,y) \nonumber\\
    I_{FLIMera}^{N+\epsilon} &= \hat{\mathbf{A}}^{-1}\mathbf{A}(\Delta x_{\epsilon} ,\Delta y_{\epsilon})\hat{\mathbf{A}}I_{FLIMera}^{N}(x',y')\nonumber\\
    \therefore \mathbf{A}(\Delta x'_{\epsilon} ,\Delta y'_{\epsilon}) &= \hat{\mathbf{A}}^{-1}\mathbf{A}(\Delta x_{\epsilon} ,\Delta y_{\epsilon})\hat{\mathbf{A}}\label{eq:FLIMerashifts}
\end{align}  
This approach can be further generalised for rotation, translation, or shear such that any affine transform, $\mathbf{A}(\Delta x, \Delta y, \theta, \Delta m, \Delta h)$, between two images recorded in the sCMOS imaging arm of the S-FLIO device can be converted into an equivalent transform of a FLIMera image:
\begin{equation}
    \mathbf{A}(\Delta x', \Delta y', \theta', \Delta m', \Delta h') = \hat{\mathbf{A}}^{-1}\mathbf{A}((\Delta x, \Delta y, \theta, \Delta m, \Delta h))\hat{\mathbf{A}}
\end{equation}
\subsubsection{\textit{Ex-Vitro} Demonstration using \textit{Convallaria majalis}}
The motion registration algorithm was demonstrated on a sample of \textit{Convallaria majalis} - referred to simply as Convallaria in the rest of the text - mounted in an eye phantom (\cref{fig:ConvallariaEyePhantom}) which could be moved within the X-Y plane during the acquisition period to mimic the image degradation seen in images of the human retina. Over a \SI{2}{\minute} integration period the sample, excited between \qtyrange{460}{467}{\nm}, is held still, translated in X, Y, and then X and Y for periods of \SI{30}{\second} each. During this period the emitted fluorescence, filtered using a \SI{500}{\nm} long-pass filter, fluorescence intensity images are continuously recorded - with integration time of \SI{30}{\ms} and frame rate of \SI{10}{\hertz} - in tandem with FLIM data using the FLIMera's RAW mode to readout histograms at a rate of \SI{24}{\kilo\hertz}. 
\begin{figure}
    \centering
    \includegraphics[width = 0.5\linewidth, trim = {2.25cm, 1cm, 1.5cm, 0.5cm}, clip]{figures/sflio-device/PhantomEyeFigConvallaria.pdf}
    \caption{The natural movements of a fixated human eye was simulated using an eye phantom with a sample of Convallaria positioned at the retinal plane and mounted to a X-Y translation stage. Saccades and drift motions are then mimic by moving the Convallaria sample with an amplitude and velocity commensurate with published values~\cite{martinez2004role}. The phantom eye serves as an analogue to the human eye in that it uses a $f=\SI{24}{\mm}, d = \SI{8}{\mm}$ lens positioned \SI{24}{\mm} from its retinal plane. The cavity in the phantom eye is filled with water to emulate the refractive properties of the vitreous humour.}
    \label{fig:ConvallariaEyePhantom}
\end{figure}
By initially having the laser shutter closed - sample is not illuminated - at the start of the acquisition period before illuminating the sample the FLIMera and sCMOS frames can then be synchronised by detecting the Heaviside like change in the photon flux, shown in \cref{fig:flimerascmossync}, by finding the peak of the gradient of this photon flux. For the FLIMera, the low detected photon flux and high frame rate resulted in a noise dominated signal (\cref{fig:flimerascmossync}a) where the frame at which the laser shutter was opened could not be readily discerned. A Butterworth filter is then used to remove the high frequency components of the signal revealing the step-like change from which the gradient could be calculated. When applying the Butterworth filter there is ordinarily some linear phase introduced into the filtered signal which would give a false location in the synchronisation process however, by using a forward and backward configuration, implemented using \textit{scipy}'s \texttt{filtfilt} function, the resulting filter does not introduce any unwanted phase to the filtered signal.

\begin{figure}
    \centering
    \begin{annotatedFigure}{\includegraphics[width =\textwidth]{figures/sflio-device/flimerasync.pdf}}
    \annotatedFigureText{0.015,0.93}{black}{0.4}{a)}
    \annotatedFigureText{0.015,0.46}{black}{0.4}{b)}
    \end{annotatedFigure}
    \caption{The step-like change in the photon flux per frame, $\Phi(n)$, for the the FLIMera (a) and sCMOS detector (b), shown as the black line, is detected by finding the peak of the gradient, $d\Phi(n)/dn$, shown as the red line. For the FLIMera a Butterworth filter was applied to the raw signal, shown as the grey line in (a), to enable the frame when the laser shutter was opened to be resolved.}
    \label{fig:flimerascmossync}
\end{figure}

Once this synchronisation process is complete the bright field or fluorescence intensity images recorded using the sCMOS detector are registered, using the above phase cross-correlation method, to the first filly illuminated image in the sequence. This is used to construct a single higher quality image, composed of corrected frames summed together spanning the acquisition period, \cref{fig:sCMOSreg}d from \cref{fig:sCMOSreg}b. It should be noted that although the image in \cref{fig:sCMOSreg}d is now free of motion artefacts it does appear to have poorer spatial resolution when compared to \cref{fig:sCMOSreg}a, an image composed of a single exposure recorded over a shorter \SI{2}{\second} exposure time. Each frame in the image sequence exhibits higher photon noise as well as higher relative contribution of read noise due to the higher frame rate - when compared to an equal but longer exposure - manifesting as small registration errors ($< \SI{2}{\pixel}$) and resulting in the motion corrected image appearing softer and losing some of the finer details

\begin{figure}
    \centering
    \begin{annotatedFigure}{\includegraphics[width = \linewidth, trim = {0.4cm, 1.1cm, 0.1cm, 0.3cm}, clip]{figures/sflio-device/sCMOSRegistration.pdf}}
    \annotatedFigureText{0.01,0.76}{white}{0.4}{a)}
    \annotatedFigureText{0.26,0.76}{white}{0.4}{b)}
    \annotatedFigureText{0.51,0.76}{white}{0.4}{c)}
    \annotatedFigureText{0.76,0.76}{white}{0.4}{d)}
    \end{annotatedFigure}
    \caption{Fluorescence intensity images recorded of \textit{Convallaria} mounted in a eye phantom (a-d). The first image, a), represents a single \SI{2}{\second} exposure. For b-d), the images are comprised of frames recorded at \SI{10}{\hertz} with \SI{30}{\ms} integration time with c) and d) being the result of summing multiple frames. The sample is initially held still for \SI{30}{\second} before being translated in the X and Y axes throughout the remainder of a total \SI{2}{\minute} acquisition period to mimic the typical movements the human eye undergoes while fixating on a target. b) shows a typical frame used in the registration process with grey values between 0 and 5. In c) the image is degraded by the movement over the entire acquisition period and in d) this motion is corrected for using phase-cross correlation to detect movement and an affine transform to correct the translational shifts.}
    \label{fig:sCMOSreg}
\end{figure}

The translational shifts in the fluorescence intensity images are then converted to equivalent shifts on the FLIMera using \cref{eq:FLIMerashifts} and applied to FLIMera histograms that cover the same time period as the sCMOS acquisition, $T = 1/\SI{10}{\hertz} = \SI{100}{\ms}$. When applying the shifts to each histogram a pixel mask is applied to eliminate the effect of "hot" pixels and faulty pixels on the fluorescence decay. This yields a single $XYT$ - data cube spanning the entire \SI{2}{\minute} acquisition period that as shown in integrated intensity images in \cref{fig:FLIMerareg}d is now free of motion artefacts and the vascular bundles can be clearly resolved albeit with a lower resolution when compared to the fluorescence intensity images shown in \cref{fig:sCMOSreg}. Furthermore, in these intensity images, motion of the sample causes features that would ordinarily be imaged by a masked faulty pixel (See \cref{fig:FLIMerareg}b) in a still sample now illuminates a different. Typically this would result in an uneven image since every pixel in the final image would be not be masked an equal number of times. This is accounted for by multiplying each pixel grey value by the factor, $\Gamma$, which weights the total number of frames, $N$, in a sequence with the number of times that pixel is masked, $M$ to yield the high quality image shown in \cref{fig:FLIMerareg}d.

\begin{equation}
    \Gamma(x,y) = \frac{N}{N-M(x,y)}
\end{equation}


\begin{figure}
    \centering
    \begin{annotatedFigure}{\includegraphics[width = \linewidth, trim = {0.5cm, 1cm, 0.2cm, 0},clip]{figures/sflio-device/FLIMeraRegistration.pdf}}
    \annotatedFigureText{0.015,0.77}{white}{0.4}{a)}
    \annotatedFigureText{0.345,0.77}{white}{0.4}{b)}
    \annotatedFigureText{0.675,0.77}{white}{0.4}{c)}
    \end{annotatedFigure}
    \caption{Demonstration of motion registration of time resolved fluorescence decays recorded with the FLIMera of a \textit{Convallaria} slide mounted in an eye phantom under going motion throughout a \SI{2}{\minute} acquisition period. Images are formed by integrating the histogram over time to mimic a fluorescence intensity image where: a) represents the entire \SI{2}{minute} acquisition period; b) is the initial \SI{30}{\second} where the sample is held still; and c) is the same \SI{2}{minute} period but with the motion corrected for using the movement detected in the fluorescence intensity images (see \cref{fig:sCMOSreg}).}
    \label{fig:FLIMerareg}
\end{figure}
