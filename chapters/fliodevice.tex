%\setstretch{2}

\section{Construction of \textit{En-face} S-FLIO system}
\subsection{Device Methodology}
The S-FLIO device is based upon an off-the-shelf fundus camera which is then modified to allow for exciting fluorescence with a pulsed laser source and  
\subsection{Illumination Arm}
\subsection{Imaging Arm}
\subsubsection{FLIMera}




\section{Evaluation of Performance}
\begin{itemize}
    \item We need good spatial resolution, and signal to noise ratio in the high resolution camera to account for movement of the eye
    \item SPAD has much fewer pixels and much larger spatial resolution (influenced by the irregular sparse array architecture) and there is a magnification difference so as long as spatial res of the registration camera is 5-10X the FLIMera and we can register with a 5 pixels then our registration will be limited by he quantisation of the digital imaging system
    \item The purpose of this device is to proof of concept the measurement of FAD in the retina using lifetimes - not construct a diffraction limited system.
    \item to quantify the spatial resolution of the system we can use the knife edge test consisting of a fluorescent slide (gives a uniform scene) that has a slanted hard edge places in front of it all housed in an artificial eye
    \item calibration is achieved by replacing the edge with an Allen key with known diameter. The number of pixels between the parallel edges yield the pixel to um conversion factor i.e the width and height that a single pixel subtends on the retina.
    \item this test evaluate the spatial resolution by averaging an aliased edge (in the case of the FLIMera - pixels are small enough on the FI camera that optics is the limit not quantisation) to yield the Edge Spread Function. Differential of the ESF is the LSF and the spatial resolution is defined as the FWHM of this LSF after a Gaussian has been fitted.
    \item This was repeated through multiple foci to ascertain where the best focus is and what is its spatial resolution
    \item on the FI camera the spatial resolution is 40um and for the FLIMera it is 400um.
    \item This resolution is larger than what was previously reported in the Harvey, McNaught artificial eye paper (spatial res was $\approx 15$um) which uses the same artificial eye and an optically similar imaging system (fundus camera with SLR lenses in a imaging relay configuration to focus the image onto a scientific camera.
    \item The system is optically limited but the source is likely to originate in misalignment of the SLR lenses, other optical elements in the beam path etc. 
    \item To attempt to ascertain the source of these issues the fundus camera was set up to image the room with some targets attached to the wall at infinity and optical elements (emission filter, beam splitter) and the spatial resolution was estimated using the rise-time method
    \begin{itemize}
        \item spatial res $\approx$ width in pixels between 90 and 10 percent of the fast edge
    \end{itemize}
    \item this was then compared to two much simpler imaging systems - and iPhone 13, and the thorcam with an SLR lens attached set to the same scene.
    \item CONCLUSION: spatial resolution is X and this is sufficient for co-registering images with the FLIMera. There are multiple sources of aberration in the system that can be harming our spatial res that could be eliminated given the a significant time investment but since a good signal to noise ratio is favoured over diffraction limited (or near it) this is sufficient.
\end{itemize}
\subsection{Spatial Resolution of S-FLIO device}
While the purpose of this thesis is not to construct a diffraction limited retinal imaging system but to record fluorescence lifetime images and quantitatively measure the distribution of fluorophores across the retina - a spatial resolution, with respect to images recorded on the retina with the FLIMera, sufficient to resolve features, and markers, common with retinal disease is key. For example, as has been discussed in Sec.~\ref{intro:AMD} the average diameter of drusen, a common marker of disease progression in AMD, is on the order of YY - to resolve drusen with the FLIMera the spatial resolution of the S-FLIO system must be lower than this value. Additionally, the spatial resolution of both detectors will influence the accuracy required when reconstructing images from fluorescence histograms of a retina undergoing motion.
\subsubsection{Spatial Resolution vs. Pixel Size}
In a sampled imaging system the spatial resolution is not simply dictated by the size of a single pixel on the detector but instead how accurately the imaging system and detector can describe a scene e.g the retina. Due the optical characteristics of the eye and the aberrations inherent with the fundus camera and the additional optics required for fluorescence lifetime imaging as well as the sampled nature of the system this will be imperfect. Mathematically the response of the imaging system is denoted as Point-Spread Function (PSF) which describes how a point source in a scene is blurred in a recorded image. 
\begin{equation}\label{eq:psf}
    I_{IMAGE}(x,y) = PSF(x,y)\circledast I_{SCENE}(x',y')
\end{equation}
Additionally in the Fourier domain the Optical Transfer Function (OTF) or the approximately equivalent Modulation Transfer Function (MTF) can be used to described how well an imaging system resolves increasing spatial frequencies~\cite{vollmerhausen2000analysis}. Here the OTF and MTF are connected to the PSF by a Fourier transform.
\begin{align}\label{eq:mtf}
    J_{IMAGE}(\nu,\mu) &\approxeq MTF(\nu,\mu)\circledast K_{SCENE}(\nu',\mu')\\
    MTF(\nu,\mu) &= \mathcal{F}\{PSF(x,y)\}(\nu,\mu)
\end{align}
Conventionally, the spatial resolution of an imaging system is defined by the full-width at half-maximum of the PSF or the spatial frequency where the MTF is half its maximum value. An example of a image blurred by a PSF is shown in \ref{fig:PSFsimulation} alongside its respective MTF.
\begin{figure}
    \centering
    \includegraphics[width = 0.4\textwidth]{figures/sflio-device/PSFMTFDiagram.jpeg}
    \caption{TEMP. Here a sample image, a), has been blurred using a the simulated PSF shown in c) to produce the image in b). The resulting modulation transfer function is shown in d).}
    \label{fig:PSFsimulation}
\end{figure}

\subsubsection{Knife Edge Test}
The Knife-Edge test is a simple yet robust method of measuring the spatial resolution of an imaging system by determining the Edge-Spread Function from an image recorded of a sharp edge positioned in the object plane and rotated off-vertical or off-horizontal by a few degrees.





\subsection{Motion Registration}
The long acquisition times of over 1 minute required to record fluorescence decays, using the FLIMera, with a SNR high enough to allow discrimination of fluorophores introduces the challenge of compensating for the natural movements of the eye that occur when fixating for long periods of time. These eye movements would result in a fluorescence lifetime image with severely reduced contrast where features such as blood vessels will not be discernible. Due to low number of photons recorded these movements cannot be corrected for using only integrated intensity images generated by the FLIMera at video rate and registered using conventional image registration algorithms to construct a single unblurred histogram.
\\
These eye movements are categorised into three different types of movement: tremors which are high frequency movements (\SI{90}{\hertz}) with amplitude on order $<1$~arc-minute; drift which is a continuous random movement that occurs between tremors and saccades but with lower frequency ($<\SI{1}{\hertz}$) but longer duration ($>\SI{0.5}{\second}$) and larger in amplitude ($\approx 0.5$~degrees); and saccades which appear as fast jerks lasting \SI{25}{\milli\second} and occurring every \SI{2}{\second} of fixation time with varying amplitude~\cite{martinez2004role}.
In the S-FLIO imaging system the affects of tremors are be ignored since their amplitude is much smaller than the angular resolution ($\approx 10$~arc-minutes) of our system tremors exhibit small enough amplitudes that they would be much smaller than the spatial resolution of the system. The resultant saccades and drift movements can be compensated by determining any translation between successive brightfield images recorded using the sCMOS camera at a high framerate (\SI{10}{\fps}) and short exposure time (\SI{30}{\milli\second}) and then applying the opposite translation to  histograms from the FLIMera that can be read out at up to \SI{12}{\kilo\hertz}.
Mathematically this is modelled using the a Affine Transform which can be used to map the co-ordinates, in image space, for each pixel on the CMOS camera and the FLIMera.
\begin{align}\label{eq:mapping}
    \begin{bmatrix}
        x\\
        y\\
        1\\
    \end{bmatrix}
     & = \mathbf{A}
     \begin{bmatrix}
         x'\\
         y'\\
         1
     \end{bmatrix}\\
    \implies I_{FLIMera}(x,y) &= \mathbf{A}I_{sCMOS}(x',y')
\end{align}
Where the affine transform is composed of four separable transforms : Translation $\mathbf{T}(\Delta x, \Delta y)$, Magnification/Scaling, $\mathbf{M}(m_{x},m_{y})$; Rotation, $\mathbf{R}(\theta)$;and Shear, $\mathbf{S}(h_{x}, h_{y})$. In the images recorded using the S-FLIO system Shear was not significantly present of either the sCMOS camera or the FLIMera and is not considered in the registration process.
\begin{equation}\label{eq:affine}
    \mathbf{A} = \mathbf{T}\mathbf{R}\mathbf{M}
\end{equation}
\begin{align}\label{eq:TransMatrices}
    T &= \begin{bmatrix}
        1 & 0 & \Delta x\\
        0 & 1 & \Delta y\\
        0 & 0 & 1
    \end{bmatrix}
    &
    R &= \begin{bmatrix}
        \cos\theta & -\sin\theta & 0\\
        \sin\theta & \sin\theta & 0\\
        0 & 0 & 1\\
    \end{bmatrix}
    &
    M &= \begin{bmatrix}
        m_{x} & 0 & 0 \\
        0 & m_{y} & 0 \\
        0 & 0 & 1\\
    \end{bmatrix}
    &
    S &= \begin{bmatrix}
        0 & h_{x} & 0\\
        h_{y} & 0 & 0\\
        0 & 0 & 1\\
    \end{bmatrix}   
\end{align}
Using this affine transform the registration process can be further into separated into two stages: The affine transform, $\hat{\mathbf{A}}$ which defines the mapping between the two detectors for a static scene ; and a time dependent affine transform relating to translational motion in the retina, $\mathbf{A}(\Delta x(t), \Delta y(t))$.

\subsection{Static Image Registration}
\subsubsection{Fourier-Merlin Method}
While feature based registration algorithms are currently employed for the purpose of registering retinal images these techniques are predominately used on collections of images with similar resolution, field of view, and magnification~\cite{chanwimaluang2006hybrid,faisan2011scanning}.  
Since the FLIMera images exhibit high noise, as well as dead and hot pixels, and are of low resolution compared to the sCMOS brightfield images the Fourier Merlin algorithm has been utilised \cite{padfield2011masked} to register these image. The Fourier-Merlin uses the phase cross-correlation method of image registration in two stages where first the magnification and scaling between the images is recovered and corrected for and then finally any translational shifts is corrected.
Briefly, the phase cross-correlation registration method identifies the translation between the two images, $k(x,y)$, and $l(x','y) = k(x+\Delta x, y + \Delta y)$ by finding the location of the peak of phase correlation, $\Omega$ between the two images. This is typically implemented using the FFT:
\begin{align}
    K(\mu, \nu) &= \mathcal{F}\{k(x,y)\} & L(\mu',\nu') &= \mathcal{F}\{l(x',y')\}\\
    \Omega(\mu,\nu) &= \frac{K L^{*}}{\lvert K L^{*} \rvert} & (\Delta x, \Delta y) &= argmax\big\{{\mathcal{F}^{-1}}\{\Omega(\mu,\nu)\}\big\}
\end{align}
To recover the magnification and scaling parameters the Fourier transform of both images is computed and then the log-polar transformation is applied (\cref{eq:logpolar}). Here, any magnification and rotation $(\Delta m ,\Delta\theta)$ between the images manifests as a translation in the log-polar transformed images along the $\rho$ and $\xi)$ axes.
\begin{gather}
    I(x,y) \rightarrow I(\rho,\xi)\\
    \begin{aligned}\label{eq:logpolar}
    \rho &= \arctan{\bigg(\frac{y}{x}\bigg)} &\xi &= \ln{(\sqrt{x^{2} + y^{2}})}
    \end{aligned}  
\end{gather}
The parameters $\Delta m$ and $\Delta\theta$ can then be recovered using Equation \ref{eq:magrot} where $U$ and $V$ refer to the number of samples in each axis of the Fourier transformed image, and $U / k$ is the radius over which the log-polar transform is applied. 
\begin{align}\label{eq:magrot}
    \Delta\theta &= \frac{2\pi\Delta\rho}{U} & \Delta m &= \frac{1}{\exp\Delta\xi / \Gamma} & \Gamma = \frac{V}{\ln\lfloor U / k\rfloor}
\end{align}
\subsubsection{S-FLIO Device}
The static calibration between the two detectors was then found using the above process and fluorescence images recorded of a United States Air Force (USAF) test target using the FLIMera and sCMOS camera in tandem. Images were recorded using the standard excitation wavelengths (\qtyrange{460}{467}{\nm}) and a \SI{500}{\nm} long pass to filter the excitation light with an integration time of \SI{10}{\second} to ensure a high SNR. For the images recorded with the FLIMera, the measured histograms were integrated over time to yield a intensity image and the  pixel aspect ratio is corrected for by twinning adjacent columns of pixels resulting in a image with resolution \qtyproduct[product-units=single]{192x252}{\pixel}.
Before the Fourier-Merlin algorithm was applied, the images were first filtered using a difference-of-Gaussian bandpass filter and a Hanning window to increase the contrast of edge features which gives a more prominent peak in the cross-correlation process. The scaling and rotation between the images was found using the FFT and log-polar transform process (Fig. \ref{fig:USAFlp}) described above and returned values of $\Delta M = \num[round-mode = figures, round-precision = 3]{4.202255063883409}$ and $\Delta\theta = \SI[round-mode = figures, round-precision = 3]{-0.098775}{\degree}$. 

\begin{figure}
    \centering
    \begin{annotatedFigure}
    {\includegraphics[width = 0.95\textwidth]{figures/sflio-device/LogPolarUSAF.pdf}}
    \annotatedFigureText{0.16,0.86}{white}{0.3}{a)}
    \annotatedFigureText{0.39,0.86}{white}{0.3}{b)}
    \annotatedFigureText{0.63,0.86}{white}{0.3}{c)}
    \annotatedFigureText{0.16,0.38}{white}{0.3}{d)}
    \annotatedFigureText{0.39,0.38}{white}{0.3}{e)}
    \annotatedFigureText{0.63,0.38}{white}{0.3}{f)}
    \end{annotatedFigure}
    \caption{Visualisation of how the scaling and rotation between the FLIMera and sCMOS can be recovered using the Fourier-Merlin: a) and d) The input images are filtered using a difference-of-Gaussians bandpass filter and a Hanning window; b) and e) The images are then Fourier transformed; and finally the Log-Polar transform (Eq.\ref{eq:logpolar}) is applied to project the relative magnification and rotation into translations in the horizontal and vertical axes of c) and f) which are recovered using standard phase cross-correlation techniques.}
    \label{fig:USAFlp}
\end{figure}


A intermediate FLIMera image is then constructed from these $\Delta m$, and $\Delta\theta$ values as well as a translation matrix to re-centre the image using an affine warp and nearest neighbour interpolation. Until now, the image origin (0,0) is taken as the upper-left corner which results in undesired shifting of the scaled image upon the application of a scaling and rotation. To account for this a further translation matrix is used to shift the image by an amount equal to \SI{-6558}{\pixel} and \SI{-3458}{\pixel} in the horizontal and vertical axes, respectively. The translations between this intermediate FLIMera image and the sCMOS image are found using standard phase cross-correlation to be $\Delta x = \SI[round-mode = figures, round-precision = 3]{-205.9709}{\pixel}$ and $\Delta y = \SI[round-mode = figures, round-precision = 3]{-298.35}{\pixel}$.
Finally, the magnification, rotation, and translations are used to construct a single Affine transform (see Eq.~\ref{eq:SFLIOaff}) and then the registered FLIMera image is then produced. The accuracy of this registration is assessed using a normalised line profile taken through an identical region in both images, as shown in Fig.~\ref{fig:USAFregistered}, and calculating the cross-correlation of both line profiles. For a perfectly registered image the location of the peak would occur at \num{0}, however, for the SFLIO system this location is \num{1} indicating a registration accuracy of \num{1} or fewer pixels on sCMOS detector. This greatly exceeds the minimum accuracy needed for sub-pixel registration on the FLIMera where a registration accuracy of less than $1/\Delta m$ is required.


\begin{figure}
    \centering
    \begin{annotatedFigure}
        {\includegraphics[width = 0.85\textwidth]{figures/sflio-device/AirForceTarget-RegisteredImage.pdf}}
        \annotatedFigureText{0.04,0.94}{black}{0.4}{a)}
        \annotatedFigureText{0.04,0.50}{black}{0.4}{c)}
        \annotatedFigureText{0.52,0.94}{black}{0.4}{b)}
    \end{annotatedFigure}
    \caption{Results of static image registration the sCMOS and FLIMera on the S-FLIO device using a USAF test target and the Fourier-Merlin algorithm for recovering scale, rotation, and translation transforms. a) Shows the original, high resolution, image recorded with the sCMOS camera; b) Shows the FLIMera image after being scaled, rotated, and translated using a affine transform and nearest neighbour interpolation. The accuracy of the registration process is assessed using the cross-correlation peak of two line profiles (cyan and magenta) through identical regions of a) and b) plotted in c) and yields an accuracy of 1 pixel}
    \label{fig:USAFregistered}
\end{figure}
\begin{equation}\label{eq:SFLIOaff}
\hat{\mathbf{A}} = \begin{bmatrix}
    \num[round-mode = figures, round-precision = 3]{4.202248819333833} & \num[round-mode = figures, round-precision = 3]{-0.007244469694988512}& \num[round-mode = figures, round-precision = 3]{-6856.35}\\
    \num[round-mode = figures, round-precision = 3]{0.007244469694988512} & \num[round-mode = figures, round-precision = 3]{4.202248819333833}& \num[round-mode = figures, round-precision = 3]{-3663.9709}\\
    \num[round-mode = figures, round-precision = 3]{0} & \num[round-mode = figures, round-precision = 3]{0}& \num[round-mode = figures, round-precision = 3]{1}
\end{bmatrix}    
\end{equation}

\subsection{Movement Registration}
Correcting for eye movement in the fluorescence decays recorded with the FLIMera is achieved by detecting translational shifts in successive brightfield or fluorescence intensity images, recorded at \SI{10}{\hertz}, with phase cross-correlation and then using the affine transform, $\hat{\mathbf{A}}$, to obtain equivalent shifts on the FLIMera. These shifts are then applied to successive histograms generated by the FLIMera operating in RAW mode at the same frame rate of as the sCMOS detector and combined to form a single histogram spanning the entire acquisition period. From this histogram a high quality fluorescence lifetime image can be constructed that is now free of motion artefacts.

\subsubsection{sCMOS shifts to FLIMera shifts}
The shifts in the sCMOS detector are converted to shifts in the FLIMera detector using the attribute of the affine transform (Eq.~\ref{eq:affreg}) and considering the transform between pairs of images:
\begin{align}\label{eq:affreg}
I_{FLIMera}^{N+\epsilon}(x',y') &= I_{FLIMera}^{N}(x' + \Delta x'_{\epsilon} ,y'+ \Delta y'_{\epsilon}) = \mathbf{A}(\Delta x'_{\epsilon} ,\Delta y'_{\epsilon})I_{FLIMera}^{N}(x',y')\\
I_{sCMOS}^{N+\epsilon}(x,y) &= I_{sCMOS}^{N}(x + \Delta x_{\epsilon} ,y+ \Delta y_{\epsilon}) = \mathbf{A}(\Delta x_{\epsilon} ,\Delta y_{\epsilon})I_{sCMOS}^{N}(x,y)
\end{align}
Where $N$ represents a reference frame and $\epsilon$ is a frame recorder later in the acquisition period that has underwent a translational shift. The $N+\epsilon$ frames in each detector can be described using the existing relationship $I_{FLIMera}^{N+\epsilon}(x',y') = \hat{\mathbf{A}}^{-1}I_{sCMOS}^{N+\epsilon}(x,y)$ and the translational term, $\mathbf{A}(\Delta x_{\epsilon} ,\Delta y_{\epsilon})$, found using image registration is converted to shifts in the FLIMera using Eq.~\ref{eq:affreg}:

\begin{align}
    I_{FLIMera}^{N+\epsilon} &= \hat{\mathbf{A}}^{-1}I_{sCMOS}^{N+\epsilon}(x,y)= \hat{\mathbf{A}}^{-1}\mathbf{A}(\Delta x_{\epsilon} ,\Delta y_{\epsilon})I_{sCMOS}^{N}(x,y) \nonumber\\
    I_{FLIMera}^{N+\epsilon} &= \hat{\mathbf{A}}^{-1}\mathbf{A}(\Delta x_{\epsilon} ,\Delta y_{\epsilon})\hat{\mathbf{A}}I_{FLIMera}^{N}(x',y')\nonumber\\
    \therefore \mathbf{A}(\Delta x'_{\epsilon} ,\Delta y'_{\epsilon}) &= \hat{\mathbf{A}}^{-1}\mathbf{A}(\Delta x_{\epsilon} ,\Delta y_{\epsilon})\hat{\mathbf{A}}\label{eq:FLIMerashifts}
\end{align}  
This approach can be further generalised for rotation, translation, or shear such that any affine transform, $\mathbf{A}(\Delta x, \Delta y, \theta, \Delta m, \Delta h)$, between two images recorded in the sCMOS imaging arm of the S-FLIO device can be converted into an equivalent transform of a FLIMera image:
\begin{equation}
    \mathbf{A}(\Delta x', \Delta y', \theta', \Delta m', \Delta h') = \hat{\mathbf{A}}^{-1}\mathbf{A}((\Delta x, \Delta y, \theta, \Delta m, \Delta h))\hat{\mathbf{A}}
\end{equation}
\subsubsection{\textit{Ex-Vitro} Demonstration using \textit{Convallaria majalis}}
The motion registration algorithm was demonstrated on a sample of \textit{Convallaria majalis} - referred to simply as Convallaria in the rest of the text - mounted in an eye phantom which could be moved within the X-Y plane Fig.~\ref{fig:ConvallariaEyePhantom} during the acquisition period to mimic the image degradation seen in images of the human retina. Over a \SI{2}{\minute} integration period the sample, excited between \qtyrange{460}{467}{\nm}, is held still, translated in X, Y, and then X and Y for periods of \SI{30}{\second} each. During this period the emitted fluorescence, filtered using a \SI{500}{\nm} long-pass filter, brightfield or fluorescence intensity images are continuously recorded - with integration time of \SI{30}{\ms} and frame rate of \SI{10}{\hertz} - in tandem with FLIM data using the FLIMera's RAW mode to readout histograms at a rate of \SI{24}{\kilo\hertz}. 
\begin{figure}
    \centering
    \includegraphics[width = 0.5\linewidth, trim = {2.25cm, 1cm, 1.5cm, 0.5cm}, clip]{figures/sflio-device/PhantomEyeFigConvallaria.pdf}
    \caption{The natural movements of a fixated human eye was simulated using an eye phantom with a sample of Convallaria positioned at the retinal plane and mounted to a X-Y translation stage. Saccades and drift motions are then mimic by moving the Convallaria sample with an amplitude and velocity commensurate with published values~\cite{martinez2004role}. The phantom eye serves as an analogue to the human eye in that it uses a $f=\SI{24}{\mm}, d = \SI{8}{\mm}$ lens positioned \SI{24}{\mm} from its retinal plane. The cavity in the phantom eye is filled with water to emulate the refractive properties of the vitreous humour.}
    \label{fig:ConvallariaEyePhantom}
\end{figure}
By initially having the laser shutter closed - sample is not illuminated - at the start of the acquisition period before illuminating the sample the FLIMera and sCMOS frames can then be synchronised by detecting the Heaviside like change in the photon flux, shown in Fig.~\ref{fig:flimerascmossync}, by finding the peak of the gradient of this photon flux. For the FLIMera, the low detected photon flux and high frame rate resulted in a noise dominated signal (Fig.~\ref{fig:flimerascmossync}a) where the frame at which the laser shutter was opened could not be readily discerned. A Butterworth filter is then used to remove the high frequency components of the signal revealing the step-like change from which the gradient could be calculated. When applying the Butterworth filter there is ordinarily some linear phase introduced into the filtered signal which would give a false location in the synchronisation process however, by using a forward and backward configuration, implemented using \textit{scipy}'s \texttt{filtfilt} function, the resulting filter does not introduce any unwanted phase to the filtered signal.

\begin{figure}
    \centering
    \begin{annotatedFigure}{\includegraphics[width =\textwidth]{figures/sflio-device/flimerasync.pdf}}
    \annotatedFigureText{0.015,0.93}{black}{0.4}{a)}
    \annotatedFigureText{0.015,0.46}{black}{0.4}{b)}
    \end{annotatedFigure}
    \caption{The step-like change in the photon flux per frame, $\Phi(n)$, for the the FLIMera (a) and sCMOS detector (b), shown as the black line, is detected by finding the peak of the gradient, $d\Phi(n)/dn$, shown as the red line. For the FLIMera a Butterworth filter was applied to the raw signal, shown as the grey line in (a), to enable the frame when the laser shutter was opened to be resolved.}
    \label{fig:flimerascmossync}
\end{figure}

Once this synchronisation process is complete the bright field or fluorescence intensity images recorded using the sCMOS detector are registered, using the above phase cross-correlation method, to the first filly illuminated image in the sequence (Fig.~\ref{fig:sCMOSreg}). The translational shifts in the brightfield or fluorescence intensity images are then converted to equivalent shifts on the FLIMera using Eq.~\ref{eq:FLIMerashifts} and applied to FLIMera histograms that cover the same time period as the sCMOS acquisition, $T = 1/\SI{10}{\hertz} = \SI{100}{\ms}$. When applying the shifts to each histogram a pixel mask is applied to eliminate the effect of "hot" pixels and faulty pixels on the fluorescence decay.
This yields a single histogram spanning the entire \SI{2}{\minute} acquisition period that as shown in integrated images in Fig~\ref{fig:FLIMerareg} is now free of motion artefacts and the large primary ring as well as the first inner ring of vascular bundles can be resolved clearly and is identical to an image when the sample is held stationary albeit the edges of the vascular bundles appear softer indicating a slightly lower resolution in the recovered images.

\begin{figure}
    \centering
    \begin{annotatedFigure}{\includegraphics[width = \linewidth, trim = {0.4cm, 1.1cm, 0.1cm, 0.3cm}, clip]{figures/sflio-device/sCMOSRegistration.pdf}}
    \annotatedFigureText{0.01,0.76}{white}{0.4}{a)}
    \annotatedFigureText{0.26,0.76}{white}{0.4}{b)}
    \annotatedFigureText{0.51,0.76}{white}{0.4}{c)}
    \annotatedFigureText{0.76,0.76}{white}{0.4}{d)}
    \end{annotatedFigure}
    \caption{Fluorescence intensity (a), and brightfield images recorded of \textit{Convallaria} mounted in a eye phantom (b),c), and d)). In b) the sample is held still for an initial \SI{30}{\second} before being translated in the X and Y axes throughout the remainder of a total \SI{2}{\minute} acquisition period to mimic the typical movements the human eye undergoes while fixating on a target. In c) the image is degraded by the movement over the entire acquisition period and in d) this motion is corrected for using phase-cross correlation to detect movement and an affine warp to correct it.}
    \label{fig:sCMOSreg}
\end{figure}
\begin{figure}
    \centering
    \begin{annotatedFigure}{\includegraphics[width = \linewidth, trim = {0.5cm, 1cm, 0.2cm, 0},clip]{figures/sflio-device/FLIMeraRegistration.pdf}}
    \annotatedFigureText{0.015,0.77}{white}{0.4}{a)}
    \annotatedFigureText{0.345,0.77}{white}{0.4}{b)}
    \annotatedFigureText{0.675,0.77}{white}{0.4}{c)}
    \end{annotatedFigure}
    \caption{Demonstration of motion registration of time resolved fluorescence decays recorded with the FLIMera of a \textit{Convallaria} slide mounted in an eye phantom under going motion throughout a \SI{2}{\minute} acquisition period. Images are formed by integrating the histogram over time to mimic a fluorescence intensity image where: a) represents the entire \SI{2}{minute} acquisition period; b) is the initial \SI{30}{\second} where the sample is held still; and c) is the same \SI{2}{minute} period but with the motion corrected for using the movement detected in brightfield images (see Fig.~\ref{fig:sCMOSreg}).}
    \label{fig:FLIMerareg}
\end{figure}
