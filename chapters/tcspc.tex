\setstretch{2}
\section{outline}
\begin{itemize}
    \item back scattered light from a short pulse of light incident on a surface experiences some material dependent temporal broadening
    \item \citeauthor{patterson1989time}~\cite{patterson1989time} proposed this idea for non-contact deep tissue imaging.
    \item Being able to discriminate materials using this opto-temporal broadening has many applications in defensive situations
    \item IR images is incapable of imaging through glass windows, lidar sensors can only give a depth map using time-off-flight sensing. 
    Using single photon counting one could in principle additionally detect a human behind an obscurant verify if they are an enemy combatant with a weapon using the opto-temporal characteristics of materials like flesh and the various metallic alloys found in guns. 
    \item For applications to retinal imaging, this technique could form a part of an imaging system where the opto temporal broadening is used to discriminate a patients face from their eyes allowing the formation of retinal images from a distance. The time-of-flight nature of the TCSPC imaging could also facilitate the filtering of corneal reflections and lens fluorescence although a detector with a temporal resolution lower than the FLIMera can offer as well as a laser source iwth
        \begin{itemize}
            \item Diameter of the eye is 24mm. Total path difference between a retinal image and a corneal reflection is 48mm X 1.337 (refractive index of vitreous humour) = 64mm $\approx$ 213ps. 
            \item peaks would appear with a separation of 213ps. In order to distinguish this difference one would need to a SPAD + imaging system with an impulse response function much narrower than this
            \item There's probably some metric akin to the Rayleigh criterion I can use to be more quantitative about this.
        \end{itemize}
    \item A proof of concept imaging system using an SLR lens attached to the SPAD array to image a scene from X m away with an approx field of view of Y cm x Z cm was epi-illuminated using the super-continuum source. Epi-illumination was chosen to allow for greater flexibility of the illuminated field of view and minimise direct reflections onto the SPAD array.
    \item TCSPC data was recorded of several subjects including: flesh, aluminium, a plaster bust of Max Planck, a glass teapot, a human hand, and plant matter. 
    \item For the glass teapot both surfaces could be resolved from the TCSPC histograms
    \item In the case of the human hand and the plant subject the illumination wavelength was varied to also allow some spectral discrimination. Allowing for the spectral properties of an object to be linked to the 
    \item An exponentially modified Gaussian function was heuristically chosen to model the detected TCPSC histograms corresponding to the back scattered light. One could derive an analytical function for this back scattered light using the properties of the laser pulse (Gaussian pulse with FWHM of around 100ps) and a suitable treatment of the radiation transfer function but this exponentially modified Gaussian approach results in a good fit (using chi squared as the metric) for a wide range of materials.
    \item Paul also investigated the use of neural networks and machine learning for a fit free approach to material discrimination using a combination of our experimental data and simulated data with varying SNR's
\end{itemize}


\FloatBarrier
\section{Rationale}
Throughout this thesis the technique of TCPSC has been utilised to resolve the temporal decays of excited fluorophores in the retina. However, the ability to count single photons and their arrival time can be exploited for other purposes beyond time-of-flight ranging which boasts a resolution of Xmm \addcite. Namely, the temporal broadening experienced by a back scattered or reflected ultra short pulse of light is explored for use in remote sensing to discriminate materials in a scene~\cite{patterson1989time}. Primarily this technique was potential defence applications where an unknown combatant could be identified through an obscurant like a glass window as well as if they are armed using the unique opto-temporal characteristics of a person and a weapon (steel) where ordinarily, infrared imaging systems are incapable of imaging through standard window glass due its inherent low transmission in the infrared spectrum. \addcite.
\FloatBarrier
\section{Theory}
When impinged on a medium a photon is either reflected off the medium, transmitted further into the medium, or absorbed with a probability of being re-emitted as a fluorescence or phosphorescence photon. Upon repeated scattering events - successive transmission and reflection - the trajectory of light within the medium becomes diffuse meaning that the distance that photons travel within the medium before being re-emitted towards a detector - yielding differing photon arrival times than those initially reflected from the medium. For a ultra-short pulse of light with a pulse width on the order of \SI{100}{\ps} these material dependent scattering properties result in a small but measurable broadening on the order also on the order of \qtyrange{50}{200}{\ps}.
This opto-temporal broadening has been previously employed by \citeauthor{patterson1989time} for the application of non-invasive deep tissue imaging also using picosecond laser pulses and a single pixel detector for performing the TCPSC~\cite{patterson1989time}. 
In this thesis, this opto-temporal broadening is investigated as function of material where it is suspected that more densely packed, homogeneous, and reflective materials such as aluminium or steel will show almost no temporal broadening but softer organic materials such as flesh where the mean distance that photons traverse before being re-emitted will be larger - will exhibit different broadening properties. 
This temporal broadening measured using the FLIMera SPAD array is characterised using a least-squares fitting  of the TCPSC histograms with an Exponentially-Modified Gaussian (Eq.~\ref{eq:expgauss}) which resembles a Gaussian multiplied by the complimentary error function, $\Erfc(z) = 1 - \Erf(z)$ (Eq.~\ref{eq:erf})  In practice the parameters in this model encapsulate the dominant properties of the time dependent distribution of photon arrival times, $t$, such as the width, $\sigma$, the mean arrival time $\mu$, the skew of the distribution, $\alpha$, and the amplitude of the distribution, $A$. 


\begin{equation}\label{eq:expgauss}
    I(t, A, \mu, \sigma, \alpha) = \frac{A}{\sigma\sqrt{2\pi}}\exp\Bigg(\frac{-(t-\mu)^{2}}{2\sigma^{2}}\Bigg)\Erfc\Bigg(-\frac{\alpha(t-\mu)}{\sigma\sqrt{2\pi}}\Bigg)
\end{equation}
\begin{equation}\label{eq:erf}
    \Erf(z) = \frac{2}{\sqrt{\pi}}\int_{0}^{\infty}\exp(-x^{2})dx
\end{equation}

An example shown in Fig.~\ref{fig:expgauss} consisting of TCSPC data recorded of back scattered light incident on an Aluminium block illuminated with a \SI{100}{ps} pulse width and wavelength of \SI{700}{\nano\metre} is fitted using a least-squares fitting routine and the Exponential Gaussian model gives fitted parameters of $A = \num[round-mode = figures, round-precision = 3]{10.43737301}$, $\mu = \num[round-mode = figures, round-precision = 3]{22.19703607}$, $\sigma = \num[round-mode = figures, round-precision = 3]{0.16858335}$, and $\alpha = \num[round-mode = figures, round-precision = 3]{0.16858335}$. To assess the appropriateness of this fitting model the reduced $\chi^{2}$ (Eq.~\ref{eq:redchi2}) was calculated using the photon counts, $y$, line of best fit $f(t, \overrightarrow\phi)$, and the degrees of freedom ,$N-P = 55 \text{ samples} - 4 \text{ fitted parameters} = 51$. For the example fit in Fig.~\ref{fig:expgauss} $\chi^{2}_{red} = \num[round-mode = figures, round-precision = 3]{6.682903784142764}$. indicates a the Exponential Gaussian model fits the data well since $\chi^{2}_{red} \not\gg 1$.

\begin{equation}\label{eq:redchi2}
    \chi^{2}_{red} = \frac{\sum_{n = 1}^{N}(y_{n} - f_{n}(t, \overrightarrow\phi))^{2}}{N-P}
\end{equation}
\begin{figure}
    \centering
    \includegraphics[width = 0.8\textwidth]{figures/TCSPC/ExpGaussExample.pdf}
    \caption{Example TCSPC histogram recorded from a  \SI{700}{\nm} \SI{100}{\ps}, pulse of light back scattered off a block of Aluminium where red 'x's denote the detected photon counts, and the black line is the line of best yielded when fitting with a Exponentially modified Gaussian (Eq.~\ref{eq:expgauss})}
    \label{fig:expgauss}
\end{figure}
\FloatBarrier
\section{Imaging System}
To record images for the purpose of material discrimination an imaging system was constructed, as shown in Fig.~\ref{fig:tcspcimagingsystem}, where a pulsed super continuum white light source (NKT SuperK EX12W) was filtered using an AOTF to bath illuminate a scene that is located $\approx \SI{2}{\metre}$ from the laser source. A \SI{100}{\milli\metre} focal length lens (\SI{1}{\arcsecond} diameter) is positioned to produce a spot size with a diameter between \qtyrange{10}{20}{\centi\metre} to allow for imaging larger objects in a scene at the cost of spatial resolution. The back scattered light is then imaged onto the FLIMera SPAD array using a \SI{58}{\milli\metre} diameter SLR lens with adjustable focal length between \qtyrange{28}{80}{\milli\metre}. By adjusting the focal length of this SLR lens the FOV of the FLIMera's \SI{3.2}{\milli\metre} sensor can by filled by objects ranging from \qtyrange{8}{22}{\centi\metre} in diameter (using Eq.~\ref{eq:objheight}\footnotemark) which for initial experiments involving a glass teapot ($\approx \SI{15}{\centi\metre}$ across, a collage of different materials, and a human hand ($\approx \SI{15}{\centi\metre}$)
\footnotetext{where $x_{0} = \SI{2}{\metre}, y_{i} = \SI{3.2}{\milli\metre}$ and, $f = \qtylist{28;80}{\milli\metre}$}

\begin{equation}\label{eq:objheight}
    x_{0} = \frac{y_{o}f}{y_{i}}
\end{equation}

\begin{figure}
    \centering
    \includegraphics[width = 0.8\textwidth]{figures/TCSPC/TCSPCimagingSetup.pdf}
    \caption{Diagram of imaging system used for material discrimination experiments using TCSPC. A pulsed white light super continuum source (NKT SuperK), filtered with an AOTF, is used to to flood illuminate the scene $\approx \SI{2}{\metre}$ from the source. A \SI{100}{\milli\metre} focal length lens (\SI{1}{\arcsecond} diameter) focus the fibre output of the laser source and by adjusting the position of the lens, $d$ with respect to the source, the spot size produced can be adjusted. An SLR lens with adjustable focal length \qtyrange{28}{80}{\milli\metre} and diameter \SI{58}{\milli\metre} images the scene onto the FLIMera. Image of Lord Kelvin from Royal Society of Edinburgh~\cite{LordKelvin}.} 
    \label{fig:tcspcimagingsystem}
\end{figure}



\FloatBarrier
\section{Material Discrimination}

\FloatBarrier
\section{Depth Imaging using SPAD arrays}

\FloatBarrier
\section{Imaging through Obscurations}