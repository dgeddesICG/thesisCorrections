
In conclusion, a imaging system has been built, and analysis techniques were developed, for recording spectrally-resolved Fluorescence Lifetime Images of the retina - the SFLIO device - to extract the detect and quantify the concentration of the primary retinal metabolites FAD, AGE, and A2E. The SFLIO device was subsequently used to record SFLIM images of anaesthetised rats The ability to quantify retinal fluorophore and longitudinally track their changes would enable for retinal disease to be diagnosed earlier in its progression and improve patient outcomes. 

The SFLIO device was constructed from a COTS ophthalmoscope with the imaging arm being modified for simultaneous spectral-fluorescence lifetime imaging and brightfield imaging for compensating for movement of the retina. The illumination arm was modified to use a white-light pulsed-supercontinuum laser, filtered by an AOTF, to enable optimal excitation of retinal fluorescence. The SFLIO device was characterised in terms of its: spectral response across all four emission bands, the spatial resolution of the FLIM and brightfield arms; and its safety for the potential imaging of in-vivo human retinas. The spatial resolution of the image arms was measured to be X and X which although satisfies the requirement that a images on the FLIMera can be registered it is Y percent poorer than similar ophthalmoscope based imaging systems found in the literature.
The capability of the SFLIO device to compensate for motion in the retina was then demonstrated using a moving sample of convallaria mounted at the retinal plane of an mechanical eye model. Motion was detected in the high frame-rate brightfield images and converted into equivalent motion on the FLIMera with sufficient accuracy that artefact-free fluorescence lifetime and fluorescence intensity images could be recorded with the FLIMera with a factor of 2X super-sampling in the XorY direction.
Next, the accuracy of current techniques for quantifying endmembers from linear mixtures of retina fluorophores - spectral fluorescence imaging, fluorescence lifetime imaging, and spectral-phasor FLIM - were compared and found to be inadequate for isolating and measuring the concentration of FAD in the retina. 
In \cref{chap:tensSFLIM}, a novel unmixing technique based on the principles of linear unmixing in spectral imaging and remote-sensing was proposed which utilises current research in the field of mathematics around pseudo-inverses of order 3 tensors. The accuracy of this new SFLIM unmixing technique was then tested using a set of optimised detection bands it was shown to outperform the aforementioned existing techniques. This chapter culminated in a broad discussion around how to effectively record spectrally-resolved FLIM measurements where it was highlighted that in designing an SFLIM experiment consideration should be given to the requirements for both spectral and temporal resolution. In imaging scenarios with low, and fixed, photon flux it may be the case that using a time-gated FLIM detector - lower temporal but higher spectral resolution - with a higher number of spectral bands would be more suitable for quantifying fluorophores than a similar detection scheme using a TCSPC based FLIM detection scheme with fewer spectral bands. 
Finally, the SFLIO device, motion compensation algorithms, and SFLIM unmixing technique were all utilised in SFLIM measurements of anaesthetised rats to measure changes in retinal FAD brought about by hypoxia. While a high-quality with discernible vasculature could not be formed on the FLIMera it was still possible to infer systemic changes in the retinal FAD as a response to reduced oxygen metabolisation. It was predicted that hypoxia would see the FAD signal decrease with a similar magnitude (betwee nX and Y percent) and this was observed in two different rats over 3 total imaging sessions with at least $p < 10^{-7}$ determined using a Students t-test. This positive result could not be repeated for transitioning back to normal oxygen consumption - where it was hypothesised that FAD concentrations would return to pre-hypoxia levels or after the animal is sacrificed where it was expected the magnitude would decrease to zero after all metabolic functions would cease. The results were for these scenarios were inconclusive and it was theorised that longer refractory periods between SFLIM measurements are required for these metabolic processes to stabilise. The end of this final chapter poses a potential weakness with assumption that the three dominant fluorophores AGE, FAD and A2E were simple additive mixtures is overly simplistic. Instead, fluorescence from the retina will be spectrally and temporally contaminated by scattering in retinal vasculature, the oxygenation dependence absorption of blood, and fluorescence from the eye's lens. The development of more accurate and detailed models of retinal fluorescence would be a worthy avenue for future work.
To summarise, SFLIM or SFLIO was investigated in this thesis as new, and promising, technique for enhancing the detection of retinal diseases by considering that metabolic function is disrupted at the early stages of disease formation. A prototype imaging system was constructed for safely recording SFLIM measurement and a novel unmixing technique was developed for quantifying retinal fluorophores. In SFLIM measurements of anaesthetised rats it was observed with high confidence that the concentration of what is assumed to be FAD does decrease as a response to acute hypoxia but this could not be repeated for other changes in oxygen consumption.

\section{Future Work}
Throughout this thesis several additional avenues for additional research were proposed but not explored due to either time constraints or ideas being formulated during the writing of this thesis with the benefit of hindsight. Here the most attractive are summarised.
\begin{itemize}
    \item The SFLIM unmixing technique developed, and presented in \cref{chap:tensSFLIM} is equivalent to linear unmxing in the field of spectral imaging and remote sensing and could be further extended. By extending the concepts of Non-negative Matrix Factorisation (NMF), or auto-encoder networks a ``blind'' unmxing technique could be developed where only a library of the temporal decay and spectral properties of different fluorophores would be need for unmixing.
    \item A limitation of \cref{chap:tensSFLIM} is the lack of experimental demonstrations. With more time some in-vitro experiments using samples of FAD embedded in a mechanical eye. Additionally, a biological sample, imaged with a SFLIM Microscope, where different structures in the sample are stained with multiple dyes with unique spectral and temporal decay characteristics would serve as key demonstrations of this novel technique.
    \item An exploration into comparing the suitability of the two dominant FLIM detection schemes - TCSPC FLIM and gated FLIM - for SFLIM imaging using a heuristic akin to a space-bandwith product would help answer the question of ``how best to utilise the available photon flux'' or which axis contains the most information as a function of photon flux about the fluorophore.
    \item SFLIM measurements of in-vivo human retinas were not able to be recorded throughout this project. If FAD in the human retina could be resolved this would provide the means for a good publication.
    \item A key limitation in the analysis of retinal fluorophore concentrations in \cref{chap:ratfad} is the overly simplistic model of fluorescence in the retina. Development of a multi-layered model in a ray-tracing software suchar Zemax would allow the effects of scattering and spectral contamination from the blood and vascualture in the retina as well as the effects of lens fluorescence.
\end{itemize}