In conclusion, a imaging system has been built, and analysis techniques were developed, for recording spectrally-resolved Fluorescence Lifetime Images of the retina - the SFLIO device - to extract the detect and quantify the concentration of the primary retinal metabolites FAD, AGE, and A2E. The SFLIO device was subsequently used to record SFLIM images of anaesthetised rats The ability to quantify retinal fluorophore and longitudinally track their changes would enable for retinal disease to be diagnosed earlier in its progression and improve patient outcomes. 
The SFLIO device was constructed by modifying an COTS ophthalmoscope to allow the retina to excited fluorescence in the band X-X by a white-light pulsed-supercontinuum filtered with an AOTF. The imaging path was modified to use a TCSPC-SPAD array (Horiba FLIMera) to record time-resolved fluorescence images and a standard scientific sCMOS camera to record high-framerate image for the purpose of compensating for the motion of the retina throughout the >2 minute image acquisition period. To ensure the retinal motion can be accuratley the compensated the spatial resolution of the images 

\section{Future Work}