\FloatBarrier
\section{SFLIM Unmixing Using Tensor Pseudo-Inverses}
In the previous chapter existing method for recovering FAD concentrations from SFLIM measurements were investigated and found to be unsuitable even in low noise scenarios. For both the fitting based and phasor S-FLIM method their limitations lie with the high overlap of retinal fluorophores in the spectral and lifetime domain.
\\
To address this, a new method has been developed which treats the temporal domain of SFLIM data as a pseudo-`spectra' and extends the matrix inversion approach of standard spectral unmixing to include the temporal domain. This combines the decay profile and emission spectra of the fluorophore to minimise the effects of this high overlap in both domains. The existing Moore-Penrose inverses used in the matrix inversion unmixing technique are only well defined for matrices – endmember tensor in~\cref{fig:tensor} cannot be inverted. To extend this technique for unmixing fluorophore concentration from SFLIM data the work of~\citeauthor{brazell2013solving} was adapted. In their work an algorithm for solving multi-linear systems of equations is described using Moore-Penrose inversions of third-order tensors.


\FloatBarrier
\subsection{Tensor Definitions and Nomeclature - maybe this should be an appendix}\label{subsec:tensdef}
\subsubsection{Tensor Notation}\label{subsec:tensnom}
A tensor is a mathematical object that can have more than 2 dimensions and is invariant to changes of coordinate systems. The dimensional of a tensor is referred to as its order where an order-$n$ tensor has $n$-dimensions e.g. an order-$0$ tensor is a scalar, and order-$1$ tensor is a vector or array, and a order-$2$ tensor is a matrix. An order-3 tensor, $\mathcal{D}$ of real numbers with dimensions of length $I$,$J$, and $K$ would be notated as $\mathcal{D} \in \mathbb{R}^{I \times J \times K}$ and its individual entries are addressed using lowercase notation for the object and its indices: $\mathbf{d}_{ijk}$. Likewise, order-1 tensors are notated as lowercase bold and matrices, order-2 tensors, are notated with uppercase bold symbols but their individual entries are notated in the same way as order-3 tensors.
Operations between tensors are generalised using the Einstein summation or Einstein notation conversions by summing over ``free'' indices which appear only once in an expression and element wise multiplications of entries in ``dummy'' indices that are repeated within an expression. To illustrate this the example in \cref{eq:matmul,eq:matmulein} show multiplication of two matrices, $\mathbf{A} \in \mathbb{R}^{I \times J}$ and $\mathbf{B} \in \mathbb{R}^{J \times K}$ using standard indexed summations and Einstein notation where summation symbols are conventionally omitted for simplicity and brevity.
\begin{align}
    \mathbf{C}_{ik} = (\mathbf{A}\mathbf{B}_{ik}) = \sum_{j= 1}^{N}\mathbf{A}_{ij}\mathbf{B}_{jk}\label{eq:matmul}
\end{align}
\begin{align}
    \mathbf{C}\indices{_i_k} = \mathbf{A}\indices{_i_j}\mathbf{B}\indices{_j_k}\label{eq:matmulein}
\end{align}
Here, ``j'' is the dummy index and is summed over, and element-wise multiplication is performed on the``i'' and ``k''  indices. 


\FloatBarrier
\subsubsection{Einstein Products}\label{subsec:einprod}
Einstein or contracted products between two tensors can be used to contract specific indices. For example the Einstein $N$-Product over $N$ indices for two arbitrary tensors
$\mathcal{A} \in \mathbb{R}^{I_{i}\times\cdot\times I_{L}\times K_{1}\times\cdot\times K_{N}}$ and $\mathcal{B} \in \mathbb{R}^{K_{i}\times\cdot\times K_{N}\times J_{1}\times\cdot\times J_{M}}$ and is defined as follows:
\begin{align}
    \bigg(\mathcal{A} \ast_{N} \mathcal{B}\bigg)_{i_{1}\cdots i_{L}j_{1}\cdots j_{M}} &=\sum\limits_{k_{1}\cdots k_{N}}\mathbf{a}_{i_{1}\cdots i_{L}k_{1}\cdots k_{N}}\mathbf{b}_{k_{1}\cdots k_{N}j_{1}\cdots j_{M}}
\end{align}
For the implementation of the Tensor based SFLIM unmixing technique only the `$\ast_{2}$' product is needed. For an example SFLIM endmember tensor, $\mathcal{D}\in \mathbb{R}^{t \times \lambda \times \sigma}$, with $t$ - time bins, $\lambda$ - spectral channels, and $\sigma$ - endmembers and a corresponding SFLIM measurement $\mathbf{M}\in\mathbb{R}^{t \times \lambda}$ one of the contracted products used in the calculation of the estimator of in the least-squares minimisation process is shown in \cref{eq:einprodex}:
\begin{align}\label{eq:einprodex}
\big(\mathcal{D}^{T}\ast_{2}\mathcal{D}\big)_{\tilde{\sigma}\sigma}=\sum_{t,\lambda} &= \hat{\mathbf{d}}_{\tilde{\sigma}t\lambda}\mathbf{d}_{\sigma t \lambda}
\end{align}
Here the to simplify notation the elements of the tensor transpose,~$\mathcal{D}^{T}$, is notated as $\hat{\mathbf{d}}$ and for indices that are twinned they are denoted as $\tilde{\sigma}$ where $\tilde{\gamma}$ has same dimensions as $\gamma$ but is not to be treated as ``dummy'' indices.


\FloatBarrier
\subsubsection{Tucker N-Mode Product}\label{subsec:tuckprod}
The Tucker N-Mode product is an extension of the Einstein Product in that it is simply a contracted sum (Einstein 1-Product) over a defined Mode of a Tensor but is now denoted using $\bullet_{N}$ instead of $\ast_{N}$.For a Tensor $\mathcal{T} \in \mathbb{R}^{I \times J \times K}$ then Mode 1 is the $i$-direction, Mode 2 is the $j$-direction, and Mode 3 is the $k$-direction.
\\
For this application only the Mode-3 product between a tensor and vector is considered and using the same tensor, $\mathcal{D}\in \mathbb{R}^{t \times \lambda \times \sigma}$, as above the mode 3 product with the vector $\mathbf{x} \in \mathbb{R}^{\sigma}$, denoted as $\bullet_{3}$  also evaluates to a vector (\cref{eq:tuckmodeex}) and is akin to the dot product between a matrix and a vector.
\begin{align}
   \big(\mathcal{D}\bullet_{3}\mathbf{x}\big)_{\sigma} &= \sum_{t, \lambda}\mathbf{d}_{t \lambda \sigma}\mathbf{x}_{\sigma} \label{eq:tuckmodeex}
\end{align}


\FloatBarrier
\subsubsection{Transposes of an Odd Third-Order Tensor}\label{subsec:tenstrans}
For Vectors and Matrices the transpose is well defined since there is only one permutation of a matrix or a vectors indices. However for an $N$ - dimensional tensor there are $N!$ permutations of the indices and therefore $N!$ valid transposes. For example an order-3 tensor has \num{6} different possible transposes (\cref{tab:trans}). In \citeauthor{brazell2013solving} they define the appropriate transposes for different multilinear systems of equations for the application for SFLIM unmixing the first entry of \cref{tab:trans} is used for $\mathcal{A} \in \mathbb{R}^{t \times \lambda \times \sigma}$, $\mathbf{M}\in\mathbb{R}^{t \times \lambda}$, and $\mathbf{x} \in \mathbb{R}^{\sigma}$.  
\begin{table}
    \centering
    \begin{tabular}{|c|c|c|c|}
        \hline
        $\mathcal{A}$ &  $\mathbf{x}$ & $\mathbf{B}$ & $\mathcal{A}^{t}$\\
        \hline
        $\mathbb{R}^{I \times J \times K}$ & $\mathbb{R}^{K}$ & $\mathbb{R}^{I \times J}$ & $\mathbb{R}^{K \times I \times J}$ \\
        
        $\mathbb{R}^{J \times K \times I}$ & $\mathbb{R}^{I}$ & $\mathbb{R}^{J \times K}$ & $\mathbb{R}^{I \times J \times K}$ \\
        
        $\mathbb{R}^{K \times I \times J}$ & $\mathbb{R}^{J}$ & $\mathbb{R}^{K \times I}$ & $\mathbb{R}^{J \times K \times I}$ \\

        
        $\mathbb{R}^{I \times K \times J}$ & $\mathbb{R}^{J}$ & $\mathbb{R}^{K \times I}$ & $\mathbb{R}^{J \times I \times K}$ \\

        
        $\mathbb{R}^{K \times J \times I}$ & $\mathbb{R}^{I}$ & $\mathbb{R}^{K \times J}$ & $\mathbb{R}^{I \times K \times J}$ \\

        
        $\mathbb{R}^{J \times I \times K}$ & $\mathbb{R}^{K}$ & $\mathbb{R}^{J \times I}$ & $\mathbb{R}^{K \times J \times I}$ \\

        \hline
    \end{tabular}
    \caption{Third order tensor transposes for various forms of multilinear systems. Reproduced from \citeauthor{brazell2013solving}\cite{brazell2013solving}}
    \label{tab:trans}
\end{table}

\FloatBarrier
\subsection{The SFLIM unmixing Method}
In this new approach a similar linear mixing model (\cref{eq:tensmix} ) is used. However, now a third-order tensor, $\mathcal{A}$, contains the spectral and decay properties of the endmember and a matrix, $\mathbf{M}$, contains the SFLIM measurements. The endmember abundances are still a vector, $\mathbf{x}$, where the endmember abundances all sum to unity.
\begin{align}
    \mathcal{A}\bullet_{3}\mathbf{x} &= \mathbf{M} \label{eq:tensmix}
\end{align}

\begin{figure}[htbp]
\centering
\begin{tikzpicture}[every node/.style={anchor=north east,fill=white}, scale = 3]
\matrix (mA) [draw,matrix of math nodes]
{
\mathcal{A}_{0}(\lambda_{0}, t_{i}) & \mathcal{A}_{0}(\lambda_{1}, t_{i}) & \cdots & \mathcal{A}_{0}(\lambda_{j}, t_{i}) \\
\mathcal{A}_{1}(\lambda_{0}, t_{i}) & \mathcal{A}_{1}(\lambda_{1}, t_{i}) & \cdots & \mathcal{A}_{1}(\lambda_{j}, t_{i}) \\
\vdots & \vdots & \ddots & \vdots \\
\mathcal{A}_{n}(\lambda_{0}, t_{i}) & \mathcal{A}_{n}(\lambda_{1}, t_{i}) & \cdots & \mathcal{A}_{n}(\lambda_{j}, t_{i}) \\
};

\matrix (mB) [draw,matrix of math nodes] at ($(mA.south west)+(3em,1.5em)$)
{
\mathcal{A}_{0}(\lambda_{0}, t_{1}) & \mathcal{A}_{0}(\lambda_{1}, t_{1}) & \cdots & \mathcal{A}_{0}(\lambda_{j}, t_{1}) \\
\mathcal{A}_{1}(\lambda_{0}, t_{1}) & \mathcal{A}_{1}(\lambda_{1}, t_{1}) & \cdots & \mathcal{A}_{1}(\lambda_{j}, t_{1}) \\
\vdots & \vdots & \ddots & \vdots \\
\mathcal{A}_{n}(\lambda_{0}, t_{1}) & \mathcal{A}_{n}(\lambda_{1}, t_{1}) & \cdots & \mathcal{A}_{n}(\lambda_{j}, t_{1}) \\
};

\matrix (mC) [draw,matrix of math nodes] at ($(mB.south west)+(3em,1.5em)$)
{
\mathcal{A}_{0}(\lambda_{0}, t_{0}) & \mathcal{A}_{0}(\lambda_{1}, t_{0}) & \cdots & \mathcal{A}_{0}(\lambda_{j}, t_{0}) \\
\mathcal{A}_{1}(\lambda_{0}, t_{0}) & \mathcal{A}_{1}(\lambda_{1}, t_{0}) & \cdots & \mathcal{A}_{1}(\lambda_{j}, t_{0}) \\
\vdots & \vdots & \ddots & \vdots \\
\mathcal{A}_{n}(\lambda_{0}, t_{0}) & \mathcal{A}_{n}(\lambda_{1}, t_{0}) & \cdots & \mathcal{A}_{n}(\lambda_{j}, t_{0}) \\
};

\draw[dashed](mA.north east)--(mC.north east);
\draw[dashed](mA.north west)--(mC.north west);
\draw[dashed](mA.south east)--(mC.south east);
\draw[thick,-stealth] ([xshift=1ex]mC.south east) -- ([xshift=1ex]mA.south east)
  node[midway,below] {t};
 \draw[thick,-stealth] ([yshift=-1ex]mC.south west) -- 
  ([yshift=-1ex]mC.south east) node[midway,below] {$\lambda$};
 \draw[thick,-stealth] ([xshift=-1ex]mC.north west)
   -- ([xshift=-1ex]mC.south west) node[midway,above,rotate=90] {n};
  
\end{tikzpicture}

\caption{Visualisation of the $\mathcal{A}$ tensor in Eq.~\ref{eq:tensmix} composed of,$n$, unique endmembers with known spectra ($\lambda$ - axis) and fluorescence decay profiles ($t$ - axis).}
\label{fig:tensor}
\end{figure}
In this section the SFLIM unmixing algorithm is now described reproduced and adapted from~\citeauthor{brazell2013solving}. Definitions of tensor operands are given in next 
For the endmember tensors, $\mathcal{A}\in \mathbb{R}^{I \times J \times K}$, matrix of SFLIM measurements, $\mathbf{M}$, with, $I$ time bins, $J$, spectral measurements, and $K$ endmembers the optimal recovered relative abundances , $\mathbf{x}\in \mathbb{R}^{K}$, would satisfy the following minimisation:
\begin{equation}
    \mathbf{x} = \textit{min}\bigg\{\lvert\lvert \mathcal{A}\bullet_{3}\mathbf{x} - \mathbf{M} \rvert\rvert_{F} \bigg\}
\end{equation}
For a least squares method there is then the minimising function $\varphi(\mathbf{x}) = \big\lvert\big\lvert \mathcal{A}\bullet_{3}\mathbf{x} - \mathbf{M}\big\rvert\big\rvert_{F}^{2}$ with a Jacobian, $\nabla\varphi(\mathbf{x})$. An estimator of the relative abundances, $\bar{\mathbf{x}}$ can then be found at the critical point where $\nabla\varphi(\bar{\mathbf{x}}) = 0$.
First, an analytical expression for the Jacobian, $\nabla\varphi(\mathbf{x})$ is found by expanding $\varphi(\mathbf{x})$ using the identity of the Frobenius norm for order-2 tensors - $\big\lvert\big\lvert \mathbf{B} \big\rvert\big\rvert_{F}^{2} = \langle \mathbf{B}, \mathbf{B}\rangle \equiv \mathbf{B}^{T}\mathbf{B}$.
\begin{align}
    \varphi(\mathbf{x}) &= \langle\mathcal{A}\bullet_{3}\mathbf{x} - \mathbf{M}, \mathcal{A}\bullet_{3}\mathbf{x} - \mathbf{M}\rangle \\
    &= (\mathcal{A}\bullet_{3}\mathbf{x})^{T}(\mathcal{A}\bullet_{3}\mathbf{x}) - 2\mathbf{M}^{T}(\mathcal{A}\bullet_{3}\mathbf{x}) +  \mathbf{M}^{T}\mathbf{M}
\end{align}
The Jacobian, $\nabla\varphi(\bar{\mathbf{x}})$ is then found: 
\begin{align}
    \nabla\varphi(\mathbf{x}) = \pdv{\varphi(\mathbf{x})}{\mathbf{x}} = 2\mathcal{A}^{T} \ast_{2} \mathcal{A}\bullet_{3}\mathbf{x} - 2\mathcal{A}^{T}\ast_{2}\mathbf{M} \label{eq:jacfunc}
\end{align}
Next, using this expression for $\nabla\varphi(\mathbf{x})$, and the assumption that $(\mathcal{A}^{T} \ast_{2} \mathcal{A})^{-1}$ is square and non-singular i.e its inverse exists,  the estimator is then found by solving  $\nabla\varphi(\bar{\mathbf{x}}) = 0$:
\begin{align}
    \nabla\varphi(\bar{\mathbf{x}}) = 0 &= \mathcal{A}^{T} \ast_{2} \mathcal{A}\bullet_{3}\bar{\mathbf{x}} - \mathcal{A}^{T}\ast_{2}\mathbf{M}\\
    \mathcal{A}^{T} \ast_{2} \mathcal{A}\bullet_{3}\bar{\mathbf{x}} &= \mathcal{A}^{T}\ast_{2}\mathbf{M}\\
    (\mathcal{A}^{T} \ast_{2} \mathcal{A})^{-1}\ast_{2}(\mathcal{A}^{T} \ast_{2} \mathcal{A})\bullet_{3}\bar{\mathbf{x}} &= (\mathcal{A}^{T} \ast_{2} \mathcal{A})^{-1}\ast_{2}\mathcal{A}^{T}\ast_{2}\mathbf{M}
\end{align}
For $\mathbf{B} = \mathcal{A}^{T} \ast_{2} \mathcal{A}$, $\mathbf{B}$ is a matrix as thus has the property: $\mathbf{B}^{-1}\ast_{2}\mathbf{B} = \mathbb{I}_{K} $ yielding a final expression for $\bar{\mathbf{x}}$:
\begin{align}
        \mathbb{I}\bullet_{3}\bar{\mathbf{x}} &= (\mathcal{A}^{T} \ast_{2} \mathcal{A})^{-1}\ast_{2}\mathcal{A}^{T}\ast_{2}\mathbf{M}\\
        \implies \bar{\mathbf{x}} &= (\mathcal{A}^{T} \ast_{2} \mathcal{A})^{-1}\ast_{2}\mathcal{A}^{T}\ast_{2}\mathbf{M} \label{eq:xbar}
\end{align}

Using this mathematical framework the above SFLIM unmixing method, shown in \cref{alg:sflim} is implemented in \textit{python} using a combination of least-squares minimisation as well as Einstein Summations to represent the Einsten Prodcuts, $\ast_{2}$, and the Tucker Mode Products, $\bullet_{3}$. The use of a least-squares minimiser with the Jacobian (\cref{eq:jacfunc}) and Estimator (\cref{eq:xbar}) functions ensures quicker convergence to a global minima despite the influence of noise and the similarity of the endmembers in the $\mathcal{A}$ tensor. For well conditioned and low noise problems the estimator $\bar{\mathbf{x}}$ can be evaluated directly to recover the relative abundance of endmembers.
\begin{algorithm}
    \caption{Algorithm for recovering endmember abundances from spectral FLIM data for a single spatial pixel}
    \label{alg:sflim}
    \begin{algorithmic}
    \Require$ \mathcal{A} \in \mathbb{R}^{I \times J \times K}\mbox{, } \mathbf{M} \in \mathbb{R}^{I \times J} $\mbox{ where $I = t$ - axis, $J = \lambda$ - axis, and $K =n$ - axis}
    \Require $\sum_{i,j}^{I,J}\mathcal{A} = 1$
    \Procedure{Least-Squares Minimisation}{$\mathcal{A}, \mathbf{M}$}
        \State $\mathbf{x}_{min} \gets \textit{least\textunderscore squares}(\text{Cost Function, Jacobian, Estimator})$
        \State $\mathbf{x}_{opt} = \nicefrac{\mathbf{x}_{min}}{\sum_{k}^{K}\mathbf{x}_{min}}$
    \EndProcedure
        \Procedure{Cost Function - $\varphi$}{$\mathcal{A}, \mathbf{M}, \mathbf{x}_{in}$}
            \State $\mathcal{A}\bullet_{3}\mathbf{x}_{in} \gets \textit{einsum}[ijk,k \rightarrow ij](\mathcal{A},\mathbf{x}_{in})$
            \State $\varphi \gets \sqrt{\sum_{i,j,k}^{I,J,K}\lvert\mathcal{A}\bullet_{3}\mathbf{x}_{in} - \mathbf{M}\rvert}^{2}$
        \EndProcedure
        \Procedure{Jacobian - $\nabla\varphi$}{$\mathcal{A}, \mathbf{M}, \mathbf{x}_{in}$}
            \State $\mathcal{A}^{T} \gets \text{permute axes of } \mathcal{A} : ijk \rightarrow kij$
            \State $\mathcal{A}\bullet_{3}\mathbf{x}_{in} \gets \textit{einsum}[ijk,k \rightarrow ij](\mathcal{A},\mathbf{x}_{in})$
            \State $\mathcal{A}^{T}\ast_{2}\mathcal{A}\bullet_{3}\mathbf{x}_{in} \gets \textit{einsum}[ijk,jk \rightarrow i](\mathcal{A}^{T}, \mathcal{A}\bullet_{3}\mathbf{x}_{in})$
            \State $\mathcal{A}^{T}\,\ast_{2}\mathbf{M} \gets \textit{einsum}[ijk,jk \rightarrow i](\mathcal{A}^{T},M)$
            \State $\nabla\varphi \gets 2\mathcal{A}^{T}\ast_{2}\mathcal{A}\bullet_{3}\mathbf{x}_{in} - 2\mathcal{A}^{T}\,\ast_{2}\mathbf{M}$
        \EndProcedure
        \Procedure{Estimator - $\bar{\mathbf{x}}$}{$\mathcal{A}, \mathbf{M}$}
            \State $\mathcal{A}^{T} \gets \text{permute axes of } \mathcal{A} : ijk \rightarrow kij$ 
            \State $\mathcal{A}^{T}\ast_{2}\mathcal{A} \gets \textit{einsum}[ijk,jkl \rightarrow il](\mathcal{A}^{T}, \mathcal{A})$
            \State $\mathcal{B} \gets (\mathcal{A}^{T}\ast_{2}\mathcal{A})^{-1}$
            \State $\mathcal{B}\ast_{2}\mathcal{A}^{T} \gets \textit{einsum}[ij, jkl \rightarrow ikl](\mathcal{B}, \mathcal{A}^{T})$
            \State $\bar{\mathbf{x}} \gets \textit{einsum}[ijk, jk \rightarrow i](\mathcal{B}\ast_{2}\mathcal{A}^{T},\mathbf{M})$
        \EndProcedure
    \end{algorithmic}
\end{algorithm}

\FloatBarrier
\section{Optimising Spectral Detection Bands for SFLIM Unmixing of Retinal Fluorophores}\label{sec:tensband}

The efficacy of this SFLIM unmixing method was then investigated after re-optimising the spectral detection bands. Since this new method now incorporates the fluorescence decay this gives another dimension to discriminate with – fluorophores with high spectral overlap but dissimilar fluorescence lifetimes can now be separated. 
For this optimisation process the catalogue of bandpass filters (>OD3 outside the pass-band) from Edmund Optics was exhaustively searched to find the optimal set of \num{6} filters from a total \num{43} ($~\approx\num{6e6}$ combination) different filters using 3 different Figures-of-Merit (FoMs).  The filters have a range of central wavelengths from \qtyrange{500}{700}{\nano\metre} and with bandwidths from \qtyrange{14}{84}{\nano\metre}. The best 100 filter sets from each FoM was then evaluated for sensitivity to fluorophore concentration using the abundance maps with shot-noise equivalent to \num{e5} photons and the combination with the lowest RMS error in unmixed FAD was selected. 
\subsection{Figures of Merit}
The tensor condition number, $\kappa$, was used for these FoMs and was also weighted to consider optical throughput. These additional metric~$\nicefrac{\sqrt{BW}}{\kappa}$ and $\nicefrac{BW}{\kappa}$ represent the shot-noise and read-noise limited noise scenario where $BW$ is an estimation of the optical throughput by simply calculating the fraction of the spectra covered by the filter set. The tensor condition number is directly comparable to the matrix condition number described previously (\cref{eq:matcondnum} but now the Frobenius norm is used for an third-order tensor rather than a matrix. For the order-3 tensor, $\mathcal{A}\in\mathbb{R}^{I \times J \times K}$,  this norm and condition number is calculated using~\cref{eq:tensorfrob,eq:tensorcondnum} and the Moore-Penrose inverses, $\mathcal{A}^{+} = (\mathcal{T}\ast_{2}\mathcal{A})^{-1}\ast_{2}\mathcal{A}$
\begin{equation}\label{eq:tensorfrob}
    \lvert\lvert \mathcal{A} \rvert\rvert_{F} = \sqrt{\sum_{i = 1}^{m}\sum_{j = 1}^{n}\sum_{k = 1}^{p}\lvert\mathbf{a}_{ijk}\rvert^{2}}
\end{equation}
\begin{equation}\label{eq:tensorcondnum}
    \kappa(\mathcal{A}) = \lvert\lvert \mathcal{A} \rvert \rvert_{F} \lvert\lvert \mathcal{A}^{+} \rvert \rvert_{F}
\end{equation}


\subsection{Optimal Spectral Filters for Quantifying Retinal FAD}
The best filters combination for each metric shown in~\cref{fig:tensoptband}a-c shows overlapping, redundant spectral bands which provided minimal gains in unmixing accuracy at the penalty of increased integration time. These overlapping filters were consolidated to give sets of 4 wider band filters - keeping the 3 narrow bands between \SI{500}{\nm} and \SI{575}{\nm}. The location of this fourth band was then reoptimized using the condition number and $\nicefrac{\sqrt{BW}}{\kappa}$ which gives the result in \cref{fig:tensoptband}d-e. Whilst this does increase the RMS error in recovered FAD from  $\epsilon_{FAD} = 0.618$ for 6 bands  to $\epsilon_{FAD} = 0.724$ it is still lower than the 5 band system previously optimised for spectral imaging and as shown in \cref{fig:SFLIM-RetFluor-1e5photons}. The sensitivity to relative fluorophore concentration follows a similar trend where unmixing errors are highest at low concentrations. Further, by using fewer, wider bands, this higher optical throughput can either be exploited for shorter SFLIM integration times whilst maintaining a similar SNR to the previous 6 band system or be used to increase the integration time per filter to yield an overall increased SNR per spectral channel, again, when compared to the 6-band system.
 
\begin{figure}
    \centering
        \begin{annotatedFigure}{\includegraphics[width = \textwidth]{figures/sflim/tensor-unmixing/SFLIMOptimisedBands.pdf}}
            \annotatedFigureText{0.01,0.95}{black}{0.3}{a)}
            \annotatedFigureText{0.34,0.95}{black}{0.3}{b)}
            \annotatedFigureText{0.68,0.95}{black}{0.3}{c)}
            \annotatedFigureText{0.01,0.50}{black}{0.3}{d)}
            \annotatedFigureText{0.52,0.50}{black}{0.3}{e)}
        \end{annotatedFigure}
    \caption{Sets of 6 filters optimised for SFLIM unmixing using 3 different performance metrics: $\kappa$ the tensor condition number (a); $\nicefrac{BW}{\kappa}$ which weights the condition number inversely the bandwidth of the filter set - mimicking a read-noise limited scheme (b); and $\nicefrac{\sqrt{BW}}{\kappa}$ where the condition number is now weighted with the square root of the bandwidth representing a shot-noise limited scheme (c). In d) and e) these 6 filters were further optimised to two sets of 4 filters by consolidating overlapping filters into a single wider band filter and re-optimising using the aforementioned performance metrics.}
    \label{fig:tensoptband}
\end{figure}
\begin{table}
    \centering
    \begin{tabular}{|c|c|c|c|c|}
        \hline
        \multicolumn{5}{|c|}{$\epsilon_{RMS}$ }\\
        \hline
        \multicolumn{2}{|c|}{} &  AGE & FAD & A2E \\
        \hline
       6 Filter Set & $\kappa$ & \num[round-mode = figures, round-precision = 3]{0.39716098} & \num[round-mode = figures, round-precision = 3]{0.61822781} & \num[round-mode = figures, round-precision = 3]{0.14949178}\\
         & $\nicefrac{\sqrt{BW}}{\kappa}$ & \num[round-mode = figures, round-precision = 3]{0.40467259} & \num[round-mode = figures, round-precision = 3]{0.64034744} & \num[round-mode = figures, round-precision = 3]{0.08565442}\\
         & $\nicefrac{BW}{\kappa}$ & \num[round-mode = figures, round-precision = 3]{0.31255062} & \num[round-mode = figures, round-precision = 3]{0.4163497} & \num[round-mode = figures, round-precision = 3]{0.12503993}\\
         \hline
         4 Filter Set & $\kappa$ & \num[round-mode = figures, round-precision = 3]{0.37950665} & \num[round-mode = figures, round-precision = 3]{0.72381184} & \num[round-mode = figures, round-precision = 3]{0.11899043}\\
         & $\nicefrac{\sqrt{BW}}{\kappa}$ & \num[round-mode = figures, round-precision = 3]{0.40303829} & \num[round-mode = figures, round-precision = 3]{0.7153356} & \num[round-mode = figures, round-precision = 3]{0.10026358}\\
         \hline
    \end{tabular}
    \caption{RMS error values - scaled from 0 to 1 for filter sets consisting of 6 and 4 spectral band obtained from optimising: condition number - $\kappa$; bandwidth weighted with condition number - $\nicefrac{BW}{\kappa}$ to mimic a read-noise limited detection scheme; and bandwidth weighted with condition number modified to represent the case of shot-noise limited detection schemes -$\nicefrac{\sqrt{BW}}{\kappa}$.}
    \label{tab:SFLIMRMSerrors}
\end{table}

\begin{figure}
    \centering
    \includegraphics[width = 0.8\textwidth]{figures/sflim/tensor-unmixing/SFLIM-4BandConfig-UnmixingResults.pdf}
    \caption{The sensitivity to relative fluorophores concentration was examined for unmixing retinal fluorophores using the SFLIM unmixing algorthim and the optimised 4 spectral band configuration for with shot-noise equivalent to that of \num{e5} emitted photons. The ground truth abundance is shown in the top row ranging from \numrange{0}{1} (blue to yellow) and the unmixed abundances are shown in the middle row. The bottom row shows the relative error in this unmixing quantified using~\cref{eq:recerror}.}
    \label{fig:SFLIM-RetFluor-1e5photons}
\end{figure}

\FloatBarrier
\section{Suitability for SFLIM inversions in different\\ unmixing scenarios}
While the SFLIM unmixing method described in chapter was developed for measuring concentrations of retinal FAD it could still prove useful in other SFLIM imaging scenarios. A set of dyes with Gaussian fluorescence emission spectra and mono-exponential decay profiles was simulated for varying spectral bands and unmixing method: spectral unmixing, SFLIM unmixing for TCSPC imaging, and SFLIM unmixing for Gated FLIM. The condition number was calculated for each scenario from 3 spectral bands to 15 spectral bands without the inclusion of photon noise. For the gated FLIM examples 2 contiguous gates of width \SI{2}{\ns} were used.
In the first set of dyes the emission spectra were simulated to be highly overlapping with a FWHM of \SI{15}{\nm} and central wavelengths only differing by \SI{2}{\nm} (CWL = \qtylist{548;550;552}{\nm}) and the respective fluorescence lifetimes were well separable - $\tau = \qtylist{1.50;2.00;2.50}{\ns}$. The spectral bands were simulated with "$n$" evenly-spaced rectangular filters between \qtyrange{520}{580}{\nm} to minimise the number of bands covering areas of the spectra with zero flux and thus give a better constraint on the condition number. When examining the condition numbers in this scenario (see \cref{fig:sflimunmixingoverlapspectra}) the high spectral overlap results in the spectral unmixing method exhibiting a $\approx 4$-times larger condition number when compared to the SFLIM unmixing indicating that the temporal domain does aid the unmixing process.
\begin{figure}
    \centering
    \begin{annotatedFigure}{\includegraphics[width = 0.85\textwidth]{figures/sflim/tensor-unmixing/SFLIM-ConditionNumber-ComparisonGaussian1.pdf}}
        \annotatedFigureText{0.01,0.96}{black}{0.3}{a)}
        \annotatedFigureText{0.01,0.52}{black}{0.3}{b)}
        \annotatedFigureText{0.41,0.96}{black}{0.3}{c)}
    \end{annotatedFigure}
    
    \caption{A set of 3 Gaussian spectra (a) with mono-exponential lifetimes (b) were simulated and in (c) the condition number of Spectral Unmixing, SFLIM unmixing, and gated-SFLIM unmixing were plotted as a function of the number of contiguous spectral bands.}
    \label{fig:sflimunmixingoverlapspectra}
\end{figure}


Similarly for the second scenario the spectra were simulated to be well-separable with central wavelengths of \qtylist{550;600;650}{\nm} with a FWHM of \SI{15}{\nm} but now the lifetimes were separated by just \SI{100}{\ps} ($\tau = \qtylist{1.50;1.60;1.70}{\ns}$) which would ordinarily be difficult to segment by fitting fluorescence lifetimes. From this simulation, the condition numbers for all three methods are equal across all spectral band numbers indicating that majority of the variation in the signal is in the spectral domain but that also that the SFLIM unmixing algorithm is not adversely affected by low variation in lifetime.


\begin{figure}
    \centering
    \begin{annotatedFigure}{\includegraphics[width = 0.85\textwidth]{figures/sflim/tensor-unmixing/SFLIM-ConditionNumber-ComparisonGaussian2.pdf}}
        \annotatedFigureText{0.01,0.96}{black}{0.3}{a)}
        \annotatedFigureText{0.01,0.52}{black}{0.3}{b)}
        \annotatedFigureText{0.41,0.96}{black}{0.3}{c)}
    \end{annotatedFigure}
    
    \caption{A set of 3 Gaussian spectra (a) with mono-exponential lifetimes (b) were simulated and in (c) the condition number of Spectral Unmixing, SFLIM unmixing, and gated-SFLIM unmixing were plotted as a function of the number of contiguous spectral bands.}
    \label{fig:sflimunmixingoverlaplifetime}
\end{figure}
\FloatBarrier
\section{Conclusions} \label{sec:sflim-conc}
In this chapter a new method for unmixing FAD concentrations from SFLIM measurements of the retina was developed and the optimal spectral bands for recording these SFLIM measurements were found. The SFLIM unmixing technique is adapted from the Jacobian based minimisation procedure described by~\citeauthor{brazell2013solving}.
After optimising the location and number of spectral bands the performance of the SFLIM unmixing algorithm was then shown to improve capabilities of quantifying retinal FAD, compared to spectral unmixing, by exhibiting a reduced RMS error - \SI{72.4}{\percent} for SFLIM unmixing opposed to \SI{91.8}{\percent} for spectral unmixing - for equivalent photon fluxes. Finally, the flexibility of the SFLIM unmixing algorithm was briefly explored by comparing the condition number of spectral unmixing, to SFLIM unmixing, and gated-SFLIM unmixing computed when unmixing sets of Gaussian spectra simulated to have either high spectral overlap or highly similar fluorescence lifetimes. These simulations demonstrated a limitation of TCSPC-based SFLIM methods where SFLIM exhibits a lower SNR than a spectral imaging system simply due to the same photon flux being distributed over more shot-noise and read-noise affected measurements. Despite this the SFLIM unmixing algorithm does outperform traditional spectral unmixing in recovering FAD concentrations with a lower RMS error. A possible solution to this noise issue could be to use a time-gated detection scheme where the $>250$ time bins spanning \SI{47}{\ps} are replaced by two or more \SI{2}{\ns} wide gates. This has the effect of increasing SNR at the cost of temporal resolution but as shown in \cref{fig:sflimunmixingoverlapspectra,fig:sflimunmixingoverlaplifetime} this would not necessarily culminate in poorer unmixing performance. In principle gated detection schemes also overcome the photon-pileup issue of TCSPC and fundamentally the low-pixel count of the FLIMera where megapixel gated SPAD detectors have been reported~\cite{morimoto2020megapixel}.
\FloatBarrier
\section{Future Work}\label{sec:sflim-future}
The SFLIM unmixing algorithm presented in this thesis represents the adaptation of the simplest technique in the spectral unmixing to the field of SFLIM. There is scope to also adapt the concepts of Non-negative Matrix Factorisation (NMF) ~\cite{pauca2006nonnegative}, VCA~\cite{nascimento2005vertex}, Auto-encoder network~\cite{su2019daen} spectral unmixing techniques to account for spectral contamination from blood in vessels and arteries, lens fluorescence by blindly recovering their abundance and spectra in the retina. Further, these "blind" unmixing method would also test the assumption made in this thesis and reported in \citeauthor{schweitzer2007towards} that AGE, A2E, and FAD are the dominant chromophores in the retina~\cite{schweitzer2007towards}.
Finally, a key demonstration of the SFLIM unmixing technique, beyond that of retinal imaging, would be in the scenario of a stained biological sample where different micro-structures are labelled with dyes that can be tailored for minimal spectral and temporal overlap.
