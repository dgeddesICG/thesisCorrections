\setstretch{2}
\section{Outline / Bullet Points}

\subsection{Rationale}
\begin{itemize}
    \item Key demonstration of SFLIO system by combining the SFLIM unmixing technique with the SFLIO imaging system
    \item Imaging rats at different oxygenation states enables imaging at higher intensities than would be allowed when imaging in-vivo humans
    \item Changing oxygenation state also also allows us to artificially change the FAD concentration. FAD is a metabolite of oxygen consumption in aerobic respiration
    \item By cycling the consumption of oxygen from Normoxic FiO2 21percent to Hypoxic 15percent back to Normoxia and finally at death (effectively zero) - detecting something change in the SFLIM data that is correlated to these events would signify the detection of FAD in the retina and also indicate the precision with which we can quantify its concentration
    \item Normoxia to Hypoxia should see a decrease in FAD, Hypoxia to Normoxia should see a larger change in FAD as the respiratory / metabolic system overproduces FAD for a short time (need a citation) to compensate for prior FAD starvation. Normoxia to Death should show the largest change since any FAD that was present in the tissue is used up / reduced to FADH and no-new FAD will be produced.
    \item FAD has been detected in rat cortexes where upon death it forms halos around the arteries and veins as it leaks into the surrounding tissue. [kens paper]
\end{itemize}
\subsection{Experimental Protocol}
\begin{itemize}
    \item Albino Rats were anaesthetised using isoflourane and their oxygenation was controlled using a nosecone that supplies oxygen at variable concentration.
    \item Rat is stabilised and a contact lens - contributing no refractive power - and a Viscotears - was used to prevent the eye's becoming dry and irritated. 
    \item Tropicamide was used to dilate the pupil from its usual X diameter to Y diameter
    \item A 2'' diameter 85mm field lens was fitted in front of the fundus camera to allow the smaller pupil and eye ball size to be accommodated in the SFLIO imaging system. 
    \item This lens does induce some glare due to additional reflections of the lens surface in the brightfield image which could ordinarily be corrected by imaging a blacked out room with the illumination on - reproducing the glare - and linearly subtracting it from each image using i.e I - aG where a is found by minimising the standard deviation of the image.
    \item for the purpose of registration this glare polluted area can simply be masked out during the phase cross-correlation process.
    \item Registration images were recorded in brightfield with frame rate of 10hz and a frame integration time of 50ms. This integration time in normal human imaging would be degraded by blur due to saccades but since the rats are anaesthetised this systemic is overridden and the only observed motion artefacts are from what appears to be laboured breathing 
    \item For SFLIM the static affine transformation between the two camera was re-determined using the same convallaria slide mounted in a phantom eye - without using the field lens - this was to account for any small changes in the positions of the SPAD and sCMOS wrt to the imaging path due to having transported the equipment to London from Glasgow. 
    \item SFLIM images were recorded over periods of 150seconds where the laser shutter is initially closed and opened after the a brief period illuminating the retina. This enables at least 2mins of continuous integration using the FLIMera raw mode
\end{itemize}
\subsection{Results}

\begin{itemize}
    \item While a high quality image could be formed on the sCMOs camera which clearly shows the vascular structure of the retina could be formed - albeit polluted by glare in the centre of the vision. A sufficient image could not be formed on the FLIMera but there is sufficient signal to attempt SFLIM unmixing. This could be due to the low fluorescence contrast as evidenced by the fluorescence intensity image recorded with a lp500 filter in the SCMOs camera. 
    \item Can't map FAD across the retina but can use the signal to extract a general trend in the FAD concentration wrt oxygenation.
    \item In the histograms recorded by the FLIMera there was also the addition Gaussian like thermal noise characteristic that was ultimately corrected by simply neglecting these values in the calculation of the cost function of the minimisation process (phi). While this does decrease the overall energy in the signal what is left should be capable of capturing both the long and short lifetime dynamics.
    \item A brief comparison of the possible dark field correction methods yielded no significant change in the recovered concentration of retinal FAD  - dark-field correction by fitting a modified Gaussian to the artefact and subtracting from each histogram, masking the effected time bins, and simple fitting through the artefact.
\end{itemize}


\section{Rationale}\label{sec:ratfadrationale}
As has been previously highlighted, the ability to map the concentration of FAD - and other retinal chromophores - across the retina would enhance the ability to diagnose retinal disease where now disease can be detected at the stage of biochemical dysfunction rather than at the typical experience where patients report what is often permanent loss of central vision. For the \textit{in-vivo} experiments presented in this chapter anaesthetised rats were imaged using the SFLIO device described in \cref{chap:fliodevice} whereby controlling the fraction of inspired oxygen, \ce{FiO2} can induce changes in the production, and expression, of retinal FAD larger than those associated with disease - offering the ideal conditions for detecting \textit{in-vivo} retinal FAD and testing the robustness of the SFLIM unmixing algorithm discussed in \cref{chap:sflim}. 

\subsection{FAD Concentrations as a response to oxygenation states}
The mechanism for this lies with the process of aerobic respiration \cref{eq:resp} in that when ATP (Adenosine TriPhosphate) is consumed - FAD as well as other flavo-proteins are produced as a byproduct\cite{berg2007biochemistry} and under oxygen starved - hypoxic - conditions (\SI{<15}{\percent} \ce{FiO2}) the concentration of FAD would be lower than when respirating room-air - normoxic conditions (\SI{21}{\percent}\ce{FiO2}). 
\begin{equation}\label{eq:resp}
    \ce{C6H12O6 + 6O2 -> 6CO2 + 6H2O + ATP}
\end{equation}
After death the concentration of FAD would eventually go to zero when all metabolic processes have ceased and as reported by \citeauthor{martinez2017understanding} in the case of a rats cerebral cortex the FAD leaks into the surrounding tissue forming what was dubbed 'halos', further it was thought that in the case of upon returning to normoxic conditions after a period of hypoxia the metabolic system overproduces FAD - increasing the concentration of FAD above that of standard normoxic conditions. 

\section{Experimental Protocol}\label{sec:expprotocol}
In the imaging experiments - carried out in collaboration with Kenneth J. Smith and Helen Yang of the Institute of Neruroinflammation, UCL - albino rats were anaesthetised using isofluorane and the oxygenation state was controlled using an external nitrogen supply mixed with room-air and fed into a fitted nosecone. As shown in [picture of rat in imaging system] the rat is secured to the imaging platform and its pupils are dilated with Tropicamide (from X to Y) to increase light throughput and a contact lens is fitted as well as Viscotears to prevent the cornea from becoming overly dry -  irritating the rat causing unnecessary distress and incurring potential motion artefacts. To accommodate for the smaller eyeball diameter, of X mm, a 2'' diameter $f = \SI{85}{\mm}$ field-lens was fitted in front of the objective lens of the fundus camera. While this additional does induce an appreciable amount of glare in the centre of the brightfield image due to reflections from the front surface of the objective lens and rear surface of the field lens.This is compensated for by simply masking out the affected area in the co-registration process however if the brightfield images were required to be free of glare it could be removed using a darkfield image recorded of the illumination source illuminated an dark room and 
\section{Temporal Artefact Calibration}\label{sec:hump}

\section{}