\setstretch{2}
\FloatBarrier
\section{Chapter Summary}
In this chapter the process of overcoming artefacts in the long exposure S-FLIO images due to retinal motion is discussed. The Fourier-Merlin technique is used to find the static Affine mapping between the FLIMera and the brightfield detector and using a USAF resolution test target this mapping was found to be within a single pixel (with respect to the FLIMera). Translational motion during the long exposure is detected from high frame-rate brightfield images and the corresponding correction is applied to FLIMera data recorded using the high-readout rate ``RAW'' mode. These techniques are demonstrated together using a sample of \textit{Convallaria majalis} mounted in an eye phantom to produce an artefact free, super-sampled, FLIMera image is produced.
\FloatBarrier
\section{Rational - Retinal Eye Movements and the Induction of Motion Artefacts}\label{sec:motionreg}
The long acquisition time required to record fluorescence decays in the retina introduces the challenge of compensating for natural movements of the eye. Over a single exposure these eye movements would result in a fluorescence lifetime image with severely reduced contrast and features such as blood vessels would be indiscernible. The low SNR in FLI imaging also means eye movements cannot be co-registered using multiple shorter exposure frames. Instead, the sCMOS science camera records high-contrast, high-resolution, brightfield images at video-rate. Movement in these images is detected using conventional image registration algorithms and then applied to FLIMera data operating in raw mode – producing a single unblurred histogram.
\\
Natural eye movements are categorised into three different types: tremors which are high frequency movements (\qty{90}{\hertz}) with amplitude of $<1$~arc-minute; drift which is a continuous random movement that occurs between tremors and saccades but with lower frequency ($<\qty{1}{\hertz}$) but longer duration ($>\qty{0.5}{\second}$) and larger in amplitude ($\approx \qty{0.5}{\degree}$); and saccades which appear as fast jerks lasting \qty{25}{\milli\second} and occurring every \qty{2}{\second} of fixation time with varying amplitude~\cite{martinez2004role}.
In the S-FLIO imaging system the effects of tremors are be ignored since their amplitude is much smaller than the angular resolution ($\approx 10$~arc-minutes) of our system tremors exhibit small enough amplitudes that they would be much smaller than the spatial resolution of the system. Saccade and drift can be compensated for by finding the translation between successive brightfield images recorded using the sCMOS camera at a high frame-rate (\qty{10}{\hertz}) and short exposure time (\qty{30}{\milli\second}) and then applying the opposite translation to  histograms from the FLIMera.
The Affine Transforms describes the mapping between images on the sCMOS and FLIMera.
\begin{equation}\label{eq:mapping}
    I_{FLIMera(x,y} = \mathbf{A}I_{sCMOS}(x',y')
\end{equation}
% \begin{align}\label{eq:mapping}
%     \begin{bmatrix}
%         x\\
%         y\\
%         1\\
%     \end{bmatrix}
%      & = \mathbf{A}
%      \begin{bmatrix}
%          x'\\
%          y'\\
%          1
%      \end{bmatrix}\\
%     \implies I_{FLIMera}(x,y) &= \mathbf{A}I_{sCMOS}(x',y')
% \end{align}
The affine transform is composed of four separable transforms: Translation $\mathbf{T}(\Delta x, \Delta y)$; Magnification/Scaling, $\mathbf{M}(m_{x},m_{y})$; Rotation, $\mathbf{R}(\theta)$;and Shear, $\mathbf{S}(h_{x}, h_{y})$. In the images recorded using the S-FLIO system Shear was not significantly present of either the sCMOS camera or the FLIMera and is not considered in the registration process.
\begin{equation}\label{eq:affine}
    \mathbf{A} = \mathbf{T}\mathbf{R}\mathbf{M}
\end{equation}
\begin{align}\label{eq:TransMatrices}
    T &= \begin{bmatrix}
        1 & 0 & \Delta x\\
        0 & 1 & \Delta y\\
        0 & 0 & 1
    \end{bmatrix}
    &
    R &= \begin{bmatrix}
        \cos\theta & -\sin\theta & 0\\
        \sin\theta & \sin\theta & 0\\
        0 & 0 & 1\\
    \end{bmatrix}
    &
    M &= \begin{bmatrix}
        m_{x} & 0 & 0 \\
        0 & m_{y} & 0 \\
        0 & 0 & 1\\
    \end{bmatrix}
    &
    S &= \begin{bmatrix}
        0 & h_{x} & 0\\
        h_{y} & 0 & 0\\
        0 & 0 & 1\\
    \end{bmatrix}   
\end{align}
The registration process is further separated into two stages: The affine transform, $\hat{\mathbf{A}}$ which defines the mapping between the two detectors for a static scene ; and a time dependent affine transform relating to translational motion in the retina, $\mathbf{A}(\Delta x(t), \Delta y(t))$.
\FloatBarrier
\section{Fourier-Mellin Technique for Registering Still FLIMera and sCMOS images}
Feature-based registration algorithms, such as SIFT~\cite{lowe2004distinctive} and ORB~\cite{rublee2011orb}, are currently used in retinal imaging for collections of images with low noise, similar fields of view, and magnifications~\cite{chanwimaluang2006hybrid,faisan2011scanning}.  The high noise in the FLIMera images and the relatively large magnification difference between the FLIMera and sCMOS detector make these feature-based techniques unsuitable. The phase cross-correlation technique, the Fourier-Mellin algorithm, is used to find the affine mapping between the detectors in two stages~\cite{padfield2011masked}. First the rotation and scale differences are found and corrected, and an intermediate image is formed. This intermediate image is then used to find the translational differences between the images and the affine mapping is constructed to produce the co-registered image.
Phase cross-correlation registration methods identify the translation between the two images, $k(x,y)$, and $l(x','y) = k(x+\Delta x, y + \Delta y)$ by finding the location of the peak of phase correlation, $\Omega$ between the two images and is implemented using an FFT:
\begin{align}
    K(\mu, \nu) &= \mathcal{F}\{k(x,y)\} & L(\mu',\nu') &= \mathcal{F}\{l(x',y')\}\\
    \Omega(\mu,\nu) &= \frac{K L^{*}}{\lvert K L^{*} \rvert} & (\Delta x, \Delta y) &= argmax\big\{{\mathcal{F}^{-1}}\{\Omega(\mu,\nu)\}\big\}
\end{align}
Log-polar transforms (\cref{eq:logpolar}).  applied to FFTs of the images encodes scaling and rotation differences into translational along the $\rho$ and $\xi)$ axes. These shifts are found using the phase cross-correlation method.
\begin{gather}
    I(x,y) \rightarrow I(\rho,\xi)\\
    \begin{aligned}\label{eq:logpolar}
    \rho &= \arctan{\bigg(\frac{y}{x}\bigg)} &\xi &= \ln{(\sqrt{x^{2} + y^{2}})}
    \end{aligned}  
\end{gather}
The parameters $\Delta m$ and $\Delta\theta$ can then be recovered using \cref{eq:magrot} where $U$ and $V$ refer to the number of samples in each axis of the Fourier transformed image, and $U / k$ is the radius over which the log-polar transform is applied. 
\begin{align}\label{eq:magrot}
    \Delta\theta &= \frac{2\pi\Delta\rho}{U} & \Delta m &= \frac{1}{\exp\Delta\xi / \Gamma} & \Gamma = \frac{V}{\ln\lfloor U / k\rfloor}
\end{align}
\FloatBarrier
\subsection{Static Registration with a USAF Test Target}
The static calibration between the two detectors was then found using the above process and fluorescence images recorded of a  fluorescent United States Air Force (USAF) test target using the FLIMera and sCMOS camera in parallel. The periodic “bar” structures of the USAF test target make it an attractive method for assessing the accuracy of the registration technique. Images were recorded using an excitation wavelength of (\qtyrange{460}{467}{\nm}) and a \qty{500}{\nm} long-pass emission filter over an integration time of \qty{10}{\second} to ensure a high SNR. For the images recorded with the FLIMera, the measured histograms were integrated over time to yield an intensity image and the pixel aspect ratio is corrected for by twinning adjacent columns of pixels resulting in an image with resolution \qtyproduct[product-units=single]{192x252}{\pixel}.
Before the Fourier-Merlin algorithm was applied, the images were first filtered using a difference-of-Gaussian bandpass filter and a Hanning window to increase the contrast of edge features - resulting in a more prominent peak in the cross-correlation. The scaling and rotation between the images was found using the FFT and log-polar transform process (\cref{fig:USAFlp}) described above and returned values of $\Delta M = \num[round-mode = figures, round-precision = 3]{4.202255063883409}$ and $\Delta\theta = \qty[round-mode = figures, round-precision = 3]{-0.098775}{\degree}$. 

\begin{figure}
    \centering
    \begin{annotatedFigure}
    {\includegraphics[width = 0.95\textwidth]{figures/sflio-device/LogPolarUSAF.pdf}}
    \annotatedFigureText{0.16,0.86}{white}{0.3}{a)}
    \annotatedFigureText{0.39,0.86}{white}{0.3}{b)}
    \annotatedFigureText{0.63,0.86}{white}{0.3}{c)}
    \annotatedFigureText{0.16,0.38}{white}{0.3}{d)}
    \annotatedFigureText{0.39,0.38}{white}{0.3}{e)}
    \annotatedFigureText{0.63,0.38}{white}{0.3}{f)}
    \end{annotatedFigure}
    \caption{Visualisation of how the scaling and rotation between the FLIMera and sCMOS can be recovered using the Fourier-Merlin: a) and d) The input images are filtered using a difference-of-Gaussians bandpass filter and a Hanning window; b) and e) The images are then Fourier transformed; and finally the Log-Polar transform (\cref{eq:logpolar}) is applied to project the relative magnification and rotation into translations in the horizontal and vertical axes of c) and f) which are recovered using standard phase cross-correlation techniques.}
    \label{fig:USAFlp}
\end{figure}

The intermediate FLIMera image was constructed using $\Delta m$, and $\Delta\theta$ with nearest-neighbour interpolation to fill in missing pixels. An additional translation matrix was needed to shift the image origin (0,0) from the upper-left corner of the image when applying rotations. This was equivalent to a shift of \qty{-6558}{\pixel} and \qty{-3458}{\pixel} in the horizontal and vertical axes, respectively.
The translations between this intermediate FLIMera image and the sCMOS image are found using phase cross-correlation to be $\Delta x = \qty[round-mode = figures, round-precision = 3]{-205.9709}{\pixel}$ and $\Delta y = \qty[round-mode = figures, round-precision = 3]{-298.35}{\pixel}$.
Finally, the magnification, rotation, and translations are used to construct a single Affine transform (see \cref{eq:SFLIOaff}) and then the registered FLIMera image is then produced. The accuracy of this registration is assessed using a unity-normalised line profile taken through an identical region in both images, as shown in \cref{fig:USAFregistered}, and calculating the cross-correlation of both line profiles. For a perfectly registered image the location of the peak would occur at \num{0}, however, for the SFLIO system this location is \num{1} indicating a registration accuracy of \num{1} or fewer pixels on sCMOS detector. This greatly exceeds the minimum accuracy needed for sub-pixel registration on the FLIMera where a registration accuracy of less than $1/\Delta m \approx 4$ is required.


\begin{figure}
    \centering
    \begin{annotatedFigure}
        {\includegraphics[width = 0.85\textwidth]{figures/sflio-device/AirForceTarget-RegisteredImage.pdf}}
        \annotatedFigureText{0.04,0.94}{black}{0.4}{a)}
        \annotatedFigureText{0.04,0.50}{black}{0.4}{c)}
        \annotatedFigureText{0.52,0.94}{black}{0.4}{b)}
    \end{annotatedFigure}
    \caption{Results of static image registration the sCMOS and FLIMera on the S-FLIO device using a USAF test target and the Fourier-Merlin algorithm for recovering scale, rotation, and translation transforms. a) Shows the original, high resolution, image recorded with the sCMOS camera; b) Shows the FLIMera image after being scaled, rotated, and translated using a affine transform and nearest neighbour interpolation. The accuracy of the registration process is assessed using the cross-correlation peak of two line profiles (cyan and magenta) through identical regions of a) and b) plotted in c) and yields an accuracy of 1 pixel}
    \label{fig:USAFregistered}
\end{figure}
\begin{equation}\label{eq:SFLIOaff}
\hat{\mathbf{A}} = \begin{bmatrix}
    \num[round-mode = figures, round-precision = 3]{4.202248819333833} & \num[round-mode = figures, round-precision = 3]{-0.007244469694988512}& \num[round-mode = figures, round-precision = 3]{-6856.35}\\
    \num[round-mode = figures, round-precision = 3]{0.007244469694988512} & \num[round-mode = figures, round-precision = 3]{4.202248819333833}& \num[round-mode = figures, round-precision = 3]{-3663.9709}\\
    \num[round-mode = figures, round-precision = 3]{0} & \num[round-mode = figures, round-precision = 3]{0}& \num[round-mode = figures, round-precision = 3]{1}
\end{bmatrix}    
\end{equation}

\section{Correcting Motion Artefacts in the FLIMera using Brightfield images}
Correcting for eye movement in the fluorescence decays recorded with the FLIMera was achieved by detecting translational shifts between a reference image and successive brightfield or fluorescence intensity images, recorded at \qty{10}{\hertz}, with phase cross-correlation. Applying an affine transform, $\hat{\mathbf{A}}$,  on these translational shifts gives equivalent shifts on the FLIMera which are then applied to histograms spanning the same time period as the sCMOS image. Repeating this for the entire image acquisition period yields a reconstructed fluorescence lifetime image that is high-contrast and is free of motion artefacts.

\subsection{Correlating translations in the Brightfield Image to the FLIMera image}
The shifts in the sCMOS detector are converted to shifts in the FLIMera detector using the attribute of the affine transform (\cref{eq:affreg}) and considering the transform between pairs of images:
\begin{align}\label{eq:affreg}
I_{FLIMera}^{N+\epsilon}(x',y') &= I_{FLIMera}^{N}(x' + \Delta x'_{\epsilon} ,y'+ \Delta y'_{\epsilon}) = \mathbf{A}(\Delta x'_{\epsilon} ,\Delta y'_{\epsilon})I_{FLIMera}^{N}(x',y')\\
I_{sCMOS}^{N+\epsilon}(x,y) &= I_{sCMOS}^{N}(x + \Delta x_{\epsilon} ,y+ \Delta y_{\epsilon}) = \mathbf{A}(\Delta x_{\epsilon} ,\Delta y_{\epsilon})I_{sCMOS}^{N}(x,y)
\end{align}
Where $N$ represents a reference frame and $\epsilon$ is a frame recorded later in the acquisition period that is translated. The $N+\epsilon$ frames in each detector can be described using the existing relationship $I_{FLIMera}^{N+\epsilon}(x',y') = \hat{\mathbf{A}}^{-1}I_{sCMOS}^{N+\epsilon}(x,y)$ and the translational term, $\mathbf{A}(\Delta x_{\epsilon} ,\Delta y_{\epsilon})$, found using image registration is converted to shifts in the FLIMera using \cref{eq:affreg}:

\begin{align}
    I_{FLIMera}^{N+\epsilon} &= \hat{\mathbf{A}}^{-1}I_{sCMOS}^{N+\epsilon}(x,y)= \hat{\mathbf{A}}^{-1}\mathbf{A}(\Delta x_{\epsilon} ,\Delta y_{\epsilon})I_{sCMOS}^{N}(x,y) \nonumber\\
    I_{FLIMera}^{N+\epsilon} &= \hat{\mathbf{A}}^{-1}\mathbf{A}(\Delta x_{\epsilon} ,\Delta y_{\epsilon})\hat{\mathbf{A}}I_{FLIMera}^{N}(x',y')\nonumber\\
    \therefore \mathbf{A}(\Delta x'_{\epsilon} ,\Delta y'_{\epsilon}) &= \hat{\mathbf{A}}^{-1}\mathbf{A}(\Delta x_{\epsilon} ,\Delta y_{\epsilon})\hat{\mathbf{A}}\label{eq:FLIMerashifts}
\end{align}  
This approach can be further generalised for rotation, translation, or shear such that any affine transform, $\mathbf{A}(\Delta x, \Delta y, \theta, \Delta m, \Delta h)$, between two images recorded in the sCMOS imaging arm of the S-FLIO device can be converted into an equivalent transform of a FLIMera image:
\begin{equation}
    \mathbf{A}(\Delta x', \Delta y', \theta', \Delta m', \Delta h') = \hat{\mathbf{A}}^{-1}\mathbf{A}((\Delta x, \Delta y, \theta, \Delta m, \Delta h))\hat{\mathbf{A}}
\end{equation}
\subsection{Demonstration of Motion Correction using \textit{Ex-Vitro} using \textit{Convallaria majalis}}
The motion registration algorithm was demonstrated on a sample of \textit{Convallaria majalis} - referred to as Convallaria herein - mounted in an eye phantom (\cref{fig:ConvallariaEyePhantom}) which could be moved within the “retinal” plane throughout the acquisition period to mimic the image degradation seen in images of the human retina. Over a \qty{2}{\minute} integration period the sample, excited between \qtyrange{460}{467}{\nm}, is held still, translated in X, Y, and then X and Y for periods of \qty{30}{\second} each. Fluorescence intensity images were recorded continuously with a \qty{500}{\nm} long-pass emission filter at a frame rate of \qty{10}{\hertz} and exposure time of \qty{30}{\ms} on the sCMOS detector and FLIM histograms were recorded using the RAW mode of the FLIMera at a rate of \qty{24}{\kilo\hertz}.
\begin{figure}
    \centering
    \includegraphics[width = 0.5\linewidth, trim = {2.25cm, 1cm, 1.5cm, 0.5cm}, clip]{figures/sflio-device/PhantomEyeFigConvallaria.pdf}
    \caption{The natural movements of a fixated human eye was simulated using an eye phantom with a sample of Convallaria positioned at the retinal plane and mounted to a X-Y translation stage. Saccades and drift motions are then mimic by moving the Convallaria sample with an amplitude and velocity commensurate with published values~\cite{martinez2004role}. The phantom eye serves as an analogue to the human eye in that it uses a $f=\qty{24}{\mm}, d = \qty{8}{\mm}$ lens positioned \qty{24}{\mm} from its retinal plane. The cavity in the phantom eye is filled with water to emulate the refractive properties of the vitreous humour.}
    \label{fig:ConvallariaEyePhantom}
\end{figure}
By initially having the laser shutter closed - sample is not illuminated - at the start of the acquisition period before illuminating the sample the FLIMera and sCMOS frames can then be synchronised by finding the peak of the gradient of the Heaviside-like change in photon flux, shown in \cref{fig:flimerascmossync}. For the FLIMera, the low detected photon flux and high frame rate resulted in a noise dominated signal (\cref{fig:flimerascmossync}a) where the frame at which the laser shutter was opened could not be discerned. A Butterworth filter was used to remove the high frequency noise in the signal revealing the step-like change from which the gradient could be calculated. The Butterworth filter was applied to the forward and reverse of the signal to negate the effect of linear phases inherent to some filters. Without this a false location of the peak in the photon flux gradient would be found degrading the accuracy of the synchronisation processing[add citation for forward + backward filtering].

\begin{figure}
    \centering
    \begin{annotatedFigure}{\includegraphics[width =\textwidth]{figures/sflio-device/flimerasync.pdf}}
    \annotatedFigureText{0.015,0.93}{black}{0.4}{a)}
    \annotatedFigureText{0.015,0.46}{black}{0.4}{b)}
    \end{annotatedFigure}
    \caption{The step-like change in the photon flux per frame, $\Phi(n)$, for the the FLIMera (a) and sCMOS detector (b), shown as the black line, is detected by finding the peak of the gradient, $d\Phi(n)/dn$, shown as the red line. For the FLIMera a Butterworth filter was applied to the raw signal, shown as the grey line in (a), to enable the frame when the laser shutter was opened to be resolved.}
    \label{fig:flimerascmossync}
\end{figure}

The fluorescence intensity images are then registered to the first fully illuminated frame in the sequence (\cref{fig:sCMOSreg}b) and summed together to construct a single motion corrected frame from (\cref{fig:sCMOSreg}d). When compared to an image formed from a single exposure, \cref{fig:sCMOSreg}a, with the a shorter, \qty{2}{\second}, exposure time integration time the re-constructed image does appear softer. This is due to each shorter exposure frame having a high proportion of read-noise and shot-noise. The noisier frames also introduce small errors of ($<\qty{2}{\pixel}$) in the registration process which further softens the image and results in the loss in detail of the fine vascular bundles.


\begin{figure}
    \centering
    \begin{annotatedFigure}{\includegraphics[width = \linewidth, trim = {0.4cm, 1.1cm, 0.1cm, 0.3cm}, clip]{figures/sflio-device/sCMOSRegistration.pdf}}
    \annotatedFigureText{0.01,0.76}{white}{0.4}{a)}
    \annotatedFigureText{0.26,0.76}{white}{0.4}{b)}
    \annotatedFigureText{0.51,0.76}{white}{0.4}{c)}
    \annotatedFigureText{0.76,0.76}{white}{0.4}{d)}
    \end{annotatedFigure}
    \caption{Fluorescence intensity images recorded of \textit{Convallaria} mounted in a eye phantom (a-d). The first image, a), represents a single \qty{2}{\second} exposure. For b-d), the images are comprised of frames recorded at \qty{10}{\hertz} with \qty{30}{\ms} integration time with c) and d) being the result of summing multiple frames. The sample is initially held still for \qty{30}{\second} before being translated in the X and Y axes throughout the remainder of a total \qty{2}{\minute} acquisition period to mimic the typical movements the human eye undergoes while fixating on a target. b) shows a typical frame used in the registration process with grey values between 0 and 5. In c) the image is degraded by the movement over the entire acquisition period and in d) this motion is corrected for using phase-cross correlation to detect movement and an affine transform to correct the translational shifts.}
    \label{fig:sCMOSreg}
\end{figure}

The translational shifts in the fluorescence intensity images are then converted to equivalent shifts on the FLIMera using \cref{eq:FLIMerashifts} and applied to FLIMera histograms that cover the same time period as the sCMOS acquisition, $T = 1/\qty{10}{\hertz} = \qty{100}{\ms}$. When applying the shifts to each histogram a pixel mask is applied to eliminate the effect of "hot" pixels and faulty pixels on the fluorescence decay. Any pixels that would be shifted outwith the image are discarded. This yields a single $XYT$ - data cube spanning the entire \qty{2}{\minute} acquisition period that as shown in \cref{fig:FLIMerareg}d is now free of motion artefacts and the vascular bundles can be clearly resolved albeit with a lower resolution when compared to the fluorescence intensity images shown in \cref{fig:sCMOSreg}. Further, in these intensity images, motion of the sample causes features that would ordinarily be imaged onto a masked faulty pixel (See \cref{fig:FLIMerareg}b) in a still sample are now often imaged on to a functional pixel.  Typically, this would result in an uneven image since every pixel in the final image would not be masked an equal number of times. 
This is corrected for by multiplying each pixel grey value by the factor, $\Gamma$, which weights the total number of frames, $N$, in a sequence with the number of times that pixel is masked, $M$ to yield the high quality image shown in \cref{fig:FLIMerareg}d.

\begin{equation}
    \Gamma(x,y) = \frac{N}{N-M(x,y)}
\end{equation}


\begin{figure}
    \centering
    \begin{annotatedFigure}{\includegraphics[width = \linewidth, trim = {0.5cm, 1cm, 0.2cm, 0},clip]{figures/sflio-device/FLIMeraRegistration.pdf}}
    \annotatedFigureText{0.015,0.77}{white}{0.4}{a)}
    \annotatedFigureText{0.345,0.77}{white}{0.4}{b)}
    \annotatedFigureText{0.675,0.77}{white}{0.4}{c)}
    \end{annotatedFigure}
    \caption{Demonstration of motion registration of time resolved fluorescence decays recorded with the FLIMera of a \textit{Convallaria} slide mounted in an eye phantom under going motion throughout a \qty{2}{\minute} acquisition period. Images are formed by integrating the histogram over time to mimic a fluorescence intensity image where: a) represents the entire \qty{2}{minute} acquisition period; b) is the initial \qty{30}{\second} where the sample is held still; and c) is the same \qty{2}{minute} period but with the motion corrected for using the movement detected in the fluorescence intensity images (see \cref{fig:sCMOSreg}).}
    \label{fig:FLIMerareg}
\end{figure}

