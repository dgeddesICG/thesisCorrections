\setstretch{2}
\FloatBarrier
\section{Chapter Summary}
Over the $\approx\qty{60}{\second}$ integration time required to record SFLIO images, natural movement of the eye will induce artefacts in images and degrade accuracy of recovered fluorophore concentrations. In this chapter the process of overcoming artefacts in long exposure SFLIO images induced by retinal motion is discussed. The Fourier-Mellin technique is used to find the affine transform between static scenes image with the FLIMera and the sCMOS detector. The accuracy of the registration technique was assessed using a USAF test target and found to be within a single pixel on the FLIMera. Using normalised phase cross-correlation, translational motion during the long exposure is detected between a reference brightfield image and successive brightfield images recored at a high frame-frate. A corresponding correction is applied to FLIMera data recorded using the high-readout rate ``RAW'' mode to remove artefacts of eye motion. These techniques are demonstrated together using a sample of \textit{Convallaria majalis} - to mimic the dimension of common retinal features - mounted in an eye phantom to produce an artefact free, FLIMera image super-sampled by a factor of two giving the effect of improving spatial resolution of the FLIMera at the cost of higher noise in each FLIMera image.
\FloatBarrier
\section{Rationale - Retinal Eye Movements and the Induction of Motion Artefacts}\label{sec:motionreg}
The long acquisition time $(\approx\qty{60}{\second})$ required to record fluorescence decays in the retina introduces the challenge of compensating for natural movements of the eye. Over a single exposure these eye movements would result in retinal features being smeared over the image resulting in a fluorescence lifetime image with severely reduced contrast and features such as blood vessels would be indiscernible. The low SNR in FLIM imaging, due to feint fluorescence and inefficient detection schemes for TCSPC, also means eye movements cannot be co-registered using multiple shorter exposure frames. Brightfield images are much brighter than fluorescence meaning that sCMOS science camera can used to record high-contrast, high-resolution, images of the retina at video-rate. Movement in these images is detected using conventional image registration algorithms and then applied to photon streams recorded using the ``RAW'' operating mode of the FLIMera – producing a single high contrast FLIM image.
\\
Natural eye movements are categorised into three different types: tremors which are high frequency movements $(\approx\qty{90}{\hertz})$ with amplitude of $<1$~arc-minute; drift which is a continuous random movement that occurs between tremors and saccades but with lower frequency ($<\qty{1}{\hertz}$) but longer duration ($>\qty{0.5}{\second}$) and larger in amplitude ($\approx \qty{0.5}{\degree}$); and saccades which appear as fast jerks lasting \qty{25}{\milli\second} and occurring every \qty{2}{\second} of fixation time with varying amplitude~\cite{martinez2004role}.
For the SFLIO device, the effects of tremors can be ignored since their amplitude is 10-times smaller than the angular resolution ($\approx 10$~arc-minutes) of the imaging system. Saccades and drifts can be compensated by finding the translation between successive brightfield images recorded using the sCMOS camera at a high frame-rate (\qty{10}{\hertz}) and short exposure time (\qty{30}{\milli\second}) and then applying a complimentary translation to frames recorded with the FLIMera. While translations of the retina do exhibit some change in magnification, rotation, and translation the low resolution of the FLIMera and the high frame-rate of the brightfield images mean this motion can be simplified to only translation.
The registration process is separated into two stages: The affine transform, $\hat{\mathbf{A}}$ which defines the mapping between the two images for a static scene; and a time-dependent transform relating to translational motion in the retina, $\mathbf{A}(\Delta x(t), \Delta y(t))$. 
\FloatBarrier

\section{Fourier-Mellin Technique for Registering FLIMera and sCMOS Images of a Static Scene}
Feature-based registration algorithms, such as SIFT~\cite{lowe2004distinctive} and ORB~\cite{rublee2011orb}, are currently used in retinal imaging for collections of images with high SNR, similar fields of view, and magnifications~\cite{chanwimaluang2006hybrid,faisan2011scanning}. The high noise in the FLIMera images and the relatively large magnification difference between the FLIMera and sCMOS detector make these feature-based techniques unsuitable. The relation between points on images, of the same scene, recorded using the FLIMera and the sCMOS detector can be described using an affine transform
\begin{equation}\label{eq:mapping}
    I_{FLIMera}(x,y) = \mathbf{A}I_{sCMOS}(x',y')
\end{equation}
This affine transform can be found using the Fourier-Mellin algorimth which uses phase cross-correlation to find the transformation between a pair of images in two stages~\cite{padfield2011masked}. The affine transform is composed of four separable transforms: Translation $\mathbf{T}(\Delta x, \Delta y)$; Magnification/Scaling, $\mathbf{M}(m_{x},m_{y})$; Rotation, $\mathbf{R}(\theta)$;and Shear, $\mathbf{S}(h_{x}, h_{y})$.
\begin{align}\label{eq:TransMatrices}
    T &= \begin{bmatrix}
        1 & 0 & \Delta x\\
        0 & 1 & \Delta y\\
        0 & 0 & 1
    \end{bmatrix}
    &
    R &= \begin{bmatrix}
        \cos\theta & -\sin\theta & 0\\
        \sin\theta & \sin\theta & 0\\
        0 & 0 & 1\\
    \end{bmatrix}
    &
    M &= \begin{bmatrix}
        m_{x} & 0 & 0 \\
        0 & m_{y} & 0 \\
        0 & 0 & 1\\
    \end{bmatrix}
    &
    S &= \begin{bmatrix}
        0 & h_{x} & 0\\
        h_{y} & 0 & 0\\
        0 & 0 & 1\\
    \end{bmatrix}   
\end{align}
And these individual components can be described using a single affine transform

Shear presents in images as features are displaced along a single axis e.g. right-angles in object space are not perpendicular in the image. In the images recorded using the SFLIO device, Shear was not significantly present in either the sCMOS camera or the FLIMera and is not considered in the registration process.
\begin{equation}\label{eq:affine}
    \mathbf{A} = \mathbf{T}\mathbf{R}\mathbf{M}
\end{equation}

First, the rotation and scale differences are found and corrected and applied to the FLIMera image to form an intermediate image. This intermediate image is then used to find the translation between the sCMOS and FLIMera such that an affine mapping is constructed to produce the final co-registered image. Phase cross-correlation registration methods identify the translation between the two images, $k(x,y)$, and $l(x','y) = k(x+\Delta x, y + \Delta y)$ by finding the location of the peak of phase correlation. First, the Fourier spectra,$K$ and $L$, of the images $k$, and $l$, respectively. 
\begin{align}
    K(\mu, \nu) &= \mathcal{F}\{k(x,y)\} & L(\mu',\nu') &= \mathcal{F}\{l(x',y')\}
\end{align}
And spatial coordinates $(x,y)$ and $(x',y')$ are transformed into spatial frequency components $(\mu, \nu)$ and $(\mu',\nu')$.
And then the phase cross-correlation, $\Omega(\mu,\nu)$, is calculated producing another image in frequency space.
\begin{equation}
    \Omega(\mu,\nu) = \frac{K L^{*}}{\lvert K L^{*} \rvert}
\end{equation}
The translations $\Delta x$ and $\Delta y$ by a performing an inverse FFT on $\Omega(\mu,\nu)$ and finding the point with the highest pixel grey-value.
\begin{equation}
    (\Delta x, \Delta y) = argmax\big\{{\mathcal{F}^{-1}}\{\Omega(\mu,\nu)\}\big\}
\end{equation}
To find the magnification and rotation between images, a log-polar transform is applied to the Fourier spectra, $K$ and $L$, such that $I(\mu,\nu) \rightarrow I(\rho,\xi)$
\begin{align}\label{eq:logpolar}
    \rho &= \arctan{\bigg(\frac{\nu}{\mu}\bigg)} &\xi &= \ln{(\sqrt{\mu^{2} + \nu^{2}})}
\end{align}
Where coordinates $(\mu,\nu)$ are transformed to $(\rho, \xi)$ using~\cref{eq:logpolar} and now changes in magnification between the original images now present as translations, $\Delta\rho, \Delta\xi$, between the log-polar transformed Fourier spectra, $K$ and $L$.
The phase cross-correlation process previously descirbed is used to yield,$\Delta\rho, \Delta\xi$, and the original magnification, $M$, and rotation $\theta$, are then recovered using:
\begin{align}\label{eq:magrot}
    \theta &= \frac{2\pi\Delta\rho}{U} & M &= \frac{1}{\exp\Delta\xi / \Gamma} & \Gamma = \frac{V}{\ln\lfloor U / k\rfloor}
\end{align}
Where in \cref{eq:magrot}, $U$ and $V$, are the sample rate of the Fourier spectra, and $U/k$ is the radius that the log-polar transform is applied over.
\FloatBarrier
\subsection{Finding the Affine Transform Between the FLIMera and sCMOS Images with a Static USAF Test Target}
The static calibration between images recorded with the FLIMera and sCMOS detector was then found using the above process and fluorescence images recorded of a fluorescent United States Air Force (USAF) test target. The periodic “bar” structures of the USAF test target makes it difficult to register and thus an attractive method for assessing the accuracy of the registration technique. Images were recorded with excitation wavelengths of \qtyrange{460}{467}{\nm} and a \qty{500}{\nm} long-pass emission filter over an integration time of \qty{10}{\second} to ensure a high SNR. For the images recorded with the FLIMera, the measured histograms were integrated over time to yield an intensity image and the pixel aspect ratio is corrected for by twinning adjacent columns of pixels resulting in an image with resolution \qtyproduct[product-units=single]{192x252}{\pixel}.
Before the Fourier-Mellin algorithm was applied, the images were first filtered using a difference-of-Gaussian bandpass filter and a Hanning window to improve the accuracy of the registration by increasing the contrast of edge features - giving more prominent peaks in the cross-correlation. The scaling and rotation between the images was found using the FFT and log-polar transform process (\cref{fig:USAFlp}) described above and yielded values of $\Delta M = \num[round-mode = figures, round-precision = 3]{4.202255063883409}$ and $\Delta\theta = \qty[round-mode = figures, round-precision = 3]{-0.098775}{\degree}$. 

\begin{figure}
    \centering
    \begin{annotatedFigure}
    {\includegraphics[width = 0.95\textwidth]{figures/sflio-device/LogPolarUSAF.pdf}}
    \annotatedFigureText{0.16,0.86}{white}{0.3}{a)}
    \annotatedFigureText{0.39,0.86}{white}{0.3}{b)}
    \annotatedFigureText{0.63,0.86}{white}{0.3}{c)}
    \annotatedFigureText{0.16,0.38}{white}{0.3}{d)}
    \annotatedFigureText{0.39,0.38}{white}{0.3}{e)}
    \annotatedFigureText{0.63,0.38}{white}{0.3}{f)}
    \end{annotatedFigure}
    \caption{Visualisation of how the scaling and rotation between images recorded with the FLIMera and sCMOS can be recovered using the Fourier-Merlin algorithm: a) and d) The input images are filtered using a difference-of-Gaussians bandpass filter and a Hanning window; b) and e) The images are then Fourier transformed; and finally the Log-Polar transform (\cref{eq:logpolar}) is applied to project the relative magnification and rotation into translations in the horizontal and vertical axes of c) and f) which are recovered using standard phase cross-correlation techniques.}
    \label{fig:USAFlp}
\end{figure}

The intermediate FLIMera image was constructed using $M$, and $\theta$ with nearest-neighbour interpolation to fill in missing pixels. An additional translation matrix was needed to shift the image origin (0,0) from the upper-left corner of the image when applying rotations. This was equivalent to a shift of \qty{-6558}{\pixel} and \qty{-3458}{\pixel} in the horizontal and vertical axes, respectively.
The translations between this intermediate FLIMera image and the sCMOS image are found using phase cross-correlation to be $\Delta x = \qty[round-mode = figures, round-precision = 3]{-205.9709}{\pixel}$ and $\Delta y = \qty[round-mode = figures, round-precision = 3]{-298.35}{\pixel}$.
Finally, the magnification, rotation, and translations are used to construct a single Affine transform (see \cref{eq:SFLIOaff}) 
\begin{equation}\label{eq:SFLIOaff}
\hat{\mathbf{A}} = \begin{bmatrix}
    \num[round-mode = figures, round-precision = 3]{4.202248819333833} & \num[round-mode = figures, round-precision = 3]{-0.007244469694988512}& \num[round-mode = figures, round-precision = 3]{-6856.35}\\
    \num[round-mode = figures, round-precision = 3]{0.007244469694988512} & \num[round-mode = figures, round-precision = 3]{4.202248819333833}& \num[round-mode = figures, round-precision = 3]{-3663.9709}\\
    \num[round-mode = figures, round-precision = 3]{0} & \num[round-mode = figures, round-precision = 3]{0}& \num[round-mode = figures, round-precision = 3]{1}
\end{bmatrix}    
\end{equation}
and then the registered FLIMera image is then produced. The accuracy of this registration is assessed using a unity-normalised line profile taken through an identical region in both images, as shown in~\cref{fig:USAFregistered}, and calculating the cross-correlation of both line profiles. 

\begin{figure}[htp]
    \centering
    \begin{annotatedFigure}
        {\includegraphics[width = 0.85\textwidth]{figures/sflio-device/AirForceTarget-RegisteredImage.pdf}}
        \annotatedFigureText{0.04,0.94}{black}{0.4}{a)}
        \annotatedFigureText{0.04,0.50}{black}{0.4}{c)}
        \annotatedFigureText{0.52,0.94}{black}{0.4}{b)}
    \end{annotatedFigure}
    \caption{Results of static image registration the sCMOS and FLIMera on the SFLIO device using a USAF test target and the Fourier-Merlin algorithm for recovering scale, rotation, and translation transforms. a) Shows the original, high resolution, image recorded with the sCMOS camera; b) Shows the FLIMera image after being scaled, rotated, and translated using a affine transform and nearest neighbour interpolation. The accuracy of the registration process is assessed using the cross-correlation peak of two line profiles (cyan and magenta) through identical regions of a) and b) plotted in c) and yields an accuracy of 1 pixel}
    \label{fig:USAFregistered}
\end{figure}
The accuracy of the registration was demonstrated by transforming a FLIMera image using the affine transform \cref{eq:SFLIOaff} and finding the offset between line-profiles taken over the same region in both images using phase cross-correlation. The offset is 1 pixel with respect to the sCMOS images implying that the two images are registered with sub-pixel accuracy with respect to the FLIMera images. The spatial resolution of both detectors influences the required accuracy of the co-registration between FLIM images to sCMOS images. For a relative magnification, $M$, between the FLIMera and sCMOS detector, and ratio between the spatial resolution of the FLIMera and sCMOS imaging arms $\epsilon$, (\cref{eq:resratio}) presents two cases. 
\begin{equation}\label{eq:resratio}
    \epsilon = \frac{\Delta x_{FLIMera}}{\Delta x_{sCMOS}}
\end{equation}
For $\epsilon \leq M$ the registration between the two cameras must be accurate to sub-pixel resolution, and for $\epsilon > M$ the accuracy of the registration can be more than a pixel. 
For the SFLIO device, $\epsilon = \num{4.28}$ and relative magnification of $M = \num{4.20}$ then the for $\epsilon > M$ the accuracy of the registration between FLIMera and sCMOS images must be $\approx \qty{1}{\pixel}$. While the Fourier-Mellin technique is suitable for registering the two images and can be used for correcting motion artefacts due to motion of the retina. The accuracy of the registration could be improved by performing successive iterations of the Fourier-Mellin technique on the generated registered FLIMera image.
\FloatBarrier
\section{Correcting Motion Artefacts in the FLIMera Using Brightfield Images}
Fast and high-amplitude motion of the retina occurring over the $\approx\qty{60}{\second}$ acquisition time will result in features in the retinal image to be blurred and smeared - lowering image quality and reducing accuracy when quantifying retinal fluorophores. For recovering fluorophore concentrations using the SFLIM-unmixing technique described in~\cref{chap:tensSFLIM}, motion artefacts can result in spectral contamination from neighbouring retinal features such as veins and arteries. With phase cross-correlation, This was achieved by detecting translational shifts between a reference image and successive brightfield or fluorescence intensity images, recorded at \qty{10}{\hertz}. Applying an affine transform, $\hat{\mathbf{A}}$, on these translational shifts gives equivalent shifts on the FLIMera frames spanning the same time period as the sCMOS image. Repeating this for the entire image acquisition period yields a reconstructed fluorescence lifetime image that is high-contrast and is free of motion artefacts.
\FloatBarrier
\subsection{Converting Translations in the Brightfield Image to the FLIMera Image}
Translational shifts in the sCMOS image are converted to corresponding translations in the FLIMera image by considering the transform between pairs of images $I_{FLIMera}^{N+\epsilon}(x',y')$ and $I_{sCMOS}^{N+\epsilon}(x,y)$:
\begin{align}\label{eq:affreg}
I_{FLIMera}^{N+\epsilon}(x',y') &= I_{FLIMera}^{N}(x' + \Delta x'_{\epsilon} ,y'+ \Delta y'_{\epsilon}) = \mathbf{A}(\Delta x'_{\epsilon} ,\Delta y'_{\epsilon})I_{FLIMera}^{N}(x',y')\\
I_{sCMOS}^{N+\epsilon}(x,y) &= I_{sCMOS}^{N}(x + \Delta x_{\epsilon} ,y+ \Delta y_{\epsilon}) = \mathbf{A}(\Delta x_{\epsilon} ,\Delta y_{\epsilon})I_{sCMOS}^{N}(x,y)
\end{align}
where $N$ represents a reference frame and $N+\epsilon$ is a frame recorded later in the acquisition period that is translated. The $N+\epsilon$ frames from each detector can be described using the existing relationship $I_{FLIMera}^{N+\epsilon}(x',y') = \hat{\mathbf{A}}^{-1}I_{sCMOS}^{N+\epsilon}(x,y)$ and the translational term, $\mathbf{A}(\Delta x_{\epsilon} ,\Delta y_{\epsilon})$, found using image registration is converted to shifts in the FLIMera using:

\begin{align}
    I_{FLIMera}^{N+\epsilon} &= \hat{\mathbf{A}}^{-1}I_{sCMOS}^{N+\epsilon}(x,y)= \hat{\mathbf{A}}^{-1}\mathbf{A}(\Delta x_{\epsilon} ,\Delta y_{\epsilon})I_{sCMOS}^{N}(x,y) \nonumber\\
    I_{FLIMera}^{N+\epsilon} &= \hat{\mathbf{A}}^{-1}\mathbf{A}(\Delta x_{\epsilon} ,\Delta y_{\epsilon})\hat{\mathbf{A}}I_{FLIMera}^{N}(x',y')\nonumber\\
    \therefore \mathbf{A}(\Delta x'_{\epsilon} ,\Delta y'_{\epsilon}) &= \hat{\mathbf{A}}^{-1}\mathbf{A}(\Delta x_{\epsilon} ,\Delta y_{\epsilon})\hat{\mathbf{A}}\label{eq:FLIMerashifts}
\end{align}  
Although not explored in this project, this approach can be further generalised for time-varying rotation, translation, or shear in image sequences. Thus the affine transform, $\mathbf{A}(\Delta x, \Delta y, \theta, \Delta m, \Delta h)$, between successive images recorded in the sCMOS imaging arm of the SFLIO device can be converted into an equivalent transform of a FLIMera image using \cref{eq:motionaffine}. 
\begin{equation}\label{eq:motionaffine}
    \mathbf{A}(\Delta x', \Delta y', \theta', \Delta m', \Delta h') = \hat{\mathbf{A}}^{-1}\mathbf{A}((\Delta x, \Delta y, \theta, \Delta m, \Delta h))\hat{\mathbf{A}}
\end{equation}
\subsection{Demonstration of Motion Correction using \textit{Ex-Vitro} using \textit{convallaria majalis}}\label{sec:motionreg-convallaria}
The motion-registration algorithm was demonstrated on a sample of \textit{convallaria majalis} - referred to as convallaria herein - mounted in an eye phantom (\cref{fig:ConvallariaEyePhantom}) which could be moved within the “retinal” plane throughout the acquisition period to mimic the image degradation seen in images of the human retina. Fluorescence in the sample is excited using the pusled white-light supercontinuum laser source (NKT SuperK Extreme) filtered between \qtyrange{460}{467}{\nm} with an AOTF, and the emission is filtered with a \qty{500}{\nm} long-pass filter to simultaneously image onto the FLIMera and sCMOS detector. Over a \qty{2}{\minute} integration period the convallaria sample is held still, translated in X, Y, and finally X and Y for periods of \qty{30}{\second} each. Fluorescence intensity images were recorded continuously at a frame rate of \qty{10}{\hertz} and exposure time of \qty{30}{\ms} on the sCMOS detector. Photon streams were recorded using the RAW mode of the FLIMera at a rate of \qty{24}{\kilo\hertz}.
\begin{figure}
    \centering
    \includegraphics[width = 0.5\linewidth, trim = {2.25cm, 1cm, 1.5cm, 0.5cm}, clip]{figures/sflio-device/PhantomEyeFigConvallaria.pdf}
    \caption{The natural movements of a fixated human eye was simulated using an eye phantom with a sample of Convallaria positioned at the retinal plane and mounted to a X-Y translation stage. Saccades and drift motions are then mimic by moving the Convallaria sample with an amplitude and velocity commensurate with published values~\cite{martinez2004role}. The phantom eye serves as an analogue to the human eye in that it uses a $f=\qty{24}{\mm}, d = \qty{8}{\mm}$ lens positioned \qty{24}{\mm} from its retinal plane. The cavity in the phantom eye is filled with water to emulate the refractive properties of the vitreous humour.}
    \label{fig:ConvallariaEyePhantom}
\end{figure}
By initially having the laser shutter closed - sample is not illuminated - at the start of the acquisition period before illuminating the sample the FLIMera and sCMOS frames can then be synchronised by finding the peak of the gradient of the Heaviside-like change in photon flux, shown in~\cref{fig:flimerascmossync}. For the FLIMera, the low detected photon flux and high frame rate resulted in a noise dominated signal (\cref{fig:flimerascmossync}a) where the frame at which the laser shutter was opened could not be discerned. A Butterworth filter was used to attenuate the high frequency noise in the signal, revealing the step-like change in photon flux when the shutter is opened. The Butterworth filter was applied to the forward and reverse of the signal to negate the effect of linear phases inherent to some filters. Without this a false location of the peak in the photon flux gradient would be found degrading the accuracy of the synchronisation processing~\cite{robertson2003design}.

\begin{figure}
    \centering
    \begin{annotatedFigure}{\includegraphics[width =\textwidth]{figures/sflio-device/flimerasync.pdf}}
    \annotatedFigureText{0.015,0.93}{black}{0.4}{a)}
    \annotatedFigureText{0.015,0.46}{black}{0.4}{b)}
    \end{annotatedFigure}
    \caption{The step-like change in the photon flux per frame, $\Phi(n)$, for the the FLIMera (a) and sCMOS detector (b), shown as the black line, is detected by finding the peak of the gradient, $d\Phi(n)/dn$, shown as the red line. For the FLIMera a Butterworth filter was applied to the raw signal, shown as the grey line in (a), to enable the frame when the laser shutter was opened to be resolved.}
    \label{fig:flimerascmossync}
\end{figure}

The fluorescence intensity images are then registered to the first fully illuminated frame in the sequence (\cref{fig:sCMOSreg}b) and summed together to construct a single motion corrected frame from (\cref{fig:sCMOSreg}d). When compared to an image formed from a single exposure, \cref{fig:sCMOSreg}a, with the a shorter, \qty{2}{\second}, integration time the re-constructed image does appear softer. This is due to each shorter exposure frame having a high proportion of read-noise and shot-noise. The noisier frames also introduce small errors of $<\qty{2}{\pixel}$, with respect to the sCMOs image, in the registration process which further softens the image and results in the loss in detail of the fine vascular bundles.
\begin{figure}
    \centering
    \begin{annotatedFigure}{\includegraphics[width = \linewidth, trim = {0.4cm, 1.1cm, 0.1cm, 0.3cm}, clip]{figures/sflio-device/sCMOSRegistration.pdf}}
    \annotatedFigureText{0.01,0.76}{white}{0.4}{a)}
    \annotatedFigureText{0.26,0.76}{white}{0.4}{b)}
    \annotatedFigureText{0.51,0.76}{white}{0.4}{c)}
    \annotatedFigureText{0.76,0.76}{white}{0.4}{d)}
    \end{annotatedFigure}
    \caption{Fluorescence intensity images recorded of convallaria mounted in a eye phantom (a-d). The first image, a), represents a single \qty{2}{\second} exposure. For b-d), the images are comprised of frames recorded at \qty{10}{\hertz} with \qty{30}{\ms} integration time with c) and d) being the result of summing multiple frames. The sample is initially held still for \qty{30}{\second} before being translated in the X and Y axes throughout the remainder of a total \qty{2}{\minute} acquisition period to mimic the typical movements the human eye undergoes while fixating on a target. b) shows a typical frame used in the registration process with grey values between 0 and 5. In c) the image is degraded by the movement over the entire acquisition period and in d) this motion is corrected for using phase-cross correlation to detect movement and an affine transform to correct the translational shifts.}
    \label{fig:sCMOSreg}
\end{figure}

The translational shifts in the fluorescence intensity images are then converted to equivalent shifts on the FLIMera, using \cref{eq:FLIMerashifts}, and applied to FLIMera histograms that cover the same time period as the sCMOS acquisition - $T = 1/\qty{10}{\hertz} = \qty{100}{\ms}$. When applying the shifts to each histogram a pixel mask is applied to eliminate the effect of screamers and dead pixels on the fluorescence decay. Any pixels that would be shifted outwith the image are discarded. This yields a single $XYT$ - data cube spanning the entire \qty{2}{\minute} acquisition period that as shown in~\cref{fig:FLIMerareg}d is now free of motion artefacts and the vascular bundles can be clearly resolved albeit with a lower resolution when compared to the fluorescence intensity images shown in~\cref{fig:sCMOSreg}. 
\begin{figure}
    \centering
    \begin{annotatedFigure}{\includegraphics[width = \linewidth, trim = {0.5cm, 1cm, 0.2cm, 0},clip]{figures/sflio-device/FLIMeraRegistration.pdf}}
    \annotatedFigureText{0.015,0.77}{white}{0.4}{a)}
    \annotatedFigureText{0.345,0.77}{white}{0.4}{b)}
    \annotatedFigureText{0.675,0.77}{white}{0.4}{c)}
    \end{annotatedFigure}
    \caption{Demonstration of motion registration of time resolved fluorescence decays recorded with the FLIMera of a convallaria slide mounted in an eye phantom under going motion throughout a \qty{2}{\minute} acquisition period. Images are formed by integrating the histogram over time to mimic a fluorescence intensity image where: a) represents the entire \qty{2}{minute} acquisition period; b) is the initial \qty{30}{\second} where the sample is held still; and c) is the same \qty{2}{minute} period but with the motion corrected for using the movement detected in the fluorescence intensity images (see \cref{fig:sCMOSreg}).}
    \label{fig:FLIMerareg}
\end{figure}
Further, in these intensity images, motion of the sample causes features that in a static scene would be imaged onto a masked faulty pixel (See~\cref{fig:FLIMerareg}b) are now imaged on to a functional pixel.  Typically, this would result in an uneven image since every pixel in the final image would not be masked an equal number of times. 
This is corrected for by multiplying each pixel grey value by the factor, $\Gamma$,
\begin{equation}
    \Gamma(x,y) = \frac{N}{N-M(x,y)}
\end{equation}
which weights the total number of frames, $N$, in a sequence with the number of times that pixel is masked, $M$ to yield the high quality image shown in~\cref{fig:FLIMerareg}d.

\FloatBarrier
\section{Conclusion}
The long acquisition time required for \textit{in-vivo} FLIM imaging of the retina requires natural movements of the eye to be detected and compensated to prevent retinal features appearing blurred and smeared and reducing accuracy in recovering retinal fluorophore concentrations. In this chapter, algorithms were descried and demonstrated to enable motion of the retina to compensated in photon streams recorded with the FLIMera by recording synchronised brightfield images with the sCMOS detector. The frame-rate of the sCMOS detectors was chosen such that dominate modes of retinal motion can be detected as translations. 
First, the transform between FLIMera and sCMOS images recorded of a static USAF test target is found with single pixel accuracy using the Fourier-Mellin algorithm. Next, translational motion detected in the sCMOS images is converted to equivalent motion in the FLIMera images. 
The motion compensation algorithm is then demonstrated on a slide of \textit{convallaria majalis} mounted in an eye phantom and translated throughout a \qty{2}{\minute} acquisition period to mimic retinal motion. Due to the irregular pixel architecture and sufficient performance of the registration algorithm, the FLIMera image was super-sampled by a factor of two revealing fine detail in the convallaria. For \textit{in-vivo} imaging of retinas, this approach can account for motion of the eye such that key retinal features such as the vasculature and drusen should be resolvable. Efforts in discriminating, and quantifying, retinal fluorophores would not then be hindered by retinal motion.