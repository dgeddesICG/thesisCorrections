\setstretch{2}
\FloatBarrier
\section{Chapter Summary}
Measuring the concentration of retinal FAD would enable the detection of retinal disease at the initial stages where normal metabolic processes are disrupted. The three most fluorescent fluorophores in the retina AGE, FAD, and A2E, all have high overlap in fluorescence emission spectra. This high overlap necessitates a robust unmixing method to reliably quantify fluorophore concentrations.
In this chapter three methods of unmixing FAD concentrations from retinal autofluorescence are simulated using additive mixtures of the retina fluorophores, AGE,FAD, and A2E. Each method is compared in terms of the error in recovered fluorophore concentration for varying fluorophore concentration and with decreasing photon flux. A fitting-based method is described which uses a 6-exponential fit of the time-resolved fluorescence to recover fluorophore concentration. A second, blind method ``Phasor S-FLIM'' is also simulated for retinal fluorophores, which uses spectrally resolved FLIM measurements. Due to poor sensitivity at photon fluxes experienced when imaging the retina these methods were deemed unsuitable. Finally, the best-performing method, spectral unmixing using matrix inversions, is tuned by finding the optimal set of detection bands to serve as a benchmark for the SFLIM unmixing technique developed in the next chapter (\cref{chap:tensSFLIM}).

\FloatBarrier
\section{Rationale}
Quantitative measurements of retinal biomarkers, such as FAD, would enhance detection of retinal disease and improve patient outcomes. FAD is a metabolite produced from the consumption of oxygen and present throughout the retina. At the formation of retinal disease the consumption of oxygen in retinal tissue changes causing an abnormal distribution of FAD in retina. The change in FAD could then be detected through a corresponding change in the emission spectra and fluorescence lifetime of the retinal autofluorescence signal. In \citeauthor{schweitzer2007towards}\cite{schweitzer2007towards}, fluorescence lifetimes of the major retinal fluorophores were measured in an \textit{ex-vivo} porcine eye using a bi-exponential decay model. They demonstrated that retinal FAD could be detected but did not map its concentration across the retina or study the changes in FAD concentration with the progression of retinal disease. With routine monitoring longitudinal changes in FAD concentration could be connected to the development of retinal disease at the point of biochemical dysfunction. This could allow retinal disease to be detected before, often permanent, physical damage to central vision is reported by patients. Additionally, using FAD as a biosensor for metabolic health could also address issues with current methods of measuring metabolic function in the retina. In retinal oximetry, measurements of vascular retinal oxygenation exhibit uncertainties in oxygenation larger than the variations typically associated with disease e.g. Glaucoma~\cite{shughoury2020retinal,mordant2014oxygen}. 

Techniques for discriminating fluorophores using their distinct spectral signatures or their intrinsic fluorescence lifetime do exist. However, existing fluorophore recovery methods are plagued by high errors due a number of factors : high overlap of fluorescence emission spectra; high photon counts required for fitting complex decay models; and fluorescent clutter in the eye corrupting the already weak return signal causes prohibitively high error in the recovered fluorophore concentration. This motivates the need for a new efficient and robust unmixing method that addresses the shortcomings of phasor analysis, spectral unmixing using matrix inversions, and unmixing directly using fluorescence lifetime.

\FloatBarrier
\section{Recovering Retinal Fluorophore Concentration Using a Multi-Exponential Fitting Model}\label{sec:lifeunmix}
The concentration of retinal fluorophores could be quantified using only measurements of the fluorescence lifetime in a single, broad, spectral band. Compared to sequentially scanning multiple spectral bands, a single-band fluorescence lifetime-base method would reduce the influence of shot-noise in dim spectral bands as well as reducing image acquisition times. The process of fitting complex fluorescence decays does however require large photon fluxes. To model combinations of the 3 retinal biomarkers of interest – FAD, AGE, and A2E/Lipofuscin a 6-exponential fit is needed to capture the bi-exponential decays of all 3 fluorophores.

\begin{table}
    \centering
    \begin{tabular}{|c|c|c|c|c|c|c|}
    \hline
    Fluorophore & $\alpha_{1}\,(\unit{\percent})$ & $\tau_{1}\,(\unit{\nano\second})$ & $\alpha_{2}\,(\unit{\percent})$ & $\tau_{2}\,(\unit{\nano\second})$ & $\langle \tau \rangle_{A}\,(\unit{\nano\second})$  & $\langle \tau \rangle_{I}\,(\unit{\nano\second})$\\
    \hline
    AGE  & 62 & 0.865 & 38 & 4.17 & 3.33 & 2.12\\
    FAD & 18 & 0.33 & 82 & 2.81 & 2.75 & 2.36\\
    A2E & 48 & 0.39 & 52 & 2.24 &  1.98 & 1.35 \\
    \hline
    \end{tabular}
    \caption{Fluorescence lifetime parameters of common retinal fluorophores FAD, AGE, and A2E / Lipofuscin resolved as bi-exponential decays as well as their associated intensity weighted, and amplitude weighted average lifetimes lifetimes, $\langle \tau \rangle_{I}$ and $\langle \tau \rangle_{A}$, respectively. Transcribed from Fig. 9 of~\citeauthor{schweitzer2007towards}\cite{schweitzer2007towards}}
    \label{tab:retlifetimes}
\end{table}

An additive mixture of FAD, AGE, and A2E with relative concentrations $c_{FAD}$, $c_{AGE}$, and $c_{A2E}$, respectively gives the following 6-exponential decay model such that $c_{FAD} + c_{AGE} + c_{A2E} = 1$:

\begin{multline}
     I(t) =  c_{F}\alpha_{F,1}\exp(-t / \tau_{F,1}) 
       + c_{F}\alpha_{F,2}\exp(-t / \tau_{F,2})\\    
      + c_{G}\alpha_{G,1}\exp(-t / \tau_{G,1}) + 
      c_{G}\alpha_{G,2}\exp(-t / \tau_{G,2})\\
      + c_{A}\alpha_{A,1}\exp(-t / \tau_{A,1}) +
      c_{A}\alpha_{A,2}\exp(-t / \tau_{A,2})
      \label{eq:retmodel}
\end{multline}
where $\alpha_{F,1}$ refers to the amplitude of the first lifetime for FAD, and $c_{F}$ is the relative abundance of FAD. These concentrations can then be recovered from a conventional 6-exponential fit: 
\begin{equation}
     I_{FIT}(t) = \delta + \sum^{n = 6}_{n \geq 1} \alpha_{n}\exp(-t / \tau_{n})
     \label{eq:6expfit}
\end{equation}
where $\delta$ represents the DC noise in the measurement. Subsequently, the concentration of each fluorophore can be recovered: 
\begin{align}
    \bar{c}_{F} &= \frac{1}{2}\Bigg(\frac{\alpha_{1}}{\alpha_{F,1}} + \frac{\alpha_{2}}{\alpha_{F,2}}\Bigg) &     \bar{c}_{G} &= \frac{1}{2}\Bigg(\frac{\alpha_{3}}{\alpha_{G,1}} + \frac{\alpha_{4}}{\alpha_{G,2}}\Bigg)\\
     \bar{c}_{A} &= \frac{1}{2}\Bigg(\frac{\alpha_{5}}{\alpha_{A,1}} + \frac{\alpha_{6}}{\alpha_{A,2}}\Bigg)\label{eq:fitrecovery}
\end{align}
where $\alpha_{n}$ refers to the amplitudes recovered from the fitting procedure and $\alpha_{X,1}$ and $\alpha_{X,2}$ refers to the known amplitudes of the respective fluorophore. The fitting parameters estimated by the the fitting routine had the same ordering as those in \cref{eq:retmodel} and best fit parameters preserve this ordering i.e. parameters are ordered FAD, AGE, A2E.
This technique was initially tested using a fluorescence decay - free from photon shot-noise - simulated using the fitting model in \cref{eq:retmodel} and with fluorophore concentrations of $c_{FAD} = c_{A2E} = 0.4$, $c_{AGE} = 0.2$ and fitted with a Levenberg-Marquardt least-squares fitting routine. The exponential decays were constructed to match the attributes of the FLIMera and supercontinuum source as follows: the interval between each time step was set to match the temporal resolution of the FLIMera ($\delta t = \qty{47}{\pico\second}$); and the time window over which the decay was simulated matched the laser period of the supercontinuum source ($T = \qty{12.8}{\ns}$) - equating to \num{266} time samples. The recovered fluorophore abundances (\cref{tab:retfitexampleresults}) shows a infinitesimal deviations from the ground truth as well as the fitting model showing good agreement with the data. This implies this technique can work in ideal circumstances where magnitude of noise is on the order of precision losses due to a \qty{32}{\bit} floating-point representation.
\\
The sensitivity to fluorophore concentration was then evaluated by repeating this simulation but now the ratios of FAD, AGE, and A2E were varied from \qtyrange{0}{100}{\percent} in steps of \qty{10}{\percent} with the condition that $c_{FAD} + c_{AGE} + c_{A2E} = \qty{100}{\percent}$. The relative error, shown in \cref{eq:recerror}, between the recovered abundance the ground truth can then be used to assess the sensitivity of the technique to changes in fluorophore concentration.

\begin{equation}\label{eq:recerror}
    \mathbf{\epsilon} = \Bigg\lvert\frac{\mathbf{c}_{n} - \bar{\mathbf{c}}_{n}}{\mathbf{c}_{n}}\Bigg\rvert
\end{equation}

The abundance maps shown in \cref{fig:retfittingnoiseless} illustrate that even for noiseless data the error in recovery grows larger than \qty{10}{\percent} for low fluorophore concentrations and for high concentrations of FAD this error is $\ll \qty{1}{\percent}$. This is likely because the optimisation procedure exits once the change in the cost-function, due to changing estimates of fluorophore concentrations, becomes smaller tolerance on the order of 10 times the floor of a 32-bit floating representation.

\begin{figure}
    \centering
    \includegraphics[width=1\linewidth]{figures//sflim//lifetime-unmixing/RetFluorFittingExampleDecay.pdf}
    \caption{Least squares fitting of a mixture of 3 retinal fluorophores FAD, AGE, A2E using a 6 exponential fitting model. The data (red crosses) and line of best fit (solid black line) is plotted (top left) as well as in log space (top right). The residuals, (line of best fit - data) is shown in the bottom plot and are on the order of \num{1e-11} indicating a good quality fit and an appropriate fitting model.}
    \label{fig:retfitexample}
\end{figure}

\begin{table}
    \centering
    \begin{tabular}{|c|c|c|c|}
        \hline
         &  Ground Truth & Recovered  &  Error $(\unit{\percent})$\\
         \hline
         $c_{FAD}$ & 0.4  & $0.4 + \num{3.1e-6}$ & \num{8e-4}\\
         $c_{AGE}$ &  0.2 & $0.2 + \num{1.8e-6}$ & \num{9e-4} \\
         $c_{A2E}$ & 0.4 & $0.4 - \num{6.3e-6}$ & \num{2e-4} \\
         \hline
    \end{tabular}
    \caption{Recovered abundances from a example fluorescence decay modelled with no noise using a 6 exponential fit}
    \label{tab:retfitexampleresults}
\end{table}


\begin{figure}
    \centering
    \includegraphics[width=1\linewidth]{figures/sflim/lifetime-unmixing/RetFittingNoiseless.pdf}
    \caption{Results from recovering the abundance of retinal fluorophores from additive mixtures using their fluorescence lifetime. The fluorescence decays were simulated without noise, and the abundances were recovered (middle row) and compared to the ground truth (top row) with an error corresponding to~\cref{eq:recerror} is plotted (bottom row). Areas shaded white represent where the fluorophore concentration is 0 and so a relative error cannot be ascribed (\cref{eq:recerror}) but from comparing the recovered abundance to the ground truth it can be seen that the relative error is $< 0.01$}
    \label{fig:retfittingnoiseless}
\end{figure}
\FloatBarrier
\subsection{Assessing the Influence of Photon Shot Noise and Fluorophore Concentration Using the Lifetime-based Recovery Method}
If this lifetime based recovery method technique is to be suitable for unmixing retinal fluorophores it needs to be robust to the low-photon fluxes, and low SNR, experienced when imaging \textit{in-vivo} human retinas where typically over an acquisition period of \qty{1}{\minute} only \numrange{e3}{e4} photons are recorded~\cite{dysli2017fluorescence}. 
The above simulations were repeated with Poissonian noise applied to simulate shot noise for a range of photon fluxes: \numlist{e3;e5;e7;e12}, representing a range of photon fluxes typical to imaging the retina (\numrange{e3}{e5}), to idealised scenarios where the performance of future-generation SPAD arrays have improved significantly (\numrange{e7}{e12}). The effects of shot-noise or read-noise were not included in these simulations due to shot noise being the dominant noise source for photon fluxes larger than \num{e3}~\cite{henderson2019192}.

% Results from fitting base unmixing
\begin{figure}
    \centering
    \begin{subfigure}[b]{\textwidth}
        \centering
        \includegraphics[width = 0.62\textwidth]{figures/sflim/lifetime-unmixing/RetFitting1e3.pdf}
        \caption{}
        \label{subfig:retfitting1e3}
    \end{subfigure}
    \begin{subfigure}[b]{\textwidth}
        \centering
        \includegraphics[width = 0.62\textwidth]{figures/sflim/lifetime-unmixing/RetFitting1e5.pdf}
        \caption{}
        \label{subfig:retfitting1e5}
    \end{subfigure}
    \caption{Results from recovering the abundance of retinal fluorophores from additive mixtures using their fluorescence lifetime with the inclusion of detector shot noise for where (\subref{subfig:retfitting1e3}),(\subref{subfig:retfitting1e5}),(\subref{subfig:retfitting1e7}),(\subref{subfig:retfitting1e12}) denote photon fluxes of \numlist{e3;e5;e7;e12} respectively. The recovered abundances (middle row) are plotted with the ground truths (top row) and the error in the unmixing corresponding to ~\cref{eq:recerror} is plotted (bottom row). Areas shaded white represent where the fluorophore concentration is 0 and so a relative error cannot be ascribed (\cref{eq:recerror}).}
\end{figure}
\begin{figure}\ContinuedFloat
    \begin{subfigure}[b]{\textwidth}
        \centering
        \includegraphics[width = 0.62\textwidth]{figures/sflim/lifetime-unmixing/RetFitting1e7.pdf}
        \caption{}
        \label{subfig:retfitting1e7}
    \end{subfigure}
    \begin{subfigure}[b]{\textwidth}
        \centering
        \includegraphics[width = 0.62\textwidth]{figures/sflim/lifetime-unmixing/RetFitting1e12.pdf}
        \caption{}
        \label{subfig:retfitting1e12}
    \end{subfigure}
    \caption{Results from recovering the abundance of retinal fluorophores from additive mixtures using their fluorescence lifetime with the inclusion of detector shot noise for where (\subref{subfig:retfitting1e3}),(\subref{subfig:retfitting1e5}),(\subref{subfig:retfitting1e7}),(\subref{subfig:retfitting1e12}) denote photon fluxes of \numlist{e3;e5;e7;e12} respectively. The recovered abundances (middle row) are plotted with the ground truths (top row) and the error in the unmixing corresponding to \cref{eq:recerror} is plotted (bottom row). Areas shaded white represent where the fluorophore concentration is 0 and so a relative error cannot be ascribed (\cref{eq:recerror}).}
    \label{fig:retfitting}
\end{figure}


At the lowest photon flux, the recovered abundances show no apparent correlation with the ground truth (\cref{subfig:retfitting1e3},\cref{subfig:retfitting1e5}) and only at the higher photon fluxes do these abundance maps resemble the ground truth (\cref{subfig:retfitting1e12}). In the regions with low fluorophore concentration,
the error in the unmixed abundances is only lower than \qty{50}{\percent} at high fluorophore concentrations and at prohibitively large photon fluxes of \num{e12}. In the current literature, a photon flux of \numrange{e3}{e4} is achievable for \textit{in-vivo} retinal FLIM imaging for practical integrations times of $\approx\qty{60}{\second}$. Using this fitting-based technique an integration time in excess of 2 millennia would be required for this technique to be suitable. These results also demonstrate the common rule-of-thumb in FLIM where $10^{2k +1}$ photons are required to confidently fit a decay with $k$ lifetimes.

To summarise, a method of recovering the relative concentrations of additive mixtures of retinal fluorophores AGE, A2E, and FAD using non-linear least-squares fitting of a 6-exponential decay model was proposed and its sensitivity to fluorophore concentration, and photon shot noise, was investigated. It was found that even for exceedingly large photon fluxes the associated error in recovering these abundances was too high to reliably discriminate AGE, A2E or FAD. While improvements to SPAD array technology would increase detection efficiency by an order-of-magnitude this technique is still incapable of mapping FAD concentrations in the retina within a reasonable image acquisition period.

\FloatBarrier
\section{Recovering Retinal Fluorophore Concentration Using Phasor SFLIM} \label{sec:phasor-sflim}
The Phasor SFLIM technique, developed by,~\citeauthor{scipioni2021phasor}\cite{scipioni2021phasor}, allows for measuring fluorophore concentrations without prior knowledge of their emission spectra or fluorescence lifetime. While this method was developed for FLIM imaging for stained biological samples- where dyes can be chosen to have well separated emission spectra and fluorescence lifetime - its suitability for measuring retinal FAD concentrations was investigated. The reported robustness to low photon-flux and spectral contamination due to sample autofluorescence makes this technique attractive for measuring retinal FAD where the retinal autofluorescence signal is degraded from absorption and scattering from blood vessels and lens autofluorescence.
\FloatBarrier
\subsection{The Phasor SFLIM Unmixing Algorithm}
The Phasor SFLIM algorithm operates as follows. The time-resolved fluorescence of the scene is recorded over multiple spectral bands and the phasor transform is applied along the time-axis to produce two phasors for each pixel and each spectral band i.e. ($I(x, y, \lambda,t)\mapsto I(x, y, \lambda,2)$) to give the spectral phasors $g(x, y, \lambda)$ and $s(x, y, \lambda)$. At this stage the spectral and temporal instrument response functions can then be applied. For each pixel, Principle-Components Analysis (PCA) is applied to orient the phasors such that variability is maximised and the location of the endmembers, in phasor space is more pronounced. The PCA transformed spectral phasors are then fitted with a generalised double-logistic function
\begin{equation}
    f(x)  =  I_{-\infty,1}   +\frac{  I_{+\infty,1} + \frac{  I_{+\infty,2} - I_{-\infty,1}  }{\big(1 - \exp[-g_{2}(x - \Delta x_{2})]\big)^{\nicefrac{1}{v_{2}}}} - I_{-\infty,1}}{\bigg(1 - \exp[-g_{1}(x - \Delta x_{1})]\bigg)^{\nicefrac{1}{v_{1}}}}\label{eq:dublog}
\end{equation}
where the endmember phasor locations are recovered from the $I_{+\infty,1}$, $I_{-\infty,1}$, and $I_{+\infty,2}$ terms.
From the location of each endmember in phasor space the fluorescence decay profile and spectra are then reconstructed weighted by their relative abundance - enabling the quantification of fluorophore concentration. 


\begin{figure}
    \centering
    \begin{annotatedFigure}{\includegraphics[width = 0.8\textwidth]{figures/sflim/phasor-unmixing/SpectralPhasors.pdf}}
    \annotatedFigureText{0.02,0.99}{black}{0.3}{a)}
    \annotatedFigureText{0.51,0.99}{black}{0.3}{b)}
    \annotatedFigureText{0.02,0.49}{black}{0.3}{c)}
    \end{annotatedFigure}
    \caption{Fluorescence decay profiles, and emission spectra for retinal fluorophores AGE, A2E, and FAD are shown in a) and b) respectively with an example additive mixture shown as the dashed black line. In c) the phasor representation of the spectrally resolved fluorescence decays of this mixture are represented as circular markers where the colour represents the spectral band. The phasors of the endmembers FAD, AGE, and A2E are shown as magenta, orange, and cyan triangles respectively.}
    \label{fig:spectralphasors}
\end{figure}
\cref{fig:spectralphasors} illustrates this for an additive mixture of AGE, FAD, and A2E (\cref{fig:spectralphasors}a) with fluorescence emission spectra shown in (\cref{fig:spectralphasors}b) where each pure endmember occupies a unique location in phasor space. When the spectral phasors of the mixture are calculated they form a triangular path around the endmembers. Due to the high overlap between the fluorescence lifetimes and emission spectra of FAD and AGE, the path form by spectral phasors does not intersect the endmember phasor for FAD - there is no spectral band where FAD is the only contributor to the fluorescence signal. The authors note in the supplementary material, that the double-generalised logistic function was chosen to model the phasor representation of the endmember spectra to minimise the assumptions made about the broadness, emission peak, and general shape the endmember spectra. 
%got to here 24/09/2024
\FloatBarrier
\subsection{Phasor SFLIM Performance for Quantifying \\ Retinal Fluorophores}
As with the fitting-based method, the sensitivity of the phasor SFLIM method to fluorophore concentration and noise was investigated. In these simulations a MATLAB implementation of the algorithm, supplied in the original publication~\cite{scipioni2021phasor}, is adapted to use 32 contiguous spectral bands with rectangular transmission functions spanning \qtyrange{500}{800}{\nm} and \num{266} time bins. The fluorescence-emission spectra of AGE, A2E, and FAD was modelled by re-sampling the plotted spectra of these fluorophores from autofluorescence emission spectra recorded of porcine eyes transcribed from Fig.~9 of~\citeauthor{schweitzer2007towards}\cite{schweitzer2007towards}. In the ideal case where no shot-noise is present the phasor SFLIM method shows poor performance for measuring FAD concentrations (\cref{fig:phasorresultnoiseless}). 
\begin{figure}
    \centering
    \includegraphics[width = 0.8\textwidth]{figures/sflim/phasor-unmixing/PhasorNoiselessResults.pdf}
    \caption{Results of unmixing additive mixtures of retinal fluorophores without the inclusion of any read-noise or shot-noise using the Phasor-SFLIM technique. The top row shows the ground truth relative abundances of the retinal fluorophores AGE, A2E (lipofuscin), and FAD which are varied between 0 and 1. The middle row shows the recovered abundances after the reconstructed spectra have been correctly ordered with the associated error - calculated using \cref{eq:recerror} - shown in the bottom row and is also scaled from 0 to 1.}
    \label{fig:phasorresultnoiseless}
\end{figure}

The close proximity of the AGE and FAD endmember phasors resulted in an error in the unmixed abundances of over \qty{100}{\percent} for low fluorophore concentrations. Further, the recovered spectra are often unpredictably miscategorised – e.g FAD signal unmixes as AGE – due to the `blind’ nature of the technique. This mis-categorisation was corrected by matching the recovered spectra ($\mathbf{a}$) to the ground truth emission spectra ($\mathbf{b}$) using the Spectral Angle Map as a matching metric:

\begin{equation}\label{eq:sam}
    \varphi = \frac{\mathbf{a}\cdot \mathbf{b}}{\lvert \mathbf{a} \rvert\,\lvert \mathbf{b} \rvert}
\end{equation}

The poor performance of this technique is shown in \cref{fig:phasorresultsnoisy} where \num{e6} photons were shared between 32 spectral bands and 266 time-bins and shot-noise was applied. At low fluorophore concentrations the error in recovered abundances is $\geq\qty{100}{\percent}$ throughout the entire ``image’’ and the mean error in quantifying FAD concentration is \qty{224}{\percent}.

\begin{figure}
    \centering
    \includegraphics[width = 0.8\textwidth]{figures/sflim/phasor-unmixing/PhasorResultsNoisy.pdf}
    \caption{Results of unmixing additive mixtures of retinal fluorophores using the Phasor - SFLIM technique with the inclusion of shot-noise. For each ``pixel'' the simulated SFLIM data has \num{e6} photons distributed over 32 contiguous spectral bands and each combination of fluorophore concentrations are sampled \num{100} times to mitigate the effect of random outliers. The top row shows the ground truth relative abundances, varying from 0 to 1 of AGE, A2E, and FAD while the middle row shows the recovered abundances. The error - calculated using \cref{eq:recerror} - is shown in the bottom row and is also scaled from 0 to 1.}
    \label{fig:phasorresultsnoisy}
\end{figure}

\subsection{Appraising the Feasibility of Using Phasor SFLIM for Measuring FAD Concentrations in the Retina}
In the literature, the phasor-SFLIM technique was reported to robustly unmix fluorophores at low photon fluxes~\cite{scipioni2021phasor,scipioni2022phasor}  – for these 3 retinal fluorophores this could not be replicated. In the above simulation the unmixing method showed poor performance even without the inclusion of photon shot noise and would miscategorise the retinal fluorophores in the recovered spectra. This poor performance is most likely due to the high spectral overlap and similar decay profiles of FAD, AGE, and A2E which in phasor spaces gives endmembers that are not well separated and in the spectral phasor plot no prominent vertices are found. While this analysis did not attempt to model spectra contamination due to absorption in blood and vasculature or the effects the lens fluorescence it can be concluded that the phasor-SFLIM algorithm is not suitable for measuring concentrations of the retinal FAD and was not pursued further.
Where phasor-SFLIM does perform well is for spectral-fluorescence lifetime imaging of stained samples where fluorophores can be chosen to minimise spectral overlap and exhibit differing fluorescence lifetimes. 
\FloatBarrier
\section{Spectral Unmixing Using Matrix Inversions}\label{sec:specunmix}
In remote sensing endmembers are discriminated using variations in spectral signatures to separate buildings, from foliage, from rivers in the scene where generally: buildings will often appear spectrally flat; foliage will peak in the green; and rivers and water features will peak in the blue spectral regions. In these applications the endmembers are unmixed by assuming the following linear mixing model
\begin{align}
    I(x,y) = I_{0}\sum^{N}_{n \geq}\alpha_{n}S_{n}(\lambda) \text{ such that } \sum_{n \geq 1}^{N}\alpha_{n} = 1 \text{ and } \int_{0}^{\infty}S_{n}d\lambda = 1
\end{align}
and recovering abundances with methods such as non-Negative Matrix Factorisation (NMF), Vertex / Geometric Component Analysis (VCA / GCA), Auto-encoder networks or simple matrix inversions.
For measuring FAD concentrations this method was investigated to benchmark the SFLIM technique presented in the next chapter. Compared to SFLIM measurements in spectral imaging same photon flux is distributed over fewer measurements in only the spectral domain. In principle this reduces the proportion of noise per measurement compared to SFLIM measurements at the cost of not recovering the information included in the fluorescence lifetime. This poses the question to be addressed in the next chapter~\cref{chap:tensSFLIM}: Does the additional information from fluorescence lifetimes justify the high noise present in SFLIM measurements?
Despite the sensitivity to noise and overlapping endmember spectra, spectral imaging could be an attractive technique for measuring retinal FAD concentrations with careful optimisation of the number of spectral bands, and their location and width. In this chapter the simplest method – matrix inversion – is considered. For a given pixel with endmember abundances, $\mathbf{a}(n)$ and known endmember spectra, $\mathbf{S}(\lambda,n)$ the spectral measurements $\mathbf{M}(\lambda)$ are simply a linear combination: 
\begin{align}
    \mathbf{M} &= \mathbf{S}\mathbf{a}\label{eq:specmix}
\end{align}
\begin{align}
    \mathbf{S} &= \begin{bmatrix} S_{0}(\lambda_{0}) & S_{0}(\lambda_{1}) & \cdots & S_{0}(\lambda_{j})\\ S_{1}(\lambda_{0}) & S_{1}(\lambda_{1}) & \cdots & S_{1}(\lambda_{j}) \\ \vdots & \vdots & \ddots & \vdots \\ S_{n}(\lambda_{0}) & S_{n}(\lambda_{1}) & \cdots & S_{n}(\lambda_{j}) \end{bmatrix} &
    \mathbf{M} &= \begin{bmatrix} M(\lambda_{0}) & M(\lambda_{1}) & \cdots & M(\lambda_{j}) \end{bmatrix} &
    \mathbf{a} &= \begin{bmatrix} a_{0} \\ a_{1} \\ \vdots \\ a_{n}\end{bmatrix}
\end{align}

The recovered endmember abundances, $\bar{\mathbf{a}}$ are then found using a simple matrix multiplication:
\begin{align}
    \bar{\mathbf{a}} &= \mathbf{S}^{-1}\mathbf{M} \label{eq:matinv}
\end{align}
For over determined systems - where the number of spectral measurements is greater than the number of unknown endmember abundances ($\lambda > n $)  - the matrix $\mathbf{S}$ is non-square and so the inverse $\mathbf{S}^{-1}$ doesn't exist. The Moore-Penrose Pseudo Inverse, denoted as $\mathbf{S}^{+}$ (\cref{eq:mpinvmatrix}), extends some useful properties of the traditional inverse (\cref{eq:mpinvidentity}):
\begin{align}
    \mathbf{S}^{+} &= (\mathbf{S}^{T}\mathbf{S})^{-1}\mathbf{S}^{T} \label{eq:mpinvmatrix}
\end{align}
\begin{align}
    \mathbf{A}^{-1} &: \mathbf{A}^{-1}\mathbf{A} = \mathbb{I} \implies \mathbf{A}^{+}: \mathbf{A}\mathbf{A}^{+}\mathbf{A} = \mathbf{A} \label{eq:mpinvidentity}
\end{align}
The matrix condition number (see \cref{eq:matcondnum}) is a convenient metric for assessing the influence of noise and spectral overlap on the error in recovered fluorophore concentrations.
\begin{equation}\label{eq:matcondnum}
    \kappa(\mathbf{S}) = \big\lvert\big\lvert \mathbf{S} \big\rvert\big\rvert_{F}\big\lvert\big\lvert \mathbf{S}^{+} \big\rvert\big\rvert_{F} 
\end{equation}
It encapsulates how sensitive the inversion process is to noise and is dependent on the endmember spectra that are being unmixed, the number of detection bands as well as the location of these bands and their width. In the inversion process, a noise in the measurement $\mathbf{M}$ of $\nicefrac{\delta \mathbf{M}}{\mathbf{M}}$ will cause an error in the unmixed abundance, $\mathbf{a}$, on the order of $\nicefrac{\delta \mathbf{a}}{\mathbf{a}}$ - scaled by the condition number, $\kappa(\mathbf{S})$. Generally, a condition number of $\kappa(\mathbf{S}) \propto 10^{k}$ results in a loss of precision of $k$ digits~\cite{cheney2012numerical}. For recovering the abundances of retinal fluorophores, which will vary between \qtylist{0; 100}{\percent} or \numlist{0; 1}, a condition number of $\kappa(\mathbf{S}) < \num{e2}$ is required to have an error of less than \qty{10}{\percent}.
\begin{equation}
        \bigg\lvert\bigg\lvert \frac{\delta \mathbf{a}}{\mathbf{a}} \bigg\rvert\bigg\rvert \leq \kappa(\mathbf{S})\bigg\lvert\bigg\lvert \frac{\delta \mathbf{M}}{\mathbf{M}} \bigg\rvert\bigg\rvert
\end{equation}

\FloatBarrier
\subsection{Optimisation of Spectral Detection Bands for\\ Recovering Concentrations of Retinal Fluorophores}
The condition number was then used to find the optimal location and number of detection bands. While narrow, contiguous bands do give higher spectral resolution each band does exhibit higher noise or longer image acquisition times. For this proposed spectral imaging scheme, a system of \num{6} bands was optimised using an exhaustive search of all $\binom{40}{6} \approxeq \num{3.8e6}$ possible groups of 6, \qty{5}{\nm}-wide, rectangular bands spanning \qtyrange{500}{700}{\nm} that minimises the matrix condition number. The search gave an optimal combination of filters with cut-on wavelengths of \qtylist{500;535;630;640;655;660}{\nm} and a condition number of $\kappa = \num{8.101}$ indicating an at most one digit of precision will be lost.

\begin{equation}\label{eq:nchoosek}
    \binom{n}{m} = \frac{n!}{k!(n-k)!}
\end{equation}

\begin{figure}
    \centering
    \includegraphics[width = 0.8\textwidth]{figures/sflim/spectral-unmixing/MatrixBandsSpectra.pdf}
    \caption{Optimal detection bands for quantifying FAD in the retina using spectral imaging. The detection bands are superimposed onto the fluorescence emission spectra of AGE, FAD, and A2E (blue, green and red lines, respectively) and were found by minimising the matrix condition number.}
    \label{fig:retspectrabands}
\end{figure}

The performance of this detection scheme is then quantified using the previously utilised abundance maps for both noiseless data and data with shot-noise equivalent to \num{e5} detected photons. In the noiseless data, the RMS errors are on the order of \qty{e-12}{\percent} however when photon noise is introduced the average error in recovering the abundance of FAD rises to \qty{91.8}{\percent}


\begin{equation}\label{eq:RMSerror}
    \mathbf{\epsilon}_{RMS} = \sqrt{\frac{\sum_{n \geq 1}^{N}\mathbf{\epsilon}^{2}}{N}}
\end{equation}


\begin{figure}[htb]
    \begin{subfigure}[b]{\textwidth}
        \centering
        \includegraphics[width = 0.62\textwidth]{figures/sflim/spectral-unmixing/SpectUnmixOptBandsNoiseless.pdf}
        \subcaption{}
        \label{subfig:optbandsnoiseless}
    \end{subfigure}
    \vfill
    \begin{subfigure}[b]{\textwidth}
        \centering
        \includegraphics[width = 0.62\textwidth]{figures/sflim/spectral-unmixing/SpectUnmixOptBands1e5.pdf}
        \subcaption{}
        \label{subfig:optbands1e5}
    \end{subfigure}
    \caption{Relative abundances of retinal fluorophores recovered using spectral unmixing and 6 detection bands optimised such that the matrix condition number is minimised under (\subref{subfig:optbandsnoiseless}) noiseless conditions and (\subref{subfig:optbands1e5}) \num{e5} photons with shot noise applied. For both (\subref{subfig:optbandsnoiseless}) and (\subref{subfig:optbands1e5}) the top rows represent the ground truth abundances, the middle rows are the recovered abundances, and the bottom row is the error defined using~\cref{eq:recerror}.}
    \label{fig:spectunmixoptbands}
\end{figure}
\FloatBarrier
\subsection{Appraising the Feasibility of Using Spectral Imaging for Measuring FAD Concentrations in the Retina}
While spectral unmixing can in principle be used to unmix retinal fluorophores the large RMS error limits its efficacy for estimating the concentration of FAD in the retina beyond asserting the presence of FAD with low confidence. While the goal of this simulation was simply to benchmark a new SFLIM unmixing method discussed in the next chapter it could be made more rigorous by using non-rectangular filters or by considering off-the-shelf filters. Further, with the discussed unmixing methods suffering from high errors in recovering the relative abundance of FAD it highlights the fundamental questions - ``What benefit does fluorescence lifetime bring to assessing retinal health,'' and ``Does its benefit outweigh the additional noise incurred in each measurement bin.''
\FloatBarrier
\section{Conclusions}
In this chapter, the feasibility of measuring FAD concentrations in retina using three methods. Simulations were constructed with linear combinations of the retinal fluorophores AGE, A2E, and FAD with Poissionian noise applied to simulate the effects of photon fluxes experience in \textit{in-vivo} SFLIO imaging of the retina. The fitting-based approach was highly sensitive to noise due to the similar lifetimes of the 3 retinal fluorophores and the 6-exponential fit required to model their decay properties (\cref{sec:lifeunmix}). Even at unrealistically high photon fluxes of \num{e12} this method showed prohibitively high errors in the recovered fluorophore concentrations and was deemed unsuitable. The reportedly phasor S-FLIM technique~\cite{scipioni2021phasor} was also investigated and was also found to have high error in recovered fluorophore concentrations. At low photon fluxes of \num{e5} the algorithm would misidentify the unmixed fluorophores meaning AGE could be unmixed as FAD which would introduce large errors in the the recovered concentration of FAD. Under the phasor transform the high spectral overlap and decay properties of AGE, A2E, and FAD resulted in the endmembers not being well separated in phasor space or forming prominent vertices for the unmixing procedure to identify (\cref{fig:phasorresultnoiseless}). 
To provide a benchmark for the unmixing method developed in~\cref{chap:tensSFLIM} a set of spectral bands were optimised for a matrix inversion spectral technique using matrix condition number related metrics (\cref{sec:specunmix}). The performance of this set of 6 detection bands was simulated and found to yield good unmixing performance at high relative fluorophores concentrations but at low concentrations of FAD this error rises to \qty{91.8}{\percent}. This motivates the need for a better unmixing technique that is sensitive to small changes in concentration of key retinal biomarkers.

% \subsection{Conclusions}
% In this chapter, the feasibility of measuring FAD concentrations in retina using three methods. Simulations were constructed with linear combinations of the retinal fluorophores AGE, A2E, and FAD with Poissionian noise applied to simulate the effects of photon fluxes experience in \textit{in-vivo} SFLIO imaging of the retina. The fitting-based approach was highly sensitive to noise due to the similar lifetimes of the 3 retinal fluorophores and the 6-exponential fit required to model their decay properties (\cref{sec:lifeunmix}). Even at unrealistically high photon fluxes of \num{e12} this method showed prohibitively high errors in the recovered fluorophore concentrations and was deemed unsuitable. The reportedly phasor S-FLIM technique~\cite{scipioni2021phasor} was also investigated and was also found to have high error in recovered fluorophore concentrations. At low photon fluxes of \num{e5} the algorithm would misidentify the unmixed fluorophores meaning AGE could be unmixed as FAD which would introduce large errors in the the recovered concentration of FAD. Under the phasor transform the high spectral overlap and decay properties of AGE, A2E, and FAD resulted in the endmembers not being well separated in phasor space or forming prominent vertices for the unmixing procedure to identify (\cref{fig:phasorresultnoiseless}). To provide a benchmark for the unmixing method developed in~\cref{chap:tensSFLIM} a set of spectral bands were optimised for a matrix inversion spectral technique using matrix condition number related metrics (\cref{sec:specunmix}). The performance of this set of 6 detection bands was simulated and found to yield good unmixing performance at high relative fluorophores concentrations and at low concentrations of FAD this error rises to \qty{91.8}{\percent}. This motivates the need for a better unmixing technique that is sensitive to small changes in concentration of key retinal biomarkers.