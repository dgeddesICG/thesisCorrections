\FloatBarrier
\section{SFLIM Imaging of FAD}\label{sec:retfad}
A successful unmixing method must be primarily be capable of detecting the presence of FAD at concentrations similar to the retina but also have the sensitivity to robustly changes in FAD concentration. To address this primary goal, the expected concentration of FAD in the retina is estimated by modelling the average metabolisation of oxygen and using this to infer the rate of production of FAD.
FAD in the body is produced through aerobic respiration wherein glucose and oxygen are metabolised produce carbon dioxide, water, and ATP the bodies usable energy source. Through the usage of ATP for other metabolic processes FAD and other flavoproteins such as Riboflavin are created as a byproduct.
\begin{equation}\label{eq:resp}
    \ce{C6H12O6 + 6O2 -> 6CO2 + 6H2O + ATP}
\end{equation}
The rate at which oxygen is metabolised, $Q_{\ce{O2}}$, is determined using ~\cref{eq:o2rate} from the average blood flow through the ophthalmic artery, $Q_{ret}$, and the relative change in saturation of blood entering, ${S\ce{O2}}_{in}$ and exiting it ${S\ce{O2}}_{out}$. The rate of blood flow through the ophthalmic artery was taken as $Q_{ret} = \SI{5}{\micro\litre\per\minute}$ by averaging two reported independent measurements recorded using different methods~\cite{dai2013absolute,riva1985blood}. Further, the oxygen saturation - the ratio of oxygenation to deoxygenated haemoglobin in the blood - has been reported as \SI{92.20}{\percent} and \SI{55.60}{\percent} for blood entering and exiting the ophthalmic artery~\cite{dunn2016physiology}.
\begin{equation}\label{eq:o2rate}
    Q_{\ce{O2}} = Q_{ret}({S\ce{O2}}_{in}-{S\ce{O2}}_{out})H
\end{equation}
Using~\cref{eq:o2rate} the rate of oxygen metabolisation was estimated to be $Q_{\ce{O2}}=\SI{3.155}{\micro\litre\per\minute}$ where $H=\num{0.285}$ is Hüffners constant which is the volumetric ratio of oxygen to haemoglobin in the blood~\cite{dunn2016physiology}. The rate of FAD production, $Q_{FAD}$, can then be determined by considering that for every metabolised \ce{O2} molecule $k_{FAD} = 6$ molecules of FAD are produced with an efficiency of $\eta_{FAD} = \SI{1.3}{\percent}$~\cite{berg2007biochemistry, ames1992energy}.
\begin{equation}
    Q_{FAD} = Q_{\ce{O2}}N_{FAD}\eta_{FAD}
\end{equation}
This yields a production rate of $Q_{FAD} = \SI{0.123}{\micro\litre\per\min}$ which is then be converted to a steady state volume of FAD by assuming that all FAD in the retina is produced in the time taken for blood to flow through the retina and any FAD that is consumed is replaced upon the next circulation period of blood. With a blood flow period of $\delta T = \SI{4.7}{\second}$ measured as the time interval between blood entering and exiting the ophthalmic artery~\cite{khoobehi1990measurement}.
\begin{equation}
    V_{FAD} = Q_{FAD}\delta T
\end{equation}
With this calculated volume of FAD,$V_{FAD} = \SI{9.639}{\micro\litre}$, corresponds to $N_{FAD} = \SI{12.27}{\nano\mole}$ using \cref{eq:nummoles} with a $gfm_{FAD} =\SI{785.56}{\gram\per\mole}$, a density equal to that of water of $\rho_{FAD} = \SI{1} {\gram\per\milli\litre}$, and $N_{A} = \SI{6.022e23}{\per\mole}$.
\begin{equation}\label{eq:nummoles}
    N_{FAD} = V_{FAD}\rho_{FAD}gfm_{FAD}
\end{equation}
Finally, the result of~\cref{eq:nummoles} is expressed as a concentration by approximating the retina as a cuboid with thickness of \SI{344}{\um} and surface area of \SI{1361}{\milli\metre\squared} to estimate its volume~\cite{nagra2017determination, myers2015retinal}.
\begin{equation}\label{eq:FADconc}
    C_{FAD} = \frac{N_{FAD}}{V_{ret}} = \SI{26.9921}{\micro\mole\per\litre}
\end{equation}
While this analysis does represent a simplified interpretation of the biological functions in the retina it is commensurate with reported measurements of flavoproteins in rabbit eye's. In \citeauthor{batey1991analysis}, the concentration is defined as \SI{39.3}{\pico\mole\per\milli\gram} by weight of the retina which is within \SI{10}{\percent} of the estimation above when converted to equivalent units - $C_{FAD} = \SI{37.64}{\pico\mole\per\milli\gram}$ obtained using the result of \cref{eq:nummoles} and taking the mass of the retina as \SI{326}{\milli\gram}~\cite{batey1991analysis,feke1989blood}.

\subsection{\textit{Ex-Vitro} SFLIM Imaging of FAD}
The results of the above analysis were then used to imitate the expected photon flux produced by FAD in the retina and assess any additional challenge with unmixing FAD from two other selected dyes - Fluorescein and Rose Bengal. Aqueous solutions of FAD were prepared with concentrations equal to 0.1, 1, and 10 times the value of $C_{FAD}$ and the remaining solutions of dye were mixed into solutions of fluorescein and Rose Bengal where the relative photon 


\subsection{Optimisation of detection bands for retinal fluorophores}\label{sec:specunmixopt}
Additionally this condition number can utilised to optimise the 2 predominant approaches to recording spectral images: contiguous bands of equal width that span a designated wavelength range - this approach evenly samples the spectra with lower sensitivity to noise but sacrifices spectral resolution; and multiple narrower discrete bands which are chosen for the scene and the endmembers to be recovered in order to maximise the spectral specificity and reduce image acquisition time.
A simple exhaustive search for the optimal detection bands for unmixing retinal fluorophores was carried out to find the optimum location of the \num{6} detection bands for the S-FLIO system described in \cref{chap:fliodevice}. The fluorescence emission bandwidth (\qtyrange{500}{700}{\nm}) that the S-FLIO system can image over was divided into \num{40} contiguous, \SI{5}{\nm} wide, bands and the condition number was calculated for each of the $\binom{40}{6} \approxeq \num{3.8e6}$ possible detection configurations. This yields an optimal set of filters with cut-on wavelengths of \qtylist{500;535;630;640;655;660}{\nm} with a matrix condition number of $\kappa = \num{8.101}$ indicating an at most one digit of precision will be lost.
whereas the heuristically chosen bands with cut-on wavelengths of \qtylist{500;510;530;560;590;630}{\nm} gives a larger condition number of $\kappa = \num{10.761}$.
\begin{equation}\label{eq:nchoosek}
    \binom{n}{m} = \frac{n!}{k!(n-k)!}
\end{equation}
\begin{figure}
    \begin{subfigure}[b]{0.49\textwidth}
        \centering
        \includegraphics[width = \textwidth]{figures/sflim/spectral-unmixing/HeuristicBandsSpectra.pdf}
        \subcaption{}
        \label{subfig:retspectraheurstic}
    \end{subfigure}
    \hfill
    \begin{subfigure}[b]{0.49\textwidth}
        \centering
        \includegraphics[width = \textwidth]{figures/sflim/spectral-unmixing/MatrixBandsSpectra.pdf}
        \subcaption{}
        \label{subfig:retspectraopt}
    \end{subfigure}
    
    \caption{Detection bands for spectral imaging superimposed onto fluorescence emission spectra of AGE, FAD, and A2E (blue, green and red lines, respectively) where (\subref{subfig:retspectraheurstic}) represents the bands chosen heuristically for the \textit{ex-vivo} rat imaging and (\subref{subfig:retspectraopt}) represents the bands which minimise the matrix condition number.}
    \label{fig:retspectrabands}
\end{figure}
Now, the performance of these two detection schemes can then be compared using the above abundance maps with noiseless data and data consisting of \num{e5} detected photons and their associated shot noise as shown in \cref{fig:spectunmixheursiticbands,fig:spectunmixoptbands}. 

\begin{equation}\label{eq:RMSerror}
    \mathbf{\epsilon}_{RMS} = \sqrt{\frac{\sum_{n \geq 1}^{N}\mathbf{\epsilon}^{2}}{N}}
\end{equation}

\begin{figure}
    \begin{subfigure}[b]{\textwidth}
        \centering
        \includegraphics[width = 0.65\textwidth]{figures/sflim/spectral-unmixing/SpectUnmixHeuristicBandsNoiseless.pdf}
        \subcaption{}
        \label{subfig:heuristbandsnoiseless}
    \end{subfigure}
    \vfill
    \begin{subfigure}[b]{\textwidth}
        \centering
        \includegraphics[width = 0.65\textwidth]{figures/sflim/spectral-unmixing/SpectUnmixHeuristicBands1e5.pdf}
        \subcaption{}
        \label{subfig:heuristbands1e5}
    \end{subfigure}
    \caption{Relative abundances of retinal fluorophores recovered using spectral unmixing and 6 detection bands using heuristically chosen band locations under (\subref{subfig:heuristbandsnoiseless}) noiseless conditions and (\subref{subfig:heuristbands1e5}) \num{e5} photons with shot noise applied. For both (\subref{subfig:heuristbandsnoiseless}) and (\subref{subfig:heuristbands1e5}) the top rows represent the ground truth abundances, the middle rows are the recovered abundances, and the bottom row is the error defined using Eq.~\ref{eq:recerror}.}
    \label{fig:spectunmixheursiticbands}
\end{figure}


\begin{figure}
    \begin{subfigure}[b]{\textwidth}
        \centering
        \includegraphics[width = 0.62\textwidth]{figures/sflim/spectral-unmixing/SpectUnmixOptBandsNoiseless.pdf}
        \subcaption{}
        \label{subfig:optbandsnoiseless}
    \end{subfigure}
    \vfill
    \begin{subfigure}[b]{\textwidth}
        \centering
        \includegraphics[width = 0.62\textwidth]{figures/sflim/spectral-unmixing/SpectUnmixOptBands1e5.pdf}
        \subcaption{}
        \label{subfig:optbands1e5}
    \end{subfigure}
    
    \caption{Relative abundances of retinal fluorophores recovered using spectral unmixing and 6 detection bands optimised such that the matrix condition number is minimised under (\subref{subfig:optbandsnoiseless}) noiseless conditions and (\subref{subfig:optbands1e5}) \num{e5} photons with shot noise applied. For both (\subref{subfig:optbandsnoiseless}) and (\subref{subfig:optbands1e5})  the top rows represent the ground truth abundances, the middle rows are the recovered abundances, and the bottom row is the error defined using Eq.~\ref{eq:recerror}.}
    \label{fig:spectunmixoptbands}
\end{figure}

The errors in the noiseless unmixing are negligible (\num{1e-18}) and is supported by the vanishingly small RMS error: AGE = \SI{2.87e-12}{\percent}, A2E = \SI{1.04e-12}{\percent}, and FAD = \SI{5.00e-12}{\percent}. For the case of noise affected spectra - the error in the unmixing, and the difference in condition number becomes apparent where the RMS errors for optimised bands are up to \SI{4}{\percent} lower (Tab.~\ref{tab:spectunmixrms}).

\begin{table}
    \centering
    \begin{tabular}{|c|c|c|c|}
        \hline
        & AGE (\unit{\percent}) & A2E (\unit{\percent}) & FAD (\unit{\percent})\\
        \hline
        Opt. Bands - Noiseless & \num{2.87e-12} & \num{1.04e-12} & \num{5.00e-12}\\
        Opt. Bands - \num{e5} photons & \num{50.0} & \num{16.6} & \num{91.8}\\ 
         Heuristic Bands - Noiseless & \num{2.87e-12} & \num{1.04e-12} & \num{5.00e-12} \\
         Heuristic Bands - \num{e5} photons & \num{50.7} & \num{16.5} & \num{95.2} \\
         \hline
    \end{tabular}
    \caption{RMS errors of spectrally unmixing retinal fluorophores using matrix inversions using the heuristically chosen bands described in Chap.~\ref{chap:fliodevice} and bands chosen such that the matrix condition number is minimised under noiseless and shot noise affected conditions (\num{e5} photons)}
    \label{tab:spectunmixrms}
\end{table}

\subsection{Conclusions}
While spectral unmixing can in principle be used to unmix retinal fluorophores the large RMS error when unmixing FAD limits its efficacy for estimating the concentration of FAD in the retina beyond confirming the presence of FAD in a mixture of retinal fluorohpores. The RMS errors in the unmixing process also validate the bands that were chosen heuristically to coincide with the isobestic and non-isobestic points of the spectras of AGE, A2E, and FAD however the optimisation process could be improved to account for fluorescence emission filters with non-rectangular spectral response as well as the overall spectral response of the FLIO device. This simulation also only pinpoints the ideal detection bands within \SI{5}{\nm} increasing the resolution of this optimisation non-linearly increases the possible number of filter combinations and thus the computation time.
These results also highlights the key question of this thesis - ``What benefit do fluorescence lifetime bring to assessing retinal health'' which is addressed in the next section.