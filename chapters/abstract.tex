% \setstretch{2}
Early detection of retinal disease improves patient outcomes. Current imaging techniques rely on detecting structural damage in the retina after permanent damage to vision has occurred. Measurements of spectrally and temporally resolved fluorescence in the retina could enable the detection of retinal disease at the initial point of biochemical dysfunction - before patients report changes in their vision. Current publications focus on measuring qualitative changes in fluorescence lifetimes as retinal disease progress thus leaving a need for quantitative techniques for measuring metabolic biomarkers in the retina.\\
In this thesis, new generation SPAD-TCSPC array are utilised to construct the SFLIO device: a widefield imaging system capable of recording fluorescence lifetimes in the retina over 4 spectral bands optimised for photon-efficient unmixing of retinal biomarkers. Natural motion of the retina are compensated for and $2\times$ pixel super-sampling is demonstrating example of a moving \textit{Convallaria Majalis} sample mounted in an artificial eye model.
Due to high spectral and lifetime overlap of the retinal biomarkers FAD, AGE, and A2E, a new method is motivated and developed for recovering the relative concentration of retinal biomarkers from SFLIM measurements. In simulations this new technique outperforms existing lifetime unmixing techniques by multiple orders of magnitude and marginally outperforms existing spectral imaging techniques. This invites further study on the most efficient usage of available photons for enhancing chemical discrimination abilities.\\
From \textit{in-vivo} SFLIM measurements recorded of multiple anaesthetised rats it was found that with high confidence $\big(p<10^{-6}\big)$ that a change in what is believed to be FAD is detected as a response to acute hypoxia. While this change could only measured as a global change and not mapped across the retina this result shows that the SFLIO device and SFLIM unmixing technique shows promise for quantitative measurements of metabolic health in the retina.\\
To summarise, the SFLIO device demonstrated in this thesis shows promise a route towards quantitative measurements of retinal health and enabling deeper study in the the pathophysiology of retinal disease.