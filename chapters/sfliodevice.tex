
\FloatBarrier
\section{Chapter Summary}
In this chapter the process of modifying an COTS fundus camera for Spectral-Fluorescence Lifetime Imaging of the retina, is described. Fluorescence lifetime images are recorded using a Horiba FLIMera SPAD-array and a secondary science (sCMOS) camera is used for compensation of eye movements. Fluorescence is excited using a pulsed white-light super-continuum laser source filtered using an AOTF to \qtyrange{460}{467}{\nm}. The modifications to existing illumination optics of the fundus camera to couple the filtered laser source are also described. A safety assessment is described which outlines safe operating procedures for imaging \textit{in-vivo} human retinas. The SFLIO device is then calibrated and characterised to ensure key retinal features can be discriminated and fluorophore concentrations can be optimally recovered. The spectral response of the instrument is then measured and the process of calibrating the malfunction pixels and temporally misalignment between pixels of the FLIMera are described. Finally, the spatial resolution of the instrument is measured using the slanted edge test giving a resolution of \qty{50}{\um} for the sCMOS detector and \qty{214}{\um} for the brightfield detector and FLIMera. The resolution of recorded fluorescence lifetime images would then be sufficient to resolve large drusen with an average diameter and retinal vasculature.
\FloatBarrier
\section{Construction of \textit{En-face} SFLIO system}
\FloatBarrier
\subsection{SFLIO System Design Methodology}
As outlined in~\cref{sec:motivation} the purpose of this project is to combine the principles of fluorescence lifetime imaging and spectral fluorescence imaging to further develop capabilities of chemical sensing in the retina beyond what is currently possible with spectral imaging or FLIM. The high spectral overlap, and similar fluorescence lifetimes of fluorescent biomarkers prohibits quantitative measurements of retinal metabolic health using either spectral or FLIM imaging, alone. A COTS fundus camera was modified to enable spectrally-resolved fluorescence lifetime measurements of the retina with a FOV wide enough to span the fovea (\qty{20}{\degree}).
While this field-of-view is generally considered to be narrow in ophthalmic imaging - existing point scanning systems can image \qty{82}{\percent} of the retinal surface or a (\qty{200}{\degree} FOV~\cite{kato2019quantitative}) - this device is to serve as a prototype to establish the efficacy this new imaging technique. To image a larger field-of-view multiple exposure would be recorded of the eye over multiple fixation points and stitched together. The total-image acquisition time is also a key constraint in designing a retinal imaging device for routine clinical assessments. 
In fluorescence lifetime imaging of the poor quantum efficiency retina and the intensity for which it can be safely illuminated results in a high-noise and photon-starved images.
For the proposed SFLIO device, the excitation and imaging bands are be optimised to enhance the detection, and discrimination, of key fluorescent retinal biomarkers - FAD, AGE, A2E. The main aim of the SFLIO prototype is evaluate the merit of combining spectral fluorescence and lifetime imaging for quantitive assessments of retinal health. While in this thesis image acquisition times are shown to be prohibitively long for regular clinical use the technique can still be appraised. SPAD arrays allow for SFLIM measurements to be recorded in a series of snapshots. Future generations of SPAD arrays will reduce image acquisition times through higher pixel counts, reduced detector dead-time, and higher fill-factor - raster scanned single-pixel FLIM solutions are seeing modest improvements with each generation. SPAD array based imagers are thus the ideal candidate for future low-cost implementation of SFLIO in a clinical setting.

\FloatBarrier
\subsection{Description of the Conventional Fundus Camera}
The S-FLIO device, shown in~\cref{fig:sflioopticsfull}, developed in this project uses an off-the-shelf fundus camera to simplify the design and construction process allowing more time to be spent developing methods of chemical quantification in the retina using SFLIM. %This approach also allows for the SFLIO system to be more readily integrated into OPTOS's existing product lines where it could be sold as a stand-alone system or a modification for existing fundus camera.
\begin{figure}[htp]
    \centering
    \includegraphics[width = 0.8\textwidth]{figures/sflio-device/SFLIOdevicePhoto.pdf}
    \caption{Photograph of the SFLIO device showing the modifications made to the imaging and illumination paths to facilitate SFLIM imaging. The modified imaging paths (red highlighted region) uses a Thorlabs Kiralux sCMOS detector for recording brightfield of fluorescence images for compensating for motion of the retina (see~\cref{chap:motionreg}). The green highlighted region shows the modifications to the illumination arm to allow illumination of the retina using the AOTF-filtered pulsed super-continuum source. Ray diagrams of the illumination and imaging optics are shown in~\cref{fig:sfliooptics} and~\cref{fig:imagingoptics}, respectively.}
    \label{fig:sflioopticsfull}
\end{figure}
The fundus camera can record images over the fields-of-view \qtylist{20;35;50}{\degree} in multiple modalities: brightfield or reflectance imaging; autofluorescence imaging; and fluorescence imaging using an exogenous dye such as fluorescein or indocyanine green in retinal angiography. The retina is illuminated using a tungsten lamp for alignment of the eye or a Xenon flash bulb. For both light sources annular illumination, shown in~\cref{fig:annularillumination}, is used to separate the illumination and imaging volumes. Reducing this overlap of ray volumes reduces the contamination due to Purkinje reflections~\cite{lee20123d} from the lens and cornea - resulting in a higher contrast image. This is the Gullstrand principle~\cite{gullstrand1910neue}.
\begin{figure}[htp]
    \centering
    \includegraphics[width = 0.7\textwidth]{figures/sflio-device/AnnularIlluminationDiagram.pdf}
    \caption{The retina is evenly illuminated through an annulus (green) leaving the centre of the pupil unobstructed to image through (red). Reproduced from Fig.~4 of~\citeauthor{terry2022trans}\cite{terry2022trans}}
    \label{fig:annularillumination}
\end{figure}

The modifications made to the illumination arm of the fundus camera for tuneable excitation of retinal fluorescence are described in~\cref{subsec:modifyillum}. Likewise, changes made to the imaging arm to enable spectral-fluorescence lifetime measurements are discussed in~\cref{subsec:modifyimaging}. These modifications are highlighted in~\cref{fig:sflioopticsfull}.



\FloatBarrier
\subsection{Modifications for Illuminating the Retina with Pulsed Super-Continuum Source}\label{subsec:modifyillum}
The SFLIO device excites retinal fluorescence using a pulsed-supercontinuum white-light source (NKT SuperK Ex-12W and Fianium source) tuned into 8 narrow bands ($\text{FWHM} = \qty{1}{\nm}$) between \qtyrange{460}{467}{\nm} with an Acousto-Optically Tuneable Filter (AOTF). Tuning the excitation wavelength, while not thoroughly explored in this thesis, would allow for retinal fluorophores to be targeted using their unique excitation spectra. As shown in~\cref{fig:sfliooptics}, the fibre output of the AOTF is coupled into the illumination path by removing an existing filter wheel and mounting a dichroic mirror in its place. Wavelengths shorter than \qty{505}{\nm} are reflected into the illumination path and unwanted longer wavelengths are transmitted into the chassis of the fundus camera and absorbed. At higher laser powers a beam dump should be used to prevent any light reflected causing injury. For the purpose of this project dumping excess long-wavelength light into the chassis was deemed sufficiently safe. To match the Numerical Aperture (NA) 
\begin{equation}\label{eq:numericalaperture}
    NA = n\sin{\Bigg(\arctan{\bigg(\frac{D}{2f}\bigg)}\Bigg)}
\end{equation}
of the fibre output of the laser the beam is collimated ($\text{diameter}= 1", f = \qty{100}{\mm}$) and refocused ($\text{diameter} = 1", f = \qty{150}{\mm}$) to a point at the perimeter of the annular stop $A_{1}$ of \cref{fig:sfliooptics}. 
This annular stop is conjugate to the annulus projected on the cornea, at the perimeter of the pupil, and results in the excitation of fluorescence over $\qty{20}{\degree}$ of the retina (see~\cref{fig:illuminationarea}). While the fundus camera is capable of the imaging over a \qty{50}{\degree} FOV, the combination of long acquisition times required for SFLIO imaging coupled with the limited output power of the supercontinuum light-source meant the area illuminated on the retina would be reduced to minimise geometric losses.

\begin{figure}
    \centering
    \begin{annotatedFigure}{\includegraphics[width = 0.8\textwidth]{figures/sflio-device/IlluminationOpticsExample.pdf}}
    \annotatedFigureText{0.02,0.84}{white}{0.3}{a)}
    \annotatedFigureText{0.53,0.84}{white}{0.3}{b)}
    \end{annotatedFigure}
    \caption{Modified illumination arm shows a reduced illuminated FOV on the retina when compared to the existing Tungsten inspection lamp. In a) an a brightfield image was recorded of a \textit{ex-vivo} enucleated rabbit eye using the Tungsten illumination source which fills the entire \qty{35}{\degree} FOV b) A fluorescence image was recorded using the modified illumination optics where fluorescence is excited between \qtyrange{460}{467}{\nm} and imaged using a \qty{500}{\nm} long-pass filter.}
    \label{fig:illuminationarea}
\end{figure}

\begin{figure}
    \centering
    \includegraphics[width = 0.9\textwidth]{figures/sflio-device/IlluminationOpticsDiagram.pdf}
    \caption{Optical layout of the illumination arm of the SFLIO system required for illuminating the retina through the existing annular illumination system. The lens $f_{2}$ is adjusted until the a focussed point of light is produced on the patients cornea.}
    \label{fig:sfliooptics}
\end{figure}
This design approach does reduce the illuminated FOV on the retina but does use the available laser power more efficiently. The multiple annular-stops block light from the central lobe of the lasers Gaussian intensity profile - illuminating over a narrower FOV minimises these geometric losses. Whilst a more powerful laser could be used to mitigate these geometric losses this would then require more rigorous fail-safe mechanisms to prevent unsafe exposure levels if an annular stop fails.




\FloatBarrier
\subsection{Modifications of Imaging Path for Simultaneous brightfield and SFLIM imaging.}\label{subsec:modifyimaging}
In this section, the modifications to the imaging arm of the fundus camera to enable SFLIM imaging is described. The low pixel-count of current SPAD arrays presentes some interesting challenges for Widefield SFLIM imaging of the retina. A narrow FOV must be imaged to be able to resolve retinal vasulature and markers of disease - this reduced photon flux resulting in longer image acuisition time. The natural movements of the eye taking place over the acquisition period will introduce motion artefacts, degrading image quality which will need to be compensated for using an additional high frame-rate camera.
The imaging arm of the SFLIO instrument, shown in~\cref{fig:imagingoptics}, uses the Horiba FLIMera SPAD array (referred to as the FLIMera herein) to record fluorescence lifetime images over a $\cong\qty{12}{\degree}$ field-of-view. A separate scientific camera (Thorlabs Kiralux) to records fluorescence intensity or brightfield images over a \qty{35}{\degree} field-of-view to correct for motion of the retina during the long acquisition period. SLR lenses were used to relay ($f=\qty{100}{\mm}$), and magnify the latent image onto each detector: $f = \qty{50}{\mm}$  for the FLIMera; and  $f = \qty{85}{\mm}$ for the sCMOS detector. A 2”-diameter plate beamsplitter transmits \qty{10}{\percent} of the signal to the sCMOS camera and reflects \qty{90}{\percent} to the FLIMera. This allows enough light to record high-quality registration images at a frame-rate of at least \qty{10}{\hertz} and retain sufficient signal for SFLIM measurements.% Due to the difference in pixel-counts between the two detectors - $\qtyproduct{192 x 126}{\pixel}$ for the FLIMera and \qtyproduct{4096 x 2160}{\pixel} the field-of-view covered by the FLIMera was reduced by a linear factor of $\approx 3$ to improve the spatial resolution of the acquired images with the intention that if SFLIM images of larger areas of the retina the patient can fixate on their eyes away from centre.


\begin{figure}[htp]
    \centering
    \includegraphics[width = 0.9\textwidth]{figures/sflio-device/ImagingOpticsDiagramFinalVersion.pdf}
    \caption{Optical layout of the imaging arm of the SFLIO system. The retinal image at the image plane, $p_1$, is relayed using SLR lenses where: $f_1$ collimates the image; $f_{2}$ forms the reflected image from the 90:10 beam-splitter (BS) onto the FLIMera; and $f_{3}$ forms the transmitted image onto the sCMOS camera for co-registration of the moving retinal images. Registration images can be recorded using either in brightfield or fluorescence using a \qty{500}{\nm} long-pass filter.}
    \label{fig:imagingoptics}
\end{figure}


A linear cassette-style filter-wheel is fitted in front of $f_{3}=\qty{50}{\mm}$ lens - denoted by the label ``BP filter'' in~\cref{fig:imagingoptics} - to hold 5 emission bandpass filters for recording spectrally resolved fluorescence lifetime images. For recording fluorescence co-registration images a $\qty{500}{\nm}$ long-pass filter is placed at the location labelled as ``LP filter'' in~\cref{fig:imagingoptics}. These spectral filters were chosen to maximise photon throughput and capabilities to discriminate retinal fluorophores (see \cref{sec:tensband}) and the central wavelengths and bandwidths are shown in \cref{tab:sflio-filts}).
\begin{table}[htp]
    \centering
    \begin{tabular}{|c|c|c|}
        \hline
        Filter No.  & CWL (\unit{\nm}) & FWHM (\unit{\nm})\\
        \hline
        1 & 500 & 24\\
        2 & 549 & 15\\
        3 & 561 & 14\\
        4a & 640 & 75\\
        4b & 675 & 50\\
        \hline
    \end{tabular}
    \caption{Spectral filters used in the SFLIO system for enhanced discrimination of retinal fluorophores from spectrally-resolved fluorescence lifetime measurements. In spectral-FLIM measurements only 4 filters are used where either filter 4a or 4b is used}
    \label{tab:sflio-filts}
\end{table}

 Due to the stability of the rat's health under hypoxia for \textit{in-vivo} rat imaging (see \cref{chap:ratfad}), filter 4b was not used to maximise success of recording a complete SFLIM data set by reducing total image acquisition time. This filter was eliminated due to its narrower bandwidth, and lower photons throughput, when compared to filter 4a (\qty{50}{\nm} vs.\qty{75}{\nm}).

\FloatBarrier
\subsection{Assessment for Safely Imaging \textit{in-vivo} Human Retinas}
A critical design requirement for a retinal imaging device is that images are recorded with minimal discomfort. In the SFLIO device two similar pulsed-supercontinuum laser sources were used throughout this project, which both have the capability of producing broadband light at an intensity far in excess of safe operating levels. A full safety assessment was carried out, in collaboration with \textit{OPTOS Plc}, to determine the safe exposure levels and the required attenuation of the laser source such that even when operating at the maximum intensity, the exposure at the retina is safe. Although \textit{in-vivo} imaging of humans retinas was not achieved in this project, this assessment is still essential for determining the clinical efficacy of our device by comparing the laser intensity required for detecting FAD in rats \textit{in-vivo} (\cref{chap:ratfad}) to the safe exposure levels.
\subsubsection{Damage Mechanisms in the Retina - Phototoxic and Photothermal Effects}
When the eye is exposed to light there are two, wavelength-dependent mechanisms from which damage can occur - phototoxic and photothermal damage. Phototoxic damage predominantly occurs at wavelengths $ < \qty{500}{\nm} $ where absorbed photons cause oxidative reactions within the cells of the RPE, choroid, and the retina causing cellular death, and with sufficient exposure, results in permanent loss of vision~\cite{glickman2002phototoxicity}. Further, in the deep UV range of the spectrum ($<\qty{200}{\nm}$) these absorbed photons can directly cause the breaking of covalent bonds and at short pulse lengths ($<\qty{1}{\micro\second}$) tissue is ablated. This is the mechanism for LASIK surgery where the cornea is reshaped to correct cases of myopia, hyperopia, and astigmatism~\cite{reinstein2012history}. In photothermal damage, longer-wavelength photons cause heating of the tissue causing the tissue to become necrotic and, for severe rapid heating, the vitreous humour cavitates at the retina causing lesions~\cite{glickman2002phototoxicity,roberts2001ocular}.
The supercontinuum laser source used in the SFLIO device does not emit light below \qty{400}{\nm} so photochemical ablation was not a concern however the risks due to photochemical oxidation or photothermal damage mechanisms were considered.  The safe exposure thresholds, defined by the two established safety standard documents - \citeauthor{ISO15004,IEC60825} -were used to evaluate the safety of the instrument\cite{ISO15004,IEC60825}.

\FloatBarrier
\subsubsection{Safe Exposure Thresholds for Illuminating the Retina with a \\Pulsed-Supercontinuum Laser}
The safety of the SFLIO device was evaluated against two sets of safety standards: \citeauthor{ISO15004} and \citeauthor{IEC60825}~\cite{ISO15004, IEC60825}. The \citeauthor{ISO15004} standard describes the evaluation of safety for ophthalmic devices categorised as either Group 1 devices, where there is no risk of ocular damage, and Group 2 where some risk does exist but is still safe to use under strict guidelines. The \citeauthor{IEC60825} standard is defines the safety of laser systems designated from Class 1, which are considered eye-safe, up to Class 4B where even stray reflections from diffuse surfaces can pose a risk to eyesight. In the ISO standard the instrument can be evaluated against the criteria for a Group 1 continuous wave device due to a clause (see Note 1 in Table 3 of the ISO document) in stating that for total exposures longer than \qty{20}{\second}, the safe exposure limits are the same as if the supercontinuum laser was a continuous wave source. Under the IEC document, the emission from the SFLIO device was evaluated against the criteria for a Class 1 pulsed laser source. The IEC document contains stricter criteria for pulsed sources and, unlike the ISO standard, considers the additional damage mechanisms arising from the use of a short pulse-length and high repetition-rate laser source such as micro-cavitation, and self-focusing of the laser beam~\cite{rockwell2010ultrashort}.
A detailed description of the approach used determine the multiple safe exposure limits are listed in~\cref{app:sfliosafety}, however for both the IEC and ISO standards the exposure limits, the measured exposure, and the resulting safety factor are given and defined here.
\paragraph*{Assessing SFLIO Device Safety Against Criteria for a Group 1 Ophthalmic Device -\citeauthor{ISO15004}\\}
In the ISO standard the SFLIO instrument is evaluated as a Group 1 device where the exposure limits for this classification are defined in terms of a wavelength dependent irradiance, $E(\lambda)$, pertaining to a particular section of the eye - retina, cornea and lens, and the anterior segment. For each of these sections, where relevant, there are also weighting functions applied to account for the photothermal and photochemical damage mechanisms, as shown in \cref{fig:safetyspectra}c. In the safety assessment the spectral irradiance is defined using the power spectrum, $P(\lambda)$, and the area illuminated,$A$. 
\begin{equation}\label{eq:irrad}
    E(\lambda) = \frac{P(\lambda)}{A} \cong \frac{P_{M}S(\lambda)}{A \int_{0}^{\infty}S(\lambda)d\lambda}
\end{equation}
The power spectrum (power per unit wavelength) is determined from a unity-normalised measurement of the spectra of light emitted from the fundus camera ($S(\lambda)$) and a  measurement of the emitted optical power ($P_{M}$). The optical power meter applies a wavelength-dependent correction function to overcome the non-uniform response across the visible spectrum. However, the bandwidth of the excitation is sufficiently narrow (FWHM $<\qty{11.5}{\nm}$ - see \ref{fig:safetyspectra}b) that a single power measurement was recorded with the bias function set to be centred on \qty{468}{\nm} - the peak of the excitation. 
\begin{figure}
    \centering
    \begin{annotatedFigure}{\includegraphics[width = 0.8\textwidth]{figures/sflio-device/SafetyAssessmentSpectras.pdf}}
    \annotatedFigureText{0,0.95}{black}{0.3}{a)} 
    \annotatedFigureText{0.34,0.95}{black}{0.3}{b)} 
    \annotatedFigureText{0.68,0.95}{black}{0.3}{c)} 
    \end{annotatedFigure}
    \caption{The spectral output of the broadband output of the supercontinuum source was resampled from the calibration report supplied with the source in a) and in b) the output is filtered between \qtyrange{460}{467}{\nm} using an AOTF and the emission incident on the retina is measured using a spectrometer placed at the objective lens of the SFLIO device. The hazard weighting functions using in the assessment of the SFLIO device as a Group 1 continuous wave ophthalmic device is shown.}
    \label{fig:safetyspectra}
\end{figure}
The safety thresholds (see \cref{tab:conthresh}) were calculated with the optical power measured when the laser source was set to its maximum output power - $P_{M} = \qty{1.1}{\milli\watt}$. This exceeds the safe exposure limits for a Group 1 device by a factor \num{20} and thus a neutral density filter with $\text{OD} = \num{1.3}$, calculated using \cref{eq:OD}
\begin{equation}\label{eq:OD}
    P = P_{0}~10^{-\text{OD}}
\end{equation}
can be used to reduce the emission to satisfy the Group 1 safety limits. The neutral density filter was fitted into the illumination arm to attenuate the optical power incident on the retina to eye-safe levels of $P_{M}=\qty{46}{\micro\watt}$ and the safe exposure thresholds were recalculated with a safety factor.

\begin{table}[ht]
    \centering
    \resizebox{0.98\textwidth}{!}{
    \begin{tabular}{|p{3cm}|c|c|c|c|c|c|c|}
        \hline
        \textbf{Description} & \textbf{Symbol} & \textbf{Equation} & \textbf{Threshold} & \textbf{Measured} & \textbf{Unit} & \textbf{Safety Factor} & \textbf{Ref}\\
        \hline
        Weighted retinal irradiance & $E_{A-R}$ & $\sum_{350}^{700}E(\lambda)A(\lambda)\Delta\lambda$ & \num{220} & \num {178} & $\unit{\micro\watt\per\centi\metre^{2}}$ & \num{1.24} & $5.3.1.3a$ \\
        \hline
        Unweighted corneal and lenticular infrared irradiance & $E_{IR-CL}$ & $\sum_{770}^{2500}E(\lambda)\Delta\lambda$ & \num{20} & \num{0.061} & $\unit{\milli\watt\per\centi\metre^{2}}$ & \num{326} & $5.3.1.4$ \\
        \hline
        Unweighted anterior segment visible and infrared radiation irradiance & $E_{VIR-AS}$ & $\sum_{380}^{1200}E(\lambda)\Delta\lambda$ & \num{4} & \num{0.0026} & $\unit{\watt\per\centi\metre^{2}}$ & \num{1565} & $5.4.1.5$\\
        \hline
        Weighted retinal visible and infrared thermal irradiance & $E_{VIR-R}$ & $\sum_{380}^{1400}E(\lambda)R(\lambda)\Delta\lambda$ & \num{0.7} & \num{0.00028} &$\unit{\watt\per\centi\metre^{2}}$ & \num{2458} & $5.3.1.3a$ \\
        \hline
    \end{tabular}
    }
    \caption{Safe exposure limits for a Group 1 device under the \citeauthor{ISO15004} standard and the resulting exposure measured from the SFLIO device with the laser source operating at maximum power - filtered between \qtyrange{460}{467}{\nano\metre} - and a $\text{OD} = 1.3$ neutral density filter is fitted. The safety factor is calculated as $\nicefrac{\text{Threshold}}{\text{Value}}$. In the final column the reference to the equation to calculate the safe exposure limits is given.}
    \label{tab:conthresh}
\end{table}
When a neutral density filter with $\text{OD} = 1.3$ is fitted to the fibre output of the laser source the SFLIO device satisfies the criteria for a Group 1 ophthalmic device and can be used for continuous imaging for up to 2 hours.

\paragraph*{Assessing SFLIO Device Safety Against Criteria for A Class 1 Pulsed Laser Source\citeauthor{IEC60825}\\}
In the IEC standard the safe exposure limits, Accessible Emission Limits (AEL), were determined for a pulsed Class 1 laser source which is deemed safe for up to 8 hours of continuous exposure in a \qty{24}{\hour} period~\cite{IEC60825}. The AELs are calculated in terms of the energy deposited by: a single pulse; a single pulse within a train of pulses - weighted by a safety factor; and the emission from \qty{100}{\second} integrated into a single pulse. The lowest AEL is chosen for evaluation, $\text{AEL}_{s.p.T}$, and is compared against the power incident of $P=\qty{1.3}{\micro\watt}$ and converted to energy deposited per pulse - $E_{pulse} = P/f$ where $f$ is the repetition rate of the source.

\begin{table}[ht]
    \centering
    \resizebox{0.98\textwidth}{!}{
    \begin{tabular}{|p{6cm}|c|c|c|c|}
    \hline
    Description & Symbol & Threshold (\unit{\joule}) & Measured (\unit{\joule}) & Safety Factor \\
    \hline
    Accessible Emission Limit for a single pulse &  $AEL_{single}$ & \num{7.7e-8} & \num{5.75e-13} & \num{1.3e5}\\
    \hline
    Accessible Emission Limit for a single pulse averaged over a period of 100s & $AEL_{s.p.T}$ &\num{1.1e-12} & \num{5.75e-13} & \num{1.91}\\
    \hline
    Accessible Emission Limit for a single pulse in a train of pulses & $AEL_{s.p.train}$ &\num{3.08e-8} & \num{5.75e-13} & \num{5e4}\\
    \hline
    \end{tabular}
    }
    \caption{Comparison of calculated Accessible Emission Limits (AEL's) for a Class 1 laser system, with the measured energy per pulse of the SFLIO system after attenuating emitted power onto the retina with a $\text{OD} = 1.3$ neutral density filter. The safety factor is calculated as $\nicefrac{\text{Threshold}}{\text{Value}}$}
    \label{tab:pulsedvalues}
\end{table}
\subsubsection{Summary of Safety Assessment}
In this safety assessment the SFLIO device was evaluated using two safety standards where the excitation source is treated as both a continuous-wave source, and a pulsed source. To ensure to safe operation for \textit{in-vivo} imaging of human eyes a $\text{OD}=1.3$ neutral density filter was used, to attenuate the source to, ensure that the exposure is within the limits set out in both the \citeauthor{ISO15004} and \citeauthor{IEC60825} even when the laser is set to its maximum power output. 
This attenuates the emission produced by the SFLIO device within factors of 1.91 when treating the device as a Class 1 pulsed laser source according to~\citeauthor{IEC60825}. When the SFLIO device is evaluated a Group 1 ophthalmic device under the safety factor is 1.24 the instrument is evaluated as a Group 1 continuous wave sourced ophthalmic device under the~\citeauthor{ISO15004}. 
\FloatBarrier
\section{Measurement of Spectral Response of SFLIO Device}
In this Section, the spectral response of the SFLIO device and each bandpass filter is characterised to ensure robust and accurate recovery of fluorophore concentrations. The spectral response of the SFLIO device was characterised using a spectrometer positioned at the image plane of the FLIMera and illuminated by a xenon broadband white-light source positioned at the aperture of the objective lens of the fundus camera. Spectra were also measured for each bandpass filter, $I_n(\lambda)$, as well as a measurement of spectra produced by the xenon light source $D(\lambda)$. The transmission function of the optics within the SFLIO device for each spectral band, $F_n(\lambda)$ are calculated using \cref{eq:filttrans}
\begin{equation}\label{eq:filttrans}
    F_{n}(\lambda) = \frac{I_{n}(\lambda)}{D(\lambda)}
\end{equation}
the combined response of the xenon white light source ($D(\lambda)$) and these previous measurements $(I_n(\lambda))$(\cref{eq:filttrans})
\begin{equation}
    I_{n}(\lambda) = D(\lambda)F_{n}(\lambda)
\end{equation}

In \cref{fig:filtfuncs} the calibrated filter functions are shown after using \cref{eq:filttrans} and subtracting dark spectra to remove the influence of the spectrometer dark current and scattered background light. The non-uniformity in the spectral response of the fundus camera optics can be attributed to cumulative absorption from the anti-reflective coatings on the lenses within the fundus camera, as well as the SLR lenses and beam splitter in the imaging arm.
\begin{figure}[htp]
    \centering
    \includegraphics[width = 0.8\textwidth]{figures/sflio-device/FilterTransmissionFunctions.pdf}
    \caption{The calibrated filter functions for each of the spectral filters are plotted (coloured lines) are used to account for the wavelength-dependent attenuation of light transmitted through refractive optics within the fundus camera (dashed line) as well as the non-rectangular response of the spectral filters. Each filter function is normalised such that dividing the filter function by the transmission function of the fundus camera gives a flat response function with intensity of 1.}
    \label{fig:filtfuncs}
\end{figure}
In this section the spectral response of the SFLIO device was characterised and the transmission function of each spectral channel was measured. These transmission functions are used to ensure reliable and robust unmixing retinal fluorophore concentrations that will be discussed later in \cref{chap:ratfad}.
\FloatBarrier

%25-07-25 got to here refer to DanielGeddes-ThesisDraft-11Feb25 document that Andy has shared.
\subsection{Characterisation of Horiba FLIMera}\label{subsec:flimeradetails}
To record widefield FLIM images the SFLIO device uses a research prototype model of the Horiba FLIMera -  SPAD array with in-pixel TCSPC electronics for recording photon arrival times with single-photon sensitivity. Due to the pre-production nature of the FLIMera used, the SPAD array should be properly calibrated and described.
\subsubsection{FLIMera Architecture}
The FLIMera is a \qtyproduct{192 x 126}{\pixel} array of Single Photon Avalanche-Diodes with on-pixel TCSPC modules capable of widefield fluorescence lifetime imaging with \qty{47}{\ps} temporal resolution. Based upon a similar architecture to SPAD array of~\citeauthor{henderson2019192} the FLIMera has a pixel pitch of \qtyproduct{18.4 x 9.2}{\um} of which over half of the detector area is used for TCSPC electronics resulting in a fill-factor of \qty{13}{\percent}~\cite{henderson2019192}. 
\begin{figure}
    \centering
    \includegraphics[width= 0.5\textwidth]{figures/sflio-device/SPADarchitecture.png}
    \caption{Schematic layout of the architecture used to produce the FLIMera sensor. The FLIMera features \qtyproduct{192 x 126}{\pixel} individual SPADs each with its own Time-to-Digital-Converter enabling TCSPC measurements with a pixel pitch of \qtyproduct{18.4 x 9.2}{\um} or equivalently a pixel aspect ratio of 2:1. Reproduced from Fig.~2 of~\citeauthor{henderson2019192}\cite{henderson2019192}}
    \label{fig:SPADarch}
\end{figure}
Although this fill-factor is low when compared to traditional imagers it does improve upon the state-of-the-art in TCPSC imagers~\cite{cusini2022historical,gersbach2010high,ulku2018512} where the FLIMera boasts a higher fill-factor, and a scalable pixel architecture - providing a route to a mega-pixel FLIM imager.
Additionally, this fill factor can be improved upon using various established techniques at the fabrication stage. A micro-lens array would focus light incident within a pixel onto the smaller photosensitive area (see \cref{fig:SPADlayout}b) - increasing fill-factor to \qty{42}{\percent} in the case of the FLIMera~\cite{henderson2019192}. The fill-factor could also be increased using a multilayered, back-illuminated, architecture (see~\cref{fig:SPADlayout}c)  where the photon counting electronics are on a layer beneath the photosensitive area of the pixel.
\begin{figure}
    \centering
    \begin{annotatedFigure}{\includegraphics[width = \textwidth]{figures/sflio-device/SPADlayoutFigure.pdf}}
    \annotatedFigureText{0.01,0.80}{black}{0.3}{a)}
    \annotatedFigureText{0.32,0.8}{black}{0.3}{b)}
    \annotatedFigureText{0.67,0.8}{black}{0.3}{c)}
    \end{annotatedFigure}
    \caption{The FLIMera features a low fill-factor of \qty{13}{\percent} due to the TDCs being located on the same plane as the SPADs (a). To improve this up to \qty{42}{\percent} a micro-lens can be placed over each pair of pixels to focus light that would ordinarily impinge a TDC - and not be detected - onto the photosensitive area of the pixel \qty{42}{\percent}. Additionally, a multilayered fabrication approach, shown in (c), where the SPADs are layered on top of the TDCs and connected using vias (red circles) - interconnects between layers - can be used to improve not only the fill-factor but potentially allow for higher pixel densities.}
    \label{fig:SPADlayout}
\end{figure}
\FloatBarrier
\subsubsection{TCSPC Acquisition Modes}
The FLIMera has two acquisition modes: a default mode where a single $(x,y,t)$ data cube is recorded over the acquisition window; and the ``raw'' mode where now each photon detection event is streamed from the FLIMera at a rate of \qty{24}{\kilo\hertz} to form ``frames''. These photon records take the form of a 4-byte word which, from the least to most significant bit contains, the $x$ (\qty{8}{\bit}), $y$ (\qty{8}{\bit}), and time bin (\qty{10}{\bit}) coordinate of the photon detection. The final 4-bits are used as a frame marker $(1111)$ to denote the next frame in the acquisition of a photon record. This high frame-rate mode comes at a cost of larger file sizes where a 2-minute integration time in ``raw'' mode results in a \qty{25}{\giga\byte}, compared to the $\approx\qty{25}{\mega\byte}$ of the default acquisition mode. These larger file sizes requires some consideration to computational requirements for processing and storing SFLIM datasets. The high frame-rate ``raw'' mode is used in compensate for motion of the retina when recording SFLIM datasets. Continuous photon streams are recorded in parallel with synchronised high-resolution brightfield images at \qty{10}{\hertz}. Retinal motion is then compensated by detecting motion in these brightfield images and applying a correction in the $xy$ plane of the photon stream. This concept is explored and implemented in~\cref{chap:motionreg}.
\FloatBarrier
\subsubsection{Dead and Hot Pixel Masking, and Correcting for Temporal Misalignment between SPAD's}\label{sec:pixelmask}
The FLIMera used in this project is a pre-production model and exhibits malfunctioning pixels, that produce no signal or a DC signal as a response to incident photons, and pixels that are misaligned in the temporal domain. These aberrant pixels need to be characterised and corrected for to ensure the accurate recovery of fluorophore concentrations using the algorithms developed in~\cref{chap:tensSFLIM}.
Malfunctioning pixels, on the FLIMera, present as either: ``dead'' pixels which do not produce a signal in response to incident photons (cyan line of \cref{fig:pixel})c; or ``screamers'' which produce a histogram resembling a DC signal (purple line of \cref{fig:pixel}c). These errant pixels were excluded from analysis of FLIMera images using a pair of binary masks constructed from thresholding (see \cref{fig:pixel}d) the photon counts on two datasets: a uniformly fluorescent microscope slide (THORLABS - FSK4); and a dark frame. 

\begin{figure}
    \centering
    \begin{annotatedFigure}{\includegraphics[width = 0.8\textwidth]{figures/sflio-device/FLIMERA-MASK-FIGURE.pdf}}
    \annotatedFigureText{0.015,0.94}{black}{0.3}{a)}
    \annotatedFigureText{0.5,0.94}{black}{0.3}{b)}
    \annotatedFigureText{0.015, 0.4}{black}{0.3}{c)}
    \annotatedFigureText{0.5, 0.4}{black}{0.3}{d)}
    \end{annotatedFigure}
    \caption{To correct for malfunctioning pixels in the FLIMera (dead pixels and screamers) a uniformly fluorescent scene is imaged and (a) and a binary pixel mask is produced by thresholding the image (b). (c) Dead pixels (blue line) produce no output while the signal of a screamer resembles a DC signal with Heaviside activation (purple) and results in an increased number of detected photons when compared to a typical functioning pixel (green line). (d) The thresholding process to produce the pixel mask is performed using a histogram of the integrated photon counts where dead and screamers appear outside of the normally distributed intensities in the scene.}
    \label{fig:pixel}
\end{figure}
Another artefact of the pre-production nature of the FLIMera is the fluorescence decays for each pixel are not aligned in the time axis resulting in improper fitting of fluorescence lifetimes and poor recovery of retinal fluorophore concentrations. Correcting this misalignment can be achieved using a pixel offset mask (see \cref{fig:shiftmap}b) which describes how many time-bins each decay should be shifted by. This mask is produced by finding the time-location of the maxima in each pixel and subtracting it from a reference value which ensures the peak of histogram is shifted to this value.

\begin{figure}
    \centering
    \begin{annotatedFigure}{\includegraphics[width = \textwidth]{figures/sflio-device/ShiftMap-Figure.pdf}}
    \annotatedFigureText{0.03,0.96}{black}{0.3}{a)}
    \annotatedFigureText{0.53,0.96}{black}{0.3}{b)}
    \end{annotatedFigure}
    \caption{In (a) example histograms show the temporal misalignment within the FLIMera. These misalignments are corrected for by using a shift map (b) which contains the required shift along the time axis to mutually align every histogram}
    \label{fig:shiftmap}
\end{figure}

\FloatBarrier
\subsection{Measurement of Spatial Resolution using Knife Edge Test}
Accurately determining the smallest features or highest spatial frequency that can be resolved by an imaging system can often be an onerous task where best practises are continually updated ~\cite{ISO12233}. In this project only estimates of these factors are required to answer a key questions in assessing the imaging performance of the SFLIO device: Can common retinal features and disease signifiers, with dimensions on the order of \qty{50}{\um}, such as drusen~\cite{abdelsalam1999drusen}. Additionally, estimates of the of the spatial resolution of the SFLIM device helps determine the accuracy required when registering the FLIM and brighfield images.
For assessing these resolution, the knife-edge or slanted-edge test is used to measure the Modulation Transfer Function (MTF) and Line-Spread Function (LSF) of the SFLIO device. This was chosen primarily for its experimental simplicity~\cite{burns2000slanted, ISO12233}.
% Accurately measuring the Modulation Transfer Function,  of an imaging system is often an onerous task where standardised best practises are continually refined~\cite{ISO12233}. In this project only estimates of the MTF are required to determine two factors: If retinal features on the order of \SI{50}{\um} such as drusen~\cite{abdelsalam1999drusen} be resolved; and the required accuracy of the registration between the two detectors such that retinal motion can be compensated. For assessing the spation resolution of the SFLIO device the knife-edge test - often referred to as the slanted-edge test in literature - was chosen for its simplistic experimental procedure~\cite{burns2000slanted, ISO12233}. 
Briefly, this method enables the calculation of the MTF of a digital imager using an image recorded of an edge positioned in the image plane and rotated a few degrees ($<\qty{10}{\degree}$) with respect to the vertical or horizontal axes of the image. The resolution of the FLIMera arm of the SFLIO device is detector limited so image of this edge appears aliased. Line profiles through the aliased edge are then aligned to produce an Edge-Spread Function (ESF) and the spatial resolution of the imaging system can then be estimated from the LSF (\cref{eq:ESF})
\begin{equation}\label{eq:ESF}
    ESF(x) = \dv{}{x}LSF(x)
\end{equation}
or from the MTF (\cref{eq:LSFtoMTF}) which is connected to the LSF through the Fourier Transform
\begin{equation}\label{eq:LSFtoMTF}
    MTF(u) = \mathcal{F}{LSF(x)}
\end{equation}
The rotation of the edge in the image plane enables the ESF to be sampled with sub-pixel precision due each line profile - through the edge - being displaced by less than a fraction of a pixel in the direction perpendicular to the edge.
To perform the Knife-edge test a fluorescence microscope slide (Thorlabs - FSK4) was positioned at the retinal plane of a model eye and a scalpel blade was placed in front of this slide to create a fluorescence image with a high-contrast edge. Fluorescence intensity images were then recorded, in parallel, on both detectors to allow the MTF to be calculated for both imaging arms. The fluorescent slide emits uniform fluorescence that can be excited around \qty{460}{\nm} and imaged using a \qty{500}{\nm} long-pass filter. Integration times of \qty{250}{\ms} and \qty{5}{\second} were chosen for the FLIMera and sCMOS camera to ensure there was sufficient photon flux to record images with low noise. This process is shown in~\cref{fig:FIMTFresult,fig:FLMTFResult}, where in the analysis of the knife-edge images, the location of the edge, $\tilde{x}$ within the image is first estimated, to the nearest whole pixel, from line profiles, $L(x)$ through the edge
\begin{equation}\label{eq:edgelocest}
    \tilde{x} = argmax\Bigg\{\abs{\dv{}{x}L(x)}\Bigg\}
\end{equation}
Due to the effect of aliasing these locations do not represent the "true" location of the edge and so a straight line is fitted to these edge locations to increase the precision. Line profiles along the entire edge are then binned into \nicefrac{1}{4} pixel intervals and summed together to form a single ESF upsamples by a factor of $\nicefrac{1}{0.25} = \num{4}$. As above, the LSF and MTF can now be computed from the ESF in terms of cycles-per-pixel. Using an object of a known width - the LSF and MTF for both detectors are then converted to cycles-per-\si{\um}. Imagea recorded of a 0.05'' (\qty{1.27}{\mm}) Allen key positioned in the phantom eye where it was found one pixel on the FLIMera and sCMOS detector equated to \qty{66}{\um} and \qty{6.3}{\um} respectively. The spatial resolution of the system is determined by the full-width at half-maximum of a Gaussian fitted through the LSF. By fitting a Gaussian through the LSF the influence of quantisation noise is reduced. For the FLIMera the spatial resolution was found to be \qty{214}{\um} and for the sCMOS detector this found to be \qty{50}{\um}. The irregular pixel architecture of the FLIMera will induce more aliasing in the image and result in poorer spatial resolution.
\begin{figure}
    \centering
    \begin{annotatedFigure}{\includegraphics[width =\textwidth]{figures/sflio-device/FIMTFFigure.pdf}}
    \annotatedFigureText{0.025, 0.965}{black}{0.3}{a)}
    \annotatedFigureText{0.66,0.965}{black}{0.3}{b)}
    \annotatedFigureText{0.66,0.67}{black}{0.3}{c)}
    \annotatedFigureText{0.025,0.39}{black}{0.3}{d)}
    \annotatedFigureText{0.35,0.39}{black}{0.3}{e)}
    \annotatedFigureText{0.66,0.39}{black}{0.3}{f)}
    \end{annotatedFigure}
    \caption{From the recorded knife edge image using the sCMOS detector (a) the edge is detected and fitted (b) using the gradient of the edge profiles, shown in (c). The edge profiles are binned, and summed together, to form a super sampled ESF (d). The LSF is then determined from the ESF using \cref{eq:ESF} and the spatial resolution of \qty{214}{\um} is determined from the FWHM of a fitted Gaussian (black line). The MTF, shown in (f), is then calculated from this LSF using \cref{eq:LSFtoMTF} and is plotted together with the diffraction limit of a human eye.}
    \label{fig:FIMTFresult}
\end{figure}

\begin{figure}
    \centering
    \begin{annotatedFigure}{\includegraphics[width = 1\textwidth]{figures/sflio-device/FLKnifeEdgeFigure.pdf}}
    \annotatedFigureText{0.03,0.94}{black}{0.3}{a)}
    \annotatedFigureText{0.33,0.94}{black}{0.3}{b)}
    \annotatedFigureText{0.7,0.94}{black}{0.3}{c)}
    \annotatedFigureText{0.03,0.48}{black}{0.3}{d)}
    \annotatedFigureText{0.33,0.48}{black}{0.3}{e)}
    \annotatedFigureText{0.7,0.48}{black}{0.3}{e)}
    \end{annotatedFigure}
    \caption{From the recorded knife edge image using the FLIMera (a) the edge is detected and fitted (b) using the gradient of the edge profiles, shown in (c). The edge profiles are aggregated to form a super sampled ESF (d). The LSF is then determined from the ESF using \cref{eq:ESF} and the spatial resolution of \qty{50}{\um} is determined from the FWHM of a fitted Gaussian (black line). The MTF, shown in (f), is then calculated from this LSF using \cref{eq:LSFtoMTF} and is plotted with the diffraction limit of the eye.}
    \label{fig:FLMTFResult}
\end{figure}

\FloatBarrier
\subsection{Comparing Measured Spatial Resolution to Other Retinal Imaging Systems}
While the spatial resolution for both detectors is sufficient to co-register retinal images, the measured spatial resolution of the sCMOS detector is significantly poorer from the work presented by \citeauthor{mordant2011validation}\cite{mordant2011validation} where the same eye phantom was imaged using a similar system where a COTS fundus camera (Canon CF-60Z). In this case the eye phantom housed capillary tubes filled with blood for the purposes of measuring blood oxygenation in the retina~\cite{mordant2011validation}.
In~\citeauthor{mordant2011validation}\cite{mordant2011validation} the spatial resolution of their system was estimated to be $\approx\qty{30}{\um}$ or $\approx\qty{3}{\pixel}$ by examining line profiles through an image of a \qty{150}{\um} outer-diameter capillary tube and computing the number of pixels required to transition from \qty{10}{\percent} to \qty{90}{\percent} of the peak pixel grey-value - within the capillary. The peak grey value was taken as the white Spectralon on the eye phantom as shown in~\cref{fig:MordauntResolitionFigure}. The reference image used was recorded of blood at \qty{586}{\nm} which will have high-absorption and, according to the Beer-Lambert law, will have a narrow black-to-white transition width.

\begin{figure}
    \centering
    \begin{annotatedFigure}{\includegraphics[width = 0.9\textwidth]{figures/sflio-device/MordauntResolutionFigure.pdf}}
    \annotatedFigureText{0.035,0.90}{white}{0.3}{a)}
    \annotatedFigureText{0.52,0.90}{black}{0.3}{b)}
    \end{annotatedFigure}
    \caption{The line profile through a \qty{150}{\um} capillary tube filled with arterial blood and mounted in an eye phantom is shown in a) and b) record using a similar imaging system to the SFLIO device where an off-the-shelf fundus camera (Canon CF-60Z) is modified for hyper-spectral imaging for the purposes of measuring oxygen saturation of blood in the retina. In b), the number of pixels to transition from black-to-white or the rise-time, $\tau_{rise}$, is denoted by the solid black lines and was computed as $\tau_{rise} \cong \qty{30}{\um}$. Figure~a) reproduced from Fig.~3A of~\citeauthor{mordant2011validation}\cite{mordant2011validation}.}
    \label{fig:MordauntResolitionFigure}
\end{figure}

The sources of this discrepancy in spatial resolution was thought to originate from chromatic aberrations due to the 90:10 beam-splitter in the imaging arm and / or the \qty{500}{\nm} long pass emission filter degrading image contrast. A chequerboard target (pitch of \qty{10}{\mm}) was positioned at infinity ($>\qty{3}{\metre}$ from the objective lens of the fundus camera) and these sources were systematically removed from the imaging path. The black-to-white transition width was then be computed for each scenario, as shown~\cref{fig:sCMOStroubleshooting}, in terms of pixels and yielded a mean rise of \qty{4.5}{\pixel}. Since this difference in transition width ($\qty{3}{\pixel}:\qty{4.5}{\pixel}$) is proportionally similar to the discrepancy in spatial resolution ($\qty{30}{\um}:\qty{50}{\um}$) the optics of the fundus camera or the SLR lenses used in the imaging arm are the limiting factor and not the beam-splitter or emission filter.

\begin{figure}
    \centering
    \begin{annotatedFigure}{\includegraphics[width = 0.8\textwidth]{figures/sflio-device/SpatialResTroubleShootFigure.pdf}}
    \annotatedFigureText{-0.02,0.99}{black}{0.3}{a)}
    \annotatedFigureText{-0.02,0.46}{black}{0.3}{c)}
    \annotatedFigureText{0.475,0.99}{black}{0.3}{b)}
    \annotatedFigureText{0.475,0.46}{black}{0.3}{d)}
    \end{annotatedFigure}
    \caption{Images recorded of a checker board patter illuminated with a white light Xenon source using the SFLIO device to ascertain the source of degradation in the spatial resolution of the fluorescence intensity imaging arm. Line profiles were recorded across the checker board pattern and the rise time of the contrast was computed. In a) A \qty{500}{\nm} long pass filter is in place for fluorescence intensity imaging with sCMOS detector. The black-to-white transition width was \qty{5}{\pixel}, In b) the 90:10 beam-splitter used for simultaneous imaging with the FLIMera was removed which resulted in a lower rise time of \qtyrange{3}{4}{\pixel}. In c) the \qty{500}{\nm} long pass filter was removed, emulating brightfield imaging, and the rise time was computed as \qty{5}{\pixel}. In d) both the  \qty{500}{\nm} long pass filter and the 90:10 beam-splitter was removed resulting in a rise-time of \qty{4}{\pixel}}
    \label{fig:sCMOStroubleshooting}
\end{figure}

While efforts could be made to improve the spatial resolution of the SFLIO system to match this stated goal of resolving small drusen, having dimensions $<\qty{63}{\um}$ it was deemed sufficient for quantitative mapping of retinal fluorophores. Further, while the assessment of spatial resolution of the system was treated with a sufficient degree of rigour, the knife-edge test was implemented for its relatively high accuracy when compared to the time and complexity cost. Characterisations using Siemens Star and US Air force test targets embedded in the eye phantom could have been employed to determine the MTF of the imaging system for both axes of the image - the Knife-edge test only allows the MTF to be calculate for the axes perpendicular to the edge.
Additionally, the irregular pixel aspect ratio and layout of the FLIMera was not accounted for in this analysis. Existing methods using fringe patterns, produced by a Twyman-Green interferometer, projected onto the detector to accurately sample its MTF~\cite{greivenkamp1994modulation} could be used for this. In practice the SFLIO system is optically limited and not detector limited so while these methods would yield more precise and accurate measurements of the MTF and spatial resolution it would not fundamentally change the performance of the systems biochemical resolution nor the requirement for adequate image registration.
\FloatBarrier
\subsection{Assessing Sensitivity of Spatial Resolution with Changing Focus}
When adjusting the focus on the SFLIO device it was observed that near the expected optimal focus, the change in image sharpness on the FLIMera was insensitive to large changes in focus. To assess whether both imaging arms were correctly co-focussed the knife-edge test was then repeated at $\approx\qty{10}{degrees}$ intervals of the focus adjustment knob for an additional 3 measurements either side of what as initially deemed to be optimal focus. This ensures that what the was assessed to be the optimal focus is where optimal value for spatial resolution is measured.
\begin{figure}[htp]
    \centering
    \includegraphics[width = 0.6\textwidth]{figures/sflio-device/MTFFociFigure.pdf}
    \caption{The knife-edge test was repeated over a total of 7 different focii's centred on what was originally deemed to be the best focus and spaced $\approx\qty{10}{\degree}$ apart. The spatial resolution for each focii and detector was calculated, and a degree 2 polynomial was fitted.}
    \label{fig:mtffoci}
\end{figure}
The results shown in \cref{fig:mtffoci} that in fact the optimal focus was actually located $\approx\qty{10}{degree}$ from what was visually determined to be the best focus. At this minimum the spatial resolution of the sCMOS detector is \qty{50}{\um} and for the FLIMera it is \qty{214}{\um}. At the extrema, the spatial resolutions are \qty{296}{\um} and \qty{328}{\um} for the FLIMera, and for sCMOS they are \qty{129}{\um} and \qty{152}{\um}. Due to the spatial resolution worsening at the same rate, for both detectors, with focus position this shows that the focus for each SLR lens (lenses with focal lengths, $f_{1}$ to $f_{3}$) is set correctly. In the experiments described in later sections of this chapter the focus is set to this new optimal focus with only minor adjustments being made to account for variations in the position of the eye phantom with respect to objective lens of the fundus camera.
