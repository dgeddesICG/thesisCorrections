In this appendix the safety of the SFLIO device is evaluated using the \citeauthor{IEC60825} and \citeauthor{ISO15004} safety documents and is split into multiple sections.~\cref{sec:sysdes} covers the modifications made to a standard imaging ophthalmoscope to allow fluorescence lifetime imaging.~\cref{sec:expthresh} will outline the potential damage mechanisms when illuminating the retina with a laser source and the methods for ensuring safe exposure levels.~\cref{sec:pulsedsafety} and ~\cref{sec:cwsafety} will evaluate the safety of our system under the criteria for a pulsed and continuous wave laser source, respectively.
\FloatBarrier
\section{System Design}\label{sec:sysdes}
Fluorescence imaging is routinely used in the diagnosis, and monitoring of retinal disease by detecting changes in fluorescent biomarkers in the retina~\cite{bernstein2019fluorescence}. In principle, this allows for the enhanced detection of disease by instead of examining structural changes in the retina, after damage as already occurred, we can detect changes in the biochemistry. However, the feint signal produced by the fluorophores of interest is often swamped by more strongly fluorescing clutter this hinders abilities to quantify individual fluorophores and results in noisy images. 
Spectral-Fluorescence Lifetime Imaging is a new technique which can discriminate mixtures of multiple fluorescent compounds using properties inherent to the respective fluorophore, namely its fluorescence emission spectra, and its fluorescence lifetime – the average time taken for a fluorophore to decay~\cite{scipioni2021phasor}.

Our system uses a standard Topcon TRC 50DX fundus camera with the imaging, and illumination paths being modified for exciting fluorescence with a supercontinuum laser, and recording fluorescence lifetime images. A pulsed, supercontinuum laser source is coupled into the existing annular illumination system of the fundus camera using a dichroic mirror. This modification allows for illuminating the retina using our laser source, with a smaller field of view, while retaining the use of the existing inspection lamp to allow for aligning the patient/volunteer during imaging (See~\cref{fig:modification}). The supercontinuum laser source (NKT Super Extreme EX12W) produces a broadband output (\cref{fig:nktspectra}) with a pulse length $<\SI{100}{\pico\second}$ and repetition rate of $\SI{80}{\mega\hertz}$. An Acousto-Optically Tuneable Filter (AOTF) reduces the broadband output into a single excitation band between $\num{460}-\SI{467}{\nano\metre}$ with the dichroic mirror further attenuating any light from the laser source $<\SI{605}{\nano\metre}$. 
To ensure safe operation, the laser source will be set to $100\%$ power output and a neutral density filter is used to attenuate this power to eye safe levels.


\begin{figure}
    \centering
    \begin{subfigure}[b]{0.45\textwidth}
        \centering
        \includegraphics[width=\textwidth]{Safety Assessment/IlluminationOpticsDiagram.pdf}
        \caption{}
        \label{subfig:illumoptic}
    \end{subfigure}
    \begin{subfigure}[b]{0.45\textwidth}
        \includegraphics[width=\textwidth]{Safety Assessment/AnnularIllumination.pdf}
        \caption{}
        \label{subfig:annular}
    \end{subfigure}
    \caption{For FLIO imaging a Topcon TRC 50DX fundus camera is modified to allow excitation of fluorescence using a pulsed, supercontinuum white light source. a) shows the modifications made to the existing illumination optics to allow for fluorescence imaging over a $\approx\SI{20}{\degree}$ field of view. b) shows the effective illumination from the laser source (blue) and the existing tungsten inspection lamp used for aligning the patient/volunteer during imaging.}
    \label{fig:modification}
\end{figure}

\begin{figure}
    \centering
    \includegraphics[width = 0.6\textwidth]{Safety Assessment/LaserSpectra.png}
    \caption{Spectral output of NKT SuperK Extreme 12W supercontinuum laser used in modified fundus camera for fluorescence lifetime imaging of the retina.}
    \label{fig:nktspectra}
\end{figure}
\FloatBarrier
\section{Exposure Thresholds}\label{sec:expthresh}
Photochemical damage occurs predominately due to over exposure of short wavelength light where free radicals are created within cells of the RPE, the choroid, and the retina (See \cref{fig:anatomyeye}). These damaged cells are either removed as part of an immune response or die – resulting in loss of vision~\cite{glickman2002phototoxicity}. For our device the photochemical risk will be the most pertinent since we are exciting fluorescence between \SI{450}{\nano\metre} and \SI{480}{\nano\metre}.
Photothermal damage occurs due to visible, and infrared light causing a rise in temperatures within the eye. In cases of extreme heating of the vitreous humour, the retina can be permanently damaged due to cavitation~\cite{roberts2001ocular}. The safe exposure limits specify which section of the eye they relate to: retina, corner, or the anterior segment as well as the mode of damage - phototoxic or photothermal.
\begin{figure}[htbp]
    \centering
    \includegraphics[width = 0.5\textwidth]{Safety Assessment/AnatomyofEye.png}
    \caption{Anatomy of the eye. The anterior segment is comprised of structures such as the cornea, iris, lens, and the ciliary body.}
    \label{fig:anatomyeye}
\end{figure}

Two sets of guidelines are used to determine the safety of our instrument. The BS EN ISO 150004-2:2007 document describes procedures and safe exposure limits for ophthalmic imaging devices. Here devices are categorised as either a Group 1 device or a Group 2 device. A Group 1 device poses no risk to a patient/volunteers eyesight. A Group 2 device, while safe for occasional use, does pose a potential risk to the eyesight of a patient/volunteer(See \cref{sec:cwsafety}). 
This set of guidelines is used to assess the safety of our device when treated as a continuous source under the restrictions for Group 1 devices. 
The IEC 60825-1:2014 document is used to asses the safety of our instrument when considered as a pulsed source. While this document is not specifically intended for ophthalmic imaging devices it does provide more rigid definitions of safety thresholds that are better suited for the source used in our instrument - short wavelength, short pulse length, and high repetition rate. Here the safety thresholds pertain to the classification given to a laser. For example a class 1 laser is safe for direct exposure of over $\SI{8}{\hour}$ whereas a Class 2 laser is only safe up to direct exposures of $<\SI{0.25}{\second}$ due to the eyes blink response. Lasers that are Class 3 and above pose significant risk to eyesight from accidental reflections.
In this safety assessment the criteria for a Class 1 laser system will be used to evaluate the safety of our device(See~\cref{sec:pulsedsafety}).
\FloatBarrier
\section{Pulsed Source Safety Assessment - IEC}\label{sec:pulsedsafety}
For pulsed sources the IEC 60825-1:2014 document is used to assess the safety of our system using the criteria for a Class 1 laser system.
\\
The safety thresholds, or Accessible Emission Limits (AEL), for a Class 1 source are used to show our system is eye-safe - similar to he Group 1 demarcation used in the BS EN ISO 150004-2:2007 document. 
The AEL's for pulsed sources are defined in three ways 1) The AEL for a single pulse, 2) The AEL for a single pulse in a train of pulses, and 3) The AEL over a defined time base. When assessing our system the most restrictive safety limit is used. Section 4.3f) of the safety guidelines are used to aid the calculation of these terms.

\begin{table}[htbp]
    \centering
    \begin{tabular}{|c|c|c|c|c|}
        \hline
         Symbol & Description & Value & Units\\
        \hline
        $\lambda$ & Wavelength of excitation source & $460-470$ & \si{\nano\metre}\\
        $f$ & Laser Pulse rate & 80 & \si{\mega\hertz}\\
        $t_{pulse}$ & Pulse Length & $<\num{100}$ & \si{\pico\second}\\
        $P_{AVG}$ & Average laser power at excitation wavelength & \num{46} & \si{\micro\watt}\\
        $P_{PK}$ & Peak laser power at excitation wavelength & \num{1.1} & \si{\milli\watt}\\
        $T_{2}$ & Timebase that the AEL's are assessed over & \num{100} & \si{s}\\
        $T_i$ & Period which all pulses in a train are grouped together & \num{5} & \si{\micro\second}\\
        $\alpha$ & Angular subtense of the system & \num{536} & \si{\milli \radian}\\
        $\alpha_{min}$  & Minimum angular subtense for an extended source & \num{1.5} & \si{\milli\radian}\\
        $\alpha_{max}$ & Maximum angular subtense for an extended source&  \num{100}& \si{\milli\radian}\\
        \hline
    \end{tabular}
    \caption{Table of variables and quantities used in determining the Accessible Emission Limit's (AEL) for a Class 1 laser source to assess the safety of our system.}
    \label{apptab:vals}
\end{table}

\subsection{Calculations of AEL's}
The AEL's for a Class 1, pulsed, laser source are calculated using Section 4.3f) and Example B3.5 in Annex B. 

% Using a power meter the average power meter, $P_{AVG}$ was measured over the entire excitation band as \SI{1.1}{\milli\watt}. The energy per pulse, $E_{pulse}$ is then:
% \begin{align*}
%     E_{pulse} &= \frac{P_{AVG}\cdot\SI{1}{\second}}{f} = \frac{\SI{1.1e-3}{\watt}\cdot\SI{1}{\second}}{\SI{80e6}{\hertz}}\\
%     E_{pulse} &=\SI{1.375e-11}{\joule}
% \end{align*}

From Table 3 the AEL for a single pulse for a pulse length of \SI{100}{\pico\second}, and wavelength between \SI{450}{\nano\metre} and \SI{500}{\nano\metre} is:
\begin{align}
    AEL_{single} = \SI{7.7e-8}{\joule}\\
\end{align}
Since $\nicefrac{1}{f} < T_{i}$ then all the pulses within this time window of $T_{i} = \SI{5e-6}{\second}$ must be grouped together:
\begin{align}
    AEL_{T_{i}} = \num{7e-4}\cdot T_{i}^{0.75}\si{\joule} = \SI{7.4e-8}{\joule}
\end{align}
The AEL for pulses over a duration $T = \SI{100}{\second}$ averaged together to calculate the $AEL_{T}$ term:
\begin{align}
    AEL_{T} &= \num{3.9e-5}\cdot C_{3}\si{\watt}\\
    AEL_{T} &= \num{3.9e-5}\cdot 10^{0.02(468-450)}\si{\watt}\\
    AEL_{T} &= \SI{8.9e-5}{\watt}
\end{align}
As a single pulse this in terms of the energy deposited in a single 'pulse'
\begin{align}
    AEL_{s.p.T} &= \frac{AEL_{T}}{f} =\frac{\SI{8.9e-5}{\watt}}{\SI{80e6}{\hertz}}\\
    AEL_{s.p.T} &= \SI{1.1e-12}{\joule}
\end{align}

Finally, the AEL for a single pulse in the train, $AEL_{s.p.train}$:

\begin{align}
    AEL_{s.p.train} &= AEL_{single}\cdot C_{5}
\end{align}
The correction factor $C_{5}$ uses the number of pulses over the period, $T_{2}$, and has a minimum of \num{0.4}:

\begin{align}
    C_{5} &= 5\cdot N^{-0.25} = 5\cdot (f\cdot T_{2})^{-0.25}\\
    C_{5} &= 0.017 < 0.4 \implies C_{5} = 0.4
\end{align}
The $AEL_{s.p.train}$ is then:
\begin{align}
    AEL_{s.p.train} &= AEL_{single}\cdot C_{5} \\
    AEL_{s.p.train} &= \SI{3.08e-8}{\joule}
\end{align}

To summarise, the Accessible Emission limits have been calculated for a pulsed Class 1 laser system. To determine the equivalent laser class of our system, and thus assess the safety of the system, the most restrictive AEL will be used - $AEL_{s.p.T} = \SI{1.1e-12}{\joule}$. In the next section the energy emitted by our system will be measured according to the measurement conditions set out in IEC document.

\begin{table}
    \centering
    \resizebox{0.98\textwidth}{!}{
    \begin{tabular}{|c|c|c|}
    \hline
    \textbf{Symbol} & \textbf{Description} & \textbf{Value} (\si{\joule})\\
    \hline
        $AEL_{single}$ & Accessible Emission Limit for a single pulse & \num{7.7e-8} \\
        $AEL_{sp.T}$ & Accessible Emission Limit for a single pulse averaged over a period of 100s & \num{1.1e-12}\\ 
        $AEL_{s.p.train}$ & Accessible Emission Limit for a single pulse in a train of pulses & \num{3.08e-8}\\
        \hline
    \end{tabular}
    }
    \caption{Accessible Emissions Limits (AEL) pertaining to a Class 1 laser system with a pulse duration of $<\SI{100}{\pico\second}$ and wavelength of \SI{480}{\nano\metre} }
    \label{apptab:aelsum}
\end{table}

\subsection{Measurement of Accessible Emissions of the system}
The accessible emission from the system - or the energy deposited onto the retina - must be measured under two conditions for retinal imaging devices: 1) Under normal use for recording retinal images and 2) when the patient / volunteer moves \SI{100}{\milli\metre} from the ideal/usual imaging position. (See Condition 3 in \cref{apptab:conditions}). Condition 3 captures the risk to retinal health from a diverging source of light focusing onto a potentially diffraction limited sport size The classification of the system is then determined using the most restrictive condition - the lowest AEL coupled with the highest measured power.



\subsubsection{Spectral Measurements}
The NKT supercontinuum laser source used in our system has the potential to cause severe phototoxic, and photothermal damage due to its broadband emission - as shown in \cref{subfig:superk}. For the excitation of flourophores the this broadband emission is reduced down to a single \SI{11.5}{\nano\metre} FWHM band using an AOTF. The spectral output of the system is measured using a Ocean Optics Spectrometer to ensure non-desirable wavelengths are properly attenuated as shown in \cref{subfig:fcspectra}. The FWHM of the source is \SI{11.5}{\nano\metre} which is sufficiently narrow that it can be treated as a single wavelength for the purposes of measuring optical power and determining the accessible emission limits. 


\begin{figure}[htbp]
    \centering
    \begin{subfigure}[b]{0.45\textwidth}
    \centering
    \includegraphics[width = \textwidth]{Safety Assessment/LaserSpectra.png}
    \caption{NKT SuperK supercontinuum source}
    \label{subfig:superk}
    \end{subfigure}
    \hfill
    \begin{subfigure}[b]{0.45\textwidth}
    \centering
    \includegraphics[width = \textwidth]{Safety Assessment/FundusCameraSpectra-460-467nm.png}
    \caption{Filtered Spectra}
    \label{subfig:fcspectra}
    \end{subfigure}
    \caption{Spectra measured from a) NKT SuperK supercontinuum and b) output of fundus camera system when filtered between $\num{460}-\SI{467}{\nano\metre}$ using an AOTF}
\end{figure}

\subsubsection{Optical Power Measurements}
The accessible emission of the system is evaluated using an optical power meter to measure the optical power emitted from our system under two conditions - 1) Normal measurement conditions and 2) where the volunteer/ patient has moved away from the usual imaging position by \SI{100}{\milli\metre} - this captures the risk of diverging light focusing down into a diffraction limited point. 
\\
Under normal imaging conditions the accessible emission of our system is evaluated by considering the total optical power emitted from our system with the angular subtense, $\alpha$, and the angle of acceptance of the eye, $\gamma$ (See \cref{app:angularsubtense}). 
The optical power was measured as $P_{AVG} = \SI{46}{\micro\watt}$ and the angles $\alpha$, = \SI{1077}{\milli\radian} and $\gamma = \SI{259}{\milli\radian}$ were calculated. 
Under these conditions the energy per pulse form our system can be calculated as simply:
\begin{align}
    E_{pulse} &= \frac{P_{AVG}}{f} = \frac{\SI{46e-6}{\joule}}{\SI{80e6}{\hertz}}\\
    E_{pulse} &= \SI{5.75e-13}{\joule}
\end{align}
Under Condition 3 the optical power was measured by positioning the \SI{7}{\milli\metre} aperture of a mechanical eye \SI{100}{\milli\metre} away from the normal imaging position and position the detector head of a optical power meter at the retina. Due to the diverging nature of the beam out of the objective lens of the fundus camera, the measured optical power is considerably lower - $P_{AVG,3} = \SI{1.3}{\micro\watt}$. Since the angular subtense, $\alpha_{3}$ is still above the stated value of $\alpha_{max} = \SI{100}{\milli\radian}$, the calculated AEL's do not change

\begin{align}
    E_{pulse,3} &= \frac{P_{AVG,3}}{f} = \frac{\SI{1.3e-6}{\joule}}{\SI{80e6}{\hertz}}\\
    E_{pulse,3} &= \SI{1.625e-14}{\joule}
\end{align}


\begin{table}[htbp]
    \centering
    \resizebox{0.98\textwidth}{!}{
    \begin{tabular}{|p{6cm}|c|c|c|c|}
    \hline
    Description & Symbol & Threshold (\si{\joule}) & Measured (\si{\joule}) & Safety Factor \\
    \hline
    Accessible Emission Limit for a single pulse &  $AEL_{single}$ & \num{7.7e-8} & \num{5.75e-13} & \num{1.3e5}\\
    Accessible Emission Limit for a single pulse averaged over a period of 100s & $AEL_{s.p.T}$ &\num{1.1e-12} & \num{5.75e-13} & \num{1.91}\\
    Accessible Emission Limit for a single pulse in a train of pulses & $AEL_{s.p.train}$ &\num{3.08e-8} & \num{5.75e-13} & \num{5e4}\\
    \hline
    \end{tabular}
    }
    \caption{Comparison of calculated Accessible Emission Limits (AEL's) for a Class 1 laser system, with the measured energy per pulse of our FLIO system. The safety factor is defined as the ratio of the Threshold value and the measured value.}
    \label{apptab:pulsedvalues}
\end{table}

\begin{table}[htbp]
    \centering
    \begin{tabular}{|c|c|c|}
    \hline
    \multirow{2}{*}{}
         & \bm{$\alpha$} & \bm{$\gamma$}\\
         &$(\si{\milli\radian})$& $(\si{\milli\radian})$\\
         \hline
         \textbf{Condition 1}& 1077 & 259\\
         \textbf{Condition 3}& 372 & 119\\
         \hline
    \end{tabular}
    \caption{Angular subtense $\alpha$, and angle of acceptance, $\gamma$, of FLIO system under Condition 1 - normal imaging conditions, and Condition 3 - where the patient has moved \SI{100}{\milli\metre} away}
    \label{apptab:angles}
\end{table}

\begin{table}[htbp]
    \centering
    \resizebox{0.98\textwidth}{!}{
        \begin{tabular}{|c|c|c|c|c|c|}
        \hline
        &\multicolumn{2}{c|}{\textbf{Condition 1}} & \textbf{Condition 2}& \multicolumn{2}{c|}{\textbf{Condition 3}}\\
        \hline
        \multirow{3}{*}{}        \textbf{Wavelength}&\textbf{Aperture}&\textbf{Distance}&&\textbf{Aperture stop/}&\textbf{Distance}\\
        &\textbf{stop}&&&\textbf{limiting aperture}&\\      \si{\nano\metre}&\si{\milli\metre}&\si{\milli\metre}&&\si{\milli\metre}&\si{\milli\metre}\\
        \hline
        $< 302.5$ & - & - && $1$ & $0$\\
        \hline
        $\geq 302.5 \,\text{to}\, 400$ & $7$ & $2000$ & & $1$ & $100$\\
        \hline
        \multirow{2}{*}{$\geq 400\,\text{to}\, 1400$} & $50$ & $2000$  & See Note 1 under & $7$ & $100$\\
        &&& 5.4.1& & \\
        \hline 
        \multirow{3}{*}{$\geq 1400\,\text{to}\, 4000$} & $7 \times \text{Condition 3}$ & $2000$& See Note 1 under & $1 \,\text{for}\, t \leq \SI{0.35}{\second}$ & $100$\\
        &&& 5.4.1 & $1.5 t^{\nicefrac{3}{8}}\,\text{for}\, \SI{0.35}{\second}< t < \SI{10}{\second}$ & \\
        &  &  &  & $3.5 \,\text{for}\, t \geq \SI{10}{\second}\, (t \,\text{in}\, \si{\second})$& \\
        \hline
        \multirow{3}{*}{$\geq 4000\,\text{to}\, \num{1e5}$} & - & - &  & $1 \,\text{for}\, t \leq \SI{0.35}{\second}$ & $0$\\
        &  &  &  & $1.5 t^{\nicefrac{3}{8}}\,\text{for}\, \SI{0.35}{\second}< t < \SI{10}{\second}$ & \\
        &  &  &  & $3.5 \,\text{for}\,t \geq \SI{10}{\second}\, (t \,\text{in}\, \si{\second})$& \\
        \hline
        $\geq \num{1e5} \,\text{to} \, \num{1e6}$ & - & - & & $11$ & 0\\
        \hline
        \end{tabular}
        }
        \caption{Conditions for measuring the accessible emission of a laser system for determine source classification}
        \label{apptab:conditions}
\end{table}

\subsection{Conclusion}
Since we have demonstrated that the energy per pulse, $E_{pulse}$, emitted from our system is below the AEL's for a Class 1 device described in the IEC 60825-1:2014 we can conclude that our system satisfies the criteria for a Class 1 laser system and thus does not pose a risk to the eyesight of a patient/volunteer under normal imaging conditions.
\FloatBarrier
\section{Continuous Wave Source Safety Assessment- BS}\label{sec:cwsafety}
The long image acquisition times required for FLIO imaging ($>\SI{20}{\second}$) coupled with the high repetition rate, short pulse length, and short wavelength mean that as per the ISO guidelines, the device should also be considered as a continuous wave source. The document breaks down the safe exposure limits to the specific damage mechanism, phototoxic or photo thermal, and to the particular section of the eye that is most likely to be damages: retina, cornea and lens, and the anterior segment.
\\
The phototoxic and photothermal damage mechanisms are wavelength dependent and this is represented in the calculations to determine the limits of safe exposures as the weighting functions $A(\lambda)$ for the photothermal damage mechanism and $R(\lambda)$ for the phototoxic damage mechanisms. These functions, shown in \cref{fig:hazardfunc}, are weighted towards the wavelengths where the eye is susceptible to damage e.g. phototoxic damage is more prevalent in the UV-to-blue wavelengths.

\begin{figure}[htbp]
    \centering
    \includegraphics[width = 0.6\textwidth]{Safety Assessment/HazardfunctionsPlot.png}
    \caption{Photothermal and Phototoxic hazard weighting functions used in the calculations of safe exposure limits for the retina. Reproduced from ISO 15004-2:2007 Annex1 Table A.1}
    \label{fig:hazardfunc}
\end{figure}

As previously mentioned, our device will be assessed against the criteria for a Group 1 device. The limits of safe exposure are given in terms of the irradiance - the incident optical power per unit area illuminated - and are shown in \cref{apptab:conthresh}. 
For the phototoxic and photothermal hazard to the retina this is given as $E_{A-R}$, and $E_{VIR}$. In this case the threshold for short wavelength light presents a greater hazard - this is reflected in a lower limit of safe exposure.
The safe exposure limit for the cornea and lens, $E_{IR-CL}$, is unweighted but only pertains to light in the infrared area of the spectra.
Likewise, the safe exposure limit for the anterior segment, $E_{VIR-R}$, is also unweighted.


\begin{table}[htbp]
    \centering
    \resizebox{0.98\textwidth}{!}{
    \begin{tabular}{|p{6cm}|c|c|c|c|c|}
        \hline
        \textbf{Description} & \textbf{Symbol} & \textbf{Equation} & \textbf{Value} & \textbf{Unit} & \textbf{Ref}\\
        \hline
        Weighted retinal irradiance & $E_{A-R}$ & $\sum_{350}^{700}E(\lambda)A(\lambda)\Delta\lambda$ & $220$ & $\si{\micro\watt\per\centi\metre^{2}}$ & $5.3.1.3a$ \\
        \hline
        Unweighted corneal and lenticular infrared irradiance & $E_{IR-CL}$ & $\sum_{770}^{2500}E(\lambda)\Delta\lambda$ & $20$ & $\si{\milli\watt\per\centi\metre^{2}}$ & $5.3.1.4$ \\
        \hline
        Unweighted anterior segment visible and infrared radiation irradiance & $E_{VIR-AS}$ & $\sum_{380}^{1200}E(\lambda)\Delta\lambda$ & $4$ & $\si{\watt\per\centi\metre^{2}}$ & $5.4.1.5$\\
        \hline
        Weighted retinal visible and infrared thermal irradiance & $E_{VIR-R}$ & $\sum_{380}^{1400}E(\lambda)R(\lambda)\Delta\lambda$ & $0.7$ & $\si{\watt\per\centi\metre^{2}}$ & $5.3.1.3a$ \\
        \hline
    \end{tabular}
    }
    \caption{Table of safe exposure limits for a Group 1 device under the ISO 15004:2007.}
    \label{apptab:conthresh}
\end{table}

With the laser source operating at maximum power, there is an optical power of $P_{M} = \SI{1.1}{\milli\watt}$, emitted from our system which is incident on the retina. The Weighted retinal irradiance, $E_{A-R}$, for this scenario is then $E_{A-R} = \SI{4245}{\micro\watt\per\centi\metre\squared}$ - exceeding the safe exposure limit by $\approx 20 \times$. A neutral density filter, of $\text{OD} = 1.3$ is fitted to attenuate the light emitted from the system to safe levels.
The attenuated power is measured using the optical power meter to be $P_{M} = \SI{46}{\micro\watt}$. The various irradiances are calculated and shown in \cref{apptab:cwvalues}:

\begin{table}[htbp]
    \centering
    \resizebox{0.98\textwidth}{!}{
    \begin{tabular}{|p{6cm}|c|c|c|c|c|}
    \hline
    Description & Symbol & Measured & Threshold & Unit & Safety Factor\\
    \hline
    Weighted retinal irradiance & $E_{A-R}$ & \num{178} & \num{220} & \si{\micro\watt\per\centi\metre\squared} & 1.24\\
    Unweighted corneal and lenticular infrared irradiance & $E_{IR-CL}$ & \num{0.061} & \num{20} & \si{\milli\watt\per\centi\metre\squared} & 326\\
    Unweighted anterior segment visible and infrared radiation irradiance & $E_{VIR-AS}$ & \num{0.0026} & \num{4} & \si{\watt\per\centi\metre\squared} & \num{1565}\\
    Weighted retinal visible and infrared thermal irradiance & $E_{VIR-R}$ & \num{0.00028} & \num{0.7} & \si{\watt\per\centi\metre\squared} & \num{2458}\\
    \hline
    \end{tabular}
    }
    \caption{Thresholds for safe exposure, and those measured from FLIO system. A safety factor is defined as Measured value / Threshold value}
    \label{apptab:cwvalues}
\end{table}

\subsection{Conclusion}
As can be seen from the above table (\cref{apptab:cwvalues}) the measured irradiances are below the thresholds for a Group 1 device. This means that our system satisfies the criteria described in the \citeauthor{ISO15004} for a Group 1 device and thus poses no risk to the eyesight of a volunteer/patient under normal imaging conditions.

\FloatBarrier
\section{Additional Information}
\subsection{Calculation of Angular Subtense and Angle of Acceptance}\label{app:angularsubtense}

The angular subtense of our system is defined as the angle subtended by the objective lens of the fundus camera and the cornea whereas the angle of acceptance is defined as the angle of subtended by the laser source, exiting the objective lens, and the cornea. This is illustrated in \cref{fig:appangularsubtense}. 
\\
\begin{figure}[htbp]
    \centering
    \includegraphics[width = 0.6\textwidth]{Safety Assessment/AngularSubtenseDiagram.pdf}
    \caption{The angular subtense of our imaging system, $\alpha$, is defined as the angle subtended by the objective lens of the fundus camera and the cornea of a volunteer/patient. The angle of acceptance, $\gamma$, is the angle subtended by the converging laser source and the cornea.}
    \label{fig:appangularsubtense}
\end{figure}
Using calipers the diameter of the objective lens on the fundus camera, $d_{image}$, and the distance between the objective lens and the patients cornea (under normal imaging conditions), $z_{source}$
The angular subtense and the angle of acceptance can then be calculated:
\begin{align}
    \alpha &= 2\arctan{\bigg(\frac{d_{source}}{2z_{source}}\bigg)} = 2\arctan{\bigg(\frac{\SI{55}{\milli\metre}}{2\cdot \SI{46}{\milli\metre}}\bigg)} = \SI{1077}{\milli\radian}\\
    \gamma &= 2\arctan{\bigg(\frac{d_{source}}{2z_{source}}\bigg)} = 2\arctan{\bigg(\frac{\SI{12}{\milli\metre}}{2\cdot \SI{46}{\milli\metre}}\bigg)} = \SI{259}{\milli\radian}
\end{align}
For Condition 3 (see Condition 3 in Table \ref{apptab:conditions}) the volunteer moves \SI{100}{\milli\metre} away from the typical imaging position i.e $d_{source} = \SI{46}{\milli\metre} + \SI{100}{\milli\metre} = \SI{146}{\milli\metre}$
The angular subtense, and angle of acceptance, $\alpha_{3}$ and $\gamma_{3}$ are then:
\begin{align}
    \alpha_{3} &= 2\arctan{\bigg(\frac{d_{source}}{2z_{source}}\bigg)} = 2\arctan{\bigg(\frac{\SI{55}{\milli\metre}}{2\cdot \SI{146}{\milli\metre}}\bigg)} = \SI{372}{\milli\radian}\\
    \gamma_{3} &= 2\arctan{\bigg(\frac{d_{source}}{2z_{source}}\bigg)} = 2\arctan{\bigg(\frac{\SI{12}{\milli\metre}}{2\cdot \SI{146}{\milli\metre}}\bigg)} = \SI{82}{\milli\radian}
\end{align}

\subsection{Power Spectra Measurement Derivation}
The ISO document describes the safe exposure limits in terms of the wavelength dependent, spectral irradiance $E(\lambda)$. In practice these can be determined using the power spectrum, $P(\lambda)$, and the area of the retina illuminated:
\begin{align}
    E(\lambda) &= \frac{P(\lambda)}{Area}
\end{align}
The power spectrum is determined by combining a measurement of the spectral output of our system, using an Ocean Optics spectrometer (Fig. \ref{subfig:fcspectra}), and an optical power meter to measure the power emitted.
The power meter applies a wavelength dependent bias function to overcome the non-uniform sensitivity of the detector but since the spectral output of our system is sufficiently narrow ($\text{FWHM} < \SI{11.5}{\nano\metre}$) we can set this bias to be centred on our excitation ($\lambda = \SI{468}{\nano\metre}$)
The power spectrum is then expressed as:
\begin{align}
    P(\lambda)&\equiv\frac{P_{M}\cdot S(\lambda)}{\int_{0}^{\infty}S(\lambda)d\lambda}
\end{align}
As an example, the weighted retinal irradiance, $E_{A-R}$, is calculated as:
\begin{align}
    E_{A-R} &= \sum_{305}^{700}E(\lambda)A(\lambda)\Delta\lambda\\
    \implies E_{A-R} &= \frac{P_{M}}{Area}\frac{\sum_{305}^{700}S(\lambda)A(\lambda)\Delta\lambda}{\int_{0}^{\infty}S(\lambda)d\lambda}
\end{align}
This is the approach that will be used in evaluating the safety of our device according to the Group 1 criteria listed in \citeauthor{ISO15004} in \cref{sec:cwsafety}

\subsection{Method for determining illuminated areas}
To calculate the safe exposure levels for continuous wave case of our device the area of the eye that is illuminated must be determined. In the calculation of $E_{A-R}$, and $E_{VIR-R}$ the area considered is the area of the retina that is illuminated under normal imaging conditions. 
This illuminated area is calculated using the solid angle of the beam produced by the fundus camera $\Omega_{sys}$ and asserting this is equal to the solid angle of the beam, $\Omega_{eye}$:
\begin{align}
    \Omega_{sys} &=\frac{A_{source}}{z_{source}^{2}}= \Omega_{eye}\\
    \frac{A_{source}}{z_{source}^{2}}&= \frac{A_{eye}}{z_{eye}^{2}}\\
    A_{eye} &= \frac{1}{4}\pi d_{source}^{2}\bigg(\frac{z_{eye}}{z_{source}}\bigg)^{2}\\
\end{align}
With $z_{eye} = \SI{17}{\milli\metre}$, $z_{source} = \SI{46}{\milli\metre}$, and $d_{source} = \SI{12}{\milli\metre}$ the area of the retina illuminated is:
\begin{align}
    A_{eye} &=\SI{0.15}{\centi\metre\squared}
\end{align}

The area illuminated on the cornea is used in the calculation of $E_{IR-CL}$ and $E_{VIR-AS}$ and is determined by measuring the diameter of the spot illuminated in the corneal plane under normal imaging conditions. With an illuminated spot size $d_{cornea} = \SI{1.5}{\milli\metre}$:
\begin{align}
    A_{cornea} = \frac{1}{4}\pi d_{cornea} = \SI{0.0177}{\centi\metre\squared}
\end{align}

\subsection{Tables from IEC 60825-1:2014}\label{app:iectables}

\begin{figure}[htb!]
    \centering
    \includegraphics[width = 0.7\textwidth]{Safety Assessment/Figures from IEC doc/section43a.png}
    \caption{Table 4.3a from the IEC safety Standard}
    \label{fig:iecsection4.3a}
\end{figure}

\begin{figure}[htb!]
    \centering
    \includegraphics[width = 0.7\textwidth]{Safety Assessment/Figures from IEC doc/section43b.png}
    \caption{Table 4.3b from the IEC safety Standard}
    \label{fig:iecsection4.3b}
\end{figure}

\begin{figure}[htb!]
    \centering
    \includegraphics[width = 0.7\textwidth]{Safety Assessment/Figures from IEC doc/Table2.png}
    \caption{Table 2 from the IEC safety Standard}
    \label{fig:iectable2}
\end{figure}

\begin{figure}[htb!]
    \centering
    \includegraphics[width = \textwidth]{Safety Assessment/Figures from IEC doc/table3a.png}
    \caption{Table 3a from the IEC safety Standard}
    \label{fig:iiectable3a}
\end{figure}

\begin{figure}[htb!]
    \centering
    \includegraphics[width = \textwidth]{Safety Assessment/Figures from IEC doc/Table3b.png}
    \caption{Table 3b from the IEC safety Standard}
    \label{fig:iectable3b}
\end{figure}

\begin{figure}[htb!]
    \centering
    \includegraphics[width = 0.7\textwidth]{Safety Assessment/Figures from IEC doc/Table9.png}
    \caption{Table 9 from the IEC safety Standard}
    \label{fig:iectable9}
\end{figure}
